\chapter*{特征值与特征向量}

\Section{求特征值和特征向量}

\subsection{抽象矩阵}

对 $ r(A) = 1 \Leftrightarrow A =\alpha\beta^\top $ 的矩阵,有

\begin{enumerate}
    \item $ A^n = \textrm{tr}(A)^{n-1}\alpha\beta^\top; $ 
    \item $ \lambda_1 = \textrm{tr}(A), \alpha_1 = \alpha; $ 
    $ \lambda_2,\cdots,\lambda_n = 0, \alpha_i $ 是 $ \beta^\top X = 0 $ 的解,共 $ n - 1 $ 个。
    \item $ A $ 可相似对角化,当且仅当 $ \textrm{tr}(A) \neq 0; $ 
    \item $ A = \sum \lambda_i\gamma_i\gamma_i^\top = \textrm{tr}(A)\gamma_1\gamma_1^\top, $ 
    其中 $ \gamma_i $ 是经过正交化的特征向量。
\end{enumerate}

实对称矩阵特征向量相互之间是正交的。对三阶实对称矩阵,
\begin{itemize}
    \item \textbf{知二求一}
    
    求向量积 $ \alpha_1\times \alpha_2, $ 即 
    $ \begin{vmatrix}
        a_1&b_1&c_1\\ a_2&b_2&c_2\\ i&j&k
    \end{vmatrix} $ 
    \item \textbf{知一求二}
    
    取合适的 $ a,b,c $ 使得 $ \alpha_2 = (a, b, 0)^\top, \alpha_3 = (-b, a, c) $ 与已知的 $ \alpha_1 $ 
    两两正交。
\end{itemize}

\subsection{具体矩阵}

以 $ |A - \lambda E| = 0 $ 求特征值,$ |A-\lambda_0 E|X = 0 $ 求对应特征向量。

\begin{enumerate}
    \item[\textbf{例题}] 设 $ \alpha_1 = (1,1,0)^\top, \alpha_2 = (1,0,1)^\top $ 为三阶矩阵 $ A $ 对应
    $ \lambda_1 = 1, \lambda_2 = 2 $ 的特征向量,
    
    $ \beta = (-1,1,-2)^\top, $ 求 $ A^2\beta. $ 
    \item[\textbf{方法}]
    用已知的特征向量线性\textbf{表示}给定的 $ \beta. $ 具体而言,
    \begin{equation*}
        \begin{aligned}
            \beta = \alpha_1 - 2\alpha_2 
            \Leftrightarrow  A^2\beta = A^2\alpha_1 - 2A^2\alpha^2 
            = \alpha_1 - 8\alpha_2.
        \end{aligned}
    \end{equation*}
\end{enumerate} 

\Section{相似}

\begin{enumerate}
    \item[\textbf{例题}]
    设 $ 3 $ 阶矩阵 $ A $ 有三个不同的特征值 $ \lambda_i, $ 对应的特征向量是 $ \alpha_i, $ 
    $ \beta = \sum \alpha_i. $ 
    \begin{enumerate}
        \item 证明 $ \beta $ 不是 $ A $ 的特征向量。
        \item 证明 $ \beta,A\beta,A^2\beta $ 线性无关。
        \item 若 $ A^3\beta = 2A\beta + A^2\beta, $ 求 $ A $ 的特征值,求 $ |A+2E|. $ 
    \end{enumerate}
    \item[\textbf{方法}]
    \begin{enumerate}
        \item 反证即可。
        \item 
        由于 $ (\beta,A\beta,A^2\beta) = (\alpha_1,\alpha_2,\alpha_3)
        \begin{pmatrix}
            1 & \lambda_1 & \lambda_1^2\\
            2 & \lambda_2 & \lambda_2^2\\
            3 & \lambda_3 & \lambda_3^2\\
        \end{pmatrix}, $ 而后二者都满秩,故前者满秩,故三向量无关。
        \item 通过构造 $ P $ 并通过化简得到等式 $ AP = PB $ 以找到具体的相似矩阵。
        
        具体而言,令 $ P = (\beta,A\beta,A^2\beta), $ 则
        \begin{equation*}
            \begin{aligned}
                AP &= (A\beta,A^2\beta,(2A+A^2)\beta)\\ 
                &= (\beta,A\beta,A^2\beta)\begin{pmatrix}
                    0&0&0\\
                    1&0&1\\
                    0&1&2\\
                \end{pmatrix}
            \end{aligned}
        \end{equation*}
        那么有 $ P^{-1}AP = B, $ 那么,$ \lambda_A = \lambda_B, |A+2E| = |B+2E|. $ 
    \end{enumerate}
\end{enumerate}

\sssubsection{$ AB^{-1} $ 的快速求法}

\begin{equation*}    
    \begin{aligned}
        \left(
        \begin{array}{c}
            P_2 \\\hdashline  P_1
        \end{array}
        \right)
        \xlongrightarrow{\textrm{列变换}}
        \left(
        \begin{array}{c}
            E \\\hdashline P_1P_2^{-1}
        \end{array}
        \right)
    \end{aligned}
\end{equation*}

类似地,

\begin{equation*}    
    \begin{aligned}
        \left(
        \begin{array}{c:c}
            P_1 & P_2
        \end{array}
        \right)
        \xlongrightarrow{\textrm{行变换}}
        \left(
        \begin{array}{c:c}
            E & P_1^{-1}P_2
        \end{array}
        \right)
    \end{aligned}
\end{equation*}

\sssubsection{相似对角化结合传递性及其快速求法}

若有 $ P_1^{-1}AP_1 = \Lambda = P_2^{-1}AP_2, $ 则有
$ B = (P_1P_2^{-1})^{-1}A(P_1P_2^{-1}). $ 

事实上,若有三阶 $ A,B, $ 存在 $ P = (\alpha_1,\alpha_2,\alpha_3) $ 使得 $ AP = PB, $ 那么
\begin{equation*}
    \begin{aligned}
        AP = PB \Leftrightarrow A(\alpha_1,\alpha_2,\alpha_3) = (\alpha_1,\alpha_2,\alpha_3)B
    \end{aligned}
\end{equation*}

由此可以得到三个方程组,分别求出 $ \alpha_1,\alpha_2,\alpha_3, $ 
即可得到 $ P $ 使得 $ P^{-1}AP = B. $ 

\begin{enumerate}
    \item[\textbf{例题}] 设 $ A = \begin{pmatrix}
        2&0&0\\0&0&1\\0&1&0\\
    \end{pmatrix}, B=\begin{pmatrix}
        1&0&0\\0&-1&0\\0&-6&2\\
    \end{pmatrix} $ 
    求 $ P $ 使得 $ P^{-1}AP = B. $ 
    \item[\textbf{方法}]
    设 $ P = (\alpha_1,\alpha_2,\alpha_3), $ 那么有
    \begin{equation*}
        \begin{aligned}
            AP = PB &\Rightarrow
            A(\alpha_1,\alpha_2,\alpha_3) = (\alpha_1,\alpha_2,\alpha_3)B \\ 
            &\Rightarrow 
            A(\alpha_1,\alpha_2,\alpha_3) = (\alpha_1,-\alpha_2-6\alpha_3,2\alpha_3)
        \end{aligned}
    \end{equation*}
    那么有
    \begin{equation*}
        \begin{aligned}
            &(A-E)\alpha_1 = 0\\
            &(A-2E)\alpha_3 = 0\\ 
            &(A+E)\alpha_2 = -6\alpha_3
        \end{aligned}
    \end{equation*}
    分别求解,解得
    \begin{equation*}
        \begin{aligned}
            \alpha_1 = (0,1,1)^\top, \alpha_3 = (1,0,0)^\top,
            \alpha_2 = k(0,-1,1)^\top+(-2,0,0)^\top, k\in \mathbb{R}.
        \end{aligned}
    \end{equation*}
    由于 $ k = 0 $ 时,$ P $ 不满秩,令 $ k = 1, $ 得到
    $ P = \begin{pmatrix}
        0&-2&1\\1&-1&0\\1&1&0\\
    \end{pmatrix} $ 满足题设。
\end{enumerate}

\begin{enumerate}
    \item[\textbf{例题}] 设 $ A $ 为三阶矩阵,三维列向量 $ \alpha_1,\alpha_2,\alpha_3 $ 线性无关,
    且 $ A\alpha_1 = \alpha_1+\alpha_2+\alpha_3, 
    A\alpha_2 = 2\alpha_2+\alpha_3,
    A\alpha_3 = 2\alpha_2+3\alpha_3, $ 
    求可逆矩阵 $ P $ 使得 $ P^{-1}AP $ 为对角阵。
    \item[\textbf{方法}]
    由于
    \begin{equation*}
        \begin{aligned}
            A(\alpha_1,\alpha_2,\alpha_3) = (\alpha_1,\alpha_2,\alpha_3)
            \begin{pmatrix}
                1&0&0\\1&2&2\\1&1&3\\
            \end{pmatrix}
        \end{aligned}
    \end{equation*}
    设 $ \begin{pmatrix}
        1&0&0\\1&2&2\\1&1&3\\
    \end{pmatrix} = B, $ 
    那么 $ A,B $ 相似。

    由于 $ B = \begin{pmatrix}
        0&0&&\\1&1&2\\1&1&2
    \end{pmatrix}+E=(0,1,1)^\top(1,1,2) + E, $ 
    前者特征值为 $ 3,0,0, $ 有 $ B $ 的特征值为 $ 4,1,1. $ 
\end{enumerate}

\sssubsection{相乘为零的矩阵的性质}

若 $ AB = O, $ 有
\begin{itemize}
    \item $ r(A)+r(B)\leq n; $ 
    \item $ B $ 的列向量是 $ AX=0 $ 的解;
    \item $ B $ 的非零列向量是 $ A $ 的特征值为 $ 0 $ 对应的特征向量;
    \item $ A $ 的行向量与 $ B $ 的列向量正交。
\end{itemize}

\sssubsection{矩阵的二次方程有两个互异的根}

若 $ n $ 阶矩阵 $ A $ 满足
\begin{equation*}
    \begin{aligned}
        A^2 - (\lambda_1+\lambda_2)A + \lambda_1\lambda_2E = 
        (A-\lambda_1E)(\lambda_2-A) = 0
    \end{aligned}
\end{equation*}

则\begin{itemize}
    \item $ r(A-\lambda_1E) + r(\lambda_2E-A) = n; $ 
    \item $ A $ 可以相似对角化。
\end{itemize}

\Section{实对称矩阵的计算}

\subsection{求正交矩阵}

\begin{enumerate}
    \item 特征值 $ \Rightarrow \Lambda = \begin{pmatrix}
        \lambda_1&&\\&\ddots&\\&&\lambda_n
    \end{pmatrix} $ 
    \item 特征向量 $ \Rightarrow P = (\alpha_1,\cdots,\alpha_n) $ 
    \item 单位化 $ \Rightarrow Q = (\gamma_1,\cdots,\gamma_n) $ 
\end{enumerate}

\subsection{求实对称矩阵}

\begin{itemize}
    \item $ P^{-1}AP = \Lambda, A = P\Lambda P^{-1} $ 
    \item $ Q^\top AQ = \Lambda, A = Q\Lambda Q^\top $ 
    \item 分解定理 - $ A = \dis\sum\lambda_i\gamma_i\gamma_i^\top $ 
    \item 特别地,$ r(A) = 1 $ 时,$ A = \textrm{tr}(A)\gamma_1\gamma_1^\top $ 
\end{itemize}

\begin{enumerate}
    \item[\textbf{定理}] 设 $ A $ 为 $ n $ 阶矩阵,则 $ A $ 可正交相似对角化 $ \Leftrightarrow A $ 为实对称矩阵。
    \item[\textbf{证明}] 
    由于 $ A $ 可正交相似对角化 $  \Rightarrow Q^\top AQ =\Lambda \Rightarrow
    A = Q\Lambda Q^\top \Rightarrow A^\top = Q^\top \Lambda Q = A, $ 
    故 $ A $ 为实对称矩阵。
\end{enumerate}

\begin{enumerate}
    \item[\textbf{例题}] 设 $ A $ 为三阶实对称矩阵,正交矩阵 $ Q = (\gamma_i) $ 使得 $ Q^\top AQ = \begin{pmatrix}
        2&&\\&3&\\&&4
    \end{pmatrix}, $ 求 $ A - \gamma_1\gamma_1^\top $ 的特征值。 
    \item[\textbf{方法}] 
    由于 $ A = \dis \sum\lambda_i\gamma_i\gamma_i^\top 
    \Rightarrow A - \gamma_1\gamma_1^\top = 
    1\gamma_1\gamma_1^\top + 
    3\gamma_2\gamma_2^\top + 
    4\gamma_3\gamma_3^\top $ 
    因此特征值分别为 $ 1,3,4. $ 
\end{enumerate}

\begin{enumerate}
    \item[\textbf{例题}] 令 $ A = \begin{pmatrix}
        1&1&1\\1&3&1\\1&1&1
    \end{pmatrix}, $ 求正交矩阵 $ Q $ 使得 $ Q^\top (A+A^*)Q $ 为对角矩阵。

    $ ※ $  需要 $ A $ 与 $ A^* $ 同时正交相似对角化。
    \item[\textbf{方法}]
    
    可以求出正交矩阵 $ Q = (\gamma_i) $ 令 $ Q^\top AQ \whichis \Lambda = \begin{pmatrix}
        0&&\\&1&\\&&4
    \end{pmatrix}. $ 其中,
    \begin{equation*}
        \begin{aligned}
            \gamma_1 = \dfrac{1}{\sqrt 2}(1,0,-1)^\top,
            \gamma_2 = \dfrac{1}{\sqrt 3}(1,-1,1)^\top,
            \gamma_3 = \dfrac{1}{\sqrt 6}(1,2,1)^\top.
        \end{aligned}
    \end{equation*}
    此时,由于
    \begin{equation*}
        \begin{aligned}
            A = Q\Lambda Q^\top \Rightarrow A^* = (Q\Lambda Q^\top)^*
            = (Q^\top)^*\Lambda^*Q^* \\ 
            Q^\top A^*Q = Q^\top(Q^\top)^*\Lambda^*Q^*Q = |Q^\top|\Lambda^*|Q|
            = \begin{pmatrix}
                4&&\\&0&\\&&0
            \end{pmatrix}.
        \end{aligned}
    \end{equation*}
    那么 $ Q^\top(A+A^*)Q = Q^\top AQ + Q^\top A^*Q = \begin{pmatrix}
        4&&\\&1&\\&&4
    \end{pmatrix}. $ 
    \item[\textbf{方法}] \textbf{表格法推广}
    
    对三阶 $ A $ 的伴随矩阵 $ A^*, $ 有
    \begin{table}[!htbp]\centering
        \begin{tabular}{ccc}
        \toprule
        $ A^* $ & $ \lambda $ & $ \alpha $  \\ \midrule
        1 & $ \lambda_2\lambda_3 $ & $ \alpha_1 $  \\
        2 & $ \lambda_1\lambda_3 $ & $ \alpha_2 $  \\
        3 & $ \lambda_1\lambda_2 $ & $ \alpha_3 $  \\ \bottomrule
        \end{tabular}
    \end{table}

    那么,若 $ A $ 对应的特征值是 $ 0,1,4, $ 由表可以知道
    $ A^* $ 对应的特征值是 $ 4,0,0, $ 由此可以快速地计算
    $ A $ 与 $ A^* $ 的对角矩阵。
\end{enumerate}

\begin{enumerate}
    \item[\textbf{例题}]
    设 $ A = \begin{pmatrix}
        5&-3&3\\-1&7&1\\5&5&3
    \end{pmatrix}, $ 求矩阵 $ C $ 使得 $ A = C^3. $ 
    \item[\textbf{方法}] \Attention{分解出秩为一的矩阵}
    
    对矩阵 $ A = \begin{pmatrix}
        x&\square&a\\y&\square&b\\\square&\square&\square
    \end{pmatrix}, $ 
    尝试将其分解为 $ \begin{pmatrix}
        z&\square&a\\y&\square&b\\\square&\square&\square
    \end{pmatrix}_B + kE $ 使得 $ z,y $ 与 $ a,b $ 等比例。

    此时可以通过 $ B $ 求 $ A $ 特征值特征向量。由于 $ A = C^3, C $ 的特征值 $ \lambda_C = \dsqrt[3]{\lambda_A}. $ 
\end{enumerate}

\begin{enumerate}
    \item[\textbf{例题}]
    设 $ A = \begin{pmatrix}
        5&-3&3\\-3&5&3\\3&3&5
    \end{pmatrix}, $ 求矩阵 $ C $ 使得 $ A = C^3. $
    \item[\textbf{方法}] 
    由于 $ A $ 是实对称矩阵,可以相似对角化,因此
    可以直接求 $ Q $ 使得 $ Q^\top AQ = \Lambda. $ 
\end{enumerate}