\newpage
\chapter{答案及注意事项}
\section{练习}

%\hypertarget{T1}{}
\from{E1}
\begin{Answer}[求$ n $ 时,求出一个定值\getback{E1}]
    \begin{equation*}
        \begin{aligned}
            \dis {\displaystyle\lim_{n\rightarrow \infty}}\sin\sqrt{4n^2+n}\pi &=
            {\displaystyle\lim_{n\rightarrow \infty}}\sin(2n+\sqrt{4n^2+n}-2n)\pi \\
            &= {\displaystyle\lim_{n\rightarrow \infty}}\sin(\sqrt{4n^2+n}-2n)\pi \\
            &= {\displaystyle\lim_{n\rightarrow \infty}}\sin\dis\frac{n}{\dis\sqrt{4n^2+n}+2n}\pi \\
            &= \sin \frac{\pi}{4} = \frac{\sqrt 2}{2}
        \end{aligned}
    \end{equation*}
\end{Answer}

\begin{answer}[660T9]{通过换元法将 $ x $ 剔除出积分式}
    令 $ s = xt $ ,则有
    \begin{equation*}
        \begin{aligned}
            I&={\displaystyle\lim_{x\rightarrow 0}}
            \dfrac{\dis \int_{x^3}^{x^2}\dfrac{\sin s}{s}\mathrm{d}s}{x^2}\\&=
            {\displaystyle\lim_{x\rightarrow 0}} \dfrac{\dis \dfrac{\sin x^2}{x^2}\cdot 2x - 
            \dfrac{\sin x^3}{x^3}\cdot 3x^2}{2x} \\&= {\displaystyle\lim_{x\rightarrow 0}}
            \dfrac{2\sin x^2 - 3\sin x^3}{2x^2} = 1
        \end{aligned}
    \end{equation*}
\end{answer}

\begin{answer}[660T11]{不要先想象分类讨论的结果再补画靶子}
    令原极限 $ = I $ ,$ t = \dfrac{1}{x} $ ,则有
    \begin{equation*}
        \begin{aligned}
            I &= \exp\left\{{\displaystyle\lim_{t\rightarrow +\infty}}\dfrac{\ln \dfrac{t+1}{t^2}}{t^a}\right\}\\&=
            \exp\left\{{\displaystyle\lim_{t\rightarrow +\infty}}\dfrac{\ln(t+1)-2\ln t}{t^a}\right\}\\&=
            \exp\left\{{\displaystyle\lim_{t\rightarrow +\infty}}\dfrac{\dfrac{1}{t+1}-\dfrac{2}{t}}{at^{a-1}}\right\}
            \\&= \exp\left\{{\displaystyle\lim_{t\rightarrow +\infty}}\dfrac{-t-2}{a(t+1)t^{a+1}}\right\}\\&=
            \exp\left\{{\displaystyle\lim_{t\rightarrow +\infty}}\dfrac{-1}{a(a+1)t^{a+1}+a^2t^a}\right\} = 1
        \end{aligned}
    \end{equation*}
\end{answer}

\begin{answer}[660T17]{对根式,令 $ t=-\dfrac{1}{x^6}, x\rightarrow \infty  $,然后泰勒展开 }
    由于
    \begin{equation*}
        \begin{aligned}
            \dsqrt[3]{1-x^6} &= -x^2\dsqrt[3]{1-\dfrac{1}{x^6}} \\&\xlongequal{t = -\dfrac{1}{x^6}}
            -x^2\left(1+\dfrac{1}{3}t+o(t)\right) = -x^2\left(1-\dfrac{1}{3x^6}+o(x^{-6})\right)
        \end{aligned}
    \end{equation*}
    代回原极限,发现当且仅当 $ a = -1,b = 0 $ 时原极限成立。
\end{answer}

\begin{answer}[660T21]{利用一次洛必达法则后不要忘记分母的次数为 $ n-1 $ }
    由题,\begin{equation*}
        \begin{aligned}
            {\displaystyle\lim_{x\rightarrow 0}}\dfrac{F(x)}{x^n}&=
            {\displaystyle\lim_{x\rightarrow 0}}\dfrac{\dis \int_0^{x-\sin x}\ln(1+t)\mathrm{d}t}{x^n}\\&=
            {\displaystyle\lim_{x\rightarrow 0}}\dfrac{ln(1+x-\sin x)(1-\cos x)}{nx^{n-1}}\\&=
            {\displaystyle\lim_{x\rightarrow 0}}\dfrac{\frac{1}{6}x^3\cdot\frac{1}{2}x^2 }{nx^{n-1}} = a
        \end{aligned}
    \end{equation*}
    其中 $ a $ 是常数,故 $ n-1 = 5 $ ,即 $ n = 6 $ 。
\end{answer}

\begin{answer}[660T27]{注意题目要求的是单点还是函数}
    由 $ f(x) $ 在定义域的可导性,有 $ f(x) $ 在 $ x=0 $ 处可导,因此有 $ {\displaystyle\lim_{x\rightarrow 0}}f(x) = -1 $ ,
    故 $ b = -1 $ 。当 $ x \not=0 $ 时,可以直接得出 $ f'(x) = \dfrac{x-(x-1)\ln(1-x)}{(x-1)x^2} $ ;
    $ x=0 $ 时,由定义,\begin{equation*}
        \begin{aligned}
            f'(0) &= {\displaystyle\lim_{x\rightarrow 0}}\dfrac{f(x) - f(0)}{x - 0} \\&=
            {\displaystyle\lim_{x\rightarrow 0}}\dfrac{\dfrac{ln(1-x)}{x} + 1}{x} = -\dfrac{1}{2}
        \end{aligned}
    \end{equation*}
    故可以知道,
    $$
        f'(x) = \begin{cases}
            \dfrac{x-(x-1)\ln(1-x)}{(x-1)x^2},& x < 1, x \neq 0\\ 
            -\dfrac{1}{2},& x = 0
        \end{cases}
    $$ 
\end{answer}


\begin{answer}[660T28]{在写作业的时候不要听狗叫}
    $ f(x) $ 在 $ R \backslash \{0\} $ 上的可导性显然。而 \begin{equation*}
        \begin{aligned}
            f'_+(0) &= {\displaystyle\lim_{x\rightarrow 0^+}}\dfrac{f(x) - f(0)}{x - 0} \\&=
            {\displaystyle\lim_{x\rightarrow 0^+}}\dfrac{x^a\sin \frac{1}{x}}{x} = {\displaystyle\lim_{x\rightarrow 0^+}}
            x^{a-1}\sin \dfrac{1}{x}
        \end{aligned}
    \end{equation*}
    而 $ f'(x)_- = 0 $ ,当 $ f(x) $ 在 $ x = 0 $ 可导时,显然 $ a > 1 $ 。

    $ f'(x) $ 在 $ R\backslash \{0\} $ 上的连续性显然,而 $ f'(0) = 0 $ ,当 $ f'(x) $ 在 $ x = 0 $ 连续时,有 \begin{equation*}
        \begin{aligned}
            {\displaystyle\lim_{x\rightarrow 0}}f'(x) - f'(0) &= {\displaystyle\lim_{x\rightarrow 0}}
            -x^{a-2}\cos \frac{1}{x} + ax^{a-1}\sin \frac{1}{x} = 0
        \end{aligned}
    \end{equation*}
    故此时显然 $ a > 2 $ .
\end{answer}

\begin{answer}[660T29]{注意不要漏掉分式上下同乘除的式子}
    由 $ f'(-2) = -1 $ 以及 $ f(x) $ 的偶性质,有 $f'(5) = f'(-1) = -f'(1) = -f'(-2) = 1 $ ,
    故有\begin{equation*}
        \begin{aligned}
            {\displaystyle\lim_{h\rightarrow 0}}\dfrac{h}{f(5-2\sin h)-f(5)} &= 
            {\displaystyle\lim_{h\rightarrow 0}}\dfrac{-2\sin h}{f(5-2\sin h)-f(5)} \cdot \dfrac{1}{-2}\\&=
            \dfrac{1}{-2f'(5)} = -\dfrac{1}{2}
        \end{aligned}
    \end{equation*}
\end{answer}

\begin{answer}[660T30]{对原式取对数运算后,要将 $ e $ 放回答案中}
    利用海因定理,令 $ t = \dfrac{1}{n} $ ,有 \begin{equation*}
        \begin{aligned}
            I &= \exp \left( {\displaystyle\lim_{t\rightarrow 0}} 
            \dfrac{t}{1-\cos t}\ln f(t) \right)\\&=\exp{\displaystyle\lim_{t\rightarrow 0}}
            \dfrac{2\ln f(t)}{t}\\&=\exp{\displaystyle\lim_{t\rightarrow 0}}
            \dfrac{2f'(t)}{f(t)} = e^6
        \end{aligned}
    \end{equation*}
\end{answer}

\begin{answer}[600T33]{计算时要检查是否将乘法和加法混淆}
    将 $ f(x) $ 展开,其最后一项必定为 $ 1^2\cdot 2^2\cdot 3^2\cdot x^2 = 36x^2 $,而
    其他项的次数显然大于2,故 $ f^\pprime(0) = 72$。
\end{answer}

\begin{answer}[660T34]{不要将分子和分母搞混}
    显然 $$ \dfrac{\mathrm{d}y}{\mathrm{d}x} = 
    \dfrac{\dfrac{\mathrm{d}y}{\mathrm{d}t}}{\dfrac{\mathrm{d}x}{\mathrm{d}t}}
    = \dfrac{\dfrac{1}{t^2+1}}{\dfrac{t}{t^2+1}} = \dfrac{1}{t} $$ 又,显然
    \begin{equation*}
        \begin{aligned}
            \dfrac{\mathrm{d}^2y}{\mathrm{d}x^2}&= 
            \dfrac{\dfrac{\mathrm{d}\dfrac{\mathrm{d}y}{\mathrm{d}x}}{\mathrm{d}t}}{\dfrac{\mathrm{d}x}{\mathrm{d}t}}
            =\dfrac{-\dfrac{1}{t^2}}{\dfrac{t}{t^2+1}} = -\dfrac{t^2+1}{t^3}
        \end{aligned}
    \end{equation*}
    因此有 \begin{equation*}
        \begin{aligned}
            K&= \dfrac{|y^\pprime|}{(1+(y')^2)^{\frac{3}{2}}} = 
            \dfrac{\dfrac{t^2+1}{|t|^3}}{(1+\dfrac{1}{t^2})^{\frac{3}{2}}}
            \\&= \dfrac{t^2+1}{(t^2+1)^\frac{3}{2}} = \dfrac{1}{\sqrt{1+t^2}}
        \end{aligned}
    \end{equation*}
\end{answer}

\begin{answer}[660T38]{注意区分 $ \sin^2 t $ 和 $ \sin t^2 $ 的区别}
    对方程两侧关于 $ x $ 取导数,有$$
        2x + (2+\sin y^2) \dfrac{\mathrm{d}y}{\mathrm{d}x} = 0 \Rightarrow
        \mathrm{d}y = \dfrac{-2x}{2+\sin y^2}\mathrm{d}x
    $$ 
\end{answer}

\begin{answer}[660T40]{注意记住 $ \ln (1+t) $ 和 $ \ln (1-t) $ 的泰勒展开式}
    由于\begin{equation*}
        \begin{aligned}
            \dis f(x) \ln \dfrac{1-2x}{1+3x} &= \ln (1-2x) - \ln (1+3x) \\&=
            -(\dfrac{2x}{1}+\dfrac{4x^2}{2}+\dfrac{8x^3}{3}+o(x^3))-
            (\dfrac{3x}{1}-\dfrac{9x^2}{2}+\dfrac{27x^3}{3}+o(x^3))
        \end{aligned}
    \end{equation*}
    可以知道 $ f^{(3)}(0) = -\dfrac{35\cdot 3\cdot 2}{3} = -70 $ .
\end{answer}

\begin{answer}[660T48]{注意必须严格验证每一条渐近线的存在性;必须考虑正无穷和负无穷处和每一个间断点}
    由 $ y $ 的方程,$ x \in (-\infty,-\dfrac{1}{2})\bigcup (0,+\infty) $
    而 $ x \rightarrow -\dfrac{1}{2} $ 时,$ f(x) $ 的极限不存在,因此 $ x = -\dfrac{1}{2} $ 是
    $ y $ 的一条垂直渐近线。而 $ {\displaystyle\lim_{x\rightarrow 0}}y = 0 $ ,故其不是 $ y $ 的渐近线。
    显然,$ {\displaystyle\lim_{x\rightarrow +\infty}}y $ 与 $ {\displaystyle\lim_{x\rightarrow -\infty}}y $ 
    均不存在。而
    \begin{equation*}
        \begin{aligned}
            {\displaystyle\lim_{x\rightarrow +\infty}}\dfrac{\sqrt{4x^2+x}\ln (2+\dfrac{1}{x})}{x}
            &\xlongequal{t = \dfrac{1}{x}}
            {\displaystyle\lim_{t\rightarrow 0}}\sqrt{4+t}\ln(2+t) = 2\ln 2
        \end{aligned}
    \end{equation*}
    而
    \begin{equation*}
        \begin{aligned}
            {\displaystyle\lim_{x\rightarrow +\infty}}\sqrt{4x^2+x}\ln (2+\dfrac{1}{x})-2\ln 2x
            &\xlongequal{t = \dfrac{1}{x}} \dfrac{\sqrt{4+t}\ln (2+t)-2\ln 2}{t} = 1+ \dfrac{\ln 2}{4}
        \end{aligned}
    \end{equation*}
    故有 $ y $ 的一条斜渐近线 $ y = 2\ln 2 x + 1 + \dfrac{\ln x}{4} $ .
    同理,在 $ -\infty $ 方向,有 $ y $ 的一条斜渐近线 $ y = -(2\ln 2 x + 1 + \dfrac{\ln x}{4}) $ .
    因此存在三条渐近线:
    \begin{BulletItemize}
        \item[\textbullet] $ x = -\dfrac{1}{2} $ ;
        \item[\textbullet] $ y = 2\ln 2 x + 1 + \dfrac{\ln x}{4} $ ;
        \item[\textbullet] $ y = -(2\ln 2 x + 1 + \dfrac{\ln x}{4}) $.
    \end{BulletItemize}
\end{answer}

\begin{answer}[660T49]{真的,我不知道应该写些什么}
    显然 $ {\displaystyle\lim_{x\rightarrow 0}}f(x) = f(0) = 0 $ ,且有
    \begin{equation*}
        \begin{aligned}
            {\displaystyle\lim_{x\rightarrow 0}}\dfrac{f(x)}{e^x - 1} = 2 &\Rightarrow
            {\displaystyle\lim_{x\rightarrow 0}}\dfrac{f(x)}{x} = 2
        \end{aligned}
    \end{equation*}
    $ f'(0) = {\displaystyle\lim_{x\rightarrow 0}}\dfrac{f(x) - f(0)}{x - 0} = 2 $ 
    则切线斜率为 $ 2 $ ,即法线斜率为 $ -\dfrac{1}{2} $ 。
    
    又,$ f(0) = 0 $ ,有法线方程
    $ y = -\dfrac{1}{2}x $ .
\end{answer}

\begin{answer}[660T51]{注意积分时不要遗漏常数项 $ C $ }
    对原等式两头求导,有
    \begin{equation*}
        \begin{aligned}
            xf'(x) = \dfrac{1}{1+x^2} &\Rightarrow f'(x) = \dfrac{1}{x(1+x^2)}
            \\&\Rightarrow f(x) + C = \dfrac{1}{2} \int \dfrac{1}{x} - \dfrac{x}{1+x^2} \mathrm{d}x^2
            \\&\Rightarrow f(x) = \dfrac{1}{2}\ln\dfrac{x^2}{1+x^2} + C
        \end{aligned}
    \end{equation*}
\end{answer}

\begin{answer}[660T52]{注意将答案化简为人话}
    显然\begin{equation*}
        \begin{aligned}
            I & = \int \dfrac{\dsqrt{9-4x^2}}{3+2x}\mathrm{d}x 
            \xlongequal{x = \frac{3}{2}\sin t} \dfrac{3}{2}
            \int \dfrac{3\cos t}{3+3\sin t} \cos t\mathrm{d}t
            \\&= \dfrac{3}{2}\int 1 - \sin t\mathrm{d}t \\&=
            \dfrac{3}{2}(t + \cos t) + C 
            %\\&= \dfrac{3}{2}\arcsin \dfrac{2}{3}x + \dfrac{1}{2}\sqrt{9-4x^2} + C
            \\&= \dfrac{3}{2}\arcsin \dfrac{2}{3}x + \dfrac{3}{2}\cos (\arcsin \dfrac{2}{3}x) + C
        \end{aligned}
    \end{equation*}
    注意到 $ \dfrac{3}{2}\cos (\arcsin \dfrac{2}{3}x) = \dfrac{1}{2}\sqrt{9-4x^2} $ ,
    故原积分为 $ \dfrac{3}{2}\arcsin \dfrac{2}{3}x + \dfrac{1}{2}\sqrt{9-4x^2} + C $ 
    其中 $ C $ 为任意常数。
\end{answer}

\begin{answer}[660T59]{注意不要漏掉提出去的系数}
    \begin{equation*}
        \begin{aligned}
            I &= \int_0^1 \arcsin x (\dfrac{\pi}{2} - \arcsin x) \mathrm{d} x 
            \xlongequal{x = \sin t} \int_0^{\frac{\pi}{2}}
            t(\frac{\pi}{2} - t) \mathrm{d} \sin t
            \\&= \frac{\pi}{2}\int_0^{\frac{\pi}{2}} t\mathrm{d}\sin t
            - \int_a^{\frac{\pi}{2}} t^2 \mathrm{d}\sin t
            = - \dfrac{\pi}{2} + 2
        \end{aligned}
    \end{equation*}
\end{answer}

\begin{answer}[660T60]{注意计算准确性}
    有 $ \dis \int_0^1\sqrt{2x-x^2}\mathrm{d}x-\int_0^1\sqrt{(1-x^2)^3}\mathrm{d}x$ ,由几何意义
    后式中前者显然为 $ \dfrac{\pi}{4} $ ,而\begin{equation*}
        \begin{aligned}
            \int_0^1\sqrt{(1-x^2)^3}\mathrm{d}x &= \int_0^{\frac{\pi}{2}}\cos^4 x\mathrm{d}x = \dfrac{3\pi}{16}
        \end{aligned}
    \end{equation*}
    因此原积分为 $ \dfrac{\pi}{4} - \dfrac{3\pi}{16} = \dfrac{\pi}{16} $ .
\end{answer}

\begin{answer}[660T64]{注意定积分的定义和上下限交换的意义(呃呃呃……)}
    分类讨论,有
    \begin{BulletItemize}
        \item $ x < -1 $ 时,原积分显然为 $ \dis \int_{-1}^{-x} f(t)\mathrm{d}t = -\dfrac{x^3}{3} + \dfrac{5}{3} $ ;
        \item $ x \in [-1,1] $ 时,原积分显然为 $ 1 - x $ ;
        \item $ x > 1 $ 时,原积分显然为 $ \dis \dfrac{x^3}{3} - \dfrac{1}{3} $ .
    \end{BulletItemize}
    因此,有$$
        \int_1^x f(t)\mathrm{d}t = \begin{cases}
            -\dfrac{x^3}{3} + \dfrac{5}{3}, & x < -1 \\ 
            1 - x, & -1 \leq x\leq 1\\ 
            \dfrac{x^3}{3} - \dfrac{1}{3}, & x > 1
        \end{cases}
    $$ 
\end{answer}

\begin{answer}[660T68]{注意计算准确性}
    显然 $\dis I = \int_1^{+\infty} \dfrac{(b-a)x + a}{x(2x+a)}\mathrm{d}x$,
    而 $ I $ 存在,因此必有 $ b - a = 0 $,即 $ a = b $ .

    那么有\begin{equation*}
        \begin{aligned}
            I &= \int_1^{+\infty} \dfrac{a}{x(2x+a)}\mathrm{d}x \\&=
            \int_1^{+\infty} \dfrac{1}{x} - \dfrac{2}{2x+a} \mathrm{d}x\\&=
            \ln \dfrac{x}{2x+a}\Big|^{+\infty}_0 = \ln(2+a) - \ln 2 = 1
        \end{aligned}
    \end{equation*} 
    因此有 $ a = b = 2e-2 $ .
\end{answer}

\begin{answer}[660T70]{注意所列式所求体积与待求体积的关系}
    设将摆线段向下平移 $ 2a $ 格后与$ x $ 轴围成图形绕 $ x $ 轴
    旋转一周所得立体的体积为 $ V_1 $ ;$ y = 2a,x = 2\pi,x = 0,y = 0 $ 围成的
    长方形绕 $ x $ 轴转一周得到的体积为 $ V_0 $,则显然有 $ V=V_0-V_1 $ .

    显然 $ V_0 = 8\pi^2a^3 $. 而\begin{equation*}
        \begin{aligned}
            V_1 &= \pi\int_0^{2\pi} (2a-y)^2 \mathrm{d}x
            \\&= \pi\int_0^{2\pi} a^2(2 - 1+\cos t)^2 \cdot a(1-\cos t) \mathrm{d}t
            \\& = a^3\pi\int_0^{2\pi} (1+\cos t)^2(1-\cos t) \mathrm{d}t
            \\&= a^3\pi\left[\int_0^{2\pi}\sin^2t\mathrm{d}t+
            \int_0^{2\pi}\sin^2t\mathrm{d}\sin t\right]
            \\& = a^3\pi^2
        \end{aligned}
    \end{equation*}
    因此原体积 $ V = V_0-V_1 = 7\pi^2a^3. $ 
\end{answer}

\section{考试}

\from{T1}
\begin{Answer}[化简,然后应用 $ e^x-1\sim x (x\rightarrow 0) $ \getback{T1}]
    \begin{equation*}
        \begin{aligned}
            \dis {\displaystyle\lim_{n\rightarrow +\infty}}n^2(\sqrt[n]a-\sqrt[n+1]a)
            &= {\displaystyle\lim_{n\rightarrow +\infty}}n^2a^{\frac1{(n+1)}}(a^{\frac1{n(n+1)}}-1)\\ 
            &= {\displaystyle\lim_{n\rightarrow +\infty}}n^2a^{\frac1{(n+1)}}(e^{\frac{\ln a}{n(n+1)}}-1)\\ 
            &= {\displaystyle\lim_{n\rightarrow +\infty}}\frac{n^2a^{\frac1{(n+1)}}\ln a}{n^2+n} = \ln a
        \end{aligned}
    \end{equation*}    
\end{Answer}

\from{T2}
\begin{Answer}[将上面平方开,通过等价求$ a,b $ \getback{T2}]
    由题,
    $$
       0 =  {\displaystyle\lim_{x\rightarrow +\infty}} 
       \dis\frac{(1-a^2)x^2+(1-2ab)x+(1-b^2)}{\sqrt{x^2+x+1}+ax+b}
    $$ 
    故有 $ 1-a^2 = 0 $ , $ 1-2ab = 0 $ 
    可以解得 $ a=1 $ , $ b = \dfrac{1}{2} $ ,故 $ a+b=\dfrac{3}{2} $ 。
\end{Answer}

\from{T3}
\begin{Answer}[考试前至晚1个小时喝咖啡\getback{T3}]
    由于
    \begin{equation*}
        \begin{aligned}
            f(x) &= \exp \left({\displaystyle\lim_{t\rightarrow x}}
            \dfrac{x\ln(\frac{\sin t}{\sin x})}{\sin t-\sin x}\right)\\ 
            &= \exp \left(\lim_{t\rightarrow x}\dfrac{x(\frac{\sin t}{\sin x}-1)}{\sin t-\sin x}\right)\\ 
            =& \exp(\dfrac{x}{\sin x})
        \end{aligned}
    \end{equation*}
    可以知道 $ f(x) $ 的间断点有 $ x = 2n\pi, n\in Z $ ,且其中只有 $ 0 $ 是可去的,因为
    其他点处 $ f(x) $ 左右极限均不存在。
\end{Answer}

\from{T4}
\begin{Answer}[不要认为令 $ a = b = 1 $ 的结果一定能代表答案\getback{T4}]
    设原极限为 $ I $ ,则对函数 $ f(n),n\in (-\infty,+\infty) $ ,有
    \begin{equation*}
        \begin{aligned}
            I &= \exp \left({\displaystyle\lim_{n\rightarrow +\infty}}
            \dfrac{\ln(\dsqrt[n]{a}+\dsqrt[n]{b})-\ln 2}{n} \right)\\ &=
            \exp\left(\dfrac{\dsqrt[n]{a}\ln+\dsqrt[n]{b}\ln b}{\dsqrt[n]{a}+\dsqrt[n]{b}}\right)
            \\&= \exp(\ln (ab)^\frac{1}{2}) = \dsqrt[]{ab}
        \end{aligned}
    \end{equation*}
\end{Answer}

\from{T5}
\begin{Answer}[对原式取对数运算后,要将 $ e $ 放回答案中\getback{T5}]
    设原极限为 $ I $ ,则有
    \begin{equation*}
        \begin{aligned}
            I &= {\displaystyle\lim_{x\rightarrow \frac{\pi}{4}}}\exp\left(\dfrac{\ln (\tan x)}
            {\cos x - \sin x}\right) \\&= \exp\left({\displaystyle\lim_{x\rightarrow \frac{\pi}{4}}}
            \dfrac{\tan x - 1}{\cos x - \sin x}\right) \\&=
            \exp({\displaystyle\lim_{x\rightarrow \frac{\pi}{4}}} \dfrac{1}{\cos x}) = e^{-\sqrt[]{2}}
        \end{aligned}
    \end{equation*}
\end{Answer}

\from{T6}
\begin{Answer}[利用前问结论推导后问结论\getback{T6}]
    由 $ \{x_n\} $ 满足 $ \ln x + \dfrac{1}{x_n} >0 $,有 $ x_n>0 $ 。故有
    \begin{equation*}
        \begin{aligned}
            &\ln x_n + \dfrac{1}{x_{n+1}} < 1 \leq f(x_n) =\ln x_n + \dfrac{1}{x_n} \\ 
            &\Rightarrow \dfrac{1}{x_{n+1}} < \dfrac{1}{x_n} \Rightarrow x_n < x_{n+1}
        \end{aligned}
    \end{equation*}
    即 $ \{x_n\} $ 单调递增。而 $$
        \ln x_n + \dfrac{1}{x_{n+1}} < 1 \Rightarrow \ln x_n < 1 \Rightarrow x_n < e
    $$
    故由单调有界原理,$ {\displaystyle\lim_{n\rightarrow \infty}}x_n $ 存在。

    令 $ {\displaystyle\lim_{n\rightarrow \infty}}x_n = a $,则由 $ \ln x_n + \dfrac{1}{x_{n+1}} <1  $,有
    $ \ln a + \dfrac{1}{a} \leq 1 $ ;同时 $ \ln a +\dfrac{1}{a} = f(a) \geq1 $ ,故由夹逼定理,有
    $ \ln a + \dfrac{1}{a}= f(a) = 1 $ ,此时解得 $ a = 1 $ ,故有 $ {\displaystyle\lim_{n\rightarrow \infty}}x_n = 1 $ .
\end{Answer}

\from{T7}
\begin{Answer}[构建合适的辅助函数,并对其使用零点定理\getback{T7}]
    由 $ |f(x)-f(y)|<|x-y| $ ,有 $ f(x)\in[a,b] $ 。

    令 $ F(x) = x - f(x) $ ,显然 $ F(x)\in C[a,b] $ 。又由 $ F(a)\cdot F(b) = [a-f(a)][b-f(b)]\leq 0 $ ,
    运用零点定理知,至少存在一点 $ c\in [a,b] $ ,使得 $ F(c) = 0 $ ,即 $ f(c) = c $ .

    假设 $ \exists d\in[a,b], c\neq d $ 使得 $ d = f(d) $ ,则有$$
        |c-d| = |f(c) - f(d)| \leq |c - d|
    $$ 这显然是矛盾的,故 $ c $ 唯一。

    因为 $ a \leq f(x) \leq b $ , $ x_1 \in [a,b] , x_{n+1} = \dfrac{1}{2}[f(x_n)+x_n]  $ ,
    可以由数学归纳法证明 $ \forall x_n, a\leq x_n \leq b $ ,
    此时 $ {\displaystyle\lim_{n\rightarrow \infty}}x_n $ 必然存在,假设其为 $ a $ 。
    对 $ x_{n+1} = \frac12[x_n+f(x_n)] $ 两边取极限,有 $ f(a) = a $ ,由 $ c $ 的唯一性,
    $ a = c $ 。
\end{Answer}

\begin{answer}[C2T1]{拐点第二充分条件在 $ f^\pprime(x) = 0 $ 时无法使用}
    由于 $ \dis {\displaystyle\lim_{x\rightarrow 0}}\dfrac{f^\pprime(x)}{|x|} = 1 $ ,
    可以知道 $ f^\pprime(x) = 0 $ ,此时不能运用拐点的第二充分条件。而由极限的保序性,
    当 $ x $ 趋近于 $ 0 $ 的时候,$ f^\pprime(x)>0 $ ,而 $ f'(x) = $ ,故$ f(0) $ 显然为极小值。
\end{answer}

\begin{answer}[C2T8]{不应当着急下结论}
    显然 $ {\displaystyle\lim_{x\rightarrow 0^-}}f(x) = 0 $ ,而
    由 $ \dfrac{1}{n+1} < x \leq \dfrac{1}{n} $ , $ {\displaystyle\lim_{x\rightarrow 0^+}}f(x)=0 $ ,
    故 $ f(x) $ 在 $ R $ 上连续。
    
    而\begin{equation*}
        \begin{aligned}
            f'_-(0) = {\displaystyle\lim_{x\rightarrow 0^-}}\dfrac{f(x)-f(0)}{x-0} = 1
        \end{aligned}
    \end{equation*}
    \begin{equation*}
        \begin{aligned}
            f'_+(0) &= {\displaystyle\lim_{x\rightarrow 0^+}}
            \dfrac{f(x)-f(0)}{x - 0} \\&= {\displaystyle\lim_{x\rightarrow 0^+}}
            \dfrac{1}{xn}
        \end{aligned}
    \end{equation*}
    同时 $ 1\leq \dfrac{1}{xn} \leq \dfrac{n+1}{n} $ ,因此由夹逼定理,$ f'_+(0) = 1 $ ,
    故 $ f(x) $ 在 $ x=0 $ 处可导。
\end{answer}

\begin{answer}[C2T11]{利用 $ {\dis\lim_{x\rightarrow 0}}\dfrac{y}{x} $ 求 $ {\dis\lim_{x\rightarrow 0}}y+x $ }
    由于存在斜渐近线,可以知道 $ {\displaystyle\lim_{x\rightarrow 0}}\dfrac{y}{x} $ 一定存在。而由题给方程,
    可以知道\begin{equation*}
        \begin{aligned}
            &1+(\dfrac{y}{x})^3 - \dfrac{3y}{x^2} = 0 (x\neq 0)\\ &\Rightarrow
            {\displaystyle\lim_{x\rightarrow \infty}}1+(\dfrac{y}{x})^3 - \dfrac{3y}{x^2} = 0 \\
        \end{aligned}
    \end{equation*}
    此时,显然有 $ {\displaystyle\lim_{x\rightarrow \infty}}\dfrac{y}{x} = -1 $ 。

    又,\begin{equation*}
        \begin{aligned}
            y^3+x^3-3xy = 0&\Rightarrow {\displaystyle\lim_{x\rightarrow \infty}}
            \dfrac{(x+y)(x^2 - xy + y^2)}{3xy} = 1 \\ &\Rightarrow
            {\displaystyle\lim_{x\rightarrow \infty}}(x+y) = 
            \dfrac{3\frac{y}{x}}{1 - \frac{y}{x} + (\frac{y}{x})^2}            
        \end{aligned}
    \end{equation*}
    故解得 $ {\displaystyle\lim_{x\rightarrow \infty}}x+y = -1 $ ,
    因此可以知道,斜渐近线的方程为 $ y = -x-1 $ 。
\end{answer}

\begin{answer}[C2T17]{}

\end{answer}

\begin{answer}[C2T20]{}

\end{answer}

\begin{answer}[C2T21]{}

\end{answer}

\begin{answer}[C2T22]{}

\end{answer}

\begin{answer}[C3T2]{注意存在间断点时不定积分的存在性}
    对于每一项,
    \begin{BulletItemize}
        \item $ f(x) $ 的一个原函数 $ F(x) = C $ ,其中 $ C $ 是常数,那么 $ F'(x) \equiv 0 $ ;
        \item $ f(x) $ 的一个原函数 $ F(x) = 0 $ ,那么其全部原函数为 $ F(x) = 0 + C = C $ ;
        \item 若 $ f(x) $ 在区间内有震荡间断点,则其有可能有原函数;
        \item 由于 $ F(x) $ 为 $ f(x) $ 的原函数,可以知道 $ F(x) $ 可导,而可导必连续。
    \end{BulletItemize}
    因此,正确的有3项。
\end{answer}

\begin{answer}[C3T13]{注意不要将变限积分函数求导公式错误应用}
    \begin{equation*}
        \begin{aligned}
            F'(x) &= \cos \left[\int_0^x\sin\left(\int_0^y\sin t^3 \mathrm{d}t\right)\mathrm{d}y\right] \cdot
            \left[\int_0^x\sin\left(\int_0^y\sin t^3 \mathrm{d}t\right)\mathrm{d}y\right]'
            \\&= \cos \left[\int_0^x\sin\left(\int_0^y\sin t^3 \mathrm{d}t\right)\mathrm{d}y\right]
            \cdot \sin\left(\int_0^x\sin t^3 \mathrm{d}t\right)
        \end{aligned}
    \end{equation*}
    注意,对变限积分函数求导时,不对里面的函数求导,只对积分上下限关于$ x $ 求导。
\end{answer}

\begin{answer}[C3T21]{构造辅助函数,通过辅助函数的单调性证明结论}
    令 $$ \dis F(x) = \int_a^{a+\int_a^x g(u)\mathrm{d}u}f(t)\mathrm{d}t - \int_a^x f(t)g(t)\mathrm{d}t,x\in[a,b] $$
    由于 $ f(x),g(x) $ 在 $ [a,b] $ 连续,有 $ F(x) $ 在 $ [a,b] $ 可导,且
    $$
        F'(x) = \left[f(a+\int_a^x g(u)\mathrm{d}u)-f(x)\right]g(x)
    $$ 
    又由 $ \dis a+\int_a^x g(u)\mathrm{d}u \leq x $ ,而 $ f(x) $ 单调递增,$ g(x)\leq 0 $ ,
    有 $ F'(x)\leq 0 $ ,即$ F(x) $ 在区间$ [a,b] $ 上单调不增。
    又由 $ F(a) = 0 $ ,可以知道 $ F(b)\leq 0 $ ,即
    $$
        \int_a^{a+\int_a^bg(t)\mathrm{d}t}f(x)\mathrm{d}x\leq \int_a^b f(x)g(x)\mathrm{d}x
    $$ 
\end{answer}