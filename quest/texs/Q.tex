\chapter{错题}
\section{练习}

\begin{Quest}[\goto{E1}]
    $\dis {\displaystyle\lim_{n\rightarrow \infty}}\sin\sqrt{4n^2+n}\pi=() $
    \begin{enumerate}
        \item $ 0 $ 
        \item $ 1 $ 
        \item $ \dis \frac{\sqrt 2}{2} $ 
        \item 不存在
    \end{enumerate}
\end{Quest}

\begin{quest}[660T9]
    $$
        I = {\displaystyle\lim_{x\rightarrow 0}}\dfrac{\dis \int_{x^2}^x\dfrac{\sin xt}{t}\mathrm{d}t}{x^2}
        =\qline.
    $$ 
\end{quest}

\begin{quest}[660T11]
    设 $ a>0 $ ,则$$
        {\displaystyle\lim_{x\rightarrow 0^+}}(x^2+x)^{x^a} = \qline.
    $$ 
\end{quest}

\begin{quest}[660T17]
    设 $ a,b $ 为常数,且 $ \dis {\displaystyle\lim_{x\rightarrow \infty}}(\dsqrt[3]{1-x^6}-ax^2-b) = 0 $ ,
    则 a=\qline,b=\qline.
\end{quest}

\begin{quest}[660T21]
    已知 $ x\rightarrow0 $ 时 $ \dis F(x) = \int_0^{x-\sin x}\ln(1+t)\mathrm{d}t $ 是 $ x^n $ 的同阶无穷小,
    则 $ n = \qline. $ 
\end{quest}

\begin{quest}[660T27]
    设$ \dis f(x) = \begin{cases}
        \dfrac{\ln (1+bx)}{x},& x\neq0\\-1,& x = 0
    \end{cases} $ ,其中 $ b $ 为常数,$ f(x) $ 在定义域上处处可导,则 $ f'(x) = \qline. $ 
\end{quest}

\begin{quest}[660T28]
    设 $ f(x) = \begin{cases}
        x^2,&x\leq 0\\ 
        x^a\sin \dfrac{1}{x},& x > 0
    \end{cases} $ ,若 $ f(x) $ 可导,则 $ a $ 满足\qline,若 $ f'(x) $ 连续,则 $ a $ 满足\qline.
\end{quest}

\begin{quest}[660T29]
    设 $ f(x) $ 是以 $ 3 $ 为周期的可导函数且为偶函数,$ f'(-2) = -1 $ ,则
    $ {\displaystyle\lim_{h\rightarrow 0}}\dfrac{h}{f(5-2\sin h) - f(5)} = \qline. $ 
\end{quest}

\begin{quest}[660T30]
    设 $ f(x) $ 在 $ x= 0 $ 可导且 $ f(0) = 1,f'(0)=3 $ ,则数列极限
    $$
    \dis I = {\displaystyle\lim_{n\rightarrow \infty}}
    (f(\dfrac{1}{n}))^{\dfrac{\frac{1}{n}}{1-\cos\frac{1}{n}}} =\qline.
    $$
\end{quest}

\begin{quest}[660T33]
    $ f(x) = x^2(x+1)^2(x+2)^2(x+3)^2 $,则 $ f^"(0) =$\qline. 
\end{quest}

\begin{quest}[660T34]
    设 $ y = y(x) $ 由参数方程 
    $ \begin{cases}
        x = \dfrac{1}{2}\ln (1+t^2)\\ y = \arctan t
    \end{cases} $ 确定,则 $ \dfrac{\mathrm{d}y}{\mathrm{d}x} = \qline,
    \dfrac{\mathrm{d}^2y}{\mathrm{d}x^2}=\qline,y = y(x) $ 在任意点处的曲率 $ K $ 为\qline.
\end{quest}

\begin{quest}[660T38]
    设函数 $ y=f(x) $ 为由方程 $ \dis \int_b^y(2+\sin^2t)\mathrm{d}t = 1 $ 确认的
    隐函数,则 $ \mathrm{d}y =$\qline.
\end{quest}

\begin{quest}[660T40]
    设 $ f(x) =\ln \dfrac{1-2x}{1+3x} $ ,则 $ f^{(3)}(0) = $\qline. 
\end{quest}

\begin{quest}[660T48]
    曲线 $ y = \sqrt{4x^2+x}\ln (2+\dfrac{1}{x}) $ 的全部渐近线为\qline.
\end{quest}

\begin{quest}[660T49]
    设函数 $ f(x) $ 在 $ x = 0 $ 处连续,
    且 $ {\displaystyle\lim_{x\rightarrow 0}}\dfrac{f(x)}{e^x - 1} = 2 $,
    则曲线 $ y = f(x) $ 在 $ x = 0 $ 处的法线方程为\qline.
\end{quest}

\begin{quest}[660T51]
    设 $ \dis \int xf'(x)\mathrm{d}x = \arctan x + C $ ,则 $ f(x) = \qline. $ 
\end{quest}

\begin{quest}[660T52]
    $ \dis I = \int \dsqrt[]{\dfrac{3-2x}{3+2x}}\mathrm{d}x = \qline. $ 
\end{quest}

\begin{quest}[660T59]
    $ \dis I = \int_0^1 \arcsin x \cdot \arccos x \mathrm{d}x =\qline. $ 
\end{quest}

\begin{quest}[660T60]
    $ \dis \int_0^1\left[\sqrt{2x-x^2}-\sqrt{(1-x^2)^3}\right]\mathrm{d}x = \qline. $ 
\end{quest}

\begin{quest}[660T64]
    设 $ f(x) = \max \{1,x^2\} $ ,则 $ \dis \int_1^x f(t)\mathrm{d}t = \qline. $ 
\end{quest}

\begin{quest}[660T68]
    $ \dis I = \int_1^{+\infty} \dfrac{2x^2+bx+a}{x(2x+a)}-1\mathrm{d}x $ ,则 $ a = \qline,b = \qline. $ 
\end{quest}


\section{考试}
\begin{Quest}[C1T5\goto{T1}]
    设 $ a>0 $ ,则 $\dis {\displaystyle\lim_{n\rightarrow +\infty}}n^2(\sqrt[n]a-\sqrt[n+1]a)=() $

    \begin{enumerate}
        \item 不存在且非无穷大
        \item 0
        \item $ \ln a $ 
        \item $ \infty $ 
    \end{enumerate}
\end{Quest}

\begin{Quest}[C1T11\goto{T2}]
    已知 $ {\displaystyle\lim_{x\rightarrow +\infty}}\left(\dis\sqrt{x^2+x+1}-ax-b\right)=0 $ ,
    则 $ a+b=\qline. $ 
\end{Quest}

\begin{Quest}[C1T12\goto{T3}]
    设函数 $\dis f(x) = {\displaystyle\lim_{t\rightarrow x}}
    \left(\frac{\sin t}{\sin x}\right)^{\dis \frac{x}{\sin t-\sin x}} $ ,
    则 $ f(x) $ 的可去间断点为 $ x=\qline. $ 
\end{Quest}

\begin{Quest}[C1T15\goto{T4}]
    若 $ a>0,b>0 $ ,则 $ \dis {\displaystyle\lim_{n\rightarrow +\infty}}
    \left(\frac{\sqrt[n]a+\sqrt[n]b}{2}\right)^n = \qline. $ 
\end{Quest}

\begin{Quest}[C1T16\goto{T5}]
    $ \dis {\displaystyle\lim_{x\rightarrow \frac{\pi}{4}}}(\tan x)^{\dis\frac{1}{\cos x-\sin x}} = \qline. $ 
\end{Quest}

\begin{Quest}[C1T20\goto{T6}]
    设函数 $\dis f(x)=\ln x + \frac{1}{x} $.
    \begin{itemize}
        \item[$ \blacksquare $ ] 求 $ f(x) $ 最小值;
        \item[$ \square $ ] 设数列 $ \{x_n\} $ 满足 $ \dis \ln x_n + \frac{1}{x_{n+1}}<1 $ ,证明
        $ {\displaystyle\lim_{n\rightarrow \infty}}x_n $ 存在,并求该极限。
    \end{itemize}
\end{Quest}

\begin{Quest}[C1T22\goto{T7}]
    设函数 $ f(x) $ 满足 $a \leq f(x)\leq b,\ \forall x,y\in [a,b],|f(x)-f(y)|<|x-y|  $ .

    设 $ x_1\in[a,b] $ , $\dis x_{n+1}=\frac{1}{2}\left[x_n+f(x_n)\right] $ ,
    证明极限 $ {\displaystyle\lim_{n\rightarrow \infty}}x_n $ 存在,记为且满足
    $ c=f(c) $ .
\end{Quest}

\begin{quest}[C2T1]
    设 $ f(x) $ 具有二阶连续导数,
    且 $ f'(0) = 0,{\displaystyle\lim_{x\rightarrow 0}}\dfrac{f^"(x)}{|x|} = 1 $,
    则 ()
    \begin{enumerate}
        \item $ f(0) $ 是 $ f(x) $ 的极大值
        \item $ f(0) $ 是 $ f(x) $ 的极小值
        \item $ (0,f(0)) $ 是 $ f(x) $ 的拐点
        \item $ f(0) $ 不是 $ f(x) $ 的极值,$ (0,f(0)) $ 也不是曲线 $ y=f(x) $ 的拐点
    \end{enumerate} 
\end{quest}


\begin{quest}[C2T8]
    已知函数$$
    f(x) = \begin{cases}
        x,&x\leq 0\\
        \dfrac{1}{n},&\dfrac{1}{n+1}<x\leq \dfrac{1}{n}, n = 1,2,\dots
    \end{cases}
$$ 则()
\begin{enumerate}
    \item $ x=0 $ 是 $ f(x) $ 的第一类间断点
    \item $ x=0 $ 是 $ f(x) $ 的第二类间断点
    \item $ f(x) $ 在 $ x=0 $ 连续但不可导
    \item $ f(x) $ 在 $ x=0 $ 可导
\end{enumerate}
\end{quest}

\begin{quest}[C2T11]
    已知 $ y = y(x) $ 满足 $ y^3+x^3-3xy = 0 $ ,且存在斜渐近线,则该斜渐近线为\qline.
\end{quest}

\begin{quest}[C2T17]
    设 $ f(x) = \arctan \dfrac{1-x}{1+x} $ ,求高阶导数值 $ f^{(2020)}(0) $ 与 $ f^{(2021)}(0) $ 。
\end{quest}

\begin{quest}[C2T20]
    设 $ a $ 为常数,讨论方程 $ x^2=ae^x $ 的实根个数及其所在范围。
\end{quest}


\begin{quest}[C2T21]
    设函数 $ f(x) $ 在 $ [0,1] $ 连续, $ (0,1) $ 可导, $ c\in (0,1) $ , $ f(0)\neq F(1) $ ,
    则存在 $ \xi\in (0,1) $ , $ \eta\in (0,1) $ 使得 $$
        2\eta f(1) + (c^2-1)f'(\eta) = f(\xi)
    $$ 
\end{quest}

\begin{quest}[C2T22]
    设函数 $ f(x) $ 和 $ g(x) $ 在 $ [a,b] $ 上存在二阶导数,且 $ g^"(x)\neq 0, f(a)=f(b)=g(a)=g(b) = 0 $ ,
    则\begin{itemize}
        \item[$ \blacksquare $ ] 在开区间 $ (a,b) $ 内 $ g(x)\neq 0 $ 
        \item 在开区间 $ (a,b) $ 内至少存在一点 $ \xi $ ,使得$$
            \dfrac{f(\xi)}{g(\xi)} = \dfrac{f^"(\xi)}{g^"(\xi)}
        $$ 
    \end{itemize}
\end{quest}

