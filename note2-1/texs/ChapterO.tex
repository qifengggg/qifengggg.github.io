\chapter*{概述}

\sssubsection{题型}

线性代数的题目一般有三个选择题、一个填空题和一个$ 12 $ 分大题。其中,选填部分主要涵盖

$$
    \begin{cases}
        \textrm{行列式计算(特殊;抽象)}  \\
        \textrm{矩阵}\ (A^n,A^{-1},A^*) \\
        \textrm{初等矩阵(左/右)}  \\
        \textrm{\Attention{秩}(性质)}  \\
        \textrm{相关性(系数、秩)}  \\
        \textrm{等价、相似、合同}  \\
        \textrm{惯性系数}
        \begin{cases}
            \textrm{可逆变换}\\
            p+q = r 
        \end{cases}
    \end{cases}
$$ 

大题部分主要涵盖

$$
    \begin{cases}
        \textrm{方程组}
        \begin{cases}
            \textrm{相关 }\\
            \textrm{解}\\
            \textrm{矩阵方程}\\
        \end{cases}\\
        \textrm{相似形 (二次型)}
    \end{cases}
$$ 

\sssubsection{一个中心}

可以说秩是线性代数的一个中心。其常用于考虑下列问题。
$$
    \begin{cases}
        |A| = 0 ?  \\
        \exists A^{-1} ?  \\
        A = (\alpha_i) \textrm{是否相关}? \\
        AX = 0 \textrm{的基解}? \\
        A\sim B   \\
        p + q = r \\
    \end{cases}
$$ 

\sssubsection{一种方法}

初等行变换是非常常用的一种方法,其常用于求
$$
    \begin{cases}
        A^{-1} \\ 
        \textrm{极大无关组}\\ 
        \textrm{方程组}\\ 
        \textrm{求} (\lambda_0E - A)X = 0\\ 
        \textrm{求正交变换}\\ 
    \end{cases}
$$ 