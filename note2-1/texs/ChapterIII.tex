\chapter{向量}

$$
    \begin{cases}
        \textrm{向量运算}\begin{cases}
            \textrm{服从矩阵运算规律}\\ 
            \textrm{内积}\ (\alpha,\beta) = \alpha^\top\beta = \beta^\top\alpha( = tr(A))\\ 
            \textrm{两向量内积为零时,两向量正交}\\ 
            \textrm{正交矩阵}
            \begin{cases}
                \textrm{单位 - 任意列向量为单位向量}\\ 
                \textrm{正交 - 任意两列向量正交}
            \end{cases}
        \end{cases}\\ 
        \Attention{\textbf{相关性}}\begin{cases}
            \textrm{定义}\\\textrm{判定}
        \end{cases}\\ 
        \textrm{线性表示}\begin{cases}
            \textrm{向量组表示}\\ 
            \textrm{向量组等价}\\ 
            \textrm{极大无关组}\begin{cases}
                \textrm{求法}\\\textrm{秩}
            \end{cases}
        \end{cases}
    \end{cases}
$$ 

\section{相关性}

\sssubsection{定义法}

定义法常用于证明相关/无关性。

令 $ \sum k_i\alpha_i = \vec 0, $ 通过
\begin{itemize}
    \item 乘(使等式变短)
    \item 重组
\end{itemize}
得出 $ \forall i, k_i = 0, $ 则 $ \alpha_i $ 线性无关。

\sssubsection{秩}

$$
    \begin{cases}
        r(\alpha_1,\cdots,\alpha_{\Attention{s}}) = \Attention{S} \Rightarrow (\alpha_i)\textrm{线性无关}\\
        r(\alpha_1,\cdots,\alpha_s) < S \Rightarrow (\alpha_i)\textrm{线性相关}\\
    \end{cases}
$$ 

\sssubsection{性质}

\begin{itemize}
    \item $ \alpha_i $ 线性无关 $ \Leftrightarrow (\alpha_i)\begin{pmatrix}
        x_1\\ \vdots \\ x_s
    \end{pmatrix} = 0 $ 仅有零解。
    \item $
        \begin{matrix}
        \textrm{内含零向量}\\ \textrm{内含等比例向量}\\\textrm{内含可表示向量}
        \end{matrix}\quad \textrm{中满足任一} \Leftrightarrow \textrm{线性相关}
    $
    \item 向量个数大于维数必相关(利用秩证明)。
    \item 全部无关,一部无关;一部相关,全部相关。
    \item 原本相关,缩短相关,原本无关,加长无关。(改变的是维数)
    \item 以少表多,多必相关。
\end{itemize}

对于一组能组成方阵的 $ (\alpha_i), $ 若其方阵 $ P $ 满足 $ AP = PB, $ 则
这组向量无关。此处运用了相似的性质。

\section{线性表示}

\begin{equation*}
    \begin{aligned}
        \begin{cases}
            \textrm{一个 -} \beta = \sum_{i = 1}^s k_i\alpha_i \Leftrightarrow (\alpha_i)X = \beta\\ 
            \textrm{一组 -} \beta_i = \sum_{j = 1}^s x_{ij}\alpha_j; i = 1,2,\cdots, t\\ 
            \textrm{两组 - 互相表示}\\ \textrm{向量组等价}\Leftrightarrow
            \begin{cases}
                \forall i, \beta_i = \sum_j x_{ij}\alpha_j \\ 
                \forall j, \alpha_j = \sum_i x_{ij}\beta_i \\ 
            \end{cases}
        \end{cases}
    \end{aligned}
\end{equation*}

\sssubsection{判定}

\begin{itemize}
    \item $ \beta $ 可由 $ \alpha_1,\cdots,\alpha_s $ 线性表出 $ \Leftrightarrow r(\alpha_i) = r(\alpha_i,\beta), $ 
    注意此时不一定满秩;
    \item $ \forall i, \beta_i $ 均可由 $ \alpha_1,\cdots,\alpha_s $ 表出
    $\forall i, \Leftrightarrow r(\alpha_i) = r(\alpha_i,\beta_i); $ 
    \item 向量组 $ \textrm{(I),(II)} $ 等价
    $ \Rightarrow r\textrm{(I)}=r\textrm{(II)}=r\textrm{(I,II)} $ 
\end{itemize}

