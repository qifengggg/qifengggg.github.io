\chapter{矩阵}

$$
    \begin{cases}
        \textrm{定义与运算}
        \begin{cases}
            AB \neq BA; (kE,A^{-1},A^*,A^T \Leftrightarrow A) \\ 
            AB = 0 \nLeftrightarrow A = 0 \textrm{或} B = 0; A^2 = 0 \textrm{同理} \\ 
            AB = AC \Rightarrow B = C, \textrm{当且仅当} A \textrm{可逆时成立} \\ 
        \end{cases} \\ 
        \textrm{特殊矩阵 - } E,\Lambda, A^\top = A,AA^\top = E \\ 
        \textrm{伴随矩阵} A^*\\ 
        \textrm{可逆矩阵}\begin{cases}
            \textrm{定义}\\ \textrm{求法} \\ \textrm{证明}A\textrm{可逆}
        \end{cases}\\ 
        \textrm{初等矩阵(逆,变换)}\\ 
        \Attention{\text{秩}}\textrm{(性质)}\\ 
        \textrm{应用}\begin{cases}
            \textrm{矩阵方程}\\ \textrm{求}A^n
        \end{cases}
    \end{cases}
$$ 

\Section{求解与伴随矩阵相关的问题}

$$
    \begin{cases}
        \textrm{求}A^*
        \begin{cases}
            \textrm{定义法 - } A^* = (A_{ij})^\top \\ 
            \textrm{公式法 - } A^* = |A|A^{-1}            
        \end{cases}\\
        \textrm{性质}
        \begin{cases}
            AA^* = A^*A = |A|E (\textrm{可推广为}\Delta\Delta^* = \Delta^*\Delta = |\Delta|E)\\ 
            (kA)^* = k^{n-1}A^* \\ 
            (A^*)^* = |A|^{n-2}A \\ 
            |A^*| = |A|^{n-1} \\ 
            |(A^*)^*| = |A|^{(n-1)^2} \\ 
            A^{-1,\top,*} \textrm{之间可以互换,如} (A^{-1})^\top = (A^\top)^{-1}
        \end{cases}
    \end{cases}
$$ 

\sssubsection{矩阵行列和的结论}

对矩阵 $ A, $ 若其每行元素和均为 $ k, $ 则有 $ A(1,1,1)^\top = (k,k,k)^\top = k(1,1,1)^\top(\lambda \alpha); $

若为每列元素和均为 $ k, $ 有 $ (1,1,1)A = (k,k,k), $ 转置后与前者相同。

右乘/左乘 $ A^*, $ 可以求伴随矩阵的行/列和。

\sssubsection{关于伴随矩阵和转置矩阵的结论}

\begin{itemize}
    \item $ \forall (i,j), a_{ij} = A_{ij} \Leftrightarrow A^* = A^\top \Leftrightarrow AA^T = E $
    且 $ |A| = 1; $ 
    \item $ \forall (i,j), a_{ij} = -A_{ij} \Leftrightarrow A^* = -A^\top \Leftrightarrow AA^T = E $
    且 $ |A| = -1; $ 
\end{itemize}

也可以使用矩阵表达式替代 $ A. $

\Section{可逆矩阵的应用}

\begin{equation*}
    \begin{aligned}
        \textrm{可逆矩阵的判定}
        \begin{cases}
            AB = BA = kE (\textrm{此时}A^{-1} = \dfrac{1}{k}B)\\ 
            A\textrm{可逆}\Leftrightarrow
            \begin{cases}
                |A| \neq 0 \\ 
                r(A) = n \\ 
                A \textrm{的列向量线性无关} \\
                AX = 0 \textrm{仅有零解} \\ 
                AX = b \textrm{有唯一解} \\ 
                \forall \lambda,\lambda\neq 0
            \end{cases}
        \end{cases}\\ 
        \textrm{可逆矩阵的计算}
        \begin{cases}
            \textrm{抽象}
            \begin{cases}
                \textrm{凑}AB=E\\ 
                \textrm{利用性质}\begin{cases}
                    (A^{-1})^{-1} = A\\ 
                    (kA)^{-1} = \dfrac{1}{k}A^{-1}\\ 
                    (AB)^{-1} = B^{-1}A^{-1}
                \end{cases}
            \end{cases}\\ 
            \textrm{具体}\begin{cases}
                \textrm{低阶(2-3) - } A^{-1} = \dfrac{1}{|A|}A^* \\ 
                \textrm{初等变换 - } (A:E)\xlongrightarrow{\textrm{行变换}}(E:A^{-1}) \\ 
                \textrm{分块矩阵}
            \end{cases}
        \end{cases}
    \end{aligned}
\end{equation*}

\sssubsection{分块矩阵求逆}

\begin{itemize}
    \item 主对角分块
    \begin{itemize}
        \item 仅在主对角上非零的分块
        
        $ \begin{pmatrix}
            B&O\\O&C
            \end{pmatrix}^{-1} = \begin{pmatrix}
                B^{-1}&O\\O&C^{-1}
            \end{pmatrix} $ 
        \item 仅“缺一块”的分块
        
        方法为“左乘同行,右乘同列”。先写逆矩阵的对角部分,同行同列根据逆矩阵寻找。

        $ \begin{pmatrix}
        B & D  \\ O & C   
        \end{pmatrix}^{-1} = 
        \begin{pmatrix}
        B^{-1} & -B^{-1}DC^{-1} \\ O & C^{-1}
        \end{pmatrix} $ 
        
        $\begin{pmatrix}
        B & O \\ D & C   
        \end{pmatrix}^{-1} = 
        \begin{pmatrix}
        B^{-1} & O \\ -C^{-1}DB^{-1} & C^{-1}
        \end{pmatrix} $ 
    \end{itemize}

    \item 副对角分块
    
    副对角分块取逆时,要交换对角的元素。
    \begin{itemize}
        \item 仅在副对角上非零的分块
        
        $ \begin{pmatrix}
            O&B\\C&O
        \end{pmatrix}^{-1} = 
        \begin{pmatrix}
            O & C^{-1} \\ B^{-1} & O
        \end{pmatrix} $ 
        \item 仅“缺一块”的分块
        
        $ \begin{pmatrix}
        O & B \\ C & D
        \end{pmatrix}^{-1} = 
        \begin{pmatrix}
        -C^{-1}DB^{-1} & C^{-1} \\ B^{-1} & O
        \end{pmatrix} $ 
        
        $ \begin{pmatrix}
        D & B \\ C & O
        \end{pmatrix}^{-1} = 
        \begin{pmatrix}
        O & C^{-1} \\ B^{-1} & -B^{-1}DC^{-1}
        \end{pmatrix} $ 
    \end{itemize}
\end{itemize}

\sssubsection{可交换矩阵的结论}

\begin{itemize}
    \item 线性组合 - $ AB = aA+bB \Rightarrow AB = BA; $ 
    \item 一元二次形式 - $ A^2 + aAB = E \Rightarrow AB = BA; $
\end{itemize}

证明的思路是,因为可逆矩阵可交换,因此构造互逆矩阵。

其中,对前者,
\begin{equation*}
    \begin{aligned}
        AB = aA + bB &\Rightarrow A(B-aE) - bB = O \\ 
        &\Rightarrow A(B-aE) - b(B\Attention{-aE+aE}) = O \\ 
        &\Rightarrow (A-bE)(B-aE) = abE \\
        &= (B-aE)(A-bE) = abE\\ 
        &\Rightarrow (A-bE)(B-aE) = (B-aE)(A-bE)\\
    \end{aligned}
\end{equation*}
展开,可以证明结论。

\Section{初等变换与初等矩阵之间的关系}

初等矩阵左乘做行变换,右乘做列变换。其有三种,为

\begin{table}[!htbp]\centering
    \begin{tabular}{|c|c|c|c|}
    \hline
            &  符号  & 行列式 & 逆                \\ \hline
    交换&$ E_{ij} $    & $ -1 $   & $ E_{ij}(k) $       \\ \hline
    倍乘&$ E_{i}(k) $  & $ k  $   & $ E_{i}(1/k) $ \\ \hline
    倍加&$ E_{ij}(k) $ & $ 1  $   & $ E_{ij}(-k) $      \\ \hline
    \end{tabular}
\end{table}

\sssubsection{利用性质计算矩阵}

有例子
\begin{equation*}
    \begin{aligned}
        E_{12}A = B &\Rightarrow 
        \begin{cases}
            -|A| = |B| \\ A^{-1}E_{12} = B^{-1}
        \end{cases}\\ 
        &\Rightarrow -|A|A^{-1}E_{12} = |B|B^{-1} \\ 
        &\Rightarrow -A^*E_{12} = B^*
    \end{aligned}
\end{equation*}

\Section{求矩阵的秩}

\sssubsection{秩的求解}

$$
    \begin{cases}
        \textrm{定义}\begin{cases}
            r\textrm{阶至少一非零}\\
            r+1\textrm{阶全为零}
        \end{cases}\\
        \textrm{初等变换法(行列均可)- } A\xlongrightarrow{\textrm{行变换}}\textrm{阶梯矩阵}B,
        \textrm{非零行数即为秩}\\ 
        \textrm{非零特征根的个数是秩}
    \end{cases}
$$ 

\sssubsection{性质}

\begin{itemize}
    \item 对 $ A_{m\times n}, $ 有 $ r(A) = r(A^\top A) = r(AA^\top) = r(A^\top) = r(kA). $
    \begin{itemize}
        \item 证明 $ r(A^\top A) = r(A) $ 
        
        利用同解方程组。
        \begin{equation*}
            \begin{aligned}
                A^\top AX = O &\Rightarrow X^\top A^\top AX = O \\ 
                &\Rightarrow (AX)^\top AX = O \\ &\Rightarrow |AX| = 0
                \\ &\Rightarrow AX = O \\ 
                AX = O \Rightarrow A^\top AX = O
            \end{aligned}
        \end{equation*}
        因此 $ A $ 与 $ A^\top A $ 同解,故其秩相等。
    \end{itemize} 
    \item $ r(A_{m\times n})\leq \min(m,n). $ 
    \item $ r(AB) \leq r(A); r(AB) \leq r(B). $ 对前者,$ B $ 可逆时等号成立。
    \item $ r(A:B)\leq r(A) + r(B); $ 
    
    $ r(\begin{matrix}
        A \\ B 
    \end{matrix}) \leq r(A) + r(B)$ 
    \item 对矩阵 $ A_{m\times \Attention{n}},B_{\Attention{n}\times s}, AB=O, $
    有 $ r(A) + r(B) \leq \Attention{n}. $ 
    
    此即所谓“前看列,后看行”。
    判断有关于行/列和秩的问题时,都应考虑这一句。
    \begin{itemize}
        \item 证明
        
        $AB = O \Rightarrow B $ 的列向量 $ \beta_i $ 是 $ AX=O $ 的一组解,
        此时 $ r(\beta_1,\cdots,\beta_s) = r(B) \leq n - r(A) \Rightarrow r(A) + r(B) \leq n. $ 
    \end{itemize}
    \item 对 $ A_n, $ 其伴随矩阵的秩
    $ r(A^*)=\begin{cases}
        n,& r(A) = n\\ 1,&r(A) = n-1(n\geq 2)\\ 0,& r(A)< n-1
    \end{cases} $
    \item $ r\begin{pmatrix}
        A & O \\ O & B
    \end{pmatrix} = r(A) + r(B)$ 
    \item 若 $ P,Q $ 可逆,则有
    $ r(PA) = r(AQ) = \overbrace{r(PAQ) = r(A)}^{\textrm{矩阵等价}} $ 
    
    若 $ B = PAQ $ 或 $ r(B) = r(A), $ 称 $ A,B $ 等价。
    \item 对 $ A_{m\times n}, $ 若其行满秩,则 $ r(A) = m, A\sim (E_m,O). $ 
    \item 对 $ A_{m\times n}, r(A) = n \Rightarrow r(AB); $ 

    $ r(A) = m, r(BA) = r(B). $ 
    \begin{itemize}
        \item 证明前者成立
        
        $ r(B)\geq r(AB) \geq r(A^{-1} AB) = r(B), $ 故 $ r(AB) = r(B). $ 
    \end{itemize}
\end{itemize}

\sssubsection{关于矩阵可因式分解的一元二次式的结论}

\begin{itemize}
    \item $ A^2 = A \Rightarrow r(A) + r(A-E) = n. $ 
    \item $ A^2 = E \Rightarrow r(A - E) + r(A + E) = n. $ 
    \begin{itemize}
        \item 证明后者成立
        
        \begin{equation*}
            \begin{aligned}
                (A-E)(A+E)= 0 &\Rightarrow  n \geq r(A-E) + r(A+E) = r(A+E) + r(\Attention{-(A-E)}) \geq r(2E) = n
                \\ &\Rightarrow r(A-E) + r(A+E) = n
            \end{aligned}
        \end{equation*}
    \end{itemize}
\end{itemize}

事实上,对于 $ A $ 的一元二次式,若其可因式分解,则分解后的因式都满秩。

\Section{求解矩阵方程}

\sssubsection{可逆矩阵}

\begin{itemize}
    \item $ AX = C\Rightarrow X = A^{-1}C $ 
    \item $ XA = C\Rightarrow X = CA^{-1} $ 
    \item $ AXB = C\Rightarrow X = A^{-1}CB^{-1} $ 
\end{itemize}

\sssubsection{不可逆矩阵}

此时需要将其转化为方程组。

\begin{itemize}
    \item $ AX = C \Rightarrow A(X_i) = (C_i), $ 其中 $ X_i,C_i $ 是对应矩阵的列向量,
    解得到的 $ n $ 个非齐次方程组即可。
    \item $ XA = C \Rightarrow A^\top X^\top = C^\top, $ 然后同上。
\end{itemize}

\Section{计算 $ n $ 阶矩阵高次幂}

\begin{itemize}
    \item 归纳运算。
    \item $ r(A) = 1 $ 时,有 $ A^n = tr(A)^{n-1}A. $ 
    \item $ A = \begin{pmatrix}
        0 & a & b \\ & 0 & c\\ && 0
    \end{pmatrix} \Rightarrow A^2 = \begin{pmatrix}
        &&ac\\ && \\&&
    \end{pmatrix}, A^{n\geq 3} = O.$ 
    \item 相似性
    
    若 $ P^{-1}AP = B, $ 则有 $ P^{-1}A^nP = B^n, $ 其中 $ P $ 不变,
\end{itemize}