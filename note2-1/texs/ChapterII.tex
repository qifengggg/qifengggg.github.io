\chapter{矩阵}

$$
    \begin{cases}
        \textrm{定义与运算}
        \begin{cases}
            AB \neq BA; (kE,A^{-1},A^*,A^T \Leftrightarrow A) \\ 
            AB = 0 \nLeftrightarrow A = 0 \textrm{或} B = 0; A^2 = 0 \textrm{同理} \\ 
            AB = AC \Rightarrow B = C, \textrm{当且仅当} A \textrm{可逆时成立} \\ 
        \end{cases} \\ 
        \textrm{特殊矩阵 - } E,\Lambda, A^\top = A,AA^\top = E \\ 
        \textrm{伴随矩阵} A^*\\ 
        \textrm{可逆矩阵}\begin{cases}
            \textrm{定义}\\ \textrm{求法} \\ \textrm{证明}A\textrm{可逆}
        \end{cases}\\ 
        \textrm{初等矩阵(逆,变换)}\\ 
        \Attention{\text{秩}}\textrm{(性质)}\\ 
        \textrm{应用}\begin{cases}
            \textrm{矩阵方程}\\ \textrm{求}A^n
        \end{cases}
    \end{cases}
$$ 

\section{求解与伴随矩阵相关的问题}

$$
    \begin{cases}
        \textrm{求}A^*
        \begin{cases}
            \textrm{定义法 - } A^* = (A_{ij})^\top \\ 
            \textrm{公式法 - } A^* = |A|A^{-1}            
        \end{cases}\\
        \textrm{性质}
        \begin{cases}
            AA^* = A^*A = |A|E (\textrm{可推广为}\Delta\Delta^* = \Delta^*\Delta = |\Delta|E)\\ 
            (kA)^* = k^{n-1}A^* \\ 
            (A^*)^* = |A|^{n-2}A \\ 
            |A^*| = |A|^{n-1} \\ 
            |(A^*)^*| = |A|^{(n-1)^2} \\ 
            A^{-1,\top,*} \textrm{之间可以互换,如} (A^{-1})^\top = (A^\top)^{-1}
        \end{cases}
    \end{cases}
$$ 

\sssubsection{矩阵行列和的结论}

对矩阵 $ A, $ 若其每行元素和均为 $ k, $ 则有 $ A(1,1,1)^\top = (k,k,k)^\top = k(1,1,1)^\top(\lambda \alpha); $

若为每列元素和均为 $ k, $ 有 $ (1,1,1)A = (k,k,k), $ 转置后与前者相同。

右乘/左乘 $ A^*, $ 可以求伴随矩阵的行/列和。

\sssubsection{关于伴随矩阵和转置矩阵的结论}

\begin{itemize}
    \item $ \forall (i,j), a_{ij} = A_{ij} \Leftrightarrow A^* = A^\top \Leftrightarrow AA^T = E $
    且 $ |A| = 1; $ 
    \item $ \forall (i,j), a_{ij} = -A_{ij} \Leftrightarrow A^* = -A^\top \Leftrightarrow AA^T = E $
    且 $ |A| = -1; $ 
\end{itemize}

也可以使用矩阵表达式替代 $ A. $

\section{可逆矩阵的应用}

\begin{equation*}
    \begin{aligned}
        \textrm{可逆矩阵的判定}
        \begin{cases}
            AB = BA = kE (\textrm{此时}A^{-1} = \dfrac{1}{k}B)\\ 
            A\textrm{可逆}\Leftrightarrow
            \begin{cases}
                |A| \neq 0 \\ 
                r(A) = n \\ 
                A \textrm{的列向量线性无关} \\
                AX = 0 \textrm{仅有零解} \\ 
                AX = b \textrm{有唯一解} \\ 
                \forall \lambda,\lambda\neq 0
            \end{cases}
        \end{cases}\\ 
        \textrm{可逆矩阵的计算}
        \begin{cases}
            \textrm{抽象}
            \begin{cases}
                \textrm{凑}AB=E\\ 
                \textrm{利用性质}\begin{cases}
                    (A^{-1})^{-1} = A\\ 
                    (kA)^{-1} = \dfrac{1}{k}A^{-1}\\ 
                    (AB)^{-1} = B^{-1}A^{-1}
                \end{cases}
            \end{cases}\\ 
            \textrm{具体}\begin{cases}
                \textrm{低阶(2-3) - } A^{-1} = \dfrac{1}{|A|}A^* \\ 
                \textrm{初等变换 - } (A:E)\xlongrightarrow{\textrm{行变换}}(E:A^{-1}) \\ 
                \textrm{分块矩阵}
            \end{cases}
        \end{cases}
    \end{aligned}
\end{equation*}

\sssubsection{分块矩阵求逆}

\begin{itemize}
    \item 主对角分块
    $ \begin{pmatrix}
    B&O\\O&C
    \end{pmatrix}^{-1} = \begin{pmatrix}
        B^{-1}&O\\O&C^{-1}
    \end{pmatrix} $ 
    \item 副对角分块
    $ \begin{pmatrix}
        O&B\\C&O
    \end{pmatrix}^{-1} = 
    \begin{pmatrix}
        O & C^{-1} \\ B^{-1} & O
    \end{pmatrix} $ 
    \item “仅缺一块的”分块
    
    方法为“左乘同行,右乘同列”。先写逆矩阵的对角部分,同行同列根据逆矩阵寻找。
    \begin{itemize}
        \item $ 
        \begin{pmatrix}
        B & D  \\ O & C   
        \end{pmatrix}^{-1} = 
        \begin{pmatrix}
        B^{-1} & -B^{-1}DC^{-1} \\ O & C^{-1}
        \end{pmatrix} $ 
        \item $ 
        \begin{pmatrix}
        B & O \\ D & C   
        \end{pmatrix}^{-1} = 
        \begin{pmatrix}
        B^{-1} & O \\ -C^{-1}DB^{-1} & C^{-1}
        \end{pmatrix} $ 
        \item $ 
        \begin{pmatrix}
        O & B \\ C & D
        \end{pmatrix}^{-1} = 
        \begin{pmatrix}
        O & C^{-1} \\ B^{-1} & -B^{-1}DC^{-1}
        \end{pmatrix} $ 
        \item $ 
        \begin{pmatrix}
        D & B \\ C & O
        \end{pmatrix}^{-1} = 
        \begin{pmatrix}
        -C^{-1}DB^{-1} & C^{-1} \\ B^{-1} & O
        \end{pmatrix} $ 
    \end{itemize}
\end{itemize}

\sssubsection{可交换矩阵的结论}

\begin{itemize}
    \item 线性组合 - $ AB = aA+bB \Rightarrow AB = BA; $ 
    \item 一元二次形式 - $ A^2 + aAB = E \Rightarrow AB = BA; $
\end{itemize}

证明的思路是,因为可逆矩阵可交换,因此构造互逆矩阵。

其中,对前者,
\begin{equation*}
    \begin{aligned}
        AB = aA + bB &\Rightarrow A(B-aE) - bB = O \\ 
        &\Rightarrow A(B-aE) - b(B\Attention{-aE+aE}) = O \\ 
        &\Rightarrow (A-bE)(B-aE) = abE \\
        &\Rightarrow (B-aE)(A-bE) = abE
    \end{aligned}
\end{equation*}
展开,可以证明结论。