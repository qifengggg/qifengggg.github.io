\chapter{线性方程组}

\begin{equation*}
    \begin{aligned}
        \begin{cases}
            \textrm{齐次方程组} AX = 0\begin{cases}
                \textrm{表示形式}\begin{cases}
                    \textrm{矩阵}\\ \textrm{向量}
                \end{cases}\\ 
                \textrm{有解的判定}\\ 
                \textrm{解的性质}\begin{cases}
                    \textrm{通解}\\ \textrm{基解}\\ \textrm{组合}
                \end{cases}\\ 
                \textrm{求解方法}
            \end{cases}\\ 
            \textrm{非齐次方程组} AX = b\begin{cases}
                \textrm{有解判定} \\ \textrm{解的性质} \\ \textrm{解的结构 - 齐通+非齐特}
            \end{cases}\\ 
            \textrm{同解与公共解}\begin{cases}
                \textrm{通解 - 解集相同}\\ \textrm{公共解 - 解集有交集}
            \end{cases}
        \end{cases}
    \end{aligned}
\end{equation*}

\section{解的判定}

对于齐次方程 $ AX = 0, \begin{cases}
    \textrm{只有零解}\Rightarrow r(A) = n\\ 
    \textrm{有非零解}\Rightarrow r(A) < n
\end{cases}. $ 

其基础解系
\begin{itemize}
    \item 为 $ AX = 0 $ 的解;
    \item 线性无关;
    \item 数量为 $ n - r(A) $ 个。
\end{itemize}

对于非齐次方程 $ AX = b,\begin{cases}
    \textrm{无解}\Rightarrow r(A) \neq r(\bar A)\\ 
    \textrm{有解}\Rightarrow r(A) = r(\bar A)
    \begin{cases}
         = n \Rightarrow \textrm{解唯一}\\ 
         < n \Rightarrow \textrm{解不唯一(无穷)}
    \end{cases}
\end{cases} $ 

\section{求通解}

对于具体的数字矩阵,通过行变换或高斯消元做。

对抽象矩阵,
\begin{itemize}
    \item 通解为齐通 + 非齐特;
    \item 齐解任意组合仍然为齐解;
    \item 若 $ \xi_1,\cdots,\xi_t $ 均为 $ AX = b $ 的解,则
    $ \dis \sum k_i\xi_i $ 为 $ \begin{cases}
        \textrm{齐解,} \sum k = 0 \\
        \textrm{非齐解,} \sum k = 1 \\
    \end{cases} $ 
\end{itemize}