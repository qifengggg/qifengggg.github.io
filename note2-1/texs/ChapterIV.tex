\chapter{线性方程组}

\begin{equation*}
    \begin{aligned}
        \begin{cases}
            \textrm{齐次方程组} AX = 0\begin{cases}
                \textrm{表示形式}\begin{cases}
                    \textrm{矩阵}\\ \textrm{向量}
                \end{cases}\\ 
                \textrm{有解的判定}\\ 
                \textrm{解的性质}\begin{cases}
                    \textrm{通解}\\ \textrm{基解}\\ \textrm{组合}
                \end{cases}\\ 
                \textrm{求解方法}
            \end{cases}\\ 
            \textrm{非齐次方程组} AX = b\begin{cases}
                \textrm{有解判定} \\ \textrm{解的性质} \\ \textrm{解的结构 - 齐通+非齐特}
            \end{cases}\\ 
            \textrm{同解与公共解}\begin{cases}
                \textrm{通解 - 解集相同}\\ \textrm{公共解 - 解集有交集}
            \end{cases}
        \end{cases}
    \end{aligned}
\end{equation*}

\Section{解的判定}

对于齐次方程 $ AX = 0, \begin{cases}
    \textrm{只有零解}\Rightarrow r(A) = n\\ 
    \textrm{有非零解}\Rightarrow r(A) < n
\end{cases}. $ 

其基础解系
\begin{itemize}
    \item 为 $ AX = 0 $ 的解;
    \item 线性无关;
    \item 数量为 $ n - r(A) $ 个。
\end{itemize}

对于非齐次方程 $ AX = b,\begin{cases}
    \textrm{无解}\Rightarrow r(A) \neq r(\bar A)\\ 
    \textrm{有解}\Rightarrow r(A) = r(\bar A)
    \begin{cases}
         = n \Rightarrow \textrm{解唯一}\\ 
         < n \Rightarrow \textrm{解不唯一(无穷)}
    \end{cases}
\end{cases} $ 

\Section{求通解}

对于具体的数字矩阵,通过行变换或高斯消元做。

对抽象矩阵,
\begin{itemize}
    \item 通解为齐通 + 非齐特;
    \item 齐解任意组合仍然为齐解;
    \item 若 $ \xi_1,\cdots,\xi_t $ 均为 $ AX = b $ 的解,则
    $ \dis \sum k_i\xi_i $ 为 $ \begin{cases}
        \textrm{齐解,} \sum k = 0 \\
        \textrm{非齐解,} \sum k = 1 \\
    \end{cases} $ 
\end{itemize}

\begin{itemize}
    \item[\textbf{例题}] 三阶行列式 $ A = (\alpha_1,\alpha_2,\alpha_3) $ 有 $ 3 $ 个不同的特征值,且 $ \alpha_3 = \alpha_1+
    2\alpha_2, $ 则
    \begin{enumerate}[label = \Roman*.]
        \item $ r(A) = 2; $ 
        \item 若有 $ \beta = \alpha_1+\alpha_2+\alpha_3, $ 求方程组 $ AX = b $ 的通解。
    \end{enumerate}
    \item[\textbf{方法}]
    \begin{enumerate}[label = \Roman*.]
        \item $ \alpha_3 = \alpha_1 + 2\alpha_2\Rightarrow $ 可列消 $ \alpha_3 \Rightarrow r(A) \leq 2. $  
        
        有三个互异特征值 $ \Rightarrow $ 可相似对角化 $ \Rightarrow r(A) \geq 2 $ 因为至多一个特征值为零。

        故 $ r(A) = 2. $ 
        \item 由于 $ r(A) = 2, X $ 的基解有一个向量。
        
        $ \alpha_3 = \alpha_1 + 2\alpha_2\Rightarrow \alpha_1 + 2\alpha_2-\alpha_3 = 0
        \Rightarrow $ 一齐解为 $ (1,2,-1)^\top. $ 

        $ \beta = \alpha_1+\alpha_2+\alpha_3 \Rightarrow A(1,1,1)^\top = \beta, $ 
        故一非齐解为 $ (1,1,1)^\top. $ 

        故通解为 $ (1,1,1)^\top + k(1,2,-1)^\top, $ 其中 $ k\in \R. $ 
    \end{enumerate}
\end{itemize}

\begin{itemize}
    \item[\textbf{例题}] 设矩阵 $ A = \begin{bmatrix}
        1&-2&3&-4\\
        0&1&-1&1\\
        1&2&0&-3\\
    \end{bmatrix}, $ 求满足 $ AB = E_3 $ 的所有矩阵 $ B. $ 
    \item[\textbf{方法}] 不妨令 $ B = (\beta_1,\beta_2,\beta_3), $ 
    则有 $ A(\beta_1,\beta_2,\beta_3) = \begin{pmatrix}
        1&0&0\\0&1&0\\0&0&1\\
    \end{pmatrix} = (\xi_1,\xi_2,\xi_3), $ 此时对 $ A\beta_i = \xi_i $ 求解。

    注意到可以通过对 $ \begin{pmatrix}
        A & \xi_1 & \xi_2 & \xi_3
    \end{pmatrix} $ 做初等行变换一起完成高斯消元。解出 $ \beta_i, $
    则 $ (\beta_i) $ 即为所求 $ B. $ 
\end{itemize}

注意,若待求式为 $ XA = B, $ 则转置为 $ A^\top X^\top = B^\top. $ 

\Section{方程组的同解与公共解}

\sssubsection{公共解}

即两方程解集交集非空。具体而言,
\begin{itemize}
    \item 方程组已知 - 联立;
    \item 知一方程组与另一通解 - 用通解表示公共解,代回方程求解;
    \item 知两组通解 - 待定系数设通解,使通解相等求系数。
\end{itemize}

\sssubsection{同解}

即两方程解集相同。

当 $ r(A) = r(B)\red{=r\begin{pmatrix}
    A \\ B
\end{pmatrix}} $ 时,$ AX = 0 $ 与 $ BX = 0 $ 同解。

\begin{itemize}
    \item[\textbf{例题}] 设有方程组 $ (I):\left\{\begin{matrix}
        x_1+x_2 = 0\\ x_2 - x_4 = 0
    \end{matrix}\right. $ 与方程组 $ (II) $ 的通解为 $ k_1(0,1,1,0)^\top+k_2(-1,2,2,1)^\top, $ 
    \begin{enumerate}[label = \Roman*.]
        \item 求 $ (I) $ 基础解系;
        \item 求 $ (I),(II) $ 是否有非零公共解;若有则将其列出。
    \end{enumerate}
    \item[\textbf{方法}]
    \begin{enumerate}[label = \Roman*.]
        \item 由高斯消元法,其基础解系为 $ (0,0,1,0)^\top,(-1,1,0,1)^\top. $ 
        \item 不妨设公共解为 $ k_1(0,1,1,0)^\top + k_2(-1,2,2,1)^\top, $ 即
        $ (-k_2,k_1+2k_2,k_1+2k_2,k_2)^\top, k_1,k_2\in \R; $ 
        
        将其代入 $ (I), $ 有 $ k_1 = -k_2, $ 此时原公共解为 $ k(1,-1,-1,-1)^\top, k\in \R\backslash\{0\}. $ 
    \end{enumerate}
\end{itemize}

\begin{itemize}
    \item[\textbf{例题}] 设有$ 2\times 4 $ 矩阵 $ A,B, $ 且 $ (I) AX = 0 $ 通解为 $ k_1(1,2,2,-1)^\top + 
    k_2(0,-1,-3,2)^\top, (II) BX = 0 $ 通解为 $ \mu_1(2,-1,a+2,1)^\top + \mu_2 (-1,2,4,a+8)^\top, $ 
    其中 $ k_1,k_2,\mu_1,\mu_2\in \R, $ 若 $ (I),(II) $ 有非零公共解,求 $ a $ 及所有非零公共解。
    \item[\textbf{方法}] 设公共解为 $ X = k_1\alpha_1 + k_2\alpha_2 = l_1\beta_1 + l_2\beta_2, $ 
    即方程 $ (\alpha_1,\alpha_2,\beta_1,\beta_2)(k_{1},k_{2},l_{1},l_2)^\top = 0 $ 
    有非零解,因而 $ (\alpha_1,\alpha_2,\beta_1,\beta_2) $ 不满秩。
    $$
    (\alpha_1,\alpha_2,\beta_1,\beta_2)\xlongrightarrow{\textrm{行变换}}
    \begin{pmatrix}
        1&0&-2&1\\
        0&-1&5&-4\\
        0&0&-a-15&7\\
        0&0&7&-a-15\\
    \end{pmatrix}
    $$
    $$
    \xlongequal{\textrm{不满秩因而}a \neq -15}
    \begin{pmatrix}
        1&0&-2&1\\
        0&-1&5&-4\\
        0&0&-a-15&7\\
        0&0&0&-\dfrac{(a+8)(a+22)}{(a+15)}\\
    \end{pmatrix}
    $$
    因此 $ a = -8 $ 或者 $ a = -22. $ 
    
    $ a = -8 $ 时,$ X = k_1(1,1,-2,1)^\top;k_1 \in \R\backslash\{0\}; $

    $ a = -22 $ 时,$ X = k_2(-1,1,8,-5)^\top;k_2\in \R\backslash\{0\}. $ 
\end{itemize}

\begin{itemize}
    \item[\textbf{例题}] 已知齐次线性方程组 
    \begin{equation*}
        \begin{aligned}
            (I)\quad{}\begin{cases}
                x_1+2x_2+3x_3 = 0\\
                2x_1+3x_2+5x_3 = 0\\
                x_1+x_2+ax_3 = 0\\
            \end{cases};\quad{}
            (II)\quad{}\begin{cases}
                x_1+bx_2+cx_3 = 0\\
                2x_1+b^2x_2+(c+1)x_3 = 0\\
            \end{cases};
        \end{aligned}
    \end{equation*}
    同解,求 $ a,b,c. $ 
    \item[\textbf{解法}] 显然 $ r(B)\leq 2 < 3; $ 
    $ (I) $ 有非零二阶子式,即 $ r(A)\geq 2; $ 又由于 $ \rm (I),(II) $ 同解,可以知道
    $ r(A) = r(B) = r\begin{pmatrix}
        A \\ B
    \end{pmatrix} = 2. $ 

    因此,
    \begin{equation*}
        \begin{aligned}
            \begin{pmatrix}
                A\\ B
            \end{pmatrix}\xlongrightarrow{\textrm{初等行变换}}
            \begin{pmatrix}
                1&2&3\\0&1&1\\0&0&a-2\\0&0&c-b-1\\0&0&c-b^2-1\\
            \end{pmatrix}
        \end{aligned}
    \end{equation*}
    且其秩为 $ 2, $ 因此 $ a - 2 = c - b - 1 = c - b^2 - 1 = 0, $ 即
    $ a = 2,\begin{pmatrix}
        b\\c 
    \end{pmatrix}=\begin{pmatrix}
        1\\2
    \end{pmatrix} $ 或 $ \begin{pmatrix}
        0 \\ 1
    \end{pmatrix}. $ 
\end{itemize}

\begin{itemize}
    \item[\textbf{例题}] 设有 $ A_{m\times n},B_{m\times n}, $ 则
    $ AX = 0,BX = 0 $ 同解的充要条件为 $ \qline. $ 
    \begin{enumerate}[label = \Alph*)]
        \item $ A,B $ 向量组等价;
        \item $ A,B $ 行向量组等价;
        \item $ A,B $ 列向量组等价;
        \item $ A^\top x = 0,B^\top x = 0 $ 同解。
    \end{enumerate}
    \item[\textbf{方法}] 
    \begin{enumerate}[label = \Alph*)]
        \item 仅有 $ r(A) = r(B) = r(A,B), r\begin{pmatrix}
            A\\B
        \end{pmatrix} $ 未知;
        \item 有 $ r(A) = r(B) =r\begin{pmatrix}
            A\\B
        \end{pmatrix}, $ 符合定义,因而正确;
        \item 仅有 $ r(A) = r(B); $
        \item 仅有 $ r(A) = r(B) =r\begin{pmatrix}
            A^\top\\B^\top
        \end{pmatrix}; $
    \end{enumerate}
\end{itemize}

