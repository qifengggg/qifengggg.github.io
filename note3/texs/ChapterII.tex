\chapter{随机变量及其分布}

\section{随机变量及其分布函数}

将样本空间 $ \Omega $ 上的实值单值函数 $ X = X(\omega),\omega \in \Omega $ 
称为随机变量。

$ F(x) = P\{X\leq x\},x\in (-\infty,+\infty) $ 是随机变量的分布函数。

分布函数具有以下性质。
\begin{itemize}
    \item 非负性 - $ 0\leq F(x)\leq 1 $ ;
    \item 规范性 - $ F(-\infty) = {\displaystyle\lim_{x\rightarrow -\infty}}F(x) = 0;
    F(+\infty) = {\displaystyle\lim_{x\rightarrow +\infty}}F(x) = 1; $ 
    \item 单调不减性 - $ \forall x_1 < x_2, F(x_1)\leq F(x_2) $ ;
    \item 右连续性 - $ \forall x_0 \in R, F(x_0)={\displaystyle\lim_{x\rightarrow x_0^+}}F(x) = F(x_0+0) $ .
\end{itemize}

其中规范性可以优先考虑,因为其与微积分有关。

当已知随机变量 $ X $ 的分布函数 $ F(x) $ 时,有
\begin{itemize}
    \item $ P(X\leq b) = F(b) $ ;
    \item $ P(X = b) = F(b) - F(b-0)$ ;
    \item $ P(X < b) = F(b-0) $ ;
    \item $ P(X > b) = 1 - P(X \leq b) = 1 - F(b) $ ;
    \item $ P(a < X \leq b) = P(X\leq b) - P(X\leq a) = F(b) - F(a) $ ;
    \item $ P(a \leq X < b) = P(X < b) - P(X < a) = F(b-0) - F(a-0) $ ;
    \item $ P(a \leq X \leq b) = P(X\leq b) - P(X < a) = F(b) - F(a-0) $ ;
    \item $ P(a < X < b) = P(X < b) - P(X \leq a) = F(b-0) - F(a) $ ;
\end{itemize}

其中,前三条的应用最为广泛。

\sssubsection{离散型随机变量}

离散型随机变量的概率分布形如下表。

\begin{table}[!htbp]\centering
    \begin{tabular}{l|lllll}
    X & $x_1$ & $x_2$ & $\dots$ & $x_k$ & $\dots$ \\ \hline
    P & $p_1$ & $p_2$ & $\dots$ & $p_k$ & $\dots$
    \end{tabular}
\end{table}

其中 $ p_k>\geq0,k\in N^*,\sum_{i=1}^{\infty}p_i = 1 $.

离散型随机变量的分布函数为右连续的阶梯型函数,区间左开右闭,为概率的累加。

\sssubsection{连续型随机变量}

\begin{Def}[连续型随机变量概率密度]

    设随机变量 $ X $ 的分布函数为 $ F(x) $ ,
    若存在非负可积函数 $ f(x)\geq 0, x\in R $ 使得对任意实数 $ x $ ,都有
    $ \dis F(x) = P\{X \leq x\} = \int_{-\infty}^x f(t)\mathrm{d}t $, 则称 $ X $ 为
    连续型随机变量,函数 $ f(x) $ 为 $ X $ 的概率密度函数。
\end{Def}

\begin{Theo}[f(x)为密度函数的充要条件]

    $$
        f(x)\textrm{是概率密度}\Leftrightarrow\begin{cases}
            f(x)\geq 0 ;\\
            \dis \int_{-\infty}^{+\infty} f(x)\mathrm{d}x = 1.
        \end{cases}
    $$ 
\end{Theo}

连续型随机变量 $ X $  具有以下性质。
\begin{itemize}
    \item $ X $ 的分布函数是连续函数,因此 $ \forall a\in R, P\{X = a\} = 0 $;
    \item $ \forall a,b \in R, P\{a<X\leq b\} = \int_a^b f(x)\mathrm{d}x $;
    \item 在 $ f(x) $ 的连续点处,有 $ F'(x) = f(x) $ .
\end{itemize}

对连续型随机变量的题目,$ f(x) $ 简单或者具有特殊性质意味着作图解。

\section{常见分布}

\subsection{离散型}

离散型随机变量需要注意其取值(尤其是第一个值)以及其对应的概率。

\sssubsection{0-1分布}

\begin{table}[!htbp]\centering
    \begin{tabular}{c|cc}
    $ X $  & $ 0 $  & $ 1 $  \\ \hline
    $ P $  & $ 1-p $  & $ p $ 
    \end{tabular}
\end{table}
其中 $ 0 < p < 1 $ .

\sssubsection{二项分布}

设事件 $ A $ 在任意一次试验中出现的概率均为 $ 0<p<1 $,而 $ X $ 为$ n $ 重伯努利试验中 $ A $ 发生的次数,则
$ X $ 所有可能取值为 $ 0,1,\dots,n $ ,对应的概率为
$ \dis P\{X = k\} = C_n^kp^k(1-p)^{n-k} $ .

\sssubsection{几何分布 $ G(p) $ }

若 $ X $ 的概率分布为 $$ \dis P\{X=k\} = (1-p)^{k-1}p, k = 1,2,\dots, $$ 
则称 $ X $ 服从参数为 $ p $ 的几何分布,记为 $ X\sim G(p) $ .

\sssubsection{泊松分布 $ P(\lambda) $ }

若随机变量 $ X $ 的概率分布满足 $$ \dis P\{X = k\} \dfrac{\lambda^k}{k!}e^{-\lambda k},\lambda > 0,k=0,1,2,\dots,$$
则称 $ X $ 服从参数为 $ \lambda $ 的泊松分布,记为 $ X\sim P(\lambda) $ .

\sssubsection{超几何分布 $ H(N,M,n) $ }

若随机变量 $ X $ 的概率分布为
$$
    P\{X=k\} = \frac{\dis C_M^kC_{N-M}^{n-k}}{C_N^n},k = 0,1,2,\min(M,n), M,N,n\in Z^+
$$ 

则称 $ X $ 服从参数为 $ N,M,n $ 的超几何分布,记为 $ X\sim H(N,M,n) $ .

\subsection{连续型}

连续性随机变量的密度函数非零区间即其定义区间。

\sssubsection{均匀分布 $ U(a,b) $ }

$ \dis f(x) = \begin{cases}
    \dfrac{1}{b-a},&a<x<b,\\0,&\textrm{其他}
\end{cases} $ 

\sssubsection{指数分布 $ E(\lambda) $ }

$ \dis f(x) = \begin{cases}
    \lambda e^{-\lambda x},& x>0,\\0,&x\leq 0
\end{cases}$ 

注意,此处可能应用泊松过程的增量平稳性,即\newline 
$ \dis \forall s,t\leq0,n\leq0, P\{N(s+t)-N(s) = n\} = P\{N(t) = n\} $.

\sssubsection{正态分布 $ N(\mu,\sigma^2) $ }

$ \dis f(x) = \dfrac{1}{\sqrt{2\pi}\sigma}\exp\left\{-\dfrac{(x-\mu)^2}{2\sigma^2}\right\},x\in R $ 

特别地,$\dis X\sim N(0,1)\Rightarrow \phi(x) = \dfrac{1}{\sqrt{2\pi}}\exp\left\{-\dfrac{x^2}{2}\right\} $,
此时其分布函数为 $ \Phi(x) $.

对于一般的正态分布 $ X\sim N(\mu,\sigma^2) $,有 $ F(x) = \Phi(\dfrac{x-\mu}{\sigma}) $.

利用正态分布密度函数的规范性,可求泊松积分 $ \dis \int_{-\infty}^{+\infty} e^{-x^2}\mathrm{d}x = \sqrt{\pi}$ 

\section{随机变量函数的分布}

\subsection{离散型随机变量函数的分布}

对离散型随机变量函数,采用列表法。

\subsection{连续型随机变量函数的分布}

对已知概率密度为 $ f_X(x) $ 的随机变量 $ X $ 有 $ Y = g(X) $ ,需要求
$ Y $ 分布函数 $ F_Y(y) $ 和概率密度函数 $ f_Y(y) $时,有两种办法。

\sssubsection{公式法}

若 $ y = g(x) $ 严格单调,其反函数 $ x = h(y) $ 有一阶连续导数,则
$ Y = g(X) $ 也是连续型随机变量,其密度函数为\newline 
$ \dis f_Y(y) = \begin{cases}
    f_X(h(y))|h'(y)|,& \alpha < y < \beta\\0,&\textrm{其他}
\end{cases} $ 
其中 $ (\alpha,\beta) $ 为 $ y = g(x) $ 在 $ X $ 上可能取值的区间上的值域。

\sssubsection{分布函数法}

先按分布函数的定义求得 $ Y $ 的分布函数,再求导得到密度函数。即求 $ \dis F_Y(y) = P(Y\leq y) = P(g(X) 
\leq y) = \mathop{\int}\limits_{g(x)\leq y}f_X(x)\mathrm{d}x $,再求 $ f_Y(y) = F_Y'(y) $.  

具体而言,连续性随机变量的函数的分布函数法如下。
\begin{enumerate}
    \item 由 $ X $ 取值范围$ (a,b) $ 确定 $ y $ 的取值范围$ (c,d) $;
    \item 由分布函数的定义,确定 $ F_Y $ 的左右两头,即对 $ F_Y(y) = P(Y\leq y) $,\begin{itemize}
        \item $ y < c, F_Y(y) = 0 $ ;
        \item $ y > d, F_Y(y) = 1 $ .
    \end{itemize}
    \item 定中间,即 $ y \in (c,d) $ .
    \begin{equation*}
        \begin{array}{c}
            F_Y(y) = P(Y\leq y) = P(g(X)\leq y)\\\Downarrow\\
            P(X\leq h(y)),\textrm{此时若$ g(x) $ 分段则分段处理}\\\Downarrow\\
            \dis \int_{-\infty}^{h(y)} f_X(x)\mathrm{d}x,\textrm{此处取交集}
        \end{array}
    \end{equation*}     
\end{enumerate}

注意,若连续型随机变量分布函数为 $ F(x) $ ,若 $ Y = F(X) $ ,则 $ Y\sim U(0,1) $ .



