\chapter{随机事件和概率}

\section{随机事件、古典与几何概型}

\sssubsection{随机事件和样本空间}

\begin{itemize}
    \item 样本空间 $ \Omega $ - 随机试验所有可能结果组成的集合;
    \item 样本点 $ \omega $ - 样本空间的元素;
    \item 随机事件 - 样本空间 $ \Omega $ 的子集;
    \item 事件发生 - 当且仅当一子集中一样本点出现时称其发生;
\end{itemize}

\sssubsection{古典概型}

若随机试验 $ E $ 
\begin{itemize}
    \item 只有有限个样本点(有限性);
    \item 每个样本点出现的可能性相等(等可能性);
\end{itemize}
则称 $ E $ 为古典型试验。

若事件 $ A $ 中含有 $ k $ 个样本点,则其概率为
$ \dis P(A) = \dfrac{A\textrm{样本点个数}}{\Omega\textrm{中样本点个数}} = \dfrac{k}{n} $ 。

若随机试验 $ E $ 
\begin{itemize}
    \item 有无限个样本点;
    \item 每个样本点出现的可能性相等;
\end{itemize}
则称 $ E $ 为几何型试验。

对事件 $ A $ ,$ P(A) = \dfrac{L(A)}{L(\Omega)} $ ,
其中 $ L $ 代表对应事件的几何度量。

\section{事件关系与概率性质及公式}

\sssubsection{事件运算的性质}

进行事件运算时,一般先逆后积再和差;运算还有性质如下。
\begin{itemize}
    \item 交换律 - $ \dis A \cup B = B \cup A; AB = BA $ ;
    \item 结合律 - $ \dis (A\cup B) \cup C = A\cup(B\cup C);\dis (A\cap B) \cap C = A\cap(B\cap C) $ ;
    \item 分配律 - $ \dis (A\cap B)C = (AC)\cap(BC); A\cup(BC) = (A\cup B)(A\cup C) $ ;
    \item 德摩根律 - $ \dis \overline{A\cup B} = \overline A\cap\overline B;
    \overline{A\cap B} = \overline A\cup\overline B; $ 
\end{itemize}

\sssubsection{概率的定义、性质与公式}

\begin{Def}[概率的公理化定义]

    设 $ E $ 是一随机事件,$ \Omega $ 是其样本空间,$ P(A) $ 是一映射将每一个事件 $ A $ 映射到一实数,
    若集合函数 $ P\{\bullet\} $ 满足
    \begin{itemize}
        \item 非负性 - 对任意事件 $ A $ 有 $ P(A)>0 $ ;
        \item 规范性 - 对必然事件 $ \Omega $ 有 $ P(\Omega) = 1 $ ;
        \item 可列可加性 - $ \forall i\neq j, i,j \in N^*, A_iA_j = \emptyset $,
        有 $ P(A_1\cup A_2\cup \dots) = P(A_1) + P(A_2) + \dots $ ;
    \end{itemize}
    则称 $ P(A) $ 为事件 $ A $ 的概率。
\end{Def}

概率有以下性质。
\begin{itemize}
    \item 非负性 - $ \forall A \in \Omega, 0 \leq P(A) \leq 1 $ ;
    \item 规范性 - $ P(\emptyset) = 0; P(\Omega) = 1 $ ;
    \item 有限可加性 - $ \forall i\neq j, i,j = 1,2,\dots,n, A_iA_j = \emptyset $,\newline
    有 $ P(A_1\cup A_2\cup \dots \cup A_n) = P(A_1) + P(A_2) + \dots + P(A_n)$ ;
\end{itemize}

概率有以下公式。
\begin{itemize}
    \item 求逆公式 - 对任意事件 $ A , P(\overline A) = 1 - P(A) $,常用于正难则反;
    \item 加法公式 - $ P(A\cup B) = P(A) + P(B) - P(AB) $ ;
    \item 减法公式 - 对任意二事件 $ A,B $ 有 $ P(A-B) = P(A\overline B)=P(A)-P(AB) $ ;
    特别地,若有 $ B\subset A $ ,则有 $ P(A-B)=P(A)-P(B) $ , $ P(B)\leq P(A) $ ;
\end{itemize}

\begin{Field}[概率不等式]

    \begin{enumerate}
        \item $ 0 \leq P(A) leq 1 $ ;
        \item $ B\subset A =  P(B)\leq P(A) $ ;
        \item $ P(A\cup B)\leq P(A)+P(B) $ .
    \end{enumerate}
\end{Field}

\section{条件概率与乘法公式}

\begin{Def}[条件概率]

    设 $ A,B $ 为二事件,且 $ P(A) > 0 $ ,则称 $ P(B|A) = \dfrac{P(AB)}{P(A)} $ 
    为在事件 $ A $ 发生的条件下,事件 $ B $ 发生的条件概率。
\end{Def}

注意,条件概率满足概率的一切性质。

条件概率具有以下性质。
\begin{itemize}
    \item $ 0\leq P(B|A)\leq 1 $ ;
    \item $ P(\emptyset|A) = 0,P(\Omega|A) = 1 $ ;
    \item $ P(\overline B|A) = 1-P(B|A) $ ;
    \item $ P(B_1\cup B_2|A) = P(B_1|A) + P(B_2|A) - P(B_1B_2|A) $ .
\end{itemize}

计算条件概率时,抽象问题用定义;
对具体问题,将概率空间从 $ \Omega $ 缩小到 $ A $ ;
对逆概问题,利用贝叶斯公式。

\begin{Field}[条件概率的乘法公式]

    \begin{itemize}
        \item 若 $ P(A)>0 $ ,则 $ P(AB) = P(A)P(B|A) $ ;
        \item 对事件 $ A,B,C $ ,若 $ P(AB)>0 $ ,则
        $ P(ABC) = P(A)P(B|A)P(C|AB) $ .
    \end{itemize}
\end{Field}

\section{独立性与伯努利概型}

\sssubsection{独立性}

\begin{Def}[独立性]

    若 $ P(AB)=P(A)P(B) $ 则称事件 $ A,B $ 相互独立。

    若 $ A,B $ 相互独立,则 $\overline A,B $和$ A,\overline B $还有$ \overline A,\overline B $ 都独立。
\end{Def}

\sssubsection{三事件独立性}

\begin{equation*}
    A,B,C\textrm{相互独立}\Leftrightarrow\left.\begin{array}{r}
        \begin{cases}
            P(AB) = P(A)P(B)\\
            P(AC) = P(A)P(C)\\
            P(BC) = P(B)P(C)\\
        \end{cases}\\P(ABC)=P(A)P(B)P(C)
    \end{array}\right\}\Leftrightarrow A,B,C \textrm{相互独立}
\end{equation*}

若 $ A,B,C $ 相互独立,则 $ A,B $ 经过和、积、差运算后得到的事件与 $ C,\overline C $ 独立,
但是 $ A,B,C $ 经过运算的事件不一定。

二事件独立的等价条件

\begin{itemize}
    \item $ P(AB)=P(A)P(B) $ ;
    \item $ \dfrac{P(AB)}{P(A)} = P(B) = P(B|A), P(A)>0 $ ;
    \item $ P(B|\overline A) = P(B|A),0<P(A)<1 $ ;
    \item $ \Attention{P(B|A)+P(\overline B|\overline A) = 1,0<P(A)<1} $ ;
    \item $ \Attention{P(B|\overline A)+P(\overline B|A) = 1,0<P(A)<1} $ .
\end{itemize}

\sssubsection{$ n $ 重伯努利概型}

若试验 $ E $ 只有 $ A $ 和 $ \overline{A} $ 两个可能结果,称
$ E $ 为伯努利试验,每次实验中,$ P(A) = p,P(\overline A) = 1-p $ ;
将伯努利试验独立重复 $ n $ 次,则其中成功 $ k $ 次的概率为
$$
    P_n(k) = C_n^kp^k(1-p)^{n-k},k = 0,1,\dots,n
$$ 

\section{全概率公式与贝叶斯公式}

\begin{Field}[完备事件组]

    若一组事件 $ A_n $ 满足$$
        \bigcup_{i=1}^nA_i = \Omega,A_iA_j = \emptyset, 1 \leq i \neq j \leq n
    $$ 则称其为完备事件组。
\end{Field}

\begin{Field}[全概率公式]

    若一组事件 $ A_n $ 是完备事件组,且 $ P(A_i) > 0,i = 1,2,\dots,n $ ,则$$
        P(B) = \sum_{i=1}^nP(A_i)P(B|A_i)
    $$ 
\end{Field}

\begin{Field}[贝叶斯公式]

    若一组事件 $ A_n $ 是完备事件组,且 $P(B)>0, P(A_i) > 0,i = 1,2,\dots,n $ ,则
    $$
        P(A_j|B) = \dfrac{P(A_j)P(B|A_j)}{\dis \sum_{i=1}^nP(A_i)P(B|A_i)},j = 1,2,\dots,n
    $$ 
\end{Field}

