\chapter{随机变量的数字特征}

\section{数学期望}

\subsection{离散型随机变量的数学期望}

\sssubsection{一维离散型随机变量的数学期望}

$$
    E(X) = \sum_{i=1}^{\infty} x_ip_i
$$ 

\sssubsection{一维离散型随机变量函数的数学期望}

$$
    E(g(X)) = \sum_{i=1}^{\infty} g(x_i)p_i
$$ 

\sssubsection{二维离散型随机变量的数学期望}

$$
    E(g(X,Y)) = \sum_{i=1}^{\infty}\sum_{j=1}^{+\infty} g(x_i,y_i)p_{ij}
$$ 

\subsection{连续型随机变量的数学期望}

\sssubsection{一维连续型随机变量的数学期望}

$$
    E(x) = \int_{-\infty}^{+\infty}xf(x)\mathrm{d}x = \int_{-\infty}^{+\infty}x\mathrm{d}F(x)
$$ 

\sssubsection{一维连续型随机变量函数的数学期望}

$$
    E(g(X)) = \int_{-\infty}^{+\infty}g(x)f(x)\mathrm{d}x = \int_{-\infty}^{+\infty}g(x)\mathrm{d}F(x)
$$ 

\sssubsection{二维连续型随机变量的数学期望}

$$
    E(g(X,Y)) = \int_{-\infty}^{+\infty}\int_{-\infty}^{+\infty}
    g(x,y)f(x,y)\mathrm{d}x\mathrm{d}y
$$ 

\subsection{数学期望的性质}

对常数 $ c,c_1,c_2 $ ,随机变量 $ X,Y $ ,
\begin{enumerate}
    \item $ E(c) = c $;
    \item $ E(cX) = cE(X) $ ;
    \item $ E(c_1X+c_2Y) = c_1E(X)+c_2E(Y) $ ;
    \item 若 $ X,Y $ 独立,则 $ E(XY) = E(X)E(Y) $.
\end{enumerate}

\section{方差}

\begin{Def}[方差]

    $ D(x) = E[X-E(X)]^2. $ 
\end{Def}

计算方差时,常用的公式为$$
    D(x) = E(X^2) - [E(X)]^2
$$ 

对常数 $ a,b $,随机变量 $ X,Y $ ,有\begin{itemize}
    \item $ D(c) = 0 $ ;
    \item $ D(aX+b) = a^2D(X) $ ;
    \item $ D(X\pm Y) = D(X) + D(Y) \pm 2\Cov(X,Y) $ ,
    其中 $ \Cov(X,Y) = E(XY) - E(X)E(Y) $.
\end{itemize}

常见分布的数学期望和方差如下表。

\begin{table}[!htbp]\centering
    \begin{tabular}{clcc}
    \toprule
    分布名称  & 分布记号                  & 数学期望           & 方差                         \\ \midrule
    0-1分布   &$  X\sim B(1,p) $           & $ p   $             & $ p(1-p)    $               \\
    二项分布  & $ X\sim B(n,p)   $         & $ np   $            & $ np(1-p)  $             \\
    泊松分布  &$  X\sim P(\lambda) $       & $ \lambda    $      & $ \lambda    $           \\
    几何分布  & $ X \sim G(p)  $           & $ \frac 1p  $      & $ \frac{1-p}{p^2}  $    \\
    超几何分布 & $ X \sim H(n,M,N)    $     &$  n\frac MN  $     &                         \\
    均匀分布  & $ X\sim U(a,b) $           & $ \frac{b+a}2  $   &$  \frac{(b-a)^2}{12} $  \\
    指数分布  &$  X\sim E(\lambda)   $     & $ \frac1\lambda $  & $ \frac1{\lambda^2} $   \\
    正态分布  & $ X\sim N(\mu,\sigma^2) $  & $ \mu $             &$  \sigma^2    $          \\
    卡方分布  & $ \chi^2\sim \chi^2(n)  $  & $ n $               & $ 2n    $                \\ \bottomrule
    \end{tabular}
\end{table}

\sssubsection{注意事项}

当给定一个含参数的概率密度函数时,向常见分布上凑。

对求 $ E(g(x)) $ ,可以将其化为 $ \dis \int_{-\infty}^{+\infty}g(x)f(x)\mathrm{d}x $,然后尝试
将其整理为 $ \dis \int_{-\infty}^{+\infty}g_1(x)f_1(x)\mathrm{d}x $,其中 $ f_1(x) $ 是
一常见分布 $ T $ 的密度函数,从而将其化为 $ E(g_1(T)) $ .

对于求分布函数$ F(x) $ 为不同分布的分布函数和的随机变量的期望,
将期望转化为 $ \dis \int x\mathrm{d}F(x) $ ,将 $ F(x) $ 拆开,
分别计算积分。

有时,可以利用等式
$ \dis E(X^2) = E(X(X-1)+X) $.

\section{随机变量的协方差和相关系数}

\subsection{协方差}

\begin{Def}[协方差]

    $ \Cov(X,Y) = E[X-E(X)][Y-E(Y)]. $ 
\end{Def}

计算协方差时,常用公式
$ \dis \Cov(X,Y) = E(XY) - E(X)E(Y) $ .

协方差有如下性质。
\begin{itemize}
    \item $ \Cov(X,X) = D(X) $ ;
    \item $ \Cov(X,Y) = \Cov(Y,X) $ ;
    \item $ \Cov(aX+b,cY+d) = ac\cdot\Cov(X,Y) $ .
\end{itemize}

\subsection{相关系数}

\begin{Def}[相关系数]

    $ \dis \rho_{XY} = \dfrac{\Cov(X,Y)}{\sqrt{D(X)}\sqrt{D(Y)}} $ .
\end{Def}

$ \rho_{XY} = 0 $ 时称$ X,Y $ 不相关,否则称其相关。

注意,独立 $ \Rightarrow $ 不相关,特别地,对二维正态分布的两边缘分布有不相关性 $ \Rightarrow $ 独立性。

对随机变量 $ X,Y $ ,相关系数有如下性质。
\begin{itemize}
    \item $ |\rho_{XY} \leq 1| $ ;
    \item $ \exists a\neq0,b,P\{Y = aX+b\} = 1 \Leftrightarrow \rho_{XY} = \dfrac{|a|}{a} $.
\end{itemize}

\sssubsection{注意事项}

对于难以快速计算的题目,应用验证法,即通过已知性质排除备选项。

\subsection{切比雪夫不等式}

\begin{Theo}[切比雪夫不等式]

    设随机变量 $ X $ 期望和方差 $ E(X) = \mu,D(X) = \sigma^2 $ 都存在,
    则对任意 $ \varepsilon > 0 $ 都有 $$
        P\left\{|X-E(X)|\geq \varepsilon\right\} \leq \dfrac{D(X)}{\varepsilon^2}
    $$ 
\end{Theo}

\begin{proof}
    \begin{equation*}
        \begin{aligned}
            P\{|X-E(X)|\geq \varepsilon\} &= \int\limits_{|X-E(X)|\geq\varepsilon}f(x)\mathrm{d}x&
            \\&\leq\int\limits_{|X-E(X)|\geq\varepsilon}\dfrac{|X-E(X)|}{\varepsilon}^2f(x)\mathrm{d}x
            &\textrm{(放大被积函数)}\\&\leq
            \dfrac{1}{\varepsilon^2}\int_{-\infty}^{+\infty}(X-E(X))^2f(x)\mathrm{d}x&
            \textrm{(放大积分区间)}\\&=\dfrac{1}{\varepsilon^2}E((X-E(X))^2)
            =\dfrac{D(X)}{\varepsilon^2}
        \end{aligned}
    \end{equation*}
    整理即为待证结论。
\end{proof}