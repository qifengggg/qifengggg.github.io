\chapter{大数定律和中心极限定理}

\section{依概率收敛}

\begin{Def}[依概率收敛]

    设 $ X_i,i=1,2,\dots,n $ 是一列随机变量序列,$ a $ 是一实数,\newline 若对任意正数$ \varepsilon $ 都有
    $ \dis {\displaystyle\lim_{n\rightarrow +\infty}}P\{|X_n-a|\leq\varepsilon\} = 1 $,则
    称上述序列依概率收敛于 $ a, $ 记为 $ X_n\xrightarrow{P}a $ .
\end{Def}

依概率收敛有如下性质。
\begin{itemize}
    \item 若 $ X_n\xrightarrow{P}X $ ,而 $ g(x) $ 是 $ R $ 上的连续函数,则
    $ g(X_n)\xrightarrow{P}g(X) $ .
    \item 若 $ X_n\xrightarrow{P}a,Y_n\xrightarrow{P}b $ 而 $ g(x,y) $ 在 $ (a,b) $ 处连续,
    则 $ g(X_n,Y_n)\xrightarrow{P}g(a,b) $ .
\end{itemize}

\section{大数定律}

大数定律的本质是算数平均依概率收敛于统计平均,也即数学期望。

\subsection{切比雪夫大数定律}

设 $ X_i,i=1,2,\dots $ 其中任意的两随机变量独立,若其方差 $ DX_i $ 均存在且有一致的上界,
也即存在 $ C\in R $ 使得 $ DX_i\leq C $ ,则$$
    {\displaystyle\lim_{n\rightarrow +\infty}}P\left\{
        \left|\dfrac{1}{n}\sum_{i=1}^bX_i - \dfrac{1}{n}\sum_{i=1}^nEX_i\right|\leq \varepsilon
    \right\} = 1
$$ 

事实上,有 $ \dis \dfrac{1}{n}\sum X_i\xrightarrow{P}\dfrac{1}{n}\sum EX_i $.

\subsection{辛钦大数定律}

设随机变量 $ X_i,i=1,2,\dots $ 相互独立同分布,且其数学期望 $ EX = \mu $ ,则对
任意 $ \varepsilon > 0 $ 都有
$$
    {\displaystyle\lim_{n\rightarrow +\infty}}P\left\{
        \left|\dfrac{1}{n}\sum_{i=1}^bX_i - \mu\right|\leq \varepsilon
    \right\} = 1
$$ 

\subsection{伯努利大数定律}

设 $ n_A $ 是 $ n $ 次独立重复试验中 $ A $ 发生的次数,$ p $ 是事件 $ A $ 在
每次试验中发生的概率,则
$$
    {\displaystyle\lim_{n\rightarrow +\infty}}P\left\{
        \left|\dfrac{n_A}{n} - p\right|\leq \varepsilon
    \right\} = 1
$$ 

事实上,令 $ \dis X_i = \begin{cases}
    1,&A\\0,&\overline A
\end{cases} $,其中 $ X_i\xlongequal{i.i.d}X_j,i\neq j $ ,容易发现其是辛钦大数定律的推论。

\section{中心极限定理}

其本质为足够多的$ (n\geq30) $ 相互独立的随机变量和近似地服从正态分布。

\subsection{列维—林德伯格中心极限定理}

设随机变量 $ X_i,i=1,2,\dots $ 相互独立同分布,有 $ EX=\mu,DX = \sigma^2 $ ,则当
$ n\rightarrow\infty $ 时有
\begin{equation*}
    \begin{aligned}
        &\sum_{i=1}^nX_i\sim N(n\mu,n\sigma^2)\\\Rightarrow&
        \dfrac{\sum X_i-\sum EX_i}{\sqrt{\sum DX_i}}\sim N(0,1),n\rightarrow\infty
        \\\Rightarrow&{\displaystyle\lim_{n\rightarrow \infty}}
        P\left\{
            \dfrac{\sum X_i-\sum EX_i}{\sqrt{\sum DX_i}}\leq x
        \right\}=\Phi(x)
    \end{aligned}
\end{equation*}

\subsection{隶莫佛—拉普拉斯中心极限定理}

设随机变量 $ X_n $ 服从参数为 $ n,p $ 的二项分布,则当 $ n\rightarrow\infty(\geq30) $ 时,
有

\begin{equation*}
    \begin{aligned}
        X_n\sim N(np,np(1-p))
    \end{aligned}
\end{equation*}
