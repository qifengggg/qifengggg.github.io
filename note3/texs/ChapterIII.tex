\chapter{多维随机变量及其分布}

\section{二维随机变量及其分布}

\begin{Def}[二维随机变量]

    设 $ X = X(\omega),Y=Y(\omega) $ 是定义在样本空间 $ \Omega $ 上的两实值单值函数,则称
    向量 $ (X,Y) $ 为二位随机变量或随机向量。
\end{Def}

二维随机变量的分布函数定义为 $ F(x,y) = P(X\leq x, Y\leq y) $ ,其具有以下性质。

\begin{enumerate}
    \item 单调性 - $ F(x,y) $ 是变量 $ x $ 或变量 $ y $ 的单调不减函数,即对$ x_1 < x_2, y_1<y_2 $,有
    $ F(x_1,y)\leq F(x_2,y);F(x,y_1)\leq F(x,y_2) $ ;
    \item 有界性 - 对任意 $ x,y $ 有 $ 0\leq F(x,y)\leq 1 $ ,且
    $ F(-\infty,y) = 0; F(x,-\infty) = 0; F(-\infty,-\infty) = 0; F(+\infty,+\infty) = 1 $;
    \item 右连续性 -  $ F(x,y) $ 分别对 $ x,y $ 右连续,即
    $ F(x+0,y) = F(x,y) = F(x,y+0) $;
    \item 非负性 - $ \dis \forall x_1 < x_2, y_1 < y_2, $ 有
    $ \dis P(x_1< X\leq x_2,y_1<Y\leq y_2) F(x_2,y_2) - F(x_1,y_2) - F(x_2,y_1)+F(x_1,y_1) $,即
    $ (X,Y) $ 落入矩形 $ \left(x_1,x_2\right]\times \left(y_1,y_2\right] $ 区域内的概率。
\end{enumerate}

其具有边缘分布函数,即
\begin{itemize}
    \item $ F_x(x) = P(X\leq x) = {\displaystyle\lim_{y\rightarrow +\infty}}F(x,y) $ ;
    \item $ F_Y(y) = P(Y\leq y) = {\displaystyle\lim_{x\rightarrow +\infty}}F(x,y) $ ;
\end{itemize}

当 $ F(x,y) = F_X(x)F_Y(y) $ 时,$ X,Y $ 独立。

\section{二维离散型随机变量}

二维离散型随机变量的分布为 $ P\{X = x_i,Y = y_j\} = p_{ij},i,j\in N^* $ .

二维离散型随机变量的题目需要列表,其事件的概率从表中找对应点。

其边缘密度分布的计算方法为将表按行或列求和,即
\begin{itemize}
    \item $\dis p_i = P\{X = x_i\} = \sum_jp_{ij} $;
    \item $\dis p_j = P\{Y = y_i\} = \sum_ip_{ij} $.
\end{itemize}

$ X,Y $ 独立的定义为 $ P\{X = x_i,Y = y_i\} = P\{X = x_i\}P\{Y = y_i\} $ 对任意 $ p_{ij} $ 成立。

当 $ P\{Y=y_i\}>0 $ 时,在 $ Y = y_i $ 的条件下,$ X $ 的条件概率为
$$
    P\{X = x_i|Y = y_j\} = \dfrac{P\{X = x_i,Y = y_j\}}{P\{Y = y_j\}} = \dfrac{p_{ij}}{p_j}.
$$ 

$ Y $ 的条件概率同理。

注意,在联合分布列中,对两随机变量,\begin{itemize}
    \item $ \exists p_{ij} = 0\Rightarrow $ 不独立;
    \item 独立 $ \Leftrightarrow $ 分布列按行或列成比例。
\end{itemize}

\section{二维连续型随机变量}

\subsection{定义与性质}

\begin{Def}[二维连续型随机变量的概率密度]

    设有二维随机变量 $ (X,Y) $ ,其分布函数为 $ F(x,y) $ ,若存在
    非负可积的二元函数 $ f(x,y) $ 使得对任意实数 $ x,y $ 都有
    $ \dis F(x,y) = \int_{-\infty}^{+\infty}\int_{-\infty}^{+\infty}
    f(u,v)\mathrm{d}u\mathrm{d}v $ ,则称 $ (X,Y) $ 为二维连续型随机变量,
    称 $ f(x,y) $ 为其概率密度函数,$ F(x,y) $ 是其分布函数。
\end{Def}

$ f(x,y) $ 具有以下性质。
\begin{itemize}
    \item 非负性 - $ f(x,y)\leq 0 $ ;
    \item 规范性 - $ \int_{-\infty}^{+\infty}\int_{-\infty}^{+\infty} f(x,y)\mathrm{d}x\mathrm{d}y = 1$;
    \item 若$ f(x,y) $ 在 $ (x,y) $ 处连续,则有 $ f(x,y) = 
    \dfrac{\partial^2 F(x,y)}{\partial x\partial y} $.
\end{itemize}

随机点落在一区域 $ G $ 内的概率为 $ \dis P\{(X,Y)\in G\} = \mathop{\iint}\limits_G 
f(x,y)\mathrm{d}x\mathrm{d}y $ 

$ X $ 的边缘概率密度为 $ f_X(x) \int_{-\infty}^{+\infty}f(x,y)\mathrm{d}y $,对 $ Y $ 同理。

若 $ X,Y $ 独立,则 $ f(x,y) = f_X(x)f_Y(y)\forall x,y $ .

对给定的实数 $ y $ ,$ Y $ 的边缘概率密度 $ f_Y(y)>0 $ ,则此时$ X $ 的条件概率密度为
$$ \dis f_{X|Y}(x|y) = \frac{f(x,y)}{f_Y(y)} $$ 同理可以得到 $ Y $ 的条件概率密度。

注意,对 $ x,y $ 使得 $ f_Y(y)>0 $ ,有 $ \dis f_{X|Y}(x|y) = \dfrac{f(x)f_{Y|X}(y|x)}{f_Y(y)} $ .

对二维连续随机变量区域划分时,需要划分的是概率密度非零区域。需要找到边界点,然后向上向右作射线。

对二维连续随机变量,
\begin{equation*}
    \begin{aligned}
        \textrm{X,Y独立}\Leftrightarrow\begin{cases}
            \textrm{$ f(x,y) $ 非零区域为矩形区域;}\\\textrm{$ f(x,y) $ 变量可分离.}
        \end{cases}
    \end{aligned}
\end{equation*}

\subsection{常见分布}

\sssubsection{二维均匀分布}

若 $ (X,Y) $ 的概率密度为 $ \dis f(x,y)=\begin{cases}
    \dfrac{1}{S_D},&(x,y)\in D\\0,&(x,y)\not\in D
\end{cases} $ ,则称 $ (X,Y) $ 服从区域 $ D $ 上的二维均匀分布。

注意,\begin{enumerate}
    \item 对区域 $ G \subset D $ ,有 $ F = \dfrac{S_G}{S_D} $;
    \item 二维均匀分布的边缘分布为一维均匀分布,当且仅当其非零区域为矩形;
    \item 二维均匀分布的条件分布一定是均匀分布。
\end{enumerate}

\sssubsection{二维正态分布}

若 $ (X,Y) $ 概率密度为
$$
    f(x,y) = \exp\left\{-\dfrac{1}{2(1-\rho^2)}\right\}
    \left\{
        \dfrac{(x-\mu_1)^2}{\sigma^2_1}+
        \dfrac{(y-\mu_2)^2}{\sigma^2_2}-
        \dfrac{2\rho(x-\mu_1)(y-\mu_2)}{\sigma_1\sigma_2}
    \right\}\Big\slash\left\{
        2\pi\sigma_1\sigma_2\dsqrt{1-\rho^2}
    \right\}
$$ 

其中 $ x,y\in R, \mu_i,\sigma_i,\rho_i > 0, -1<\rho<1 $ 均为常数,
则称 $ (X,Y) $ 服从参数为 $ \mu_1,\mu_2,\sigma_1,\sigma_2,\rho $ 的
二维正态分布,记为 $ (X,Y) \sim N(\mu_1,\mu_2;\sigma_1^2,\sigma_2^2;\rho) $.

二维正态分布具有性质如下。

\begin{itemize}
    \item $ X,Y $ 独立 $ \Rightarrow \rho = 0 $ ;
    \item 两个边缘分布服从一维正态分布,即 $ X\sim N(\mu_1,\sigma_1^2) $ 
    和 $ Y\sim N(\mu_2,\sigma_2^2) $,且$ \rho $ 为二者相关系数 ;
    \item $ X,Y $ 的任意非零线性组合 $ aX+bY $ 服从一维正态分布,即$$
        aX + bY \sim N(a\mu_1+b\mu_2,a^2\sigma_1^2 + b^2\sigma_2^2 + 2ab\sigma_1\sigma_2\rho)
    $$ 
    \item 若 $ Z_1 = aX + bY $ 与 $ Z_2 = cX + dY $ 为 $ X,Y $ 的非零线性组合,
    若 $ \begin{vmatrix}
        a&b\\c&d
    \end{vmatrix}\neq0 $ ,则 $ (Z_1,Z_2) $ 仍然服从二维正态分布。
\end{itemize}

\section{二维随机变量函数的分布}

\subsection{分布的独立可加性}

\begin{itemize}
    \item 若 $ X\sim B(m,p), Y\sim B(n,p) $ ,且 $ X,Y $ 相互独立,则 $ X+Y\sim B(m+n,p) $ ;
    \item 若 $ X\sim P(\lambda_1), Y\sim P(\lambda_2) $ ,
    且 $ X,Y $ 相互独立,则 $ X+Y\sim P(\lambda_1+\lambda_2) $ ;
    \item 若 $ X\sim N(\mu_1,\sigma_1^2), Y\sim N(\mu_2,\sigma_2^2) $ ,
    且 $ X,Y $ 相互独立,则 $ X+Y\sim N(\mu_1+\mu_2,\sigma_1^2+\sigma_2^2) $ ;
\end{itemize}

更一般地,若 $ X_i\sim N(\mu_i,\sigma_i^2),i=1,2,\dots,n $ ,且 $ X_i,i=1,2,\dots,n $ 相互独立,则
$$
    Y = \sum_{i=1}^n C_iX_i + C \sim N(\sum_{i=1}^nC_i\mu_i+C,\sum_{i=1}^nC_i^2\sigma_i^2)
$$ 

其中 $ C_i,i=1,2,\dots,n $ 是不全为零的常数。

\subsection{二维离散型随机变量函数的分布}

使用表格法列出关键取值对及其对应的概率。

\subsection{二维连续型随机变量函数的分布}

已知二维连续型随机变量 $ (X,Y) $ 的概率密度 $ f(x,y) $ ,求连续函数 $ Z = g(X,Y) $ 的概率密度 $ f_Z(z) $ .

可以采取分布函数法或公式法。

线性规划最值时,注意可行域的边界点。

\subsubsection{分布函数法}

对于分布函数法,具体而言,\begin{enumerate}
    \item 由 $ (x,y)\in D $ 得到 $ z = g(x,y) \in [a,b] $ ;
    \item 由 $ F_z(z) = P(Z\leq z),z\in R $ 定两边,即
    $ z<a, F_z(z) = 0; z>b F_z(z) = 1; $ 
    \item 对 $ z\in [a,b] $ ,\begin{equation*}
        \begin{aligned}
            F_Z(z) &= P(Z\leq z) = P(g(X,Y)\leq z)
            \\&= \iint\limits_{g(x,y)\leq z \cup D}f(x,y)\mathrm{d}x\mathrm{d}y
        \end{aligned}
    \end{equation*}
    注意,不要忘记与非零区间取交集。
\end{enumerate}

\subsubsection{公式法}

对于公式法,设 $ (X,Y) $ 的联合概率密度为 $ f(x,y) $,有以下公式。

\sssubsection{求和 $ Z = X + Y $ }

$ Z = X + Y $ 的概率密度为$$
    f_z(z) = \int_{-\infty}^{+\infty}f(x,z-x)\mathrm{d}x
    = \int_{-\infty}^{+\infty}f(z-y,y)\mathrm{d}y
$$ 
若 $ X,Y $ 还相互独立,则适用卷积公式$$
f_z(z) = \int_{-\infty}^{+\infty}f_X(x)f_Y(z-x)\mathrm{d}x
= \int_{-\infty}^{+\infty}f_X(z-y)f_Y(y)\mathrm{d}y
$$ 

\sssubsection{求差 $ Z = X - Y $ }

$ Z = X - Y $ 的概率密度为$$
    f_z(z) = \int_{-\infty}^{+\infty}f(x,x-z)\mathrm{d}x
    = \int_{-\infty}^{+\infty}f(z+y,y)\mathrm{d}y
$$ 

\sssubsection{求积 $ Z = XY $ }

$ Z = XY $ 的概率密度为$$
    f_z(z) = \int_{-\infty}^{+\infty}\dfrac{1}{|x|}f(x,\frac{z}{x})\mathrm{d}x
$$ 

\sssubsection{求商 $ Z = \dfrac{X}{Y} $ }

$ Z = \dfrac{X}{Y} $ 的概率密度为$$
    f_z(z) = \int_{-\infty}^{+\infty}|y|f(yz,y)\mathrm{d}y
$$ 

\sssubsection{公式法的要点}

以求和为例,对$ \dis f(z) =\int_{-\infty}^{+\infty} f(x,z-x)\mathrm{d}x, z $
变动意味着非零区间变动,\newline 此时非零区间为 $ (x,z-x)\in D $.

\subsection{一离散型一连续型随机变量函数的分布}

结合全概率公式对离散型变量进行全集分解,也即分类讨论。

\subsection{最值函数}

对极大值 $ U = \max(X,Y) $ ,当 $ X,Y $ 独立同分布时,
$ F_U(u) = [F_x(u)]^2 $ ,因此密度为 $ f_U(u) F'_U(u) = 2F_X(u)f_X(u) $ .

对极小值 $ U = \min(X,Y) $ ,当 $ X,Y $ 独立同分布时,
$ F_U(u) = 1-[1-F_x(u)]^2 $ ,因此密度为 $ f_U(u) F'_U(u) = 2[1-F_X(u)]f_X(u) $ .

事实上,这是顺序统计量:对于一组 $ n $ 个独立同分布的随机变量,这组中第$ k $ 大的密度函数为
$$
    f_k(x) = n!\dfrac{[F(x)]^{n-1}}{(n-1)!}\dfrac{[1-F(x)]^{n-k}}{(n-k)!}f(x)
$$ 
据此可求分布函数。

注意,若 $ X\sim E(\lambda_1),Y\sim E(\lambda_2)$ 且二随机变量独立,
又有 $ Z = \min(X,Y) $ ,则 $ Z \sim E(\lambda_1+\lambda_2) $.

对二维随机变量 $ (X,Y) $ ,令 $ U = \max(X,Y),V = \min(X,y) $ ,则有
\begin{itemize}
    \item $ U + V = X + Y $ ;
    \item $ U - V = |X - Y| $ ;
    \item $ UV = XY $ .
\end{itemize}

