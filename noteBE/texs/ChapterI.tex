\chapter{第一段}

第一段写到40个词语就足够了。

\section{单幅漫画}

A/An 

\begin{itemize}
    \item \colorbox{lime!40!white}{encouraging / inspiring / elevating / uplifting}
    \item \colorbox{magenta!40!white}{discouraging / distrubing / frustrating / annoying}
    \item \colorbox{cyan!40!white}{thought-provoking / thought stimulating}
\end{itemize}

\noindent story [surfaces/emerges] from the image(presented above):
2-3句.

\noindent Apparently, the picture does [reflect/mirror] a common phenomenon in our contemporary society.

\section{对比漫画}

可以将单幅漫画的开头改造成为复数形式。

The images pose a [sharp/striking/stark] contrast:

\noindent in the first picture, \textrm{(I)}.
[Conversely/Oppositely], in the second one, \textrm{(II)}.

\noindent Apparently, the drawing do reflect distinct attitudes and approaches 

\noindent (adopted by different people in our contemporary society).

\section{例子 - 单幅漫画}

\subsection{2010第一段}

there is a huge cultural hotpot which involves numerous 
Chinese cultural elements, such as Beijing Opera, Kung fu and Taoism, 
as well as foreign counterparts.

\subsection{2007第一段}

there is a fierce football match which involves a goalkeeper and 
a shooter. Both of them regard their own task as mission impossible.

\subsection{2023第一段}

there is a fierce dragon boat race, and a great number of 
audiences on the bridge are apperciating the race. Two 
elderly folks are delighted because the race is better and 
better in their village.

\subsection{2009第一段}

there is a huge Net which involves numerous chambers 
in which individuals are staring at their own computer screens without 
face-to-face communication.

\section{例子 - 对比漫画}

\subsection{2016第一段}

\begin{enumerate}
    \item a father is watching television programs and requiring his son to work hard.
    \item both a father and his son are concentrating on their work together.
\end{enumerate}

\subsection{2020第一段}

\begin{enumerate}
    \item a girl is concentrating on her homework/assignments, 
    intending to finish it/them as soon as possible.
    \item a boy does not touch on his tasks, deciding to finish them before the deadline.
\end{enumerate}

\chapter{第二段}

\section{模板}

\sssubsection{多方向 - 不常用}

As a matter of fact, the picture
attempts to [convey/deliver] a crucial message:
\textrm{(I)} [counts/matters] in \textrm{(I)} .

\noindent To put it more explicitly, its importance can be illustrated
from following aspects. Firstly, \dots
In addition, \dots Finally, \dots

\sssubsection{单方向}

As a matter of fact, the picture
attempts to [convey/deliver] a crucial message:
\textrm{(I)} [counts/matters] in \textrm{(I)} .
To put it more explicitly, it is undeniable that 
\textrm{(II)} (为什么需要XXX/XXX的重要性).
In such cases, \textrm{(III)} ought to \textrm{(III)}
(in order to \dots).
By contrast, if \textrm{(IV)} lack such awareness, \textrm{(IV)}
will be more likely to \textrm{(IV)} in the end.
A host of convincing examples can be found to illustrate my viewpoint.
(写完这句也可以不举例子)

\section{例子}

\subsection{2007第二段}

\begin{enumerate}
    \item self-confidence; our success.
    \item we may [encounter/be confronted with/be trapped into]
    some unexpected [challenges/difficulties/setbacks/adversities/failures]
    during the course of life.
    \item we ought to [remain self-confident/focus on our own strength]
    in order to [conquer/weather] tough moments.
    \item if we lack such awareness, we will be more likely 
    to [suffer from failure/taste the bitterness of failure in the end].
    \item[例.] During the process of a football game, team members must believe in 
    themselves to defeat mighty rivals.
\end{enumerate}

\subsection{2008第二段}

\begin{enumerate}
    \item cooperation; our success.
    \item 参考2007.
    \item we ought to cooperate with others in order to weather tough moments.
    \item 参考2007.
    \item[例.] During the process of a football game, team members must cooperate/collaborate with 
    each other to defeat mighty rivals.
\end{enumerate}

\subsection{2012第二段}

\begin{enumerate}
    \item positive attitude.
    \item 参考2007.
    \item \dots [remain optimistic/focus on silver linings] \dots
    \item 参考2007.
\end{enumerate}

\subsection{2021第二段}

\begin{enumerate}
    \item parents' encouragement; in kids' daily life.
    \item children may be trapped into some unexpected setbacks
    during the course of life.

    children's hobbies may be denied by peers.
    \item parents ought to [encourage/motivate/inspire] their kids to stick to 
    their own pursuits.
    \item if parents lack such awareness, children will be more 
    likely to lose themselves in the end.
\end{enumerate}

\subsection{2013第二段}

\begin{enumerate}
    \item making a suitable choice; our daily life.
    \item we may be confronted with numerous options during the 
    course of life. 
    
    graduates may be confronted with numerous options when they leave the campus.
    \item we/graduates ought to make a [sensible/wise/proper]
    choice according to our/their own situations.
    \item if we/graduates lack such awareness, we/they will be more likely to lose
    ourselves/themselves in the end.
    \item[例.] graduates with strong academic interest should be encouraged to pursue 
    further study.
\end{enumerate}

\subsection{2016第二段}

\begin{enumerate}
    \item parents' role model; kids' daily life.
    \item children will [copy/imitate] parents' behaviors, whether 
    consciously or unconsciously.

    parents' behaviors will have [dramatic/profound/subtle]
    effects on children's growth.
    \item parents ought to set a positive example for kids.
    \item if parents lack such awareness, children will be more likely to 
    form undesirable habits in the end.
\end{enumerate}

\subsection{2018/2022第二段}

\begin{enumerate}
    \item acquiring updated knowledge; our campus life.
    \item new knowledge enables us to boost our competitive edge and 
    expand our vision.
    \item we ought to seize every chance to obtain more knowledge.
    \item if we lack such awareness, we will be more likely to harvest nothing in the end.
\end{enumerate}

\subsection{2023第二段}

\begin{enumerate}
    \item tradition culture.
    \item traditional festivals document lifestyles, values, and wisdom
    of our ancestors, serving as a bridge between the past and the present. 
    \item we ought to cherish and celebrate traditional festivals.
    \item if we \dots, traditional festivals will be more likely to be 
    forgotten by the generations of tomorrow in the end.

    Ritual n. 仪式
\end{enumerate}

\subsection{影响 - 2015第二段}

As a matter of fact, the picture
attempts to convey a crucial message:
smartphones do have dramatic effects on our 
interpersonal relationship. 
To put it more explicitly, it is undeniable that 
although smartphones bring much convenience for our life,
we can not turn blind eyes to their negative influences.
In other words, even though we get together with friends or 
relatives, we tend to [be addicted to/be sbsorbed into]
our own devices, without any face-to-face communication. 
As a consequence, our interpersonal relationship will be 
more likely to be weakened in the end.

\chapter{第三段}

\section{模板}

On the whole, according to the analysis mentioned above, we could draw a reasonable 
conclusion that \textrm{(I)} are advised to cultivate/foster a sense of \textrm{(I)}
and put it into practice. Only in this manner can \textrm{(II)}.

\section{例子}

\begin{itemize}
    \item \begin{enumerate}
        \item we; self-confidence;
        \item we [succeed/achieve success/reap the joy of success/
        embrace a fulfilling and promising prospect.]
    \end{enumerate}
    \item \begin{enumerate}
        \item parents; role model;
        \item children develop desirable habits.
    \end{enumerate}
    \item \begin{enumerate}
        \item we; protecting and promoting traditional culture;
        \item we witness cultural diversity and prosperity.
    \end{enumerate}
\end{itemize}

