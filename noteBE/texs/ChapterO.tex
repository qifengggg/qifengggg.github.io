\chapter*{大作文的格式}

\section{行文逻辑}

\begin{itemize}
    \item \textbf{第一段}
    
    描述漫画内容。

    \begin{enumerate}[label = \Alph*.]
        \item 极简主义 - 抓住与漫画主题相关的细节;
        \item 时态 - 现在进行时、一般现在时;
        \item 汉语提示 - 不要直译
    \end{enumerate}
    \item \textbf{第二段}
    
    \textbf{论证某种品质的重要性;}

    阐述现象重要性;
    \item \textbf{第三段}
    
    提出建议,说明怎么做。
\end{itemize}

\newpage

\section{模板}

\subsection{单幅漫画}

A/An encouraging and thought-provoking story 
emerges from the image presented above: \textrm{(I)}
Apparently, the picture does mirror a common phenomenon in our contemporary society.

As a matter of fact, the picture
attempts to convey a crucial message:
\textrm{(I)} does count in \textrm{(I)} .
To put it more explicitly, it is undeniable that 
\textrm{(II)} (为什么需要XXX/XXX的重要性).
In such cases, \textrm{(III)} ought to \textrm{(III)}
(in order to \dots).
By contrast, should \textrm{(IV)} lack such awareness, \textrm{(IV)}
would be more likely to \textrm{(IV)} in the end.
A host of convincing examples can be found to illustrate my viewpoint.

On the whole, according to the analysis mentioned above, we could draw a reasonable 
conclusion that \textrm{(I)} are advised to cultivate/foster a sense of \textrm{(I)}
and put it into practice. Only in this manner can \textrm{(II)}.

%\newpage

\subsection{对比漫画}

The images pose a stark contrast:
in the first picture, \textrm{(I)}.
Conversely, in the second one, \textrm{(II)}.
Apparently, the drawing do reflect distinct attitudes and approaches 
adopted by different people in our contemporary society.

As a matter of fact, the pictures
attempt to convey a crucial message:
\textrm{(I)} does count in \textrm{(I)} .
To put it more explicitly, it is undeniable that 
\textrm{(II)} (为什么需要XXX/XXX的重要性).
In such cases, \textrm{(III)} ought to \textrm{(III)}
(in order to \dots).
By contrast, should \textrm{(IV)} lack such awareness, \textrm{(IV)}
would be more likely to \textrm{(IV)} in the end.
A host of convincing examples can be found to illustrate my viewpoint.

On the whole, according to the analysis mentioned above, we could draw a reasonable 
conclusion that \textrm{(I)} are advised to cultivate/foster a sense of \textrm{(I)}
and put it into practice. Only in this manner can \textrm{(II)}.