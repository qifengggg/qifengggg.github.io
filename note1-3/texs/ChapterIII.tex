\chapter{一元函数积分学}

\section{定积分的计算}

计算定积分时,考虑

$$
\begin{cases}
    \textrm{不定积分}\begin{cases}
        \textrm{凑微分}\\ 
        \textrm{分部}\\ 
        \textrm{换元}\\     
    \end{cases}\\
    \textrm{定积分}\begin{cases}
        \textrm{奇偶性}\\
        \textrm{周期性}\\ 
        \textrm{Wallis公式} \\    
    \end{cases}    
\end{cases}
\textrm{其中,换元包括}
\begin{cases}
    \textrm{不定积分}
    \begin{cases}
        \textrm{三角代换}\\\textrm{根式代换}\\\textrm{倒代换}\\\textrm{万能代换}\\\textrm{整体代换}    
    \end{cases}\\\textrm{定积分}    
    \begin{cases}
        \textrm{区间再现公式}\\\textrm{平移变换}
    \end{cases}
\end{cases}
$$

\sssubsection{反函数的积分 T3.15}

$$
    \int_a^b g(y)\mathrm{d}y \xlongequal{y = f(x)}
    \int_\alpha^\beta \overbrace{g(f(x))}^{x}f'(x)\mathrm{d}x
$$

事实上,设 $ y = f(x) $ 在 $ [a,b] $ 上单调连续,反函数为 $ x = f^{-1}(y) = g(y), $ 
且 $ f(a) = c, f(b) = d, $ 
则有
$$
    \int_a^bf(x)\mathrm{d}x + \int_c^d g(y)\mathrm{d}y = bd - ac.
$$

反函数的二重积分,本质上是变限积分函数的积分。

\sssubsection{平移变换结合奇偶性 T3.16}

令 $ x = \dfrac{a+b}{2} + t, $ 使得区间变为对称区间,以便使用奇偶性。

\sssubsection{区间再现公式}

\Attention{山穷水尽}时,使用区间再现公式。

令 $ x = a + b - t, $ 则

$$
    \int_a^bf(x)\mathrm{d}x = \int_a^bf(a+b-x)\mathrm{d}x = 
    \dfrac{1}{2}\int_a^b\textcolor{red}{[f(x)+f(a+b-x)]}\mathrm{d}x
$$

\begin{enumerate}
    \item[\textbf{例题}] 求 $ \dis \dis \int_0^\frac{\pi}{2} \ln \sin x \mathrm{d}x = 
    \dis \int_0^\frac{\pi}{2} \ln \cos x \mathrm{d}x. $ 
    \item[\textbf{方法}] 可以知道 \begin{equation*}
        \begin{aligned}
            \int_0^\frac{\pi}{2}\ln\sin x \mathrm{d}x &= 
            \dfrac{1}{2}\int_0^\frac{\pi}{2}[\ln \cos x + \ln\sin x] \mathrm{d}x 
            = \dfrac{1}{2}\int_0^\frac{\pi}{2}\ln \left(\sin x\cos x\right) \mathrm{d}x \\ 
            &= \dfrac{1}{2}\int_0^\frac{\pi}{2}\ln \left(\dfrac{1}{2}\sin 2x\right)\mathrm{d}x 
            = \dfrac{1}{2}\int_0^\frac{\pi}{2}\ln \sin 2x\mathrm{d}x - \dfrac{\pi}{4}\ln 2 \\ 
            (\textrm{令}t = 2x)&= \dfrac{1}{4}\int_0^\pi\ln \sin t\mathrm{d}t - \dfrac{\pi}{4}\ln 2 
            = \dfrac{1}{2}\int_0^\frac{\pi}{2}\ln \sin t\mathrm{d}t - \dfrac{\pi}{4}\ln 2 \\ 
        \end{aligned}
    \end{equation*}
    故 $ \dis \int_0^\frac{\pi}{2}\ln\sin x \mathrm{d}x = -\dfrac{\pi}{2}\ln 2. $ 
\end{enumerate}

\begin{enumerate}
    \item[\textbf{例题}] 求 $ \dis \int_0^1 \dfrac{\ln (1+x)}{1+x^2}\mathrm{d}x. $ 
    \item[\textbf{方法}] 可以知道
    \begin{equation*}
        \begin{aligned}
            \int_0^1 \dfrac{\ln (1+x)}{1+x^2}\mathrm{d}x &= 
            \int_0^\frac{\pi}{4} \dfrac{\ln (1+\tan x)}{\sec^2 x}\mathrm{d}\tan x 
            = \int_0^\frac{\pi}{4} \ln (1+\tan x)\mathrm{d}x \\ 
            &= \dfrac{1}{2}\int_0^\frac{\pi}{4} \left[\ln (1+\tan x) + \ln(1+\tan(\dfrac{\pi}{4}-x))\right]\mathrm{d}x 
            \\ &= \int_0^\frac{\pi}{4} \left[\ln (1+\tan x) + 
            \ln\left(1+\dfrac{1-\tan x}{1+\tan x}\right)\right]\mathrm{d}x \\ 
            &= \dfrac{1}{2}\int_0^\frac{\pi}{4} \left[\ln (1+\tan x) + 
            \ln\left(\dfrac{2}{1+\tan x}\right)\right]\mathrm{d}x \\ 
            &= \dfrac{1}{2}\int_0^\frac{\pi}{4}\ln 2\mathrm{d}x = \dfrac{1}{2}\cdot \dfrac{\pi}{4}\ln 2
            = \dfrac{\pi}{8}\ln 2.
        \end{aligned}
    \end{equation*}
    此外,$$
    \int_0^\frac{\pi}{4} \ln (1+\tan x)\mathrm{d}x = 
    \int_0^\frac{\pi}{4} \ln \dfrac{\sin x+\cos x}{\cos x}\mathrm{d}x
    = \int_0^\frac{\pi}{4} \ln \sqrt{2}+\ln \left[\sin (x+\dfrac{\pi}{4})\right]
    -\ln \cos x\mathrm{d}x 
    $$
    而后二项在积分完毕后正好相等。
\end{enumerate}

\sssubsection{Wallis公式}

$ \dis I_n = \int_0^\frac{\pi}{2}\sin^n x \mathrm{d}x = \int_0^\frac{\pi}{2}\cos^n x \mathrm{d}x. $ 

证明:递推公式 $ I_n = \dfrac{n-1}{n}I_{n-2}, $ 递推至 $ n = 1 $ 或 $ n = 0 $ 即可。

$$
    {\displaystyle\lim_{n\rightarrow \infty}}\dfrac{1}{2n+1}\left[\dfrac{(2n)!!}{(2n-1)!!} \right]^2
    = \dfrac{\pi}{2}.
$$

证明:由 $ {\displaystyle\lim_{n\rightarrow \infty}}\dfrac{I_{2n+1}}{I_{2n}} = 1 $ 
推出,右边由夹逼准则得到。

\section{变限积分函数的计算}

考虑 $ \begin{cases}
    \textrm{分部积分法}\\\textrm{交换积分次序}
\end{cases} $ 

\sssubsection{经典错误:变限积分求导结合微分方程 - T3.22}

\begin{enumerate}
    \item[\textbf{例题}] 设 $ f(x) $ 在 $ (-1,+\infty) $ 内连续,
    且有 $ \dis f(x)\left[\int_0^xf(t)\mathrm{d}t+1\right]=\dfrac{xe^x}{2(x+1)^2}, $
    求 $ f(x). $ 
    \item[\textbf{方法}] 此题不能求导,因为求导无法消出有效的微分方程。
    
    令 $ \dis F(x) = \int_0^xf(t)\mathrm{d}t + 1, $ 
    则有 $ \dis 2F'(x)F(x) = \left(F(x)^2\right)' = \dfrac{xe^x}{(1+x)^2}, $ 
    积分计算 $ F^2(x)\Rightarrow F(x)\Rightarrow f(x). $ 
\end{enumerate}

对于 $ \dis f(x)\int_0^x f(t)\mathrm{d}t, $ 令 $\dis F(x) = \int_0^xf(t)\mathrm{d}t, $ 
则有 $ \dis f(x)\int_0^x f(t)\mathrm{d}t = \dfrac{1}{2}[F(x)^2]'. $ 

\begin{enumerate}
    \item[\textbf{例题}] 设 $ f(x) $ 在 $ [0,1] $ 连续,且 $ \dis \int_0^1 f(x)=A, $ 
    求 $ \dis \int_0^1 \mathrm{d}x\int_x^1 f(x)f(y)\mathrm{d}y. $ 
    \item[\textbf{方法}] 可以知道
    \begin{equation*}
        \begin{aligned}
            \int_0^1 \mathrm{d}x\int_x^1 f(x)f(y)\mathrm{d}y &= 
            \int_0^1 f(x)\int_x^1f(y)\mathrm{d}y\mathrm{d}x \\ 
            (\textrm{令}F(x) = \int_x^1f(y)\mathrm{d}y)&= 
            -\int_0^1 \dfrac{1}{2}\left[F(x)^2\right]'\mathrm{d}x \\ 
            &= -\dfrac{1}{2}\left[F(x)^2\right]\Big|_0^1 = \dfrac{A^2}{2}.
        \end{aligned}
    \end{equation*}
    \item[\textbf{方法}] 可以知道
    $$
        \iint\limits_{D_1} = \iint\limits_{D_2} = \dfrac{1}{2}\iint\limits_{D}
        = \dfrac{1}{2}\int_0^1f(x)\mathrm{d}x\int_0^1f(y)\mathrm{d}y = \dfrac{A^2}{2}.
    $$
    此处利用轮换对称性的推广,即对 $ D:(x,y) $ 与 $ D':(y,x), $ 
    $$
        \iint\limits_{D}f(x,y)\mathrm{d}x\mathrm{d}y = 
        \iint\limits_{D'}f(y,x)\mathrm{d}y\mathrm{d}x
    $$
\end{enumerate}

\section{反常积分的计算}

\sssubsection{迪利克雷积分}

\begin{equation*}
    \begin{aligned}
        \int_0^{+\infty}\dfrac{\sin x}{x}\mathrm{d}x &= 
        \int_0^{+\infty}\mathrm{d}x\int_0^{+\infty}e^{-xy}\sin x \mathrm{d}y \\ 
        &= \int_0^{+\infty}\mathrm{d}y\int_0^{+\infty}e^{-xy}\sin x \mathrm{d}x \\ 
        &= -\int_0^{+\infty}\dfrac{e^{-xy}(y\sin x +\cos x)}{1+y^2}\Big|_0^{+\infty}
        \mathrm{d}y\\ &= \int_0^{+\infty}\dfrac{\mathrm{d}y}{1+y^2} = \arctan y\Big|_0^{+\infty}
        = \dfrac{\pi}{2}.
    \end{aligned}
\end{equation*}

\begin{equation*}
    \begin{aligned}
        \int_0^{+\infty}\dfrac{\sin^2 x}{x^2}\mathrm{d}x &= 
        -\int_0^{+\infty}\sin^2 x\mathrm{d}\dfrac{1}{x} =
        -\dfrac{\sin^2 x}{x}\Big|_0^{+\infty} + \int_0^{+\infty}
        \dfrac{2\sin x\cos x}{x}\mathrm{d}x
        \\&= \int_0^{+\infty}\dfrac{\sin 2x}{2x}\mathrm{d}2x
        = \dfrac{\pi}{2}.
    \end{aligned}
\end{equation*}

\section{定积分的几何应用}

$$ \begin{cases}
    \textrm{面积}\begin{cases}
        \textrm{直角坐标}\\\textrm{极坐标}\\\textrm{参数方程}
    \end{cases}\\
    \textrm{体积}\begin{cases}
        \textrm{x轴}\\\textrm{y轴}\\\textrm{平移}
    \end{cases}\\
    \textrm{弧长}\begin{cases}
        \textrm{直角坐标}\\\textrm{极坐标}\\\textrm{参数方程}
    \end{cases}\\
    \textrm{侧面积}
\end{cases} $$ 

\sssubsection{区域绕直线旋转的体积}

区域 $ D $ 绕直线 $ L:Ax+By+C=0 $ 旋转一周所得旋转体的体积为
$$
    V = 2\pi \iint\limits_{D}r(x,y)\mathrm{d}\sigma
$$

其中 $ r(x,y) $ 为 $ D $ 中点到 $ L $ 距离 $ \dfrac{|Ax+By+C|}{\dsqrt{A^2+B^2}}. $ 
要求 $ D $ 与 $ L $ 无公共点。

特别地,若 $ L: y-C = 0, $ 则有
$$
    V = 2\pi\iint\limits_D (y-c)\mathrm{d}\sigma = |2\pi(\bar y - C)S_D|
$$

其中 $ \bar y $ 是 $ D $ 的形心,$ S_D $ 是其面积。
注意,体积是非负的。

\section{定积分物理应用}

微元法考虑
$ 
\begin{cases}
    \textrm{功}\quad{} W = Fs \\ 
    \textrm{力}\begin{cases}
        \textrm{万有引力} F = \dfrac{GMm}{r^2}\\ 
        \textrm{液体压力} F = PS = \overbrace{\rho gh}^{\textrm{液体压力}} S 
    \end{cases}
\end{cases} 
$ 

\section{切比雪夫不等式和柯西不等式}

\sssubsection{切比雪夫不等式}

设 $ f(x),g(x),p(x) $ 在 $ [a,b] $ 上连续,且 $ p(x) $ 不变号,
\begin{itemize}
    \item 若 $ f(x),g(x) $ 单调性相同,则有
    $$
        \int_a^bf(x)p(x)\mathrm{d}x\int_a^bg(x)p(x)\mathrm{d}x\leq
        \int_a^bp(x)\mathrm{d}x\int_a^bf(x)g(x)p(x)\mathrm{d}x
    $$
    \item 若 $ f(x),g(x) $ 单调性不同,则有
    $$
        \int_a^bf(x)p(x)\mathrm{d}x\int_a^bg(x)p(x)\mathrm{d}x\geq
        \int_a^bp(x)\mathrm{d}x\int_a^bf(x)g(x)p(x)\mathrm{d}x
    $$
\end{itemize}

\sssubsection{柯西不等式}

设 $ f(x),g(x) $ 在 $ [a,b] $ 连续,则有$$
    \left[\int_a^bf(x)g(x)\mathrm{d}x\right]^2\leq
    \int_a^bf^2(x)\mathrm{d}x\int_a^bg^2(x)\mathrm{d}x
$$