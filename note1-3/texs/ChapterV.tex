\chapter{二重积分}

\section{交换积分次序}

$ \begin{cases}
    \textrm{直角 - 穿针法}\\
    \textrm{极坐标}\begin{cases}
        xy\textrm{坐标系 - 作圆弧}\\ 
        \theta r\textrm{坐标系 - 穿针法(当成直角坐标)}
    \end{cases}
\end{cases} $ 

二重变限积分求极限时,
$ \begin{cases}
    \textrm{洛必达结合交换积分次序}\\\textrm{二重积分中值定理(非零因子)}
\end{cases} $ 

\sssubsection{例题T6.2.2}

\begin{equation*}
    \begin{aligned}
        {\displaystyle\lim_{t\rightarrow 0^+}}\dfrac{1}{t^3}\int_0^t\mathrm{d}x
        \int_{x^2}^{t^2}\arctan(\cos(2x+3\dsqrt{y}))\mathrm{d}y &= 
        {\displaystyle\lim_{t\rightarrow 0^+}}\dfrac{1}{t^3}\iint\limits_{D}
        \arctan(\cos(2x+3\dsqrt{y}))\mathrm{d}x\mathrm{d}y
        \\&= {\displaystyle\lim_{t\rightarrow 0^+}}\dfrac{1}{t^3}
        \arctan(\cos(2\xi+3\dsqrt{\eta}))\cdot S_D \,(\xi,\eta \in D)
        \\&= {\displaystyle\lim_{t\rightarrow 0^+}}\dfrac{1}{t^3}
        \arctan(\cos(2\xi+3\dsqrt{\eta}))\cdot\dfrac{2}{3}t^3
        \\&= \dfrac{\pi}{4}\cdot\dfrac{2}{3} = \dfrac{\pi}{6}.
    \end{aligned}
\end{equation*}

\sssubsection{例题T6.3.2}

\begin{equation*}
    \begin{aligned}
        \int_0^\frac{\pi}{2}\dfrac{1}{\sqrt{x}}\mathrm{d}x\int_{\sqrt{x}}^{\sqrt{\frac{\pi}{2}}}
        \dfrac{1}{1+(\tan y^2)^{\sqrt 2}}\mathrm{d}y 
        &= \int_0^{\sqrt{\frac{\pi}{2}}}\dfrac{1}{1+(\tan y^2)^{\sqrt 2}}
        \mathrm{d}y\int_0^{y^2}\dfrac{1}{\sqrt x}\mathrm{d}x
        \\&= \int_0^{\sqrt{\frac{\pi}{2}}}\dfrac{2y}{1+(\tan y^2)^{\sqrt 2}}\mathrm{d}y
        \\&= \int_0^{\frac{\pi}{2}}\dfrac{\mathrm{d}t}{1+(\tan t)^{\sqrt 2}}
        \\&= \int_0^{\frac{\pi}{2}}\dfrac{(\cos t)^{\sqrt 2}\mathrm{d}t}{(\sin t)^{\sqrt 2}+(\cos t)^{\sqrt 2}} 
        \\\textrm{区间再现}&=\dfrac{1}{2}\cdot\dfrac{\pi}{2} = \dfrac{\pi}{4}.
    \end{aligned}
\end{equation*}

\sssubsection{例题T6.3.4}

\begin{equation*}
    \begin{aligned}
        \int_0^1\dfrac{x^b-x^a}{\ln x}\mathrm{d}x &= 
        \int_0^1 \mathrm{d}x\int_a^b x^y\mathrm{d}y \\ 
        &= \int_a^b \mathrm{d}y \int_0^1 x^y\mathrm{d}x
        \\&= \int_a^b \dfrac{\mathrm{d}y}{1+y} = \ln\dfrac{1+b}{1+a}.
    \end{aligned}
\end{equation*}

\sssubsection{例题T6.3.5}

此处以 $ \theta - r $ 坐标系交换积分次序。

\begin{equation*}
    \begin{aligned}
        \int_0^{2\pi}(\theta^2-1)\mathrm{d}\theta\int_{\frac{\theta}{2}}^\pi e^{r^2}\mathrm{d}r &=
        \int_0^\pi e^{r^2}\mathrm{d}r\int_0^{2r}(\theta^2-1)\mathrm{d}\theta \\ 
        &= \dfrac{8}{3}\int_0^\pi r^3e^{r^2}\mathrm{d}r - 2\int_0^\pi re^{r^2}\mathrm{d}r
        \\ &= \dfrac{1}{3}(4\pi^2-7)e^{\pi^2}+\dfrac{7}{3}.
    \end{aligned}
\end{equation*}

\section{二重积分的计算}

考虑
$ \begin{cases}
    \textrm{直角坐标系}\\
    \textrm{极坐标变换}\\
    \textrm{奇偶性}\\
    \textrm{轮换对称性}\\
    \textrm{形心公式}\\
    \textrm{平移变换}\\
    \textrm{雅可比行列式}
\end{cases} $ 

\sssubsection{被积函数为齐次函数 $ f(\dfrac{y}{x}) $ T6.6}

\begin{enumerate}
    \item[\textbf{例题}] 设有区域 $ D = \{(x,y)|x|+|y|\leq 1\}, $ 
    计算 $ \dis \iint\limits_D e^{\dfrac{|y|}{|x|+|y|}}\mathrm{d}x\mathrm{d}y. $ 
    \item[\textrm{\textbf{方法}}] 可以知道,
    \begin{equation*}
        \begin{aligned}
            \textrm{原积分} &= 4\iint\limits_{D_1}e^{\dfrac{y}{x+y}}\mathrm{d}x\mathrm{d}y \\ 
            &= 4\int_0^{\frac{\pi}{2}}\mathrm{d}\theta\int_0^{\frac{1}{\sin \theta + \cos \theta}}
            e^{\frac{\sin \theta}{\sin \theta + \cos \theta}}r\mathrm{d}r
            \\&= 2\int_0^{\frac{\pi}{2}} e^{\frac{\sin \theta}{\sin \theta + \cos \theta}} 
            \dfrac{1}{(\cos \theta + \sin \theta)^2}\mathrm{d}\theta
            \\&= 2\int_0^{\frac{\pi}{2}} e^{\frac{\sin \theta}{\sin \theta + \cos \theta}} 
            \mathrm{d}\dfrac{\sin \theta}{\sin \theta + \cos \theta}
            \\&= 2e^{\frac{\sin \theta}{\sin \theta + \cos \theta}}
            \Big|_0^\frac{\pi}{2} = 2(e-1).
        \end{aligned}
    \end{equation*}
\end{enumerate}

事实上,对于包含齐次函数 $ f(\dfrac{y}{x}) $ 的二重积分,都应该考虑利用极坐标变出 
$ \mathrm{d}\dfrac{A\sin\theta+B\cos\theta}{\sin\theta+\cos\theta}. $

对于 $ f(\dfrac{y^2}{x^2}),f(\dfrac{ax+by}{x+y}) $ 也是类似。

\sssubsection{极坐标压轴题 - T6.7}

\begin{enumerate}
    \item[\textbf{例题}] 设 $ f(x,y) $ 在区域 $ D = \{(x,y)|x^2+y^2\leq 1\} $ 
    上有一阶连续偏导数,在 $ D $ 的边界上 $ f(x,y) $ 恒为零,求
    $$
        {\displaystyle\lim_{\varepsilon\rightarrow 0^+}}\iint\limits_{\varepsilon^2\leq x^2+y^2\leq 1}
        \dfrac{xf'_x+yf'_y}{x^2+y^2}\mathrm{d}x\mathrm{d}y.
    $$
    \item[\textbf{方法}] 引入极坐标,则有
    \begin{equation*}
        \begin{aligned}
            &\dfrac{\partial f}{\partial \rho} = 
            \dfrac{\partial f}{\partial x}\dfrac{\partial x}{\partial \rho}
            + \dfrac{\partial f}{\partial y}\dfrac{\partial y}{\partial \rho}
            = f'_x\cos\theta + f'_y\sin\theta \\ 
            &\dfrac{\partial f}{\partial \theta} = 
            \dfrac{\partial f}{\partial x}\dfrac{\partial x}{\partial \theta}
            + \dfrac{\partial f}{\partial y}\dfrac{\partial y}{\partial \theta}
            = -f'_x(\rho \sin\theta)+f'_y(\rho \cos\theta)\\ 
            \Rightarrow & \rho \dfrac{\partial f}{\partial \rho} = 
            \rho\left[f'_x\cos\theta + f'_y\sin\theta\right]
            =xf'_x+yf'_y.
        \end{aligned}
    \end{equation*}
    那么,对于题设,有
    \begin{equation*}
        \begin{aligned}
            \textrm{原极限}&= {\displaystyle\lim_{\varepsilon\rightarrow 0^+}}
            \int_0^{2\pi}\mathrm{d}\theta\int_{\varepsilon}^1
            \dfrac{\partial f}{\partial \rho}\mathrm{d}\rho \\ 
            &= {\displaystyle\lim_{\varepsilon\rightarrow 0^+}}
            \int_0^{2\pi}f(\rho\cos\theta,\rho\sin\theta)\Big|_\varepsilon^1\mathrm{d}\theta\\
            &= -{\displaystyle\lim_{\varepsilon\rightarrow 0^+}}
            \int_0^{2\pi}f(\varepsilon\cos\theta,\varepsilon\sin\theta)\mathrm{d}\theta \\ 
            &= -\int_0^{2\pi}f(0,0)\mathrm{d}\theta = -2\pi f(0,0).
        \end{aligned}
    \end{equation*}
\end{enumerate}

\sssubsection{极坐标逆问题 - T6.8}

在极坐标不好做的情况下,将 $ (\rho,\theta) $ 转换回 $ (x,y) $ 即可。

\newpage

\sssubsection{奇偶性 - T6.9-10}

\begin{itemize}
    \item \textbf{奇偶性的推广}
    
    设 $ f(x,y) $ 在区域 $ D = D_1\bigcup D_2 $ 上连续,其中 $ D_1,D_2 $ 关于原点对称,则
    $$
        \iint\limits_Df(x,y)\mathrm{d}x\mathrm{d}y
        =\begin{cases}
            2\dis\iint\limits_{D_1}f(x,y)\mathrm{d}x\mathrm{d}y,\quad{} f(-x,-y) = f(x,y)\\
            0,\quad{} f(-x,-y) = -f(x,y)\\
        \end{cases}
    $$
\end{itemize}

\begin{enumerate}
    \item[\textbf{例题}]
    设区域 $ D $ 由曲线 $ y = x^3,y = x $ 围成。求
    $$
        \iint\limits_D\left[e^{x^2}+\sin(x+y)\right]\mathrm{d}x\mathrm{d}y.
    $$
    \item[\textbf{方法}] 
    由奇偶性,可以知道
    \begin{equation*}
        \begin{aligned}
            \textrm{原积分}&=2\int_0^1\mathrm{d}x\int_{x^3}^x e^{x^2}\mathrm{d}y\\ 
            &= 2\int_0^1e^{x^2}(x-x^3)\mathrm{d}x 
            \\&= \int_0^1(1-x^2)e^{x^2}\mathrm{d}x^2 = e-2.
        \end{aligned}
    \end{equation*}
\end{enumerate}

\begin{enumerate}
    \item[\textbf{例题}]
    对 $ D = \{(x,y)|x^2+y^2\leq 1, x+y\geq 0\}, $ 求
    $$
        \iint\limits_D\dfrac{1+2x^2+xy^2}{1+x^2+y^2}\mathrm{d}x\mathrm{d}y.
    $$
    \item[\textbf{方法}] 
    原积分等于 $ \dis \iint\limits_D\dfrac{1+2x^2}{1+x^2+y^2}\mathrm{d}x\mathrm{d}y+
    \iint\limits_D\dfrac{xy^2}{1+x^2+y^2}\mathrm{d}x\mathrm{d}y. $ 
    前一部分利用轮换对称性。对后一部分,以 $ y = x $ 将 $ D $ 
    划分为 $ D_1,D_2, $ 以 $ y = 0 $ 将 $ D_1 $ 划分为 $ D_{11},D_{12}. $ 
    那么,
    \begin{equation*}
        \begin{aligned}
            \textrm{后半部分积分} &= \iint\limits_{D_1}+\iint\limits_{D_2}
            \dfrac{xy^2}{1+x^2+y^2}\mathrm{d}x\mathrm{d}y\\ 
            &= 2\iint\limits_{D_12}\dfrac{xy^2}{1+x^2+y^2}\mathrm{d}x\mathrm{d}y\\ 
            &= 2\int_{0}^\frac{\pi}{4}\mathrm{d}\theta\int_0^1
            \dfrac{r^4\sin^2\theta\cos\theta}{1+r^2}\mathrm{d}r = \dfrac{1}{3\sqrt 2}
            \left(\dfrac{\pi}{4}-\dfrac{2}{3}\right).
        \end{aligned}
    \end{equation*}
\end{enumerate}

\sssubsection{轮换对称性的推广 - T6.11}

\begin{itemize}
    \item \textbf{轮换对称性的推广}
    
    $$
        \iint\limits_Df(x,y)\mathrm{d}\sigma = 
        \iint\limits_{D'}f(y,x)\mathrm{d}\sigma
    $$
    其中 $ D' = \{(y,x)|(x,y)\in D\}. $ 
\end{itemize}