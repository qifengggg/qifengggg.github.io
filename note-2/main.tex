\documentclass[oneside]{ctexbook}
\usepackage{ctex}
\usepackage[utf8]{inputenc}
\usepackage{amsmath}
%\usepackage{amsthm}
\usepackage{ntheorem}
\usepackage{booktabs}
\usepackage{caption}
\usepackage{listings}
\usepackage[dvipsnames]{xcolor}
\usepackage{xeCJK}
\usepackage{bm}
\usepackage{fancyhdr}
\usepackage{graphicx}
\usepackage{amssymb}
\usepackage{mathrsfs}
\usepackage{titlesec, blindtext, color}
\usepackage{arydshln}
\usepackage{hyperref} 
\usepackage[OT1]{fontenc}
\usepackage{geometry}
\usepackage{comment}
\usepackage{extarrows}
\usepackage{inconsolata}
\usepackage{yhmath}
\usepackage{enumitem}
\usepackage[titletoc]{appendix}

\geometry{a4paper,scale=0.8}

\hypersetup{hidelinks,
	colorlinks=true,
	allcolors=black,
	pdfstartview=Fit,
	breaklinks=true
}

\ctexset{
	 chapter = {
		name = {第,章},
		format = \linespread{1.0}\zihao{1}\bfseries\sffamily\centering,
		nameformat = {\fontsize{30pt}{0pt}\centering\vspace{15pt}},
		titleformat = {\centering},
		number = \chinese{chapter},
		numberformat = \sffamily,
		aftername = ,
		beforeskip = {7pt},
		afterskip = {18pt},
		aftername = {\\},
		pagestyle = plain,
	},
	section={
		%format用于设置章节标题全局格式,作用域为标题和编号
		%字号为小三,字体为黑体,左对齐
		%+号表示在原有格式下附加格式命令
		format+ = \zihao{-1} \heiti \raggedright\centering,
		%name用于设置章节编号前后的词语
		%前、后词语用英文状态下,分开
		%如果没有前或后词语可以不填
		name = {,.\,},
		%number用于设置章节编号数字输出格式
		%\chinese输出section编号为中文,\arabic输出为阿拉伯数字
		number = \Roman{section},
		%beforeskip用于设置章节标题前的垂直间距
		%ex为当前字号下字母x的高度
		%基础高度为1.0ex,可以伸展到1.2ex,也可以收缩到0.8ex
		beforeskip = 1.0ex plus 0.2ex minus .2ex,
		%afterskip用于设置章节标题后的垂直间距
		afterskip = 1.0ex plus 0.2ex minus .2ex,
		%aftername用于控制编号和标题之间的格式
		%\hspace用于增加水平间距
		aftername = \hspace{0pt}
	},
	subsection={
		format+ = \zihao{-2} \kaishu  \raggedright,
		%仅输出subsection编号且为中文
		number = \roman{subsection},
		name = {,.~},
		beforeskip = 1.0ex plus 0.2ex minus .2ex,
		afterskip = 1.0ex plus 0.2ex minus .2ex,
		aftername = \hspace{0pt}
	},
	subsubsection={
		%设置对齐方式为居中对齐
		format+ = \zihao{-3} \fangsong,
		%仅输出subsubsection编号,格式为阿拉伯数字,打字机字体
		number = \ttfamily\arabic{subsubsection},
		name = {,.},
		%beforeskip = 1.0ex plus 0.2ex minus .2ex,
		afterskip = 1.0ex plus 0.2ex minus .2ex,
		aftername = \hspace{0pt}
	}
}

\newenvironment{Appendices}[1][]
{

	\ctexset{
		chapter = {
			name = {#1},
			format = \linespread{1.0}\zihao{1}\bfseries\sffamily\centering,
			nameformat = {\fontsize{30pt}{0pt}\centering\vspace{15pt}},
			titleformat = {\centering},
			%number = \chinese{chapter},
			number = ,
			numberformat = \sffamily,
			aftername = ,
			beforeskip = {7pt},
			afterskip = {18pt},
			aftername = {\\},
			pagestyle = plain,
		},
		section={
			%format用于设置章节标题全局格式,作用域为标题和编号
			%字号为小三,字体为黑体,左对齐
			%+号表示在原有格式下附加格式命令
			format+ = \zihao{-1} \heiti \raggedright\centering,
			%name用于设置章节编号前后的词语
			%前、后词语用英文状态下,分开
			%如果没有前或后词语可以不填
			name = {,.\,},
			%number用于设置章节编号数字输出格式
			%\chinese输出section编号为中文,\arabic输出为阿拉伯数字
			number = \Roman{section},
			%beforeskip用于设置章节标题前的垂直间距
			%ex为当前字号下字母x的高度
			%基础高度为1.0ex,可以伸展到1.2ex,也可以收缩到0.8ex
			beforeskip = 1.0ex plus 0.2ex minus .2ex,
			%afterskip用于设置章节标题后的垂直间距
			afterskip = 1.0ex plus 0.2ex minus .2ex,
			%aftername用于控制编号和标题之间的格式
			%\hspace用于增加水平间距
			aftername = \hspace{0pt}
		},
		subsection={
			format+ = \zihao{-2} \kaishu  \raggedright,
			%仅输出subsection编号且为中文
			number = \roman{subsection},
			name = {,.},
			beforeskip = 1.0ex plus 0.2ex minus .2ex,
			afterskip = 1.0ex plus 0.2ex minus .2ex,
			aftername = \hspace{0pt}
		},
		subsubsection={
			%设置对齐方式为居中对齐
			format+ = \zihao{-3} \fangsong,
			%仅输出subsubsection编号,格式为阿拉伯数字,打字机字体
			number = \ttfamily\arabic{subsubsection},
			name = {,.},
			%beforeskip = 1.0ex plus 0.2ex minus .2ex,
			afterskip = 1.0ex plus 0.2ex minus .2ex,
			aftername = \hspace{0pt}
		}
	}
}

\fancyhf{}
\fancyhead[L]{
	\begin{minipage}[c]{0.06\textwidth}
		\hyperlink{Index}{\includegraphics[height=7.5mm]{1.jpg}}
	\end{minipage}
	\begin{minipage}[c]{0.4\textwidth}
	高等数学笔记
	\end{minipage}}
\fancyhead[R]{
    \begin{minipage}[r]{0.1\textwidth}
	\href{https://qifengggg.github.io/}{奇峰}
    \end{minipage}
}
\cfoot{\thepage}

\usepackage{titletoc}
\titlecontents{section}[6em]{\bfseries \zihao{5} \vspace{2pt}}
{\contentslabel{2.5em}}{\hspace*{-4em}}{~\titlerule*[0.6pc]{$.$}~\contentspage}
\titlecontents{subsection}[7em]{\zihao{5}}{\contentslabel{2em}}{\hspace*{-2em}}{~\titlerule*[0.6pc]{$.$}~\contentspage}
\titlecontents{subsubsection}[8em]{\zihao{5}}{\contentslabel{3em}}{\hspace*{-2em}}{~\titlerule*[0.6pc]{$.$}~\contentspage}
\titlecontents{paragraph}[11em]{\zihao{5}}{\contentslabel{4em}}{\hspace*{-2em}}{~\titlerule*[0.6pc]{$.$}~\contentspage}

\pagestyle{fancy}
%\renewcommand{\qedsymbol}{$\blacksquare$}

\renewcommand{\labelenumi}{\roman{enumi}.}
\renewcommand{\labelitemii}{$ \circ $ }
\newcommand{\dis}{\displaystyle}
\newcommand{\Attention}[1]{\textcolor{red}{\bf{#1}}}
\newcommand{\red}[1]{\textcolor{red}{{#1}}}
\newcommand{\sssubsection}[1]{\noindent\textbf{#1}}
\newcommand{\nextline}{\vspace{5pt}\newline}
\newcommand{\dsqrt}{\dis\sqrt}
\newcommand{\whichis}{\mathop=\limits^{\exists}}
\newcommand{\pprime}{{\prime\prime}}
\newcommand{\important}{\textcolor{red}{\textbullet}}

\title{\includegraphics[scale=0.6]{1.jpg}\\ \textsf{高等数学笔记}}
\author{奇峰}
\date{之前}

\begin{document}

\theoremseparator{}
\newtheorem{def1}{定义}[section]
\newtheorem{theo1}{定理}[section]
\newtheorem{func1}{方法}[section]
\newtheorem{infer1}{推论}[section]
\newenvironment{proof}{\begin{itemize}\item \textbf{证明}
	
	}{\end{itemize}}
%\newenvironment{<环境名称>}[<参数个数>][<首参数默认值>]{<环境前定义>}
%							{<环境后定义>}


\newenvironment{Def}[1][\quad{}]{\begin{def1}\textbf{#1}}{\end{def1}}
\newenvironment{Theo}[1][\quad{}]{\begin{theo1}\textbf{#1}}{\end{theo1}}
\newenvironment{Func}[1][\quad{}]{\begin{func1}\textbf{#1}}{\end{func1}}
\newenvironment{Infer}[1][\quad{}]{\begin{infer1}\textbf{#1}}{\end{infer1}}
\newenvironment{Field}[1][\quad{}]{\noindent\newline\textbf{#1}}{}

\setlength{\parskip}{3pt}
\setenumerate[1]{topsep=2pt}
\setitemize[1]{topsep=2pt}

\newcommand{\from}[1]{\hypertarget{#1}{}}
\newcommand{\goto}[1]{\quad{}\hyperlink{#1}{$ \diamondsuit $ }\hypertarget{K#1}{}}
%\newcommand{\getback}[1]{\begin{flushright}\hyperlink{K#1}{$ \blacksquare $ }\end{flushright}}

\newcommand{\getback}[1]{\quad{}\hyperlink{K#1}{$ \blacksquare $ }}
\newcommand{\qline}{\underline{~~~~~~~~}}

\frontmatter
\maketitle
\hypertarget{Index}{}
\tableofcontents

\mainmatter

\chapter{函数~~极限~~连续}

\section{函数的性态}

\subsection{有界性的判定}

\begin{itemize}
    \item 若 $ {\displaystyle\lim_{x\rightarrow x_0}}f(x)=A, $ 则存在 $ \delta>0, $ 当
    $ 0<|x-x_0|<\delta $ 时,$ f(x) $ 有界;
    \item  若 $ f(x) $ 在 $ [a,b] $ 连续,则其在 $ [a,b] $ 有界;
    \item[\important]  若 $ f(x) $ 在 $ (a,b) $ 连续,且 $ {\displaystyle\lim_{x\rightarrow a^+}}f(x),
    {\displaystyle\lim_{x\rightarrow b^-}}f(x) $ 均存在,则其在 $ (a,b) $ 有界;
    \item $ f'(x) $ 在\underline{有限区间}有界 $ \Rightarrow f(x) $ 在该区间有界。
\end{itemize}

\subsection{导函数、原函数的奇偶性与周期性}

\sssubsection{导函数的奇偶性与周期性}

\begin{itemize}
    \item 可导奇函数的导函数为偶函数;
    \item 可导偶函数的导函数为奇函数;
    \item 可导周期函数的导函数为周期函数;
\end{itemize}

\sssubsection{原函数的奇偶性与周期性}

\begin{itemize}
    \item 连续奇函数的原函数均为偶函数;
    \item 连续偶函数的原函数仅有一个为奇函数,即 $ C = 0 $ 时;
    \item 周期函数的原函数为周期函数 $ \Rightarrow \dis \int_0^T f(t)\mathrm{d}t = 0. $ 
\end{itemize}

\section{极限的概念}

讨论数列最值,将其拆分为前 $ N $ 个与后无穷个,前者求最值,后者利用极限定义
可知其接近极限值。

讨论同时包含 $ \sin(x_n),\cos(x_n) $ 的抽象数列时,可以考虑
令 $ x_n = \begin{cases}
    \pi/2,& 2i + 1\\ -\pi/2,& 2i
\end{cases}, $ 利用 $ \sin,\cos $ 奇偶性的不同。

\section{重点 - 函数极限的计算}

\subsection{$ 0/0 $ 形}

\sssubsection{洛必达法则}

若 $ f(x),g(x) $ 
\begin{itemize}[topsep = 0pt]
    \item $ \lim f(x)=\lim g(x)=0/\infty; $
    
    可以推广为 $ \dfrac{\blacksquare}{\infty}; $
    \item $ f(x),g(x) $ 在 $ x_0 $ 某去心邻域内可导,且 $ g'(x)\neq 0 $ ;
    
    此处注意, $ \begin{cases}
        n\textrm{阶可导}&\Rightarrow \textrm{洛}n-1\textrm{次}+\textrm{导数定义}\\
        n\textrm{阶连续导数}&\Rightarrow \textrm{洛}n\textrm{次}
    \end{cases} $ 
    \item $\dis \frac{\lim f'(x)}{\lim g'(x)}=A(\textrm{或}\infty), $ 
\end{itemize}
则 $\dis \frac{\lim f(x)}{\lim g(x)}=A(\textrm{或}\infty). $ 

~

\sssubsection{等价代换}

当 $ x\rightarrow0 $ 时,有

\begin{itemize}
    \item $ \sin x \sim \tan x \sim \arcsin x \sim \arctan x \sim e^x - 1 \sim \ln(1+x) \sim x; $ 
    \item $ e^x - 1 -x \sim x - \ln(1+x) \sim 1 - \cos x \sim \dfrac{x^2}{2}; $ 
    \item $ (1+x)^{\alpha}-1\sim \alpha x;$
    \item $ x - \sin x \sim \arcsin x - x \sim \dfrac{x^3}{6};$ 
    \item $ \tan x - x \sim x - \arctan x \sim \dfrac{x^3}{3};$ 
    \item $ \tan x - \sin x \sim \arcsin x - \arctan x \sim \dfrac{x^3}{2}; $ 
\end{itemize}

对于以上等价无穷小,有

\begin{enumerate}
    \item 可变量代换,如 $ \sin \square \sim \square,\ \tan \square \sim \square,\cdots $ 
    \item $ x\rightarrow0 $ 时,
    $\dis a^x-1=e^{x\ln a} -1\sim x\ln a,\ \log_a(1+x)=\frac{\ln(x+1)}{\ln a}\sim \frac{x}{\ln a};$
    \item 若 $ x\rightarrow a $ ,可以令 $ t = x - a \rightarrow 0. $ 
    \item 不能在复合函数的自变量处做等价代换,如 $ x\rightarrow 0\nRightarrow f(x)\sim f(\sin x). $ 
\end{enumerate}

\sssubsection{泰勒公式}

\begin{itemize}
    \item $\dis e^x = \sum_{i=0}^n \dfrac{x^n}{n!}+ o(x^n)$;
    \item $\dis \cos x = 1 - \frac{x^2}{2} + \dfrac{x^4}{24} +\dots + \frac{(-1)^{n}x^{2n}}{(2n)!} + o(x^{2n})$ ;
    \item $\dis \sin x = x - \frac{x^3}{6} + \dots + \frac{(-1)^{n}x^{2n+1}}{(2n+1)!} + o(x^{2n+1}) $ ;
    \item $\dis \arcsin x = x + \frac{x^3}{6} +o(x^3)$ ;
    \item $\dis \tan x = x + \frac{x^3}{3} + o(x^3)$ ;
    \item $\dis \arctan x = x - \frac{x^3}{3} + o(x^3)$ ;
    \item $\dis \ln (1+x) = x - \frac{x^2}{2} + \frac{x^3}{3} + \dots + \frac{(-1)^{n-1}x^n}{n} + o(x^n) $;
    \item $\dis \ln(1-x) = -(x+\frac{x^2}{2} + \frac{x^3}{3}) + o(x^3)$;
    \item $\dis (1+x)^\alpha = 1+\sum_{k=1}^n C_\alpha^kx^k + o(x^n) $ ,
    其中 $\dis C_\alpha^k=\frac{\prod_{i = 0}^{k-1}(\alpha - i)}{k!} $ 

    如,$ \dis \sqrt{1+x} = 1 + \dfrac{1}{2}x - \dfrac{1}{8}x^2 + o(x^2); $ 
    \item $\dis \frac{1}{1-x} = \sum_{i=0}^n x^i + o(x^n) $ ;
    \item $\dis \frac{1}{1+x} = \sum_{i=0}^n (-1)^i x^i + o(x^n) $;
\end{itemize}

泰勒公式求极限时,\begin{itemize}
    \item 分子阶数不小于分母阶数;
    \item 加减不抵消,“齐头并进”;
    \item 可推广为 $ \square\rightarrow 0. $
\end{itemize}

\subsection{$ \infty/\infty $ 形}

主要方法有\begin{itemize}
    \item 洛必达;
    \item 抓大头,即每个因式保留高阶无穷大;
    
    $\dis x\rightarrow 0 \Rightarrow \ln^\alpha(x)\ll x^\beta \ll a^x \ll x^x, $ 
    其中 $ \alpha,\beta > 0, a > 1. $ 
\end{itemize}

\subsection{$ \infty - \infty $ 形}

主要方法有\begin{itemize}
    \item 通分(有分式时);
    \item 有理化(有根号时);
    \item 倒代换,即令 $ t = \dfrac{1}{x}. $ 
\end{itemize}

\subsection{$ 0^0 $ 与 $ \infty^0 $ 形}

若 $ {\displaystyle\lim_{x\rightarrow x_0}}u(x) = 0(\infty),{\displaystyle\lim_{x\rightarrow x_0}}v(x) = 0, $ 则
$ {\displaystyle\lim_{x\rightarrow x_0}}u(x)^{v(x)} = \exp\left({\displaystyle\lim_{x\rightarrow x_0}}v(x)\ln u(x)\right). $ 

\subsection{$ 1^\infty $ 形}

\begin{itemize}
    \item 若 $ {\displaystyle\lim_{x\rightarrow x_0}}u(x) = 0,{\displaystyle\lim_{x\rightarrow x_0}}v(x) = \infty, $ 则
    $ {\displaystyle\lim_{x\rightarrow x_0}}[1+u(x)]^{v(x)} = 
    \exp\left({\displaystyle\lim_{x\rightarrow x_0}}v(x)u(x)\right). $ 
    \item 若 $ {\displaystyle\lim_{x\rightarrow x_0}}u(x) = 1,{\displaystyle\lim_{x\rightarrow x_0}}v(x) = \infty, $ 则
    $ {\displaystyle\lim_{x\rightarrow x_0}}u(x)^{v(x)} = 
    \exp\left({\displaystyle\lim_{x\rightarrow x_0}}v(x)[u(x)-1]\right). $ 
\end{itemize}

事实上,有
\begin{equation*}
    \begin{aligned}
        {\displaystyle\lim_{x\rightarrow 0}}
        \left(\dfrac{\sum_{i=0}^n a_i^x}{n}\right)^\frac{1}{x}
        = \sqrt{\prod a_i}
    \end{aligned}
\end{equation*}

\section{已知极限反求参数}

若 $ {\displaystyle\lim_{x\rightarrow x_0}}\dfrac{f(x)}{g(x)} $ 存在且
$ g{\displaystyle\lim_{x\rightarrow x_0}}g(x) = 0, $ 则 $ {\displaystyle\lim_{x\rightarrow x_0}}f(x) = 0. $ 

若 $ {\displaystyle\lim_{x\rightarrow x_0}}\dfrac{f(x)}{g(x)} = A \Attention{\neq 0} $ 且
$ g{\displaystyle\lim_{x\rightarrow x_0}}f(x) = 0, $ 则 $ {\displaystyle\lim_{x\rightarrow x_0}}g(x) = 0. $ 

\sssubsection{例}

$ {\displaystyle\lim_{x\rightarrow 0}}
\dis \int_b^x \dfrac{\ln(1+t^3)}{t}\mathrm{d}t = 0 \Leftrightarrow b = 0. $

\begin{itemize}
    \item 证明
    
    $ b = 0 $ 时原式显然成立。

    $ \dfrac{\ln(1+t^3)}{t} > 0 (t\neq 0)\Rightarrow b \neq 0 $ 时原式不成立。
    
    因此,$ b $ 只能为零。
\end{itemize}

\section{无穷小阶的比较}

\sssubsection{例}

设函数 $ f(x) $ 在 $ x = 0 $ 的某邻域内具有二阶连续导数,且 $ f(0)\neq 0,f'(0)\neq 0,f\pprime(0)
\neq 0, $ 则存在一组唯一的 $ \lambda_i,i=1,2,3 $ 使得 $ h\rightarrow 0 $ 时,有
$ \sum \lambda_if(ih) - f(0) $ 是 $ h^2 $ 的高阶无穷小。

\begin{itemize}
    \item 一般证明
    
    $ \sum \lambda_if(ih) - f(0) $ 是 $ h^2 $ 的高阶无穷小 $ \Rightarrow \sum \lambda_if(ih) - f(0) = 0; $ 

    对上式两边求导,有 $ \sum \lambda_i if'(ih) = 0; $ 

    对上式两边求导,有 $ \sum \lambda_i^2 if^\pprime(ih) = 0; $ 

    因此,有 $ \begin{pmatrix}
        1&1&1\\ 1&2&3\\ 1&4&9
    \end{pmatrix}\begin{pmatrix}
        \lambda_1 \\
        \lambda_2 \\
        \lambda_3 \\
    \end{pmatrix} = 
    \begin{pmatrix}
        1 \\
        2 \\
        3 \\
    \end{pmatrix}, $ 由于系数矩阵满秩,其有唯一解,因而得证。
    \item 泰勒法
    
    将 $ f(h),f(2h),f(3h) $ 展开至二阶,代入
    $ {\displaystyle\lim_{h\rightarrow 0}}\dfrac{\sum \lambda_i f(ih) - f(0)}{h^2}, $ 
    然后和前述做法一致。
\end{itemize}

\section{重点 - 数列极限的计算}

\subsection{夹逼定理}

左边缩,右边放,两边极限相等。

放缩时,有不等式

\begin{itemize}
    \item $ 0<x<\pi/2, $ 则 $ \sin x<x<\tan x;
    \sin x < x < \pi/2 \sin x; 2/\pi x < \sin x < x; $ 
    
    利用 $ f(x) = \dfrac{\sin x}{x} $ 的性质证明。
    \item $ x > 0, x > \sin x; x < 0, x < \sin x; $ 
    \item $ e^x > 1 + x, x\neq 0; $ 
    \item $ \dfrac{x}{1+x}< \ln (x+1)< x, x > -1, x \neq 0. $ 
\end{itemize}

\subsection{单调有界定理}

对数列 $ x_{n+1} = f(x_n) $ 求极限,方法如下。
\begin{itemize}
    \item 适当放缩以证明有界性;
    \item 做差、做商或求导证明单调性;
    \item 若其单调,由单调有界知$ \lim x_n $ 存在;
    \item 令 $ \lim x_n = a, $ 对原式两端取极限,有 $ a = f(a), $ 因此可以解得 $ a; $ 
    \item 若其不单调,则设 $ \lim x_n = a, $ 再利用夹逼定理证明前者确实成立。
\end{itemize}

\subsection{定积分}

\begin{equation*}
    \begin{aligned}
        \int_a^b f(x)\mathrm{d}x&={\displaystyle\lim_{d\rightarrow 0}}\sum_{i=i}^nf(\xi_i)\dfrac{b-a}{n}
    \end{aligned}
\end{equation*}

其中,$ \xi_i\in\left[a+\dfrac{i-1}{n}(b-a),a+\dfrac{i}{n}(b-a)\right]. $ 

\section{间断点的判定}

设 $ x=a $ 为 $ f(x) $ 的一间断点,
\begin{enumerate}
    \item 若 $ {\displaystyle\lim_{x\rightarrow a^+}}f(x) $ 与 $ {\displaystyle\lim_{x\rightarrow a^-}}f(x) $ 均存在,
    则称 $ x=a $ 为 $ f(x) $ 的一个第一类间断点,其还能\textbf{且必须要}分为\begin{itemize}
        \item 可去间断点 - 
        $ {\displaystyle\lim_{x\rightarrow a^+}}f(x)={\displaystyle\lim_{x\rightarrow a^-}}f(x); $
        \item 跳跃间断点 - 
        $ {\displaystyle\lim_{x\rightarrow a^+}}f(x)\neq{\displaystyle\lim_{x\rightarrow a^-}}f(x); $
    \end{itemize}
    \item 若 $ {\displaystyle\lim_{x\rightarrow a^+}}f(x) $ 与 $ {\displaystyle\lim_{x\rightarrow a^-}}f(x) $ 
    中有至少一个不存在,则称其为第二类间断点。第二类间断点不用强制细分。

    第二类间断点可以分为
    \begin{itemize}
        \item 无穷间断点 - 左右极限至少有一个为无穷;
        \item 震荡间断点 - 左右极限至少有一个不存在,但不是无穷;
    \end{itemize}
\end{enumerate}

可能存在间断点的地方:
\begin{itemize}
    \item 初等函数的无定义点;
    \item 分段函数的分段点。
\end{itemize}


\chapter{随机变量及其分布}

\section{随机变量及其分布函数}

将样本空间 $ \Omega $ 上的实值单值函数 $ X = X(\omega),\omega \in \Omega $ 
称为随机变量。

$ F(x) = P\{X\leq x\},x\in (-\infty,+\infty) $ 是随机变量的分布函数。

分布函数具有以下性质。
\begin{itemize}
    \item 非负性 - $ 0\leq F(x)\leq 1 $ ;
    \item 规范性 - $ F(-\infty) = {\displaystyle\lim_{x\rightarrow -\infty}}F(x) = 0;
    F(+\infty) = {\displaystyle\lim_{x\rightarrow +\infty}}F(x) = 1; $ 
    \item 单调不减性 - $ \forall x_1 < x_2, F(x_1)\leq F(x_2) $ ;
    \item 右连续性 - $ \forall x_0 \in R, F(x_0)={\displaystyle\lim_{x\rightarrow x_0^+}}F(x) = F(x_0+0) $ .
\end{itemize}

其中规范性可以优先考虑,因为其与微积分有关。

当已知随机变量 $ X $ 的分布函数 $ F(x) $ 时,有
\begin{itemize}
    \item $ P(X\leq b) = F(b) $ ;
    \item $ P(X = b) = F(b) - F(b-0)$ ;
    \item $ P(X < b) = F(b-0) $ ;
    \item $ P(X > b) = 1 - P(X \leq b) = 1 - F(b) $ ;
    \item $ P(a < X \leq b) = P(X\leq b) - P(X\leq a) = F(b) - F(a) $ ;
    \item $ P(a \leq X < b) = P(X < b) - P(X < a) = F(b-0) - F(a-0) $ ;
    \item $ P(a \leq X \leq b) = P(X\leq b) - P(X < a) = F(b) - F(a-0) $ ;
    \item $ P(a < X < b) = P(X < b) - P(X \leq a) = F(b-0) - F(a) $ ;
\end{itemize}

其中,前三条的应用最为广泛。

\sssubsection{离散型随机变量}

离散型随机变量的概率分布形如下表。

\begin{table}[!htbp]\centering
    \begin{tabular}{l|lllll}
    X & $x_1$ & $x_2$ & $\dots$ & $x_k$ & $\dots$ \\ \hline
    P & $p_1$ & $p_2$ & $\dots$ & $p_k$ & $\dots$
    \end{tabular}
\end{table}

其中 $ p_k>\geq0,k\in N^*,\sum_{i=1}^{\infty}p_i = 1 $.

离散型随机变量的分布函数为右连续的阶梯型函数,区间左开右闭,为概率的累加。

\sssubsection{连续型随机变量}

\begin{Def}[连续型随机变量概率密度]

    设随机变量 $ X $ 的分布函数为 $ F(x) $ ,
    若存在非负可积函数 $ f(x)\geq 0, x\in R $ 使得对任意实数 $ x $ ,都有
    $ \dis F(x) = P\{X \leq x\} = \int_{-\infty}^x f(t)\mathrm{d}t $, 则称 $ X $ 为
    连续型随机变量,函数 $ f(x) $ 为 $ X $ 的概率密度函数。
\end{Def}

\begin{Theo}[f(x)为密度函数的充要条件]

    $$
        f(x)\textrm{是概率密度}\Leftrightarrow\begin{cases}
            f(x)\geq 0 ;\\
            \dis \int_{-\infty}^{+\infty} f(x)\mathrm{d}x = 1.
        \end{cases}
    $$ 
\end{Theo}

连续型随机变量 $ X $  具有以下性质。
\begin{itemize}
    \item $ X $ 的分布函数是连续函数,因此 $ \forall a\in R, P\{X = a\} = 0 $;
    \item $ \forall a,b \in R, P\{a<X\leq b\} = \int_a^b f(x)\mathrm{d}x $;
    \item 在 $ f(x) $ 的连续点处,有 $ F'(x) = f(x) $ .
\end{itemize}

对连续型随机变量的题目,$ f(x) $ 简单或者具有特殊性质意味着作图解。

\section{常见分布}

\subsection{离散型}

离散型随机变量需要注意其取值(尤其是第一个值)以及其对应的概率。

\sssubsection{0-1分布}

\begin{table}[!htbp]\centering
    \begin{tabular}{c|cc}
    $ X $  & $ 0 $  & $ 1 $  \\ \hline
    $ P $  & $ 1-p $  & $ p $ 
    \end{tabular}
\end{table}
其中 $ 0 < p < 1 $ .

\sssubsection{二项分布}

设事件 $ A $ 在任意一次试验中出现的概率均为 $ 0<p<1 $,而 $ X $ 为$ n $ 重伯努利试验中 $ A $ 发生的次数,则
$ X $ 所有可能取值为 $ 0,1,\dots,n $ ,对应的概率为
$ \dis P\{X = k\} = C_n^kp^k(1-p)^{n-k} $ .

\sssubsection{几何分布 $ G(p) $ }

若 $ X $ 的概率分布为 $$ \dis P\{X=k\} = (1-p)^{k-1}p, k = 1,2,\dots, $$ 
则称 $ X $ 服从参数为 $ p $ 的几何分布,记为 $ X\sim G(p) $ .

\sssubsection{泊松分布 $ P(\lambda) $ }

若随机变量 $ X $ 的概率分布满足 $$ \dis P\{X = k\} \dfrac{\lambda^k}{k!}e^{-\lambda k},\lambda > 0,k=0,1,2,\dots,$$
则称 $ X $ 服从参数为 $ \lambda $ 的泊松分布,记为 $ X\sim P(\lambda) $ .

\sssubsection{超几何分布 $ H(N,M,n) $ }

若随机变量 $ X $ 的概率分布为
$$
    P\{X=k\} = \frac{\dis C_M^kC_{N-M}^{n-k}}{C_N^n},k = 0,1,2,\min(M,n), M,N,n\in Z^+
$$ 

则称 $ X $ 服从参数为 $ N,M,n $ 的超几何分布,记为 $ X\sim H(N,M,n) $ .

\subsection{连续型}

连续性随机变量的密度函数非零区间即其定义区间。

\sssubsection{均匀分布 $ U(a,b) $ }

$ \dis f(x) = \begin{cases}
    \dfrac{1}{b-a},&a<x<b,\\0,&\textrm{其他}
\end{cases} $ 

\sssubsection{指数分布 $ E(\lambda) $ }

$ \dis f(x) = \begin{cases}
    \lambda e^{-\lambda x},& x>0,\\0,&x\leq 0
\end{cases}$ 

注意,此处可能应用泊松过程的增量平稳性,即\newline 
$ \dis \forall s,t\leq0,n\leq0, P\{N(s+t)-N(s) = n\} = P\{N(t) = n\} $.

\sssubsection{正态分布 $ N(\mu,\sigma^2) $ }

$ \dis f(x) = \dfrac{1}{\sqrt{2\pi}\sigma}\exp\left\{-\dfrac{(x-\mu)^2}{2\sigma^2}\right\},x\in R $ 

特别地,$\dis X\sim N(0,1)\Rightarrow \phi(x) = \dfrac{1}{\sqrt{2\pi}}\exp\left\{-\dfrac{x^2}{2}\right\} $,
此时其分布函数为 $ \Phi(x) $.

对于一般的正态分布 $ X\sim N(\mu,\sigma^2) $,有 $ F(x) = \Phi(\dfrac{x-\mu}{\sigma}) $.

利用正态分布密度函数的规范性,可求泊松积分 $ \dis \int_{-\infty}^{+\infty} e^{-x^2}\mathrm{d}x = \sqrt{\pi}$ 

\section{随机变量函数的分布}

\subsection{离散型随机变量函数的分布}

对离散型随机变量函数,采用列表法。

\subsection{连续型随机变量函数的分布}

对已知概率密度为 $ f_X(x) $ 的随机变量 $ X $ 有 $ Y = g(X) $ ,需要求
$ Y $ 分布函数 $ F_Y(y) $ 和概率密度函数 $ f_Y(y) $时,有两种办法。

\sssubsection{公式法}

若 $ y = g(x) $ 严格单调,其反函数 $ x = h(y) $ 有一阶连续导数,则
$ Y = g(X) $ 也是连续型随机变量,其密度函数为\newline 
$ \dis f_Y(y) = \begin{cases}
    f_X(h(y))|h'(y)|,& \alpha < y < \beta\\0,&\textrm{其他}
\end{cases} $ 
其中 $ (\alpha,\beta) $ 为 $ y = g(x) $ 在 $ X $ 上可能取值的区间上的值域。

\sssubsection{分布函数法}

先按分布函数的定义求得 $ Y $ 的分布函数,再求导得到密度函数。即求 $ \dis F_Y(y) = P(Y\leq y) = P(g(X) 
\leq y) = \mathop{\int}\limits_{g(x)\leq y}f_X(x)\mathrm{d}x $,再求 $ f_Y(y) = F_Y'(y) $.  

具体而言,连续性随机变量的函数的分布函数法如下。
\begin{enumerate}
    \item 由 $ X $ 取值范围$ (a,b) $ 确定 $ y $ 的取值范围$ (c,d) $;
    \item 由分布函数的定义,确定 $ F_Y $ 的左右两头,即对 $ F_Y(y) = P(Y\leq y) $,\begin{itemize}
        \item $ y < c, F_Y(y) = 0 $ ;
        \item $ y > d, F_Y(y) = 1 $ .
    \end{itemize}
    \item 定中间,即 $ y \in (c,d) $ .
    \begin{equation*}
        \begin{array}{c}
            F_Y(y) = P(Y\leq y) = P(g(X)\leq y)\\\Downarrow\\
            P(X\leq h(y)),\textrm{此时若$ g(x) $ 分段则分段处理}\\\Downarrow\\
            \dis \int_{-\infty}^{h(y)} f_X(x)\mathrm{d}x,\textrm{此处取交集}
        \end{array}
    \end{equation*}     
\end{enumerate}

注意,若连续型随机变量分布函数为 $ F(x) $ ,若 $ Y = F(X) $ ,则 $ Y\sim U(0,1) $ .




\chapter{一元函数积分学}

\section{不定积分}

\subsection{原函数}

\begin{Def}[原函数]

    设函数 $ f(x) $ 和 $ F(x) $ 在区间 $ I $ 上有定义,若$ F'(x) = f(x) $ 在区间 $ I $ 上成立,则
    称 $ F(x) $ 为 $ f(x) $在$ I $ 上的一个原函数。
\end{Def}

注意,原函数存在 $ \Rightarrow $ 有无穷多个,每个原函数之间仅差一个常数。

设 $ f(x) $ 在区间 $ I $ 上连续,则$ f(x) $ 在区间 $ I $ 上一定存在原函数,但是其不一定是
初等函数,如 $ \int \sin x^2 \mathrm{d}x $ ,$ \int \cos x^2 \mathrm{d}x $ ,
$ \int \frac{\sin x}{x} \mathrm{d}x $ ,$ \int \frac{\cos x}{x} \mathrm{d}x $ ,
$ \int \frac{1}{\ln x} \mathrm{d}x $ , $ \int e^{-x^2} \mathrm{d}x $ 等,称其为“积不出来”。

若$ f(x) $ 在区间 $ I $ 上存在第一类间断点或第二类中的无穷间断点,则其在该区间内没有原函数。

注意,函数在区间内有震荡间断点的,可能有原函数;总结如下。

\begin{Theo}[原函数存在定理]

    \begin{itemize}
        \item 连续函数存在原函数;
        \item 含有第一类间断点或无穷间断点的函数不存在原函数;
        \item 含有震荡间断点的函数可能存在原函数。
    \end{itemize}    
\end{Theo}

\subsection{不定积分}

\begin{Def}[不定积分]

    设 $ F(x) $ 为 $ f(x) $ 在区间 $ I $ 上的一个原函数,则$ f(x) $ 
    的所有原函数 $ \{F(x)+C\},C\in \mathbb{R} $ 为 $ f(x) $在$ I $ 的不定积分,
    记为 $\dis \int f(x) \mathrm{d}x$。
\end{Def}

\begin{Field}[不定积分基本性质]

    设 $\dis \int f(x)\mathrm{d}x = F(x)+C, $ 其中 $ F(x) $ 为 $ f(x) $ 一原函数,$ C $ 为一常数,
    则\begin{itemize}
        \item $\dis \int F'(x)\mathrm{d}x=F(x)+C $ 或 $\dis \int \mathrm{d}F(x)=f(x)+C ; $
        \item $\dis \left[\int f(x)\mathrm{d}x\right]'=f(x) $ 
        或 $\dis \mathrm{d}\left[\int f(x)\mathrm{d}x\right]=f(x)\mathrm{d}x ; $
        \item $\dis \int kf(x)\mathrm{d}x = k\int f(x)\mathrm{d}x ; $
        \item $\dis \int [f(x)\pm g(x)]\mathrm{d}x = \int f(x)\mathrm{d}x \pm \int g(x)\mathrm{d}x ; $
    \end{itemize}
\end{Field}

\sssubsection{基本积分公式}

$ C $ 为任意常数,则
\begin{enumerate}
    \item 幂函数
    
    $\dis \int x^\alpha \mathrm{d}x = \frac{x^{\alpha +1}}{\alpha + 1} + C\ (\alpha \neq -1), $
    
    特别地, $\dis \int \frac{1}{x} \mathrm{d}x = \ln|x| + C; $ 
    \item 指数函数
    
    $\dis \int a^x \mathrm{d}x = \frac{a^x}{\ln a} + C\ (a>0,a\neq 1) ; $
    
    特别地,$\dis \int e^x\mathrm{d}x = e^x+C ; $
    \item 对数函数
    
    $ \dis \int\ln x\mathrm{d}x = x \ln x - x + C; $ 
    \item 三角函数
    
    $ \dis \int \cos x \mathrm{d}x = \sin x + C, \dis \int \sin x \mathrm{d}x = -\cos x + C ; $

    $ \dis \int \tan x\mathrm{d}x = -\ln|\cos x| + C,\int \cot x \mathrm{d}x = \ln|\sin x| + C; $ 

    $ \dis \int \sec x\mathrm{d}x = \ln|\sec + \tan x| + C,\int \csc x= -\ln|\csc x + \cot x| + C
    = \ln|\csc x - \tan x| + C; $ 

    $\dis \int \frac{1}{\cos^2 x}\mathrm{d}x = \tan x + C, \dis \int -\frac{1}{\sin^2 x}\mathrm{d}x = \cot x + C ; $

    $\dis \int \frac{\tan x}{\cos x} \mathrm{d}x = \frac{1}{\cos x} + C, 
    \dis \int \frac{\cot x}{\sin x}\mathrm{d}x = -\frac{1}{\sin x} + C ; $
    \item 平方和
    
    $\dis \int \frac{1}{1+x^2}\mathrm{d}x = \arctan x + C = -\textrm{arccot} x + C; $

    $ \dis \int \dfrac{\mathrm{d}x}{a^2 + x^2} = \dfrac{1}{a}\arctan \dfrac{x}{a} + C; $    
    \item 平方差
    
    $ \dis \int \dfrac{\mathrm{d}x}{1-x^2} = \dfrac{1}{2}\ln\left|\dfrac{1+x}{1-x}\right| + C; $ 

    $ \dis \int \dfrac{\mathrm{d}x}{a^2-x^2} = \dfrac{1}{2a}\ln\left|\dfrac{a+x}{a-x}\right| + C; $ 
    
    \item 根号下平方差
    
    $ \dis \int \dfrac{\mathrm{d}x}{\sqrt{1-x^2}} = \arcsin x + C,$

    $ \dis \int \dfrac{\mathrm{d}x}{\sqrt{a^2-x^2}} = \arcsin \dfrac{x}{a} + C,$
    \item 根号下平方和
    
    $ \dis \int \dfrac{\mathrm{d}x}{\sqrt{a^2+x^2}} = \ln(x + \sqrt{a^2+x^2}) + C,$

    $ \dis \int \dfrac{\mathrm{d}x}{\sqrt{x^2 - a^2}} = \ln\left|x - \sqrt{x^2-a^2}\right| + C;$
\end{enumerate}    



\subsection{不定积分法}

\begin{Field}[第一换元法(凑微分)]

    设 $ g(\varphi(x)) = f(x) $ ,则计算的原则是
    \begin{equation*}
        \begin{aligned}
            \int f(x)\mathrm{d}x &= \int g(\varphi(x))\varphi'(x)\mathrm{d}x \\
            &= \int g(\varphi(x))\mathrm{d}\varphi(x) \\ 
            &= F(\varphi(x))+ C
        \end{aligned}
    \end{equation*}
\end{Field}

常见的凑微分形式如下。

\begin{enumerate}
    \item $ \int f(ax+b)(ax+b)\mathrm{d}x = \frac{1}{a}\int f(ax+b)\mathrm{d}(ax+b) (a\neq 0) ; $
    \item $ \int f(ax^n+b)x^{n-1}\mathrm{d}x = \frac{1}{na}\int f(ax^n+b)\mathrm{d}(ax^n + b) (an\neq0) ; $
    \item $ \int \frac{f(\ln x)}{x}\mathrm{d}x = \int f(\ln x)\mathrm{d}(\ln x); $
    \item $ \int f(\frac{1}{x})\frac{1}{x^2}\mathrm{d}x = -\int f(\frac{1}{x})\mathrm{d}(\frac{1}{x}) ; $
    \item $ \int \frac{f(\sqrt x)}{\sqrt x} \mathrm{d}x = 2 \int f(\sqrt x)\mathrm{d}(\sqrt x) ; $
    \item $ \int f(a^x)a^x\mathrm{d}x = \frac{1}{\ln a}\int f(a^x)\mathrm{d}(a^x) $,
    特别地,$ \int f(e^x)e^x\mathrm{d}x = \int f(e^x)\mathrm{d}(e^x) ; $
    \item $ \int f(\sin x)\cos x \mathrm{d}x = \int f(\sin x)\mathrm{d}(\sin x) ; $
    \item $ \int f(\cos x)\sin x \mathrm{d}x = -\int f(\cos x)\mathrm{d}(\cos x); $
    \item $ \int \frac{f(\tan x)}{\cos^2 x}\mathrm{d}x = \int f(\tan x)\mathrm{d}(\tan x) ; $
    \item $ \int \frac{f(\cot x)}{\sin^2 x}\mathrm{d}x = -\int f(\cot x)\mathrm{d}(\cot x) ; $
    \item $ \int f(\sec x)\sec x \tan x \mathrm{d}x = \int f(\sec x)\mathrm{d}(\sec x) ; $
    \item $ \int f(\csc x)\csc x \cot x\mathrm{d}x = -\int f(\csc x)\mathrm{d}(\csc x) ; $
    \item $ \int \frac{f(\arcsin x)}{\sqrt{1-x^2}}\mathrm{d}x = \int f(\arcsin x)\mathrm{d}(\arcsin x) ; $
    \item $ \int \frac{f(\arccos x)}{\sqrt{1-x^2}}\mathrm{d}x = -\int f(\arccos x)\mathrm{d}(\arccos x) ; $
    \item $ \int \frac{f(\arctan x)}{1+x^2}\mathrm{d}x = \int f(\arctan x)\mathrm{d}(\arctan x) ; $
    \item $ \int \frac{f(\textrm{arccot} x)}{1+x^2}\mathrm{d}x = -\int f(\textrm{arccot} x)\mathrm{d}(\textrm{arccot} x) ; $
    \item $ \int \frac{f(\arctan \frac{1}{x})}{1+x^2}\mathrm{d}x = -\int f(\arctan \frac{1}{x})\mathrm{d}(\arctan \frac{1}{x}) ; $
    \item $ \int \frac{f|ln(x+\sqrt{x^2+a^2})|}{\sqrt{x^2+a^2}}\mathrm{d}x = 
    \int f|ln(x+\sqrt{x^2+a^2})|\mathrm{d}(\ln(x+\sqrt(x^2+a^2))) ; $
    \item $ \int \frac{f|ln(x+\sqrt{x^2-a^2})|}{\sqrt{x^2-a^2}}\mathrm{d}x = 
    \int f|ln(x+\sqrt{x^2-a^2})|\mathrm{d}(\ln(x+\sqrt(x^2-a^2))) ; $
    \item $ \int \frac{f'(x)}{f(x)}\mathrm{d}x = \ln|f(x)|+C\ (f(x)\neq0) $ .
\end{enumerate}

\begin{Field}[第二换元法]

    设 $ x=\varphi(t) $ 具有连续导数且单调,且 $ \varphi'(t)\neq 0 $ ,若
    $$
        \int f[\varphi(t)]\varphi'(t)\mathrm{d}t=G(t)+C
    $$ 
    则有$$
        \int f(x)\mathrm{d}x 
        \xlongequal{\textrm{令}x=\varphi(t)}
        \int f[\varphi(t)]\varphi'(t)\mathrm{d}t = G(t) + C = G[\varphi^{-1}(x)]+C
    $$ 
    其中 $ t=\varphi^{-1}(x) $ 为 $ x=\varphi(t) $ 的反函数。
\end{Field}

\begin{comment}
被积函数是 $ x $ 与 $ \sqrt[n]{ax+b} $ 或 $ x $ 与 $ \dis\sqrt[n]{\frac{ax+b}{cx+d}} $ 
    或由 $ e^x $ 构成的代数式的根式,则令 $ \sqrt[n]{g(x)}=t $,使得 $ x=\varphi(t) $ 中不再含有根式,
    此时做变量代换 $ x = \varphi(t) $ 即可;
    被积函数含有 $ \dis\sqrt{Ax^2+Bx+C}(A\not=0) $ 时,先根据$ A $ 的符号将其整理为
    $ \sqrt{A[(x-x_0)^2\pm l^2]} $ 或 $ \sqrt{-A[l^2-(x-x_0)^2]} $ ,然后做三角替换。   
\end{comment}

第二换元法多用于根式的被积函数,通过换元法去根式,具体而言,可以分为两类。
\begin{enumerate}
    \item 根号下一次方
    
    \begin{itemize}
        \item $ \dsqrt[n]{ax+b} = t; $ 
        \item $ \dsqrt[n]{\dfrac{ax+b}{cx+d}} = t; $ 
        \item 当式中同时存在 $ \dsqrt[n]{ax+b},\dsqrt[m]{ax+b} $ 时,取 $ t = \dsqrt[l]{ax+b}, $ 
        其中 $ l $ 为 $ m,n $ 的最小公倍数。
    \end{itemize}
    \item 根号下二次方
    
    \begin{itemize}
        \item 根式形如 $ \sqrt{a^2-x^2} $ 时,令 $ x=a\sin t, $ 此时 $ t\in[-\frac{\pi}{2},\frac{\pi}{2}]; $ 
        \item 根式形如 $ \sqrt{x^2-a^2} $ 时,令 $ x=a\sec t, $ 此时 $ t\in[0,\frac{\pi}{2})\cup(\frac{\pi}{2},\pi]; $ 
        \item 根式形如 $ \sqrt{a^2+x^2} $ 时,令 $ x=a\tan t, $ 此时 $ t\in (-\frac{\pi}{2},\frac{\pi}{2}). $ 
    \end{itemize}
\end{enumerate}

\begin{Field}[分部积分法]

    对 $ \int uv'\mathrm{d}x $ ,有$$
        \int uv'\mathrm{d}x = \int u\mathrm{d}v=uv-\int v\mathrm{d}u
    $$ 
\end{Field}

注意,\begin{itemize}
    \item 被积函数为两类函数乘积;
    \item 凑微分的优先性:指数 $ > $ 三角函数 $ > $ 幂函数。
\end{itemize}

注意,不定积分结果中一定包含常数 $ C, $ \textbf{即使在计算过程中积分号下内容相消干净。}

分部积分法可以利用列表法做。具体而言,为上导下积,对角相乘,正负相间。
以 $ \dis \int x^2e^x \mathrm{d}x $ 为例,则有表格

\begin{center}
    \begin{tabular}{c|ccc}
        $ x^2 $ & $ 2x $ & $ 2 $ & $ 0 $ \\\hline 
        $ e^x $ &$ e^x $ &$ e^x $ &$ e^x $ 
    \end{tabular}    
\end{center}
    
此时结果为  $ x^2 e^x - 2x e^x + 2 e^x + C. $ 

\subsection{特殊类型的不定积分法}

\subsubsection{通用方法}

\begin{itemize}
    \item 倒代换 - 分母次数高的,令$ x = \dfrac{1}{t}; $ 
    \item 整体代换 - 含有复杂函数的,令复杂部分为 $ t; $
\end{itemize}

\subsubsection{有理分式的不定积分}

\begin{Theo}[]

    任何实系数多项式在实数域内均可分解为一次因式和二次因式的乘积。
\end{Theo}

\begin{Theo}[]

    设 $ f(x)=\frac{P_m(x)}{Q_n(x)}$ 是有理真分式,若在实数域内分母 $ Q_n(x) $ 可因式分解为 $$
        Q_n(x)=(x-a)^\alpha(x^2+px+q)^\beta
    $$ 
    则有$$
        \frac{P_m(x)}{Q_n(x)}=\sum_{i=1}^\alpha \frac{A_i}{(x-a)^i}+
        \sum_{i=1}^\beta \frac{C_ix+D_i}{(x^2+px+q)^i}
    $$ 其中 $ A_i\ B_i\ C_i\ D_i $ 等系数都是实数。
\end{Theo}

具体而言,其步骤为
\begin{enumerate}
    \item 将假分式分为多项式和真分式;
    \item 将真分式分母整理至最简;
    \item 裂项,分别积分。
\end{enumerate}

求待定系数时,去分母,并注意\begin{itemize}
    \item 多项式相等条件:同次数的系数相等;
    \item 待定系数的个数为分母次数。
\end{itemize}

分母次数大于$ 4 $ 的时候,一般不用裂项法,而是寻求特殊方法。

待定系数时可以代入特殊值求解系数。

一个例子是,
\begin{equation*}
    \begin{aligned}
        \dfrac{x^4+2x^3+3x^2+4x+1}{(x-1)(x^2-1)(x^2+1)}=&\dfrac{x^4+2x^3+3x^2+4x+1}{(x+1)(x-1)^2(x^2+1)}\\ 
        =& \dfrac{A}{x+1} + \dfrac{B}{x-1} + \dfrac{C}{(x-1)^2} + \dfrac{Dx+E}{x^2+1}
    \end{aligned}
\end{equation*}

注意,
\begin{itemize}
    \item 分母有高次幂的,待定系数时次数从低到高都要有;
    \item 待定系数时分子次数比分母次数低一次。
\end{itemize}

\subsubsection{三角有理函数的不定积分}

三角有理函数指由 $ \sin x $ 和 $ \cos x $ 进行有限次四则运算得到的函数。

\begin{Field}[三角有理函数的不定积分一般方法]

    利用万能代换,即$\tan \dis\frac{x}{2} = t $,此时 $ \sin x = \dis\frac{2t}{t^2+1} $ ,
    $ \cos x = \dis\frac{t^2-1}{t^2+1} $ , $ x = 2\arctan t $ ,
    $ \dis\frac{\mathrm{d}x}{\mathrm{d}t} = \displaystyle\frac{2}{1+t^2} $ .
\end{Field}

注意,应用万能代换时, $ x $ 的次数不应当超过一次,形如$$
    \int \dis\frac{a_1\sin x + b_1 \cos x + c_1}{a_2\sin x + b_2 \cos x + c_2}
$$ 

\sssubsection{特殊三角有理函数的积分}

\begin{itemize}
    \item 对 $$
        I = \int \frac{a\sin x + b\cos x}{c\sin x+d\cos x}\mathrm{d}x
    $$ 
    可以使用待定系数法,令$$
        a\sin x+b\cos x = A(c\sin x+d\cos x)+B(c\sin x+ d\cos x)'
    $$ 
    其中$ A,B $ 是待定系数,将其解出并代入原式,得$$
        I=Ax+B\ln|c\sin x+d\cos x|+C
    $$ 
    \item \begin{itemize}
        \item 若$ R(-\sin x, \cos x)=-R(\sin x, \cos x) $ ,令 $ \cos x = t $ 并换元;
        \item 若$ R(\sin x, -\cos x)=-R(\sin x, \cos x) $ ,令 $ \sin x = t $ 并换元;
        \item 若$ R(-\sin x, -\cos x)=R(\sin x, \cos x) $ ,令 $ \tan x = t $ 并换元;
        \item 针对上一条,若$ R $ 形如$$
            \int \sin^{2m} x + \cos^{2n} x \mathrm{d}x
        $$ 其中 $ m,n\in N $ ,则应使用二倍角公式降幂。
    \end{itemize}
\end{itemize}


\section{定积分}

\subsection{定积分概念}

\begin{Def}[]

    将 $ [a,b] $ 任意地划分为 $ n $ 个小区间 $ a = x_1<x_2<\dots<x_{n-1}<x_n=b $ ,
    细度 $ d=\max_{1\leq i \leq n}(x_i-x_{i-1}) $ , $ \xi_i $ 为 $ [x_{i-1},x_i] $ 任意一点;
    则$ f(x) $在$ [a,b] $ 上的定积分为$$
        \int_a^b f(x)\mathrm{d}x = {\displaystyle\lim_{d\rightarrow 0}}
        \sum_{i=i}^nf(\xi_i)\Delta x_i
    $$ 
    若上述极限存在。
    若$ f(x) $在$ [a,b] $ 上确有定积分,则称 $ f(x) $在$ [a,b] $ 可积。
\end{Def}

\begin{Theo}[连续必可积]

    若$ f(x) $ 在闭区间 $ [a,b] $ 连续,则$ f(x) $在$ [a,b] $ 可积。
\end{Theo}

\begin{Theo}[]

    若$ f(x) $在$ [a,b] $ 有界,且在 $ [a,b] $ 中仅有有限个间断点,则$ f(x) $在$ [a,b] $ 可积。
\end{Theo}

\begin{Theo}[可积必有界]

    若$ f(x) $在$ [a,b] $ 可积,则其在 $ [a,b] $ 有界。
\end{Theo}

\begin{Theo}[定积分值的变量符号无关性]

    定积分的值与积分变量的符号无关,即$$
        \int_a^b f(x)\mathrm{d}x=\int_a^b f(t)\mathrm{d}t
    $$ 
\end{Theo}

\sssubsection{利用定积分定义求极限}

\begin{equation*}
    \begin{aligned}
        \int_a^b f(x)\mathrm{d}x&={\displaystyle\lim_{d\rightarrow 0}}\sum_{i=i}^nf(\xi_i)\Delta x_i\\
        &={\displaystyle\lim_{n\rightarrow \infty}}\sum_{i=1}^n f(a+\frac{i}{n}(b-a))\frac{b-a}{n}
    \end{aligned}
\end{equation*}

具体而言,其步骤为
\begin{itemize}
    \item 令 $ \Delta x_i $ 为 $ \frac{1}{n} $ 或 $ \frac{k}{n} ; $
    \item 取 $ \xi_i $ 的位置,如\begin{itemize}
        \item $ \xi_i $ 取右端点: $ a+\frac{i}{n}(b-a) ; $
        \item $ \xi_i $ 取左端点: $ a+\frac{i-1}{n}(b-a) ; $
        \item $ \xi_i $ 取中点: $ a+\frac{2i-1}{2n}(b-a) ; $
        \item $ \xi_i $ 取三等分点: $ a+\frac{3i-2}{3n}(b-a) ; $
        \item $ \xi_i $ 取三分之二等分点: $ a+\frac{3i-1}{3n}(b-a) $ 等等。
    \end{itemize}
    \item 求 $ a={\displaystyle\lim_{n\rightarrow \infty}}\xi_1,b={\displaystyle\lim_{n\rightarrow \infty}}\xi_n $
    \item 将求得的 $ a,b $ 用于验证 $ \Delta x_i $ 的取法是否正确,
    即是否有$ \sum_{i=1}^n \Delta x_i = b-a $ 。
\end{itemize}

特别地,有
$$
    {\displaystyle\lim_{n\rightarrow \infty}}\sum_{i=1}^n f(\dfrac{i}{n})\dfrac 1n =\int_0^1f(x)\mathrm{d}x
$$ 
此即所谓“零到一,$n $ 等分,取端点”的极限。

\sssubsection{利用二重积分定义求极限}

若 $ f(x,y) $ 在 $ [a,b]\times [c,d] $ 上连续,则有$$
    {\displaystyle\lim_{n\rightarrow \infty}}\sum_{i=1}^n\sum_{j=1}^n f(\xi_i,\eta_j)\dfrac{b-a}{n}
    \dfrac{d-c}{n} = \int_a^b\mathrm{d}x\int_c^df(x,y)\mathrm{d}y
$$ 
其中$$
    \xi_i\in [a+\dfrac{i-1}{n}(b-a),a+\dfrac{i}{n}(b-a)],
    \xi_j\in [c+\dfrac{j-1}{n}(d-c),a+\dfrac{j}{n}(d-c)],
$$ 
其具体的步骤与一维情况类似。

特别地,有
$$
    {\displaystyle\lim_{n\rightarrow \infty}}\sum_{i=1}^n\sum_{j=1}^n f(\dfrac{i}{n},\dfrac{j}{n})
    \dfrac{1}{n^2} = \int_0^1\mathrm{d}x\int_0^1f(x,y)\mathrm{d}y
$$ 


\sssubsection{$ n $ 项和求法}

$ n $ 项和的求和有如下的方式。
\begin{itemize}
    \item 直接求和;
    \item 夹逼定理;
    \item 定积分定义;
    \item 数项级数。
\end{itemize}

\begin{Field}[定积分的几何意义]

    定积分的几何意义为面积的代数和。
\end{Field}

\subsection{定积分的性质}

\begin{itemize}
    \item 线性性 - $\dis \int_a^b [k_1 f_1(x)]+k_2[f_2(x)]\mathrm{d}x = 
    k_1\int_a^b f_1(x)\mathrm{d}x + k_2\int_a^b f_2(x)\mathrm{d}x ; $
    \item 区间可加性 - $\dis \int_a^b f(x)\mathrm{d}x = \int_a^c f(x)\mathrm{d}x + \int_c^b f(x)\mathrm{d}x, $
    即使 $ c\not\in [a,b] ; $
    \item 比较定理 - 设 $ a\leq b, f(x)\leq g(x), $ 则 $\dis \int_a^b f(x)\mathrm{d}x \leq \int_a^b g(x)\mathrm{d}x ; $
    
    推论 - $\dis a<b \Rightarrow |\int_a^b f(x) \mathrm{d}x | < \int_a^b|f(x)|\mathrm{d}x ; $
    \item 估值定理 - $\dis a < b, m\leq f(x)\leq M \Rightarrow m(b-a)\leq\int_a^b f(x)\mathrm{d}x \leq M(b-a); $
    \item 积分中值定理 - 设 $ f(x) $在$ [a,b] $ 上连续,则$\exists \xi \in [a,b]  $ 
    使得 $\dis \int_a^b f(x)\mathrm{d}x = f(\xi)(b-a) ; $

    加强形式 - 设 $ f(x) $在$ [a,b] $ 上连续,则$\exists \xi \in (a,b)  $ 
    使得 $\dis \int_a^b f(x)\mathrm{d}x = f(\xi)(b-a) ; $

    加强形式 - 设 $ f(x),g(x) $在$ [a,b] $ 上连续,$ g(x) $ 不变号,则$\exists \xi \in (a,b)  $ 
    使得 $\dis \int_a^b f(x)g(x)\mathrm{d}x = f(\xi)\int_a^bg(x)\mathrm{d}x ; $
\end{itemize}

注意,\begin{itemize}
    \item 定积分值与被积函数在有限个点上的值无关;
    \item 定积分中值定理可证方程有根。
\end{itemize}

\subsection{重要定理、公式、关系}

\begin{Def}[变上限函数]

    设 $ f(x) $在$ [a,b] $ 可积,则函数形如 $ F(x)=\int_a^x f(t)\mathrm{d}t, x\in [a,b]  $ 
    称为变上限积分函数。
\end{Def}

\begin{Theo}[变上限积分求导定理]

    若$ f(x) $在$ [a,b] $ 可积,则$ F(x) $在$ [a,b] $ 连续。

    若$ f(x) $在$ [a,b] $ 连续,则$ F(x) $在$ [a,b] $ 上可导,且 $ F'(x) = f(x) $。
\end{Theo}

注意,\begin{itemize}
    \item 设 $$
        F(x)=\int_{\varphi_1(x)}^{\varphi_2(x)} f(t)\mathrm{d}t
    $$ 且 $ \varphi_1(x),\varphi_2(x) $ 可导,$ f(x) $ 连续,则有$$
        F'(x) = f(\varphi_2(x))\varphi_2'(x)-f(\varphi_1(x))\varphi_1'(x)
    $$ 

    变限积分求导,被积函数不含 $ x. $ 
    \item 若$ x_0 $ 为 $ f(x) $在$ [a,b] $ 上的一个跳跃间断点,则$ F(x) $在$ x_0 $ 连续,
    但不可导;
    \item 若$ x_0 $ 为 $ f(x) $在$ [a,b] $ 上的一个可去间断点,则$ F(x) $在$ x_0 $ 处可导,
    但 $ F(x) $ 不是$ f(x) $ 的原函数。
\end{itemize}

\begin{Theo}[牛顿——莱布尼茨公式]

    设$ f(x) $在$ [a,b] $可积,$ F(x) $为$ f(x) $在$ [a,b] $上的任意原函数,
    则有$$
        \int_a^b f(x)\mathrm{d}x=F(x)\Big|_a^b=F(b)-F(a)
    $$ 
\end{Theo}

\subsection{定积分求法}

\sssubsection{利用牛顿莱布尼茨公式}

$$
   \int_a^b f(x)\mathrm{d}x=F(x)\Big|_a^b=F(b)-F(a)
$$ 

\sssubsection{分部积分法}

$$
    \int_a^b u'v \mathrm{d}x = \int_a^b u\mathrm{d}v = uv\Big|_a^b - \int_a^b v\mathrm{d}u
$$ 

其利用技巧与不定积分的分部积分法完全一致。

\sssubsection{换元积分法}

设 $ f(x) $ 在 $ [a,b] $ 上连续,若变量替换 $ x=\varphi(t) $ 满足
\begin{enumerate}
    \item $ \varphi'(t) $ 在 $ [\alpha,\beta] $ (或者$ [\beta,\alpha] $ ) 上连续;
    \item $ \varphi(\alpha) = a , \varphi(\beta) = b $ ,且当 $ \alpha < t < \beta $ 时,
    有 $ a \leq \varphi(t) < b, $ 
\end{enumerate}
则有$$
    \int_a^b f(x)\mathrm{d}x = \int_\alpha^\beta f[\varphi(t)]\varphi'(t)\mathrm{d}t
$$ 

\subsection{常用的定积分公式}

\sssubsection{对称区间奇偶函数的积分公式}

\begin{enumerate}
    \item 设 $ f(x) $ 是在区间 $ [-a,a],(a>0) $ 上连续的偶函数,则$$
        \int_{-a}^a f(x)\mathrm{d}x = 2 \int_0^a f(x)\mathrm{d}x
    $$ 
    \item 设 $ f(x) $ 是在区间 $ [-a,a],(a>0) $ 上连续的奇函数,则$$
        \int_{-a}^af(x)\mathrm{d}x = 0
    $$ 
\end{enumerate}

可用于积分等式的方法:
\begin{enumerate}
    \item 含有中值:积分中值定理;
    \item 不含中值:\begin{itemize}
        \item 积分区间分割;
        \item 换元法
    \end{itemize}
\end{enumerate}

\sssubsection{周期函数的积分公式}

设 $ f(x) $ 在 $ (-\infty,+\infty) $ 内是以 $ T $ 为周期的周期函数,则对任意常数 $ a $ 、任意自然数 $ n $ 
都有\begin{enumerate}
    \item $ \dis \int_a^{a+T}f(x)\mathrm{d}x = \int_0^T f(x)\mathrm{d}x ; $
    \item $ \dis \int_a^{a+nT}f(x)\mathrm{d}x = n\int_0^T f(x)\mathrm{d}x ; $
\end{enumerate}

\sssubsection{三角函数的积分公式}

\begin{Theo}[Wallis公式 - 分部积分推导]

    $$
        \int_0^\frac{\pi}{2} \sin^{2n} x \mathrm{d} x = \int_0^\frac{\pi}{2} \cos^{2n} x \mathrm{d} x = 
        \dfrac{(2n-1)!!}{(2n)!!}\cdot \dfrac{\pi}{2}
    $$ 
    $$
        \int_0^\frac{\pi}{2} \sin^{2n+1} x \mathrm{d} x = \int_0^\frac{\pi}{2} \cos^{2n+1} x \mathrm{d} x = 
        \dfrac{(2n)!!}{(2n+1)!!}
    $$ 
    $$
        \int_0^\frac{\pi}{2} \sin^{n} x \mathrm{d} x = \int_0^\frac{\pi}{2} \cos^{n} x \mathrm{d} x = 
        \dfrac{(n-1)!!}{(n)!!}\cdot (\dfrac{\pi}{2})^\texttt{(int)(!(n\%2))}
    $$ 
\end{Theo}

注意,当被积函数中有自然数 $ n $ 时,通过分部积分法构造递推公式,并用递推公式推出结果。

三角函数的公式还有如下几个。设 $ f(x) $ 在 $ [0,1] $ 连续,则
\begin{itemize}
    \item $ \dis \int_0^\frac{\pi}{2} f(\sin x)\mathrm{d}x=\int_0^\frac{\pi}{2} f(\cos x)\mathrm{d}x $ 
    \item $\dis \int_0^\pi xf(\sin x)\mathrm{d}x = \dfrac{\pi}2\int_0^\pi f(\sin x)\mathrm{d}x = 
    \pi\int_0^\frac{\pi}{2} f(\sin x)\mathrm{d}x $ 
    \item $\dis \int_0^{\pi} f(\sin x)\mathrm{d}x = 2\int_0^\frac{\pi}{2} f(\sin x)\mathrm{d}x $ 
\end{itemize}

其中,$ ii) $ 利用了区间再现公式,具体而言,令 $ t = a + b - x $ 即可。

难以计算原函数的积分称为变态积分。对变态积分的计算,可以
\begin{itemize}
    \item 构造递推公式;
    \item 积分区间再现。
\end{itemize}

\section{反常积分}

定积分存在两种限制,由此可以引申出两种反常积分。
对 $ \dis \int_a^b f(x)\mathrm{d}x, $ 
\begin{itemize}
    \item $ [a,b] $ 必须是有限区间 $ \Rightarrow $ 无穷区间上的反常积分;
    \item $ f(x) $ 必须有界 $ \Rightarrow $ 无界函数的反常积分。
\end{itemize}

反常积分时定积分的极限。

\subsection{无穷区间上的反常积分}

$ \dis \int_a^{+\infty}f(x)\mathrm{d}x = {\displaystyle\lim_{t\rightarrow +\infty}}
\int_a^t f(x)\mathrm{d}x $ 称为无穷区间上的反常积分,
若其存在,则称该反常积分收敛,否则发散。\vspace{3pt}

对 $ \dis \int_{-\infty}^b f(x)\mathrm{d}x = {\displaystyle\lim_{t\rightarrow -\infty}}
\int_t^b f(x)\mathrm{d}x $ 同理。\vspace{3pt}

对 $ \dis \int_{-\infty}^{+\infty}f(x)\mathrm{d}x $,将其拆成以上两种情况的反常积分,若二者都收敛,
则称其收敛,否则称其发散。

注意,
\begin{enumerate}
    \item 计算方法与积分类似,如 $\dis \int_a^{+\infty} = F(x)\Big|^{+\infty}_a = 
    {\displaystyle\lim_{t\rightarrow +\infty}}F(t) - F(a); $ 
    \item 反常积分收敛时,适用对称奇偶性,否则不适用;
    \item $ p $ 积分:对 $ \dis \int_a^{+\infty}\dfrac{1}{x^p}\mathrm{d}x,\ a>0 $ ,
    当 $ p > 1 $ 时积分收敛,$ p\leq 1 $ 时发散;
    \item 利用分部积分法时,可以先进行不定积分,然后再引入上下限。
\end{enumerate}

\begin{Theo}[比较判别法]

    设 $ f $ 与 $ g $ 在 $ [a,+\infty) $ 上非负且在任意区间 $ [a,b],b<\infty $ 可积,且 $ f \leq g $ ,
    $ x\in[a,+\infty) $ ,则\begin{enumerate}
        \item 若 $ \int_a^{+\infty}g(x)\mathrm{d}x $ 收敛,则$ \int_a^{+\infty}f(x)\mathrm{d}x $收敛;
        \item 若 $ \int_a^{+\infty}f(x)\mathrm{d}x $ 发散,则$ \int_a^{+\infty}g(x)\mathrm{d}x $发散。
    \end{enumerate}
\end{Theo}

\begin{Theo}[比较判别法]

    设 $ f $ 与 $ g $ 在 $ [a,+\infty) $ 上非负,则
    \begin{enumerate}
        \item 若 $ g(x) > 0 $ 且 $ \dis {\displaystyle\lim_{x\rightarrow +\infty}}
        \dfrac{f(x)}{g(x)} = k \neq 0 $ ,则 $ \int_a^{+\infty}f(x)\mathrm{d}x $ 与
        $ \int_a^{+\infty}g(x)\mathrm{d}x $ 的敛散性相同;
        \item 若当 $ x\rightarrow+\infty $时,$ f\sim g $,则 $ \int_a^{+\infty}f(x)\mathrm{d}x $ 与
        $ \int_a^{+\infty}g(x)\mathrm{d}x $ 的敛散性相同。
    \end{enumerate}
\end{Theo}

\begin{Theo}[绝对收敛必收敛]

    若 $ f(x) \in C[a,\infty) $ 且 $ \dis \int_a^{+\infty}|f(x)|\mathrm{d}x $ 收敛,则
    $ \dis \int_a^{+\infty}f(x)\mathrm{d}x $ 收敛。
\end{Theo}

\subsection{无界函数的反常积分}

设 $ {\displaystyle\lim_{x\rightarrow b^-}}f(x) = \infty $,则称 $ b $ 为瑕点,
$ \dis \int_a^b f(x)\mathrm{d}x = {\displaystyle\lim_{t\rightarrow b^-}}
\int_a^t f(x)\mathrm{d}x $ 为无界函数的反常积分,若其存在,称其收敛,否则其发散。

瑕点在区间左端点或区间内时类似。瑕点在区间内而拆成两个积分时,二积分全部收敛则收敛,
否则发散。

注意,
\begin{enumerate}
    \item 计算方法与积分类似,如 $\dis \int_a^{b} = F(x)\Big|^{b}_a = 
    {\displaystyle\lim_{t\rightarrow b^-}}F(t) - F(a) $ ;
    \item 反常积分收敛时,适用对称奇偶性,否则不适用;
    \item $ p $ 积分:对 $ \dis \int_0^{a}\dfrac{1}{x^p}\mathrm{d}x,\ a>0 $ ,
    当 $ p < 1 $ 时积分收敛,$ p\geq 1 $ 时发散;
    \item 利用分部积分法时,可以先进行不定积分,然后再引入上下限。
\end{enumerate}

\begin{Theo}[比较判别法]

    设 $ {\displaystyle\lim_{x\rightarrow b^-}}f(x) = \infty,{\displaystyle\lim_{x\rightarrow b^-}}
    g(x) = \infty $ ,$ f,g $ 在 $ [a,b) $ 非负且对任意 $ \xi < b $ 都有在 $ [a,\xi] $ 上可积,$ f\leq g $ ,
    则\begin{enumerate}
        \item 若 $ \int_a^{b}g(x)\mathrm{d}x $ 收敛,则$ \int_a^{b}f(x)\mathrm{d}x $收敛;
        \item 若 $ \int_a^{b}f(x)\mathrm{d}x $ 发散,则$ \int_a^{b}g(x)\mathrm{d}x $发散。
    \end{enumerate}
\end{Theo}

\begin{Theo}[比较判别法]

    设 $ {\displaystyle\lim_{x\rightarrow b^-}}f(x) = \infty,{\displaystyle\lim_{x\rightarrow b^-}}
    g(x) = \infty $ ,$ f,g $ 在 $ [a,b) $ 非负
    \begin{enumerate}
        \item 若 $ g(x) > 0, x\in[a,b) $ 且 $ \dis {\displaystyle\lim_{x\rightarrow b^-}}
        \dfrac{f(x)}{g(x)} = k \neq 0 $ ,则 $ \int_a^{b}f(x)\mathrm{d}x $ 与
        $ \int_a^{b}g(x)\mathrm{d}x $ 的敛散性相同;
        \item 若当 $ x\rightarrow b^- $时,$ f\sim g $,则 $ \int_a^{b}f(x)\mathrm{d}x $ 与
        $ \int_a^{b}g(x)\mathrm{d}x $ 的敛散性相同。
    \end{enumerate}
\end{Theo}

\begin{Theo}[绝对收敛必收敛]

    若 $ {\displaystyle\lim_{x\rightarrow b^-}}f(x)=\infty $ 且 $ \dis \int_a^{b}|f(x)|\mathrm{d}x $ 收敛,则
    $ \dis \int_a^{b}f(x)\mathrm{d}x $ 收敛。
\end{Theo}

\section{定积分的几何应用}

主要应用于求面积、体积、弧长、侧面积。

\subsection{平面图形求面积}

\sssubsection{直角坐标系下平面图形求面积}

对于由 $ x = x_1, x = x_2,y = f_1(x), y = f_2(x) (x_1 < x_2,\forall x \in [x_1,x_2]f_1(x) > f_2(x)) $ 
围成的图形,有 $$
    S = \int_a^b[f_1(x)-f_2(x)]\mathrm{d}x
$$ 

对于由 $ y = y_1, y = y_2,x = f_1(y), x = f_2(y) (y_1 < y_2,\forall y \in [y_1,y_2]f_1(y) > f_2(y)) $ 
围成的图形,有 $$
    S = \int_a^b[f_1(y)-f_2(y)]\mathrm{d}y
$$ 

\sssubsection{极坐标下的平面图形求面积}

$$
    S = \int_{\theta_1}^{\theta_2}\textcolor{red}{\dfrac{1}{2}r^2}\mathrm{d}\theta
$$ 

\sssubsection{由参数方程表示的平面图形求面积}

对由 $ \begin{cases}
    x = x(t) \\ y = y(t)
\end{cases} (\alpha\leq t\leq \beta) $ 表示的平面图形,其面积为$\dis\int_\alpha^\beta |y(t)x'(t)|\mathrm{d}t. $

\sssubsection{重要的平面曲线}

重要的平面曲线如下。
\begin{itemize}
    \item 心型线 $ r = a(1+\cos\theta) ; $
    \item 摆线(旋轮线);
    \item 星型线;
    \item 双扭线;
    \item 阿基米德螺线 $ r = a\theta ; $
    \item 对数螺线;
    \item 三叶玫瑰线 $ r = \sin(3\theta),(0\leq\theta\leq \dfrac{\pi}{3}) ; $
\end{itemize}

\subsection{旋转体求体积}

\sssubsection{薄片法}

绕 $ x $ 轴旋转时,有 $ \mathrm{d}V_x = \pi y^2 \mathrm{d}x $ ,则体积为$$
    V = \int_a^b \pi f(x)^2 \mathrm{d}x
$$ 

\sssubsection{柱壳法}

绕 $ y $ 轴旋转时,有 $ \mathrm{d}V_y = 2\pi xy \mathrm{d} x $ ,则体积为$$
    V = \int_a^b 2\pi \left|xf(x)\right|\mathrm{d}x
$$ 

\subsection{平均值}

设 $ f(x) $ 在区间 $ [a,b] $ 连续,则 $ f(x) $ 在 $ [a,b] $ 的平均值为
$$
    \dfrac{\dis \int_a^b f(x)\mathrm{d}x}{b-a}
$$ 

\subsection{平行截面面积已知的立体体积}

$ \dis V = \int_a^bA(x)\mathrm{d}x,a<b $ ,其中 $ A(x) $ 是待求体积立体截面面积。

\subsection{平面曲线段的弧长}

\sssubsection{直角坐标}

对曲线段 $ y = f(x),x\in[a,b] $ ,设 $ f(x) $ 有连续导数,则给定平面曲线段弧长元素
和弧长分别为
$$ \dis \mathrm{d}s = \dsqrt{1+f'^2(x)}\mathrm{d}x;
s = \int_a^b \dsqrt{1+f'^2(x)}\mathrm{d}x. $$

\sssubsection{参数方程}

若曲线能表示为 $ x = x(t),y=y(t),t\in[\alpha,\beta] $,且其在$ (\alpha,\beta) $ 内
有连续导数,则给定平面曲线段弧长元素和弧长分别为

$$ \dis \mathrm{d}s = \dsqrt[]{y'^2(t)+x'^2(t)}\mathrm{d}t;
s = \int_a^b \dsqrt[]{[y^2(t)]^2+[x'(t)]^2}\mathrm{d}t. $$

\sssubsection{极坐标}


若曲线能表示为 $ \rho = \rho(\theta),\theta\in[\theta_1,\theta_2], $
则给定平面曲线段弧长元素和弧长分别为

$$ \dis \mathrm{d}s = \dsqrt[]{\rho^2(\theta)+\rho'^2(\theta)}\mathrm{d}\theta ;
s = \int_a^b \dsqrt[]{\rho^2(\theta)+[\rho'(\theta)]^2}\mathrm{d}\theta. $$

\subsection{旋转曲面求侧面积}

曲线$ y = f(x),x\in(a,b) $ 弧绕 $ x $ 轴旋转所得曲面表面积为
$ \dis S = 2\pi\int_a^b|f(x)|\overbrace{\dsqrt[]{1+[f'(x)^2]}\mathrm{d}x}^{\mathrm{d}s} $ 

曲线$ y = f(x),x\in(a > 0,b) $ 弧绕 $ y $ 轴旋转所得曲面表面积为
$ \dis S = 2\pi\int_a^bx\cdot\overbrace{\dsqrt[]{1+[f'(x)^2]}\mathrm{d}x}^{\mathrm{d}s} $ 

当旋转曲面使用极坐标或参数方程表示时,将 $ \mathrm{d}s $ 改为相对应的即可。

\section{定积分的物理应用}
定积分的物理应用主要有以下几种。

\begin{itemize}
    \item 做功 $ W = FS; $ 
    \item 受力 \begin{itemize}
        \item 万有引力 $ F = \dfrac{GMm}{r^2}; $ 
        \item 液体压力 $ F = PS = \rho ghS; $ 
    \end{itemize}
\end{itemize}


\chapter{多元函数微分学}

\section{多元函数微分法}

\subsection{多元函数}

\begin{Def}[二元函数极限]

    设 $ f(x,y) $ 在点 $ (x_0,y_0) $ 某去心邻域有定义,若 $ \forall \varepsilon > 0, \exists \delta > 0 $ 
    使得只要 $ 0 < \dsqrt{}{(x-x_0)^2+(y-y_0)^2}<\delta $ 就有 $ |f(x,y)-A|<\varepsilon $ ,则称
    当 $ (x,y) $ 趋近于 $ (x_0,y_0) $ 时,$ f(x,y) $ 的极限存在,值为 $ A $ ,记为
    $ {\displaystyle\lim_{\begin{subarray}{c}
        x\rightarrow x_0\\y\rightarrow y_0
    \end{subarray}}}f(x,y) = A $ 或者 $ {\displaystyle\lim_{(x,y)\rightarrow (x_0,y_0)}}f(x,y) = A $ ;
    否则,称为极限不存在。
\end{Def}

注意,\begin{enumerate}
    \item 求二重极限可用的方法有
    \begin{itemize}
        \item 极限的四则运算;
        \item 夹逼定理;
        \item 等价代换;
        \item 重要极限;
        \item 无穷小与有界量积仍为无穷小;
        \item 连续性;
        \item 极坐标变换。
    \end{itemize}
    \item 二重积分存在充要条件$$
        {\displaystyle\lim_{(x,y)\rightarrow (x_0,y_0)}}f(x,y) = A \Leftrightarrow
        \textrm{当} (x,y) \textrm{以任意形式趋近于} (x_0,y_0) \textrm{时都有} f(x,y)\rightarrow A
    $$ 
    其逆否命题常用。
\end{enumerate}

\begin{Def}[二元函数连续]

    若 $ {\displaystyle\lim_{(x,y)\rightarrow (x_0,y_0)}}f(x,y) = f(x_0,y_0) $ 
    或者$ {\displaystyle\lim_{(\Delta x,\Delta y)\rightarrow (0,0)}}
    f(x+\Delta x,y + \Delta y) - f(x,y) = 0 $ ,
    则称 $ f(x,y) $ 在 $ (x_0,y_0) $ 连续。
    若 $ f(x,y) $ 在区域 $ D $ 中每一点均连续,则称 $ f(x,y) $ 在 $ D $ 内连续。
\end{Def}

\subsection{偏导数}

\begin{Def}[偏导数]

    设二元函数 $ z = f(x,y) $ 在 $ (x_0,y_0) $ 某邻域内有定义,令 $ y = y_0 $ ,
    给 $ x $ 以增量 $ \Delta x $ ,\nextline 若 $ {\displaystyle\lim_{\Delta x\rightarrow 0}}
    \dfrac{(x_0+\Delta x,y_0)-f(x_0,y_0)}{\Delta x} $ 存在,则称其为 $ z = f(x,y) $ 
    在 $ (x_0,y_0) $ 处关于 $ x $ 的偏导数,\nextline 记为 $ f_x'(x_0,y_0) $ 
    或者 $ \dfrac{\partial z}{\partial x}\Big|_{(x_0,y_0)} $ 
    或者 $ z'_x\Big|_{(x_0,y_0)} $ 。此外,$$
        f_x'(x_0,y_0) = {\displaystyle\lim_{x\rightarrow x_0}}
        \dfrac{f(x,y_0)-f(x_0,y_0)}{x-x_0}
    $$ 
\end{Def}

$ y $ 的偏导数可以类似地定义。

注意,两偏导数在 $ (x_0,y_0) $ 处存在 $ \not\Leftrightarrow f(x,y) $ 在 $ (x_0,y_0) $ 连续。

\begin{Def}[高阶偏导数]

    设对 $ z = (x,y) $ 有 $ f_x' $ 和 $ f_y' $ ,则有
    \begin{itemize}
        \item $ \dfrac{\partial }{\partial x}\left(\dfrac{\partial z}{\partial x}\right) 
        = \dfrac{\partial^2 z}{\partial x^2} = f^\pprime_{xx} $ ;
        \item $ \dfrac{\partial }{\partial y}\left(\dfrac{\partial z}{\partial y}\right) 
        = \dfrac{\partial^2 z}{\partial y^2} = f^\pprime_{yy} $ ;
        \item $ \dfrac{\partial }{\partial y}\left(\dfrac{\partial z}{\partial x}\right) 
        = \dfrac{\partial^2 z}{\partial x\partial y} = f^\pprime_{xy} $ ;
        \item $ \dfrac{\partial }{\partial x}\left(\dfrac{\partial z}{\partial y}\right) 
        = \dfrac{\partial^2 z}{\partial y\partial x} = f^\pprime_{yx} $ ;
    \end{itemize}
    其中后二者称为混合二阶偏导数。
\end{Def}

\begin{Theo}[二阶混合偏导数相等的充分条件]

    若对 $ z = f(x,y) $ 有 $ f^\pprime_{xy} $ 与 $ f^{\prime\prime}_{yx} $ 于 $ (x_0,y_0) $ 连续,
    则有 $ f^\pprime_{xy}(x_0,y_0) = f^\pprime_{yx}(x_0,y_0) $ 。
\end{Theo}

\subsection{全微分}

\begin{Def}[全微分]

    设 $ z = f(x,y) $ 在 $ (x_0,y_0) $ 处有全增量 
    $ \Delta z = f(x_0+\Delta x,y_0 + \Delta y) - f(x_0,y_0) $ ,
    若 $ \Delta z = A\Delta x + B\Delta y + o(\rho),\rho = \dsqrt{}{(\Delta x)^2+(\Delta y)^2}\rightarrow0 $ 
    其中 $ A,B $ 不依赖于 $ \Delta x,\Delta y $ ,仅与 $ x_0,y_0 $ 有关,则称 $ z = f(x,y) $ 
    在 $ (x_0,y_0) $ 处可微,而 $ A\Delta x+B\Delta y $ 称为 $ z = f(x,y) $ 在 $ (x_0,y_0) $ 处的
    全微分,记为 $ \mathrm{d}z\big|_{(x_0,y_0)} $ 或 $ \mathrm{d}f\big|_{(x_0,y_0)} $
\end{Def}


注意,
\begin{equation*}
    \begin{array}{c}
        z = f(x,y) \textrm{在} (x_0,y_0) \textrm{可微} \\
        \Updownarrow \\
        {\displaystyle\lim_{(x,y)\rightarrow (x_0,y_0)}} 
        \dfrac{\overbrace{f(x,y)-f(x_0,y_0)}^{\dis \Delta z} - 
        \overbrace{f'_x(x_0,y_0)(x-x_0)+f'_y(x_0,y_0)(y-y_0)}^{\dis \mathrm{d}z}}{\dis \sqrt{(x-x_0)^2+(y-y_0)^2}}
    \end{array}
\end{equation*}

\begin{Theo}[可微的必要条件]

    若 $ f(x,y) $ 在 $ (x_0,y_0) $ 可微,则\begin{itemize}
        \item $ f(x,y) $ 在 $ (x_0,y_0) $ 连续;
        \item $ f'_x(x,y),f'_y(x,y) $ 在 $ (x_0,y_0) $ 处均存在。
    \end{itemize}
\end{Theo}

\begin{Theo}[可微的充分条件]

    若 $ f'_x(x,y) $ 和 $ f'_y(x_0,y_0) $ 在 $ (x_0,y_0) $ 均连续,则
    $ f(x,y) $ 在 $ (x_0,y_0) $ 可微。
\end{Theo}

\sssubsection{多元函数几个概念的联系}

% Please add the following required packages to your document preamble:
% \usepackage{booktabs}
\begin{table}[!htbp]\centering
    \begin{tabular}{@{}lcccc@{}}
        &  &  &  & \begin{tabular}[c]{@{}c@{}}$ f(x,y) $ 在 \\ $ (x_0,y_0) $ 连续\end{tabular} \\
        &  &  & $\nearrow$ &  \\
       \begin{tabular}[c]{@{}l@{}}$ f'_x(x,y) $ 和 $ f'_y(x_0,y_0) $ \\ 在 $ (x_0,y_0) $ 均连续\end{tabular} & $\longrightarrow$ & \begin{tabular}[c]{@{}c@{}}$ f(x,y) $ 在 \\ $ (x_0,y_0) $ 可微\end{tabular} &  &  \\
        & \multicolumn{1}{l}{} & \multicolumn{1}{l}{} & \multicolumn{1}{l}{$\searrow$} & \multicolumn{1}{l}{} \\
        &  &  &  & \begin{tabular}[c]{@{}c@{}}$ f'_x(x,y),f'_y(x,y) $ 在\\  $ (x_0,y_0) $ 处均存在\end{tabular}
    \end{tabular}
\end{table}
\chapter{矩阵的特征值与特征向量}

\begin{equation*}
    \begin{aligned}
        \begin{cases}
        \textrm{求}\lambda_A\textrm{与}\alpha_A\,\begin{cases}
            \textrm{抽象}\\\textrm{具体}\\\textrm{性质}
        \end{cases}\\
        \textrm{相似}\,\begin{cases}
            A\sim B\\ A\sim \Lambda(\textrm{判定与求解})\\ 
            \textrm{求} A^n,A,\textrm{相似矩阵}B
        \end{cases}\\
        \textrm{对角化}\,\begin{cases}
            \textrm{方阵}A_n(\textrm{重根})\\ 
            \textrm{对称矩阵}A(\textrm{结论})
        \end{cases}
    \end{cases}
    \end{aligned}
\end{equation*}

\Section{矩阵的特征值与特征向量}

\sssubsection{可能的情况}

对抽象矩阵 $ A, $ \begin{itemize}
    \item 凑 $ A\alpha = \lambda \alpha; $ 
    \item $ A + kE $ 不可逆(求 $ \lambda_A $ );
\end{itemize}

对具体矩阵 $ A, $ \begin{itemize}
    \item $ |\lambda E - A| = 0 \Rightarrow \lambda_A; \forall \lambda_0, (\lambda_0 E - A)X = 0\Rightarrow \alpha; $ 
    \item $ A = B + kE, r(B) = 1; $ 
\end{itemize}

\sssubsection{列表法}

可以列表表示对 $ A $ 做变换时,特征值特征向量的变化。

\begin{table}[!htbp]\centering
    \begin{tabular}{ccc}
    \toprule
    $A        $&$ \lambda      $& $\alpha                  $ \\ \midrule
    $A^k      $&$ \lambda^k    $& $\alpha                  $ \\
    $A^m+kE   $&$ \lambda^m+kE $& $\alpha                  $ \\
    $A^{-1}   $&$ 1 /\lambda   $& $\alpha                  $ \\
    $A^*      $&$ |A|/\lambda  $& $\alpha                  $ \\
    $A^\top   $&$ \lambda      $&  无法断定                   \\
    $P^{-1}AP $&$ \lambda      $& $\Attention{P^{-1}\alpha}$ \\ \bottomrule
    \end{tabular}
\end{table}

注意,可以手动将待求矩阵拆为便于运算的形式,如 $ A = B + kE $ 等。

\sssubsection{秩为一的矩阵}

对秩为一的矩阵,$ r(A) = 1\Leftrightarrow A = \alpha\beta^\top; $ 
\begin{itemize}
    \item $ \lambda_1 = tr(A) = \alpha^\top\beta = \beta^\top\alpha, \lambda_2 = \cdots,\lambda_n = 0; $ 
    \item \begin{itemize}
        \item 若 $ tr(A)\neq 0, $ 对 $ \lambda_1 = tr(A), $ 
        $ A = \alpha\beta^\top \Rightarrow A\alpha = \alpha\beta^\top\alpha = tr(A)\alpha, $ 
        故 $ \lambda_1 = tr(A) $ 对应的无关特征向量为 $ \alpha. $ 

        对其余的特征值 $ \lambda_i = 0, $ 即解 $ AX = 0 \Rightarrow \alpha\beta^\top X = 0, $ 发现
        其与 $ \beta^\top X = 0 $ 同解。由于这里有 $ n - 1 $ 个无关特征向量,
        加上$ \lambda_1 $ 的一个后共计 $ n $ 个无关特征向量,注意到矩阵 $ A $ 可以相似对角化。

        \item 若 $ tr(A) = 0, $ 则所有特征值都为 $ 0. $ 此时仍求解 $ AX = 0, $ 
        仍能解得 $ n - 1 $ 个无关特征向量;因为不够$ n $ 个,矩阵 $ A $ 无法相似对角化。
    \end{itemize}
    故 $ A $ 能否相似对角化取决于其迹是否为零。
\end{itemize}

\begin{itemize}
    \item[\textbf{例题}] 设矩阵 $ A = \begin{pmatrix}
        3&2&2\\2&3&2\\2&2&3\\
    \end{pmatrix}, P = \begin{pmatrix}
        0&1&0\\1&0&1\\0&0&1\\
    \end{pmatrix}, B = P^{-1}A^*P,$ 求 $ B + 2E $ 特征值特征向量。
    \item[\textbf{方法}] 通过表格法,发现 $ A^* $ 的特征值为 $ \dfrac{|A|}{\lambda_A}, $ 
    特征向量仍为 $ \alpha_A; $ 而 $ P^{-1}A^*P $ 特征值仍为$ \dfrac{|A|}{\lambda_A}, $ 
    特征向量为为 $ P^{-1}\alpha_A. $
    只需求解 $ A $ 特征向量、特征值、行列式并按上述式计算即可。
\end{itemize}

\begin{itemize}
    \item[\textbf{例题}] 
    设 $ A_{3\times 3} = \alpha\beta^\top + \beta\alpha^\top,
    \alpha,\beta $ 为单位列向量,$ \alpha^\top\beta = \dfrac{1}{3}, $ 则
    \begin{enumerate}[label = \Roman*.]
        \item $ 0 $ 是 $ A $ 的特征值;
        \item $ \alpha+\beta,\alpha-\beta $ 都是 $ A $ 的特征向量;
        \item $ A $ 可以相似对角化。
    \end{enumerate}
    \item[\textbf{证明}]
    \begin{enumerate}[label = \Roman*.]
        \item $ r(A) = r(\alpha\beta^\top+\beta\alpha^\top)
        \leq r(\alpha\beta^\top) + r(\beta\alpha^\top) = 2, $ 
        因而 $ A $ 不满秩,故其必有 $ \lambda_i = 0. $ 
        \item 由于
        \begin{equation*}
            \begin{aligned}
                A(\alpha+\beta) &= \alpha\beta^\top\alpha + \alpha\beta^\top\beta
                + \beta\alpha^\top\alpha + \beta\alpha^\top\beta 
                \\&= [tr(A) + |\alpha|](\alpha+\beta) = \dfrac{4}{3}(\alpha+\beta)
            \end{aligned}
        \end{equation*}
        因而为特征向量;$ \alpha-\beta $ 同理。
        \item 可以算出其有特征值 $ \lambda = 0,\dfrac43,-\dfrac{2}{3} $ 互异,故
        可以相似对角化。
    \end{enumerate}
\end{itemize}

\Section{相似性的判定}

若存在可逆矩阵 $ P $ 使得 $ P^{-1}AP = B, $ 或者 $ AP = PB, $ 则称 $ A,B $ 相似。

\sssubsection{性质}

对相似的 $ A,B, $ 

\begin{itemize}
    \item $ |A| = |B|; tr(A) = tr(B); $ 
    
    $ r(A) = r(B); \lambda_A = \lambda_B; $ 
    \item $ P^{-1}A^nP = B^{n}; P^{-1}(A+kE)P = B+kE; $ 
    \item 若 $ A\sim C, C\sim B, $ 则 $ A\sim B. $ 
\end{itemize}

\begin{itemize}
    \item[\textbf{例题}] 对相似的 $ A = \begin{pmatrix}
        1&a1\\1&5&1\\4&12&6\\
    \end{pmatrix},B = \begin{pmatrix}
        b&&\\&b&\\&&c
    \end{pmatrix}, $ 求 $ a,b,c. $ 
    \item[\textbf{方法}] 由相似性质,$ tr(A) = 12 = tr(B) = 2b + c; $
    
    $ |\lambda E-A| = (\lambda-2)(\lambda^2 - 10\lambda +13 - a)=0 $ 
    有解 $ \lambda = b,b,c. $ 
    
    分别假设 $ b = 2,c = 2, $ 得
    $ \begin{pmatrix}
        a\\b\\c
    \end{pmatrix} = \begin{pmatrix}
        -3 \\ 2 \\ -8
    \end{pmatrix} $ 或
    $ \begin{pmatrix}
        -12\\5\\2
    \end{pmatrix}. $ 
\end{itemize}

\begin{itemize}
    \item[\textbf{例题}] 设 $ A_{3\times 3} $ 的互异的特征值为 $ \lambda_i,i\in\{1,2,3\}; $ 
    对应的特征向量为 $ \alpha_i,i\in\{1,2,3\}. $ 令 $ \beta = \sum \alpha, $ 
    \begin{enumerate}[label = \Roman*.]
        \item 证明 $ \beta,A\beta,A^2\beta $ 线性无关;
        \item 若 $ A^3\beta = A\beta, $ 求 $ r(A - E) $ 与 $ |A+2E|. $ 
    \end{enumerate}
    \item[\textbf{方法}] \begin{enumerate}[label = \Roman*.]
        \item 由于特征值互异,其对应的特征向量线性无关。
        设有
        \begin{equation*}
            \begin{aligned}
                k_1\beta_1 + k_2A\beta_2 + k_3A^2\beta_3
                = \sum(k_1 + k_2\lambda_i + k_3\lambda_i^2)\alpha_i = 0;
            \end{aligned}
        \end{equation*}
        由于 $ \alpha_i $ 无关,知 $ \forall i, k_1 + k_2\lambda_i + k_3\lambda_i^2 = 0, $ 
        而 $ \lambda $ 互异,故对 $ Ak = 0, |A|\neq 0, $ 故该方程只有零解,
        故原向量组无关。
        \item 当第一问中构造了一组无关向量时,一般在第二问用到其列矩阵,
        应用\begin{itemize}
            \item 相似;
            \item 矩阵乘法;
        \end{itemize}联系已知矩阵以构建方便计算待求结论的新矩阵。

        令 $ P = (\beta,A\beta,A^2\beta), $ 则
        \begin{equation*}
            \begin{aligned}
                AP &= (A\beta,A^2\beta,A^3\beta) = (A\beta,A^2\beta,A\beta) \\ 
                &= (\beta,A\beta,A^2\beta)\begin{pmatrix}
                    0&0&0\\1&0&1\\0&1&0\\
                \end{pmatrix} = PB
            \end{aligned}
        \end{equation*}
        故 $ A,B $ 相似,有 $ r(A-E) = r(B-E)=2,|A+2E| = |B+2E| = 6. $ 
    \end{enumerate}
\end{itemize}

\begin{itemize}
    \item[\textbf{例题}] 已知矩阵 $ A = \begin{pmatrix}
        -2&-2&1\\2&x&-2\\0&0&-2\\
    \end{pmatrix},B = \begin{pmatrix}
        2&1&0\\0&-1&0\\0&0&y\\
    \end{pmatrix}, $ 相似,
    \begin{enumerate}[label = \Roman*.]
        \item 求 $ x,y; $ 
        \item 求可逆矩阵 $ P $ 使得 $ P^{-1}AP = B. $ 
    \end{enumerate}
    \item[\textbf{方法}] 
    \begin{enumerate}[label = \Roman*.]
        \item 由题,$ |A| = 4(x-2) = |B| = -2y, tr(A) = x - 4 = tr(B) = y + 1, $ 
        故解得 $ x = 3,y = -2. $ 
        \item 可以知道 $ \lambda_A = \lambda_B = 2,-1,-2. $ 故 $ A,B $ 都
        相似于 $ \Lambda = \begin{pmatrix}
            2&&\\&-1&\\&&-2\\
        \end{pmatrix}. $ 
        
        此时可以求 $ P_1^{-1}AP_1 = \Lambda,P_2^{-1}BP_2 = \Lambda; P_1,P_2$  
        各列是对应矩阵的特征向量;

        发现 $ B = P_2P_1^{-1}AP_1P_2^{-1} = \Lambda, $ 
        因此 $ P = P_1P_2^{-1}. $ 
    \end{enumerate}
\end{itemize}

\Section{相似对角化的判定与运算}

\sssubsection{定义}

若存在可逆矩阵 $ P $ 使得 $ P^{-1}AP = \Lambda \Leftrightarrow AP = P\Lambda, $ 
则称方阵 $ A $ 可以相似对角化。

\sssubsection{求解}

\begin{itemize}
    \item 可逆 $ P $ 阵
    
    $ A $ 的 $ n $ 个线性无关的特征向量 $ \alpha_A $ 构成;
    
    若特征向量的数目不够,则其不可相似对角化。
    \item 对角矩阵 $ \Lambda $
    
    其由 $ n $ 个特征值对应,特征值与特征向量的位置是对应的。
\end{itemize}

\sssubsection{判定}

\begin{itemize}
    \item \textbf{充分条件}
    
    实对称矩阵 $ \Rightarrow $ 可相似对角化;

    有 $ n $ 个互异特征值的矩阵 $ \Rightarrow $ 可相似对角化;
    \item \textbf{充要条件}
    \begin{equation*}
        \begin{aligned}
            \textrm{可相似对角化}&\Leftrightarrow \textrm{有}n\textrm{个无关特征向量}\\
            &\Leftrightarrow k\textrm{重特征值对应}k\textrm{个无关特征向量}\\ 
            &\Leftrightarrow k = n - r(\lambda_0E - A) \\
            &\Leftrightarrow \red{r(\lambda_0E - A) = n - k} \\
        \end{aligned}
    \end{equation*}
\end{itemize}

\begin{itemize}
    \item[\textbf{例题}] 设三阶矩阵 $ A $ 的特征值为 $ 1,3,-2, $ 其对应的特征向量为
    $ \alpha_1,\alpha_2,\alpha_3, $ 
    
    若 $ P = (\alpha_1,2\alpha_3,-\alpha_2), $ 
    则 $ P^{-1}A^*P = \qline. $ 
    \item[\textbf{方法}] 
    由于三阶矩阵 $ A $ 的特征值为 $ 1,3,-2, $ 其对应的特征向量为
    $ \alpha_1,\alpha_2,\alpha_3, $ 
    对 $ P = (\alpha_1,2\alpha_3,-\alpha_2), $ 其列分别对应的特征值为 $ 1,-2,3. $  
    注意,此处 $ -\alpha_2 $ 的系数 $ -1 $ \textbf{不影响特征值的正负性}。

    对 $ A^*, $ 其特征值为 $ \dfrac{|A|}{\lambda} = -6,-2,3, $ 
    因此 $ P^{-1}A^*P $ 对应的特征值的次序改变,为 $ -6,3,-2, $ 
    因而有 $ \Lambda = \begin{pmatrix}
        -6&&\\&3&\\&&-2\\
    \end{pmatrix}. $ 
\end{itemize}

\begin{itemize}
    \item[\textbf{例题}] 设 $ A $ 为二阶矩阵,$ P = (\alpha,A\alpha), $ 其中 $ \alpha $ 为非零向量且
    不是 $ A $ 的特征值,
    \begin{enumerate}[label = \Roman*.]
        \item 证明 $ P $ 为可逆矩阵;
        \item 若 $ A^2\alpha+A\alpha-6\alpha = 0, $ 求 $ P^{-1}AP $ 并判断其是否
        与对角矩阵相似。
    \end{enumerate}
    \item[\textbf{方法}]
    \begin{enumerate}[label = \Roman*.]
        \item 假设 $ A $ 不是可逆矩阵,则其不满秩,因而 $ \alpha $ 与 $ A\alpha $ 成比例。
        此时有 $ A\alpha = k\alpha, $ 也即 $ \alpha $ 为 $ A $ 的特征值,与题设矛盾,
        因而 $ A $ 是可逆矩阵。
        \item 由于\begin{equation*}
            \begin{aligned}
                P^{-1}AP &= P^{-1}A(\alpha,A\alpha) = P^{-1}(A\alpha,A^2\alpha) \\ 
                &= P^{-1}(A\alpha,-A\alpha+6\alpha) \\&= P^{-1}(\alpha,A\alpha)\begin{pmatrix}
                    0&6\\1&-1
                \end{pmatrix}_{B} = P^{-1}PB = B,
            \end{aligned}
        \end{equation*}
        而 $ \lambda_B = 2,-3 $ 互异,因而可以相似对角化,故 $ A $ 也可以相似对角化。
    \end{enumerate}
\end{itemize}

\begin{itemize}
    \item[\textbf{例题}] 设 $ A,B,C $ 为三阶矩阵,有 $ AB = -B, CA^\top 2C, $ 
    
    其中 $ B = \begin{pmatrix}
        1&2&3\\-1&1&0\\2&-1&1\\
    \end{pmatrix},C = \begin{pmatrix}
        1&-2&1\\-2&4&-2\\-1&2&-1\\
    \end{pmatrix},\xi = \begin{pmatrix}
        1\\2\\a
    \end{pmatrix}, $ 
    \begin{enumerate}[label = \Roman*.]
        \item 求矩阵 $ A; $ 
        \item 求当 $ a $ 为何值时有 $ A^{100}\xi = \xi. $ 
    \end{enumerate}
    \item[\textbf{方法}] \begin{enumerate}[label = \Roman*.]
        \item 不妨设 $ B = (\beta_i),C^\top = (\gamma_i), $ 注意到有 $ (CA^\top)^\top = 2C^\top\Rightarrow
        AC^\top = 2C^\top. $ 
        
        由于 $ A\beta_1 = -\beta_1,A\beta_2 = -\beta_2, A\gamma_1 = 2\gamma_1, $ 
        知 $ \lambda_A = -1,-1,2,\alpha_A = \beta_1,\beta_2,\gamma_1. $ 
        
        因此,$ A = P^{-1}\Lambda P, P = (\beta_1,\beta_2,\gamma_1), \Lambda = 
        \begin{pmatrix}
            -1&&\\&-1&\\&&2\\
        \end{pmatrix}. $ 

        可以解得 $ A = P\Lambda P^{-1} = \dfrac{1}{8}\begin{pmatrix}
            -5&-15&-9\\-6&22&18\\3&-15&-17\\
        \end{pmatrix} $ 
        \item 由于 $ \beta_1,\beta_2 $ 为 $ A $ 的无关特征向量,必有
        $ A^{100}(k_1\beta_1+k_2\beta_2) = (-1)^{100}(k_1\beta_1+k_2\beta_2), $ 
        而 $ A^{100}\xi = \xi, $ 可以知道 $ \xi $ 可被 $ \beta_1,\beta_2 $ 线性表出,因此
        $ \xi = x_1\beta_1+x_2\beta_2 $ 必有解,即
        \begin{equation*}
            \begin{aligned}
                (\beta_1,\beta_2,\xi)\rightarrow\begin{pmatrix}
                    1&2&1\\-1&1&2\\2&-1&a\\
                \end{pmatrix}\rightarrow\begin{pmatrix}
                    1&2&1\\0&1&1\\0&0&a+3\\
                \end{pmatrix}
            \end{aligned}
        \end{equation*}
        有解,因此 $ a = -3. $ 
    \end{enumerate}
\end{itemize}

 \Section{实对称矩阵}

\sssubsection{对角化}

\begin{itemize}
    \item 可逆矩阵 $ P $ 
    
    必定存在可逆矩阵 $ P $ 使得 $ P^{-1}AP = \Lambda, $ 其中 $ P $ 由
    特征向量组成;
    \item 正交矩阵 $ Q $ 
    
    必定存在正交矩阵 $ Q $ 使得 $ Q^\top AQ =  Q^{-1}AQ = \Lambda, $ 
    其中 $ Q $ 经过了施密特正交单位化。
\end{itemize}

施密特正交化时,对无关的一组 $ \alpha_i, $ 有
\begin{itemize}
    \item $ \beta_1 = \alpha_1; $ 
    \item $ \beta_2 = \alpha_2 - \dfrac{(\alpha_2,\beta_2)}{(\beta_2,\beta_2)}\beta_2 $ 
    \item $ \beta_3 = \alpha_3 - 
    \dfrac{(\alpha_3,\beta_2)}{(\beta_2,\beta_2)}\beta_2-
    \dfrac{(\alpha_3,\beta_1)}{(\beta_1,\beta_1)}\beta_1 $ 
\end{itemize}

\sssubsection{求原矩阵}

\begin{itemize}
    \item 对 $ P $ 有 $ A = P\Lambda P^{-1}; $ 
    \item 对 $ Q = (\gamma_i) $ 有 $ A = Q\Lambda Q^{\top}; $ 
\end{itemize}

特别地,若 $ r(A) = 1, A = tr(A)\gamma_1\gamma_1^\top. $ 

\begin{itemize}
    \item[\textbf{例题}] 已知 $ A $ 为三阶实对称矩阵,各行元素和均为$ 3, $ 且
    $ \alpha_1 = (-1,2,-1)^\top,(0,-1,1)^\top $ 是 $ AX = 0 $ 的解。
    \begin{enumerate}[label = \Roman*.]
        \item 求 $ A $ 特征值与特征向量;
        \item 求正交矩阵 $ Q $ 使得 $ Q^\top AQ = \Lambda; $ 
        \item 求 $ A $ 以及 $ (A-\dfrac{3}{2}E)^6. $ 
    \end{enumerate}
    \item[\textbf{方法}]
    \begin{enumerate}[label = \Roman*.]
        \item 求特征向量时,\textbf{若未明示求无关特征向量,则需给出全部特征向量。}
        
        由各行和均为$ 3 $ 知 $ A(1,1,1)^\top = (3,3,3)^\top = 3(1,1,1)^\top, $ 
        故 $ A $ 有特征值 $ 3, $ 其对应无关特征向量为 $ \alpha_3 = (1,1,1)^\top; $ 
        
        由题设,$ \alpha_1,\alpha_2 $ 是 $ \lambda = 0 $ 对应的无关特征向量。

        因此,$ \lambda = 0 $ 对应的特征向量为 $ k_1\alpha_1+k_2\alpha_2, k_1,k_2\in \R\backslash\{0\}; $ 
        
        $ \lambda = 3 $ 对应的特征向量为 $ k_3\alpha_3, k_3\in \R\backslash\{0\}. $ 
        \item 可以知道 $ \lambda_A = 0,0,3, $ 对应的无关特征向量为 $ \alpha_1,\alpha_2,\alpha_3. $ 
        
        因为 $ \alpha_3 $ 与另外二者正交,对 $ \alpha_1,\alpha_2 $ 做施密特正交化。
        故有\begin{itemize}
            \item $ \beta_1 = \alpha_1 = (-1,2,-1)^\top; $ 
            \item $ \beta_2 = \alpha_2 - \dfrac{(\alpha_2,\beta_1)}{(\beta_1,\beta_1)}\beta_1
            = \dfrac{1}{2}(-1,0,1)^\top; $ 
        \end{itemize}
        对全体做单位化,故有
        $ \gamma_1 = \dfrac{1}{\sqrt 6}\beta_1;  \gamma_2 = \dfrac{1}{\sqrt 2}\beta_2; 
        \gamma_3 = \dfrac{1}{\sqrt 3}\beta_3; $ 

        此时,有 $ Q = (\gamma_i) $ 使得 $ Q^{-1}AQ = Q^\top AQ = \Lambda = \begin{pmatrix}
            0&&\\&0&\\&&3
        \end{pmatrix}. $ 
        \item 显然 $ A = Q\Lambda Q^\top. $ 而 $ r(A) = 1, \lambda = 0,0,3, $ 
        
        故有 $ A = tr(A)\alpha_1\alpha_1^\top = \begin{pmatrix}
        1&1&1\\1&1&1\\1&1&1\\    
        \end{pmatrix}; $ 

        $ (A - \dfrac{3}{2}E)^6 = Q^\top(\Lambda-\dfrac{3}{2})Q $ 

        \begin{equation*}
            \begin{aligned}
                (A - \dfrac{3}{2}E)^6 &= Q^\top(\Lambda-\dfrac{3}{2})^6Q \\ 
                &= Q^\top\begin{pmatrix}
                    -\dfrac{3}{2}&&\\&-\dfrac{3}{2}&\\&&\dfrac{3}{2}\\
                \end{pmatrix}^6 Q
                \\&= (\dfrac{3}{2})^6Q^\top Q = (\dfrac{3}{2})^6 E.
            \end{aligned}
        \end{equation*}
    \end{enumerate}
\end{itemize}

\begin{itemize}
    \item[\textbf{例题}] 设 $ A = \begin{pmatrix}
        0&-1&4\\-1&3&a\\4&a&0\\
    \end{pmatrix}, $ 有正交矩阵 $ Q $ 使得 $ Q^\top AQ = \Lambda, $ 若 $ Q $ 的
    第一列为 $ \dfrac{1}{\sqrt 6}(1,2,1)^\top, $ 求 $ a,Q. $ 
    \item[\textbf{方法}] 由于存在特征值 $ \lambda_1 $ 使得 $ A\lambda_1 = \lambda_2(1,2,1)^\top, $ 
    可以解得 $ a = -1,\lambda_1 = 2. $ 

    代入 $ a = -1, $ 求 $ |\lambda E - A| = 0, $ 解得 $ \lambda_2 = -4,\lambda_3 = 5. $ 
    通过 $ A\alpha_2 = \lambda_2\alpha_2 $ 求 $ \alpha_2, $ 显然 $ \alpha_1,\alpha_2,\alpha_3 $ 正交。
    因此由正交性,$$
        \alpha_3 = \begin{pmatrix}
            i&j&k\\ a_{11}&a_{12}&a_{13}\\a_{21}&a_{22}&a_{23}\\
        \end{pmatrix} = 2i-2j+2k
    $$
    此时 $ Q = \left(\dfrac{\alpha_i}{|\alpha_i|}\right). $ 
\end{itemize}
\chapter{无穷级数}

\section{数项级数}

\subsection{数项级数概念}

\begin{Def}[数项级数]

    无穷多个数 $ u_1,u_2,\dots,u_n,\dots $ 按照一定的顺序相加得到的表达式
     $ \sum_{n=1}^\infty u_n $ 称为数项级数,简称级数,其中 $ u_n $ 称为通项。
\end{Def}

级数是无数个的,也是有序的。

\begin{Def}[级数收敛]

    称 $ S_n = \sum_{k=1}^n u_k $ 为级数的部分和,数列 $ \{S_n\} $ 即为部分和数列。

    若 $ {\displaystyle\lim_{n\rightarrow \infty}}S_n\xlongequal{\exists}S $ ,
    则称级数 $\dis \sum_{n=1}^\infty u_n $ 收敛,且和为 $ S $ ;反之,若极限不存在,则称其
    为发散的,发散级数没有和的概念。
\end{Def}

以下是几个重要级数及其敛散性。

\begin{itemize}
    \item 几何级数 - $\dis \sum_{n=0}^\infty ar^n $ ,当 $ |r|<1 $ 时,其收敛于 $ \dfrac{a}{1-r} $ ;
    若 $ |r|\geq1 $ ,则其发散。
    \item p级数 - $\dis \sum_{n=1}^\infty \dfrac{1}{n^p} $ ,若 $ p > 1 $ ,其收敛;若 $ p\leq 1 $ ,其发散。
\end{itemize}

事实上,有 $\dis \sum_{n=1}^\infty \dfrac{1}{n^2} = \dfrac{\pi^2}{6}. $ 

\section{级数基本性质及其收敛必要条件}

级数有以下性质。

\begin{itemize}
    \item 若 $ \dis \sum_{n=1}^{\infty}u_n,\sum_{n=1}^{\infty}v_n $ 都收敛,则
    $ \dis \sum_{n = 1}^{\infty}(u_n\pm v_n) $ 也收敛,且有
    $ \dis \sum_{n = 1}^{\infty}(u_n\pm v_n) = \sum_{n=1}^{\infty}u_n\pm\sum_{n=1}^{\infty}v_n $ .
    \item 在级数中增加、减少、改变有限项不会影响级数敛散性。
    \item 收敛级数具有结合律,即可以对级数的项任意地添加括号,新级数仍然收敛于同一个和。
    注意,去括号不成立。
    \item 对 $ k \neq 0 $ 有 $ \dis \sum_{n=1}^{\infty}ku_n $ 与  $ \dis \sum_{n=1}^{\infty}u_n $
    具有相同的敛散性。
    \item 级数  $ \dis \sum_{n=1}^{\infty}u_n $ 收敛的必要条件是 $ {\displaystyle\lim_{n\rightarrow \infty}}u_n = 0 $ .
\end{itemize}

\section{正项级数及其敛散性判别法}

\begin{Def}[正项级数]

    若 $ u_n\geq 0 $ ,则有 $\dis \sum_{n=1}^{\infty}u_n $ 称为正项级数。
\end{Def}

\begin{Theo}[正项级数收敛基本定理]

    正项级数 $\dis \sum_{n=1}^{\infty}u_n $ 收敛 $ \Leftrightarrow  S_n $ 有上界。
\end{Theo}

正项级数有以下几种判别法。

\begin{Theo}[直接比较法]

    设当 $ n\geq N $ 时,有 $ 0\leq u_n\leq v_n $ 总是成立,则\begin{itemize}
        \item 若 $\dis \sum_{n=1}^{\infty}v_n $ 收敛,则 $\dis \sum_{n=1}^{\infty}u_n $ 收敛;
        \item 若 $\dis \sum_{n=1}^{\infty}u_n $ 发散,则 $\dis \sum_{n=1}^{\infty}v_n $ 发散。
    \end{itemize}
\end{Theo}

\begin{Theo}[间接比较法]

    设 $ u_n\geq 0, v_n>0,n = 1,2,3,\dots $,且有 $ {\displaystyle\lim_{n\rightarrow \infty}}
    \dfrac{u_n}{v_n} = A  $ ,则\begin{itemize}
        \item 若 $ 0<A<+\infty $ ,两级数具有相同的敛散性;
        \item 若 $ A = 0 $ ,则若 $\dis \sum_{n=1}^{\infty}v_n $ 收敛,有 $\dis \sum_{n=1}^{\infty}u_n $ 收敛;
        \item 若 $ A = +\infty $ ,若 $\dis \sum_{n=1}^{\infty}u_n $ 收敛,则 $\dis \sum_{n=1}^{\infty}v_n $ 收敛。
    \end{itemize}
\end{Theo}

\begin{Theo}[比值判别法]

    设 $ u_n > 0 $ ,有 $ {\displaystyle\lim_{n\rightarrow \infty}}\dfrac{u_{n+1}}{u_{n}} = \rho $ ,
    则\begin{itemize}
        \item 若 $ \rho > 1 $ ,有 $\dis \sum_{n=1}^{\infty}u_n $ 收敛;
        \item 若 $ \rho < 1 $ ,则其发散;
        \item 若 $ \rho = 1 $ ,该判别法失效。
    \end{itemize}
\end{Theo}

\begin{Theo}[根值判别法]

    设 $ u_n \leq 0 $ ,有 $ {\displaystyle\lim_{n\rightarrow \infty}}\sqrt[n]{u_n} = \rho $ ,
    则\begin{itemize}
        \item 若 $ \rho > 1 $ 甚至为$ +\infty $ 时,有 $\dis \sum_{n=1}^{\infty}u_n $ 收敛;
        \item 若 $ \rho < 1 $ ,则其发散;
        \item 若 $ \rho = 1 $ ,该判别法失效。
    \end{itemize}
\end{Theo}

\begin{Theo}[积分准则]

    若函数在 $ [1,+\infty) $ 上非负、连续、递减,则级数 $\dis \sum_{n=1}^{\infty}f(n) $ 与
    积分 $ \dis \int_1^{+\infty}f(x)\mathrm{d}x $ 具有相同的敛散性。
\end{Theo}

注意,判别正项级数敛散性时,\begin{equation*}
    \begin{matrix}
        \textrm{间接比较法} \\\Downarrow\\ 
        \textrm{比值法}\\\Downarrow\\
         \textrm{根值法}\\\Downarrow\\
          \textrm{积分准则}\\\Downarrow\\
        \textrm{直接比较法}\\\Downarrow\\
         \textrm{定义·部分和数列}\\\Downarrow\\
          \textrm{收敛级数定义}
    \end{matrix}
\end{equation*}

此外,若通项为积分或者是抽象的,从直接比较法开始应用,试图放缩被积函数以去除积分号。

\section{任意项级数}

\subsection{交错级数及其敛散性判别法}

\begin{Def}[交错级数]

    若 $\dis u_n>0,\sum_{n=1}^{\infty}(-1)^{n+1}u_n $ 称为交错级数。
\end{Def}

\begin{Theo}[莱布尼茨判别法]

    交错级数 $\dis \sum_{n=1}^{\infty}(-1)^{n+1}u_n $ 满足
    \begin{itemize}
        \item $ u_{n+1}\leq u_n $ 
        \item $ {\displaystyle\lim_{n\rightarrow \infty}}u_n = 0 $ 
    \end{itemize}
    则该交错级数收敛,且 $\dis 0\leq \sum_{n=1}^{\infty}(-1)^{n+1}u_n\leq u_1 $ .
\end{Theo}

\subsection{条件收敛与绝对收敛}

\begin{Def}[条件收敛与绝对收敛]

    若 $ \dis \sum_{n=1}^{\infty}|u_n| $ 收敛,则称 $ \dis \sum_{n=1}^{\infty}u_n $ 绝对收敛。

    若 $ \dis \sum_{n=1}^{\infty}u_n $ 收敛而 $ \dis \sum_{n=1}^{\infty}|u_n| $ 发散,则
    $ \dis \sum_{n=1}^{\infty}u_n $ 称为条件收敛。
\end{Def}

绝对收敛的级数有交换律,而条件收敛的级数没有。

\begin{Theo}[条件收敛必收敛]

    $ \dis \sum_{n=1}^{\infty}|u_n|$ 收敛 $\dis \Rightarrow \sum_{n=1}^{\infty}u_n $ 收敛。
\end{Theo}

\section{幂级数}

\subsection{函数项级数}

\begin{Def}[函数项级数]

    设 $ u_n(x),n=1,2,\dots $ 定义在 $ I $ 上,则 $ \dis \sum_{n=1}^{\infty}u_n(x) $ 
    称为区间 $ I $ 上的函数项级数。
\end{Def}

\begin{Def}[收敛点与收敛域]

    对 $ x_0\in I $ ,若常数项级数 $\dis \sum_{n=1}^{\infty}u_n(x_0) $ 收敛,则称 $ x_0 $ 为函数项级数
    $ \dis \sum_{n=1}^{\infty}u_n(x) $ 的收敛点;反之,若其发散,则为发散点。

    函数项级数 $ \dis \sum_{n=1}^{\infty}u_n(x) $ 的全部收敛点构成的集合是收敛域,全部发散点构成的集合
    是发散域。
\end{Def}

\begin{Def}[和函数]

    和函数 $\dis S(x) =: \sum_{n=1}^{\infty}u_n(x) $ ,其定义域为收敛域。
\end{Def}

\subsection{幂级数}

\begin{Def}[幂级数]

    形如$ \dis \sum_{n=1}^\infty a_n(x-x_0)^n $ 的级数称为 $ (x-x_0) $ 的幂级数,常数 $ a_n $ 称为
    幂级数的系数。
\end{Def}

显然,当 $ x_0 = 0 $ 时,有 $ \dis \sum_{n=0}^{\infty}a_nx^n $ 称为 $ x $ 的幂级数。

\begin{Theo}[阿贝尔定理]

    若幂级数 $\dis \sum u_n(x) $ 在 $ x = x_0 $ 处收敛,则对任意 $ x $ 满足 $ |x| < |x_0| $ ,
    都有$ \dis \sum u_n(x) $ 绝对收敛;
    若在 $ x = x_0 $ 发散,则对任意 $ x $ 满足 $ |x|>|x_0| $ 都有$\dis \sum u_n(x) $ 发散。
\end{Theo}

\subsection{幂级数收敛半径及其求法}

幂级数的收敛半径 $ R $ 可以由以下公式求出。

\begin{itemize}
    \item $ \dis R = {\displaystyle\lim_{n\rightarrow \infty}}\left|\dfrac{a_n}{a_{n+1}}\right| $ 
    \item $ \dis R = {\displaystyle\lim_{n\rightarrow \infty}}\dfrac{1}{\dsqrt[n]{|a_n|}} $ 
\end{itemize}

注意,公式仅对标准幂级数成立,即要求幂函数的次数逐次增长;

收敛区间的中心为 $ x_0 $. 以 $ x = 0 $ 为例子,若存在收敛半径 $ R\geq 0 $ ,则
$ x\in (-R,R) $ 时级数绝对收敛,$ x\in R\big\backslash(-R,R) $ 时级数发散。
若级数有条件收敛点,则其只能在 $ \pm R $ 处,因而最多有两个。

在求收敛半径后,必须写出收敛区间,然后判断端点处敛散性。

\subsection{幂级数的性质}

\sssubsection{幂级数的加减运算}

设 $ \dis \sum_{n=1}^\infty a_nx^n = f(x),|x|< R_1;\sum_{n=1}^\infty b_nx^n = g(x), |x|< R_2 $,则有
$ \dis \sum_{n=1}^\infty (a_n\pm b_n)x^n = f(x)\pm g(x), |x|<\min(R_1,R_2)  $ .

\sssubsection{幂级数的逐项求导与逐项积分}

设幂级数 $ \dis \sum_{n=0}^\infty a_nx^n $ 的收敛半径 $ R>0 $ 且有和函数 $ \dis S(x) = \sum_{n=1}^\infty a_nx^n $ ,则
成立以下性质。
\begin{itemize}
    \item $ S(x) $ 在 $ \overset{\textrm{收敛区间}}{(-R,R)} $ 内可导,且有逐项求导公式$$
        S'(x) = \left(\sum_{n=0}^\infty a_nx^n\right)' = \sum_{n=0}^\infty (a_nx^n)' = \sum_{n=0}^\infty na_nx^{n-1}
    $$ 
    求导后幂级数收敛半径不变,因而 $ S(x) $ 在 $ (-R,R) $ 内有任意阶导数;
    \item $ S(x) $ 在$ \overset{\textrm{收敛区间}}{(-R,R)} $内有逐项积分公式$$
        \int_0^x S(t)\mathrm{d}t = \sum_{n=0}^\infty\int_0^x a_nt^n\mathrm{d}t = \sum_{n=0}^\infty \dfrac{a_n}{n+1}x^{n+1}
    $$ 
    积分后收敛半径不变;
    \item 幂级数的和函数 $ S(x) $ 在收敛域上连续。
\end{itemize}

\sssubsection{单位幂级数}

$ \dis \sum_{n=0}^\infty x^n $ 称为单位幂级数,显然其收敛半径是1,且收敛区间内,有 $ \dis \sum x^n = \dfrac{1}{1-x} $ .

$ \dis \sum_{n=0}^\infty (-1)^{n+1}x^n $ 收敛半径也是1, 且收敛区间内,有 $ \dis \sum x^n = \dfrac{1}{1+x} $ .


\section{函数展开成幂级数}

\subsection{概念}

设 $ f(x) $ 是给定的函数,若存在幂级数 $ \dis \sum_{n=0}^\infty a_n(x-x_0)^n $ 使得任意 $ x\in I $ 
级数 $ \dis \sum_{n=0}^\infty a_n(x-x_0)^n $ 都收敛到 $ f(x) $ ,即 $\dis \forall x\in I, f(x) = \sum_{n=0}^\infty a_n(x-x_0)^n $ ,
则称 $ f(x) $ 在区间 $ I $ 上可展开为 $ x - x_0 $ 的幂级数,称级数 $ \dis \sum_{n=0}^\infty a_n(x-x_0)^n $ 为$ f(x) $ 的幂级数。

\begin{Def}[幂级数展开唯一性]

    若 $ f(x) $ 在 $ x = x_0 $ 可以展开为幂级数 $ \dis \sum_{n=0}^\infty a_n(x-x_0)^n $ ,则该展开式是唯一的,且为
    泰勒级数,即有 $ \dis f(x) = \sum_{n=0}^\infty a_n(x-x_0)^n = \sum_{n=0}^\infty \dfrac{f^{(n)(x_0)}}{n!}(x-x_0)^n$
\end{Def}

\subsection{函数展开成幂级数的条件及其形式}

\sssubsection{泰勒级数与马克劳林级数的定义}

设函数 $ f(x) $ 在 $ x_0 $ 的某一领域 $ |x-x_0|<\sigma $ 内具有任意阶导数,则级数
$ \dis \sum_{n=0}^\infty \dfrac{f^{(n)(x_0)}}{n!}(x-x_0)^n $ 称为函数 $ f(x) $ 在 $ x = x_0 $ 处的泰勒级数。

特别地,若 $ x_0 = 0 $ ,称级数为$ f(x) $ 的马克劳林级数。

\sssubsection{函数展开成为幂级数的充要条件}

设 $ f(x) $ 在 $ |x-x_0|<R $ 内有任意阶导数,其泰勒公式为$$
    f(x) = \sum_{i=0}^n \dfrac{f^{(i)}(x_0)}{i!}(x-x_0)^i + R_n(x)
$$ 其中$ R_n(x) $为 $ n $ 阶余项,
其拉格朗日型为 $ \dis R_n(x) = \dfrac{f^{(n+1)}[x_0+\theta(x-x_0)]}{(n+1)!}(x-x_0)^{n+1},\theta\in(0,1) $ ,
则 $ \dis f(x) = \sum_{n=0}^\infty \dfrac{f^{(n)}(x_0)}{ni!}(x-x_0)^n,|x-x_0|<R $ 的充要条件为
$ {\displaystyle\lim_{n\rightarrow \infty}}R_n(x) = 0,|x-x_0|<R $ .

特别地,当 $ x_0 = 0 $ 时得到函数展开为马克劳林级数的充要条件。

\subsection{函数展开成为幂级数的方法}

\begin{itemize}
    \item 公式法 $$
     \dis f(x) = \sum_{n=0}^\infty \dfrac{f^{(n)}(x_0)}{ni!}(x-x_0)^n,|x-x_0|<R 
    $$
    \item 变量替换;
    \item 逐项求导;
    \item 逐项积分;
    \item 恒等变形、变量代换等。
\end{itemize}

幂级数有8个常用的展开式。

\begin{itemize}
    \item $ \dis e^x = \sum_{n=0}^{\infty}\dfrac{1}{n!}x^n, |x|<\infty $ 
    \item $ \dis \sin x = \sum_{n=0}^{\infty}(-1)^n\dfrac{x^{2n+1}}{(2n+1)!},|x|<\infty $ 
    \item $ \dis \cos x = \sum_{n=0}^{\infty}(-1)^n\dfrac{x^{2n}}{(2n)!},|x|<\infty $ 
    \item $ \dis (1+x)^\alpha = 1 + \sum_{n=1}^{\infty}\dfrac{\mathrm{C}_\alpha^n}{n!}x^n, |x|<1 $ 
    \item $ \dis \dfrac{1}{1-x} = \sum_{n=0}^{\infty}x^n, |x|<1 $ 
    \item $ \dis \dfrac{1}{1+x} = \sum_{n=0}^{\infty}(-1)^nx^n, |x|<1 $ 
    \item $ \dis \ln(1+x) = \sum_{n=1}^{\infty} \dfrac{(-1)^{n+1}x^n}{n!}, -1<x\leq 1 $ 
    \item $ \dis \ln(1-x) = \sum_{n=1}^{\infty} \dfrac{x^n}{n!}, -1\leq x< 1 $ 
\end{itemize}

以上都展成马克劳林级数。若要展成泰勒级数,经过适当处理后可以利用马克劳林级数的结果。

展开的结果尽量合成一个级数,级数后要注收敛范围。

可以利用幂级数求数项级数,即将数项级数视为幂级数的 $ x $ 取了定值。

\section{傅里叶级数}

\subsection{傅里叶级数及傅里叶系数}

设函数 $ f(x) $ 在区间 $ [-\pi,\pi] $ 上黎曼可积,则称公式
\begin{itemize}
    \item $ \dis a_k = \dfrac{1}{\pi}\int_{-\pi}^\pi f(x)\cos kx\mathrm{d}x $ 
    \item $ \dis b_k = \dfrac{1}{\pi}\int_{-\pi}^\pi f(x)\sin kx\mathrm{d}x $ 
\end{itemize}

为函数 $ f(x) $ 的傅里叶系数。称以 $ a_k,b_k $ 为系数的三角级数$$
    \dfrac{a_0}{2} + \sum_{n=1}^{\infty}\left(a_n\cos nx + b_n\sin nx\right)
$$ 
为 $ f(x) $ 的傅里叶级数,记为
$$
    f(x)\sim \dfrac{a_0}{2} + \sum_{n=1}^{\infty}\left(a_n\cos nx + b_n\sin nx\right)
$$ 

类似地,可以定义周期为 $ 2l $ 的函数的傅里叶级数,即

\begin{itemize}
    \item $ \dis a_k = \dfrac{1}{l}\int_{-l}^l f(x)\cos \dfrac{k\pi x}{l}\mathrm{d}x $ 
    \item $ \dis b_k = \dfrac{1}{l}\int_{-l}^l f(x)\sin \dfrac{k\pi x}{l}\mathrm{d}x $
    \item $ \dis f(x) \sim \dfrac{a_0}{2} + 
    \sum_{n=1}^{\infty}\left(a_n\cos \dfrac{n\pi x}{l} + b_n\sin \dfrac{n\pi x}{l}\right) $ 
\end{itemize}

\subsection{傅里叶级数的收敛定理}

设 $ f(x) $ 是周期为 $ 2\pi $ 的周期函数,并满足迪利克雷条件,即
\begin{itemize}
    \item 在一个周期内连续或有有限个第一类间断点;
    \item 在一个周期内有有限个极值点,即单调区间个数有限,
\end{itemize}
则$ f(x) $ 的傅里叶级数收敛,且有

\begin{equation*}
    \begin{aligned}
        \dfrac{a_0}{2} + \sum_{n=1}^{\infty}\left(a_n\cos nx + b_n\sin nx\right)
        = \begin{cases}
            f(x),&x\textrm{为$ f(x) $ 的连续点}\\\dfrac{f(x-0)+f(x+0)}{2},& x\textrm{为$ f(x) $ 的间断点}
        \end{cases}
    \end{aligned}
\end{equation*}

\subsection{奇偶函数的傅里叶级数}

若 $ f(x) $ 是奇函数,则有 $ \dis a_n = \dfrac{1}{l}\int_{-l}^l f(x)\cos \dfrac{n\pi x}{l}\mathrm{d}x = 0,
f(x) = \sum_{n=1}^{\infty}b_n\sin \dfrac{n\pi x}{l} $ ,是为正弦级数。

若 $ f(x) $ 是偶函数,则有 $ \dis b_n = \dfrac{1}{l}\int_{-l}^l f(x)\sin \dfrac{n\pi x}{l}\mathrm{d}x = 0,
f(x) =\dfrac{a_0}{2}+  \sum_{n=1}^{\infty}a_n\cos \dfrac{n\pi x}{l} $ ,是为余弦级数。

以上二级数,都有 $ x\in\{f\textrm{的连续点}\} $.

\subsection{有限区间上的函数的傅里叶级数}

将有限区间上的函数展为正弦级数时,将$ f(x) $ 延拓为周期为 $ 2\pi $ 的奇函数 $ F(x) $ .
此时 $ a_n = 0,\dis b_n = \int_{-\pi}^{\pi} f(x)\sin nx \mathrm{d}x $ ,用分部积分法求 $ b_n $ .
然后有 $ \dis f(x) \xlongequal{\textrm{f(x)延拓的部分}} F(x) \xlongequal{\textrm{连续点}} \sum_{n = 1}^\infty b_n\sin nx$
并给出 $ x $ 的区间。 

展成余弦级数同理。


\begin{Appendices}

\chapter{补充结论}

\sssubsection{反三角公式}

\begin{itemize}
    \item $ \arcsin x + \arccos x = \dfrac{\pi}{2}; $ 
    \item $ \arctan x + \textrm{arccot} x = \dfrac{\pi}{2}; $ 
    \item $ \arctan x + \arctan \dfrac{1}{x} = \begin{cases}
        \dfrac{\pi}{2},& x>0\\
        -\dfrac{\pi}{2},& x<0\\
    \end{cases} $ 
\end{itemize}

\sssubsection{$ n $ 次根式的极限}

\begin{itemize}
    \item $ {\displaystyle\lim_{n\rightarrow \infty}}\dsqrt[n]a = 1\ (a > 0); $ 
    \item $ {\displaystyle\lim_{n\rightarrow \infty}}\dsqrt[n]n = 1; $ 
    \item $ {\displaystyle\lim_{n\rightarrow \infty}}\dsqrt[n]{a_1^n+\cdots+a_n^n} = a_m\ (a_m = \max\{a_i\}); $ 
\end{itemize}

\sssubsection{递推数列求极限}

对数列 $ x_{n+1} = f(x_n) $ 求极限,方法如下。
\begin{itemize}
    \item 适当放缩以证明有界性;
    \item 做差、做商或求导证明单调性;
    \item 若其单调,由单调有界知$ \lim x_n $ 存在;
    \item 令 $ \lim x_n = a, $ 对原式两端取极限,有 $ a = f(a), $ 因此可以解得 $ a; $ 
    \item 若其不单调,则设 $ \lim x_n = a, $ 再利用夹逼定理证明前者确实成立。
\end{itemize}

\sssubsection{均值不等式}

\begin{equation*}
    \begin{aligned}
        \sqrt[n]{a_1\cdots a_n}\leq \dfrac{a_1+\cdots a_n}{n}\leq \dsqrt{\dfrac{a_1^2+\cdots a_n^2}{n}}
    \end{aligned}
\end{equation*}

\newpage

\sssubsection{函数不等式}

\begin{itemize}
    \item $ \sin x < x <\tan x\ ( x \in (0,\dfrac{\pi}{2})); $ 
    \item $ \sin x < x\ (x > 0),\ \ \sin x > x\ (x < 0); $ 
    \item $ e^x > 1 + x\ (x \neq 0); $ 
    \item $ \dfrac{x}{1+x}<\ln(1+x)<x\ (x \in (-1,0)\cup(0,\infty)); $ 
\end{itemize}

\sssubsection{零点定理应用}

若 $ f(x)\in C[0,1],f(0) = f(1), $ 则对任意 $ n\geq 2,\exists \xi\in [0,1] $ 
使得 $ f(\xi+\dfrac{1}{n}) = f(\xi). $ 
\begin{proof}
    构造 $ F(x) = f(x + \dfrac{1}{n}) - f(x), $ 
    由于 $ F(\dfrac{0}{n}),\cdots,F(\dfrac{n-1}{n}) $ 的平均值为 $ 0, $ 
    说明 $ 0 $ 是 $ F(x) $ 的函数值,因此必定有 $ \xi $ 使得
    $ F(\xi) = f(\xi + \dfrac{1}{n}) - f(\xi) = 0. $ 
\end{proof}

\sssubsection{比值的极限推导数}

设 $ f(x) $ 在 $ x = 0 $ 处连续,且 $ {\displaystyle\lim_{x\rightarrow 0}}\dfrac{f(x)}{x} = A, $ 则
$ f(0) = 0,f'(0) = A. $ 

\sssubsection{一类带绝对值函数的可导性}

设 $ f(x) = (x-x_0)^n|x-x_0|, $ 则 $ f(x) $ 在 $ x = x_0 $ 处 $ n $ 阶可导,但 $ n+1 $ 阶不可导。

\sssubsection{和差化积公式与二倍角公式}

\begin{itemize}
    \item 和差化积公式
    \begin{itemize}
        \item $\dis \sin(\alpha\pm\beta) = \sin\alpha\cos\beta+\cos\alpha\sin\beta $
        \item $\dis \cos(\alpha\pm\beta) = \cos\alpha\cos\beta-\sin\alpha\sin\beta $
        \item $\dis \tan(\alpha\pm\beta) = \dfrac{\tan\alpha+\tan\beta}{1+\tan\alpha\tan\beta}$
    \end{itemize}
    \item 二倍角公式
    \begin{itemize}
        \item $\dis \sin 2\alpha = 2\sin\alpha\cos\alpha $
        \item $\dis \cos 2\alpha = \cos^2\alpha - \sin^2\alpha $
        \item $\dis \tan 2\alpha = \dfrac{2\tan\alpha}{1+\tan^2\alpha}$
    \end{itemize}
    \item 降幂公式
    \begin{itemize}
        \item $\dis \sin^2 \alpha = \dfrac{1-\cos 2\alpha}{2} $ 
        \item $\dis \cos^2 \alpha = \dfrac{1+\cos 2\alpha}{2} $
    \end{itemize}
\end{itemize}

\sssubsection{幂指函数求导公式}

若 $ u = u(x),v = v(x) $ 均可导,且 $ u(x)>0, $ 
则有 $ (u^v)' = (e^{v\ln u})' = u^v(v\ln u)'. $ 

\sssubsection{高阶导数值的求法}

求高阶导数值时, 有如下的求法。
\begin{enumerate}
    \item 奇偶性 - 奇函数求偶阶导或偶函数求奇阶导为奇函数。
    \item 递推公式
    \begin{itemize}[parsep = 6pt]
        \item $ [(ax+b)^\alpha]^{(n)} = \dfrac{\alpha !}{(\alpha - n)!}(ax+b)^{\alpha-n}a^n = 
        \alpha(\alpha - 1)\cdots(a-n+1)(ax+b)^{\alpha-n}a^n, $ 
        
        特别地,$ \left(\dfrac{1}{ax+b}\right)^{(n)} = \dfrac{(-1)^nn!a^n}{(ax+b)^{n+1}}; $ 
        \item $ \dis (e^{ax+b})^{(n)} = a^ne^{ax+b},(a^x)^{(n)} = a^x\ln^n a;$ 
        \item $ [\ln(ax+b)]^{(n)} = a\left(\dfrac{1}{ax+b}\right)^{(n-1)} = \dfrac{(-1)^{n-1}(n-1)!a^n}{(ax+b)^n}; $ 
        \item $ [\sin(ax+b)]^{(n)} = a^n\sin(ax+b+\dfrac{n\pi}{2});
        [\cos(ax+b)]^{(n)} = a^n\cos(ax+b+\dfrac{n\pi}{2}). $ 
    \end{itemize}
    \item 莱布尼茨公式 - 乘积的高阶导数
    
    若 $ u = u(x),v = v(x) $ 均 $ n $ 阶可导,则有$$
        (uv)^{(n)} = \sum_{k=0}^n\mathrm{C}_n^k u^{(k)} v^{(n-k)}.
    $$ 
    \item 泰勒公式 - 一般而言,应用于 $ x = 0 $ 处。
\end{enumerate}

\sssubsection{拉格朗日证明包含两点导数的等式}

对区间 $ [a,c],[c,b] $ 分别应用拉格朗日,其中 $ c $ 根据题干结论确定。

\sssubsection{柯西中值定理证明包含两点导数的等式}

\begin{itemize}
    \item 对 $ f(x) $ 使用拉格朗日;
    \item 对 $ f(x),g(x) $ 使用柯西。
\end{itemize}

\sssubsection{导数与单调性的推断}

\begin{itemize}
    \item 已知一点导数符号 $ \nRightarrow $ 单调区间
    
    $ f(x) = \begin{cases}x+2x^2\sin \dfrac{1}{x},& x\neq 0\\ 0,& x = 0\end{cases} $ 
    \item 若 $ f(x) $ 在 $ x = x_0 $ 有一阶连续函数且 $ f'(x_0) > 0, $ 
    则在 $ x = x_0 $ 某邻域内有 $ f'(x)>0, f(x) $ 单调递增。
\end{itemize}

\sssubsection{一个包含 $ e^x, \sin, \cos $ 的积分的结论}

也即上导下抄。

\begin{equation*}
    \begin{aligned}
        &\int e^{\alpha x} \sin \beta x \mathrm{d}x 
        =& \dfrac{ \dis 
        \left|\begin{matrix}
            (e^{\alpha x})' & (\sin \beta x)' \\ 
            e^{\alpha x} & \sin \beta x 
        \end{matrix}\right|
        }{\alpha^2 + \beta^2} + C \\ 
        &\int e^{\alpha x} \cos \beta x \mathrm{d}x 
        =& \dfrac{ \dis 
        \left|\begin{matrix}
            (e^{\alpha x})' & (\cos \beta x)' \\ 
            e^{\alpha x} & \cos \beta x 
        \end{matrix}\right|
        }{\alpha^2 + \beta^2} + C \\ 
    \end{aligned}
\end{equation*}

\sssubsection{微分方程中“任意常数”的写法}

\begin{itemize}
    \item 等式中无 $ \ln \rightarrow C;  $ 
    \item 等式中有 $ \ln (\square) \rightarrow \ln C; $ 
    \item 等式中有 $ \ln |\square| \rightarrow \ln |C|; $ 
\end{itemize}

\sssubsection{$ \Gamma $ 积分}

$$
    \int_0^{+\infty} x^n e^{-x}\mathrm{d}x = n!
$$ 

\sssubsection{偏导数逆问题}

对偏导数求积分时,常数项为不含该未知量的函数,如
\begin{equation*}
    \begin{aligned}
        f^\pprime_{yy}(x,y) = 2 \Rightarrow
        f'_y(x,y) = \int f^\pprime_{yy}(x,y) \mathrm{d}y
        = 2y + \Attention{c(x)}
    \end{aligned}
\end{equation*}
然后利用其他已知条件求解原函数。

\sssubsection{欧拉积分、泊松积分}

欧拉积分可积,但不可求积,其主要包括
\begin{itemize}
    \item $ \dis e^{\pm x^2},e^{1/x},\dfrac{1}{\ln x}; $ 
    \item $ \sin x^2,\sin \dfrac{1}{x}, \dfrac{\sin x}{x}; $ 
    \item $ \cos x^2,\cos \dfrac{1}{x}, \dfrac{\cos x}{x} $ 等。
\end{itemize}

泊松积分为
$$
    \int_{-\infty}^\infty e^{-x^2}\mathrm{d}x = \sqrt{\pi}    
$$ 



\end{Appendices}

\end{document}
