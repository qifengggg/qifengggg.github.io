\begin{Appendices}[附录]

\chapter{补充结论}

\sssubsection{一类无穷阶可导的抽象函数}

若 $ f(x) $ 满足
\begin{itemize}
    \item $ f(x)=\int_0^xf(x)\mathrm{d}x + \Delta; $ 
    \item $ f'(x)=f(x)+\Delta; $ 
    \item $ f^\pprime(x) = f'(x) + \Delta,$ 
\end{itemize}

其中 $ \Delta $ 无穷阶可导,则 $ f(x) $ 无穷阶可导。

\sssubsection{幂与可导函数积的高阶导数}

设 $ f(x) = (x - x_0)^ng(x), $ 其中 $ g(x) $ 在 $ x = x_0 $ 处 $ n $ 阶可导且 $ g(x_0) \neq 0, $ 则
$$
    \forall  i < n, f^{(i)}(x_0) = 0, f^{(n)}\neq 0
$$  

\sssubsection{变限积分函数的初始条件}

对变限积分函数 $ g(x) = \dis \int_b^x f(t)\mathrm{d}t, $ 注意到 $ g(b) = 0. $ 

\sssubsection{三类根式的积分公式}

\begin{itemize}
    \item $ \dis \int \dfrac{\mathrm{d}x}{\dsqrt{a^2 - x^2}} =  \arcsin \dfrac{x}{a} + C \Rightarrow $
    
    $ \dis \int \dsqrt{a^2 - x^2} \mathrm{d}x = \dfrac{x}{2}\dsqrt{a^2 - x^2} + \dfrac{a^2}{2}\arcsin \dfrac{x}{a} + C $ 
    \item $ \dis \int \dfrac{\mathrm{d}x}{\dsqrt{a^2 + x^2}} =  \ln\left(x + \dsqrt{x^2+a^2}\right) + C \Rightarrow $
    
    $ \dis \int \dsqrt{a^2 + x^2} \mathrm{d}x = \dfrac{x}{2}\dsqrt{a^2 + x^2} + 
    \dfrac{a^2}{2}\ln\left(x + \dsqrt{x^2+a^2}\right) + C $ 
    \item $ \dis \int \dfrac{\mathrm{d}x}{\dsqrt{x^2 - a^2}} = \ln\left|x + \dsqrt{x^2 - a^2}\right| + C \Rightarrow $
    
    $ \dis \int \dsqrt{x^2 - a^2} \mathrm{d}x = \dfrac{x}{2}\dsqrt{x^2 - a^2} \Attention{-} 
    \dfrac{a^2}{2}\ln\left|x + \dsqrt{x^2 - a^2}\right| + C $ 

    注意此处是减不是加。
\end{itemize}

\sssubsection{区间再现公式的一个应用场景}

当出现 $ \dis \int_0^\frac{\pi}{2} \dfrac{\mathrm{d}t}{1+(\tan t)^k} = 
\int_0^\frac{\pi}{2} \dfrac{(\cos t)^k\mathrm{d}t}{(\sin t)^k+(\cos t)^k} $ 时,
常应用区间再现公式,此时原式为 $ \dfrac{1}{2}. $ 

出现 $ \cot t $ 时也类似。

\sssubsection{三角函数凑微分}

$$
    \mathrm{d}\left[\dfrac{\sin \theta}{\sin \theta + \cos \theta}\right] = 
    \dfrac{\mathrm{d}\theta}{(\sin \theta + \cos \theta)^2}
$$

\end{Appendices}
