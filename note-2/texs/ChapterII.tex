\chapter{一元函数微分学}

\section{导数与微分的概念}

\sssubsection{一极限在已知/未知导数存在情况下的区别}

已知 $ f(0) = 0,{\displaystyle\lim_{h\rightarrow 0}}\dfrac{1}{h}[f(2h)-f(h)] $ 存在不能推出
$ f(x) $ 在 $ x = 0 $ 处可导,因为此种推导过程中事实上是在导数存在的假设之下进行的。

但是,若已知 $ f(x) $ 在 $ x = x_0 $ 处可导,则 $ {\displaystyle\lim_{x\rightarrow x_0}}
\dfrac{f(x_0 - mh) - f(x_0-nh)}{h} = (m-n f'(x_0)), $ 因为此处可导性已知,故可以利用导数的定义。

\sssubsection{$ f(x)$ 可导与 $ |f(x)| $ 可导之间的关系 }

假设 $ x = x_0 $ 处 $ f(x) $ 连续。

\begin{itemize}
    \item 若 $ f(x_0)\neq 0, $ 则 $ |f(x)| $ 在此处可导 $ \Rightarrow f(x) $ 在此处可导(由保序性);
    \item 若 $ f(x_0)= 0, $ 则 $ |f(x)| $ 在此处可导 $ \Rightarrow f'(x_0)\whichis 0. $ 
\end{itemize}

\sssubsection{左右导数存在性与连续性的关系}

\begin{equation*}
    \begin{aligned}
        \left.\begin{matrix}
            f'_+(x_0)\textrm{存在}\Leftrightarrow f(x)\textrm{于此处右连续}\\
            f'_-(x_0)\textrm{存在}\Leftrightarrow f(x)\textrm{于此处左连续}\\
        \end{matrix}\right\}\textrm{但左右导数不相等}\Rightarrow f(x)\textrm{于此处连续}
    \end{aligned}
\end{equation*}

\sssubsection{一个分段函数连续、可导、有连续导数的条件}

对于$ \dis f(x) = \begin{cases}
    x^\alpha \sin \dfrac{1}{x^\beta} / x^\alpha \cos \dfrac{1}{x^\beta},
    & x > 0 \\ 0, & x \leq 0
\end{cases}, \beta > 0, $ 有

\begin{equation*}
    \begin{aligned}
        f(x)\textrm{在}x = x_0 \textrm{处}
        \begin{cases}
            \textrm{连续} \Leftrightarrow \alpha > 0\\ 
            \textrm{连续} \Leftrightarrow \alpha > 1\\ 
            \textrm{连续} \Leftrightarrow \alpha > 1 + \beta\\ 
        \end{cases}
    \end{aligned}
\end{equation*}

\section{导数与微分的计算}

\sssubsection{分段函数}

分段函数分段求,分断点处用定义。

\sssubsection{复合函数}

复合函数使用链式法则求解,对嵌套类函数 $ f(f(x)), $ 可以
考虑直接找出其表达式,也可以令 $ f(x) = u, $ 然后使用链式法则直接求解。

\sssubsection{隐函数}

\begin{itemize}
    \item \textbf{直接求导} - 对 $ F(x,y) = 0 $ 两端对 $ x $ 求导,解得 $ \dfrac{\mathrm{d}y}{\mathrm{d}x}. $ 
    \item 公式法 - 使用隐函数求导公式 $ \dfrac{\mathrm{d}y}{\mathrm{d}x} = -\dfrac{F_x'(x,y)}{F_y'(x,y)}. $
    \item 全微分 - $ F(x,y) = 0 $ 两端求全微分,解得 $ \dfrac{\mathrm{d}y}{\mathrm{d}x}. $  
\end{itemize}

\sssubsection{反函数}

设 $ x = f^{-1}(y) $ 由 $ y = f(x) $ 确定,则
\begin{itemize}
    \item 若 $ f(x) $ 可导,且 $ f'(x)\neq 0, $ 则$ \dfrac{\mathrm{d}x}{\mathrm{d}y} = 
    \dfrac{1}{\mathrm{d}y/\mathrm{d}x} = \dfrac{1}{f'(x)}. $ 
    \item 若 $ f(x) $ 二阶可导,且 $ f'(x)\neq 0, $ 则
    $ \dfrac{\mathrm{d}^2y}{\mathrm{d}x^2} = -\dfrac{f^\pprime(x)}{[f'(x)]^3}. $ 
\end{itemize}

\sssubsection{参数方程}

设 $ y = f(x) $ 由参数方程 $ \left\{\begin{matrix}
    x = x(t) \\ y = y(t)
\end{matrix}\right. $ 确定。此时,$ t = t(x),y = y(t(x)). $ 

对于其导数,有
\begin{itemize}
    \item 若 $ x(t),y(t) $ 均可导,且$ x'(t)\neq 0, $ 则 $ \dfrac{\mathrm{d}y}{\mathrm{d}x} = 
    \dfrac{\mathrm{d}y/\mathrm{d}t}{\mathrm{d}x/\mathrm{d}t} = \dfrac{y'(t)}{x'(t)}; $ 
    \item 若 $ x(t),y(t) $ 均二阶可导,且$ x'(t)\neq 0, $ 则 $ \dfrac{\mathrm{d}^2y}{\mathrm{d}x^2} = 
    \dfrac{y^\pprime(t)x'(t) - y'(t)x^\pprime(t)}{[x'(t)]^3}. $ 
\end{itemize}

\sssubsection{高阶导数}

\begin{itemize}
    \item 奇偶性
    
    奇函数偶阶导数或偶函数求奇阶导为奇函数,此时 $ f(0) = 0. $ 
    \item 递推公式
    
    \begin{itemize}[parsep = 6pt]
        \item $ [(ax+b)^\alpha]^{(n)} = \dfrac{\alpha !}{(\alpha - n)!}(ax+b)^{\alpha-n}a^n = 
        \alpha(\alpha - 1)\cdots(a-n+1)(ax+b)^{\alpha-n}a^n, $ 
        
        特别地,$ \left(\dfrac{1}{ax+b}\right)^{(n)} = \dfrac{(-1)^nn!a^n}{(ax+b)^{n+1}}; $ 
        \item $ \dis (e^{ax+b})^{(n)} = a^ne^{ax+b},(a^x)^{(n)} = a^x\ln^n a;$ 
        \item $ [\ln(ax+b)]^{(n)} = a\left(\dfrac{1}{ax+b}\right)^{(n-1)} = \dfrac{(-1)^{n-1}(n-1)!a^n}{(ax+b)^n}; $ 
        \item $ [\sin(ax+b)]^{(n)} = a^n\sin(ax+b+\dfrac{n\pi}{2});$
        \item $ [\cos(ax+b)]^{(n)} = a^n\cos(ax+b+\dfrac{n\pi}{2}). $ 
    \end{itemize}
    \item 莱布尼茨公式 - 乘积的高阶导数
    
    若 $ u = u(x),v = v(x) $ 均 $ n $ 阶可导,则有$$
        (uv)^{(n)} = \sum_{k=0}^n\mathrm{C}_n^k u^{(k)} v^{(n-k)}.
    $$ 
    \item 泰勒公式 - 一般而言,应用于 $ x = 0 $ 处
    
    求 $ n $ 阶导时,找到含有 $ x^n $ 的项,其求 $ n $ 阶导数后正好剩下常数。
\end{itemize}

\section{极重点 - 导数应用求切线和法线}

\sssubsection{直角坐标表示的曲线}

对直角坐标 $ y = f(x) $ 表示的曲线,有
\begin{itemize}
    \item \textbf{切线方程} - $ y - y_0 = f'(x_0)(x-x_0). $ 
    \item \textbf{法线方程} - $ y - y_0 = \dfrac1{f'(x_0)}(x-x_0). $ 
\end{itemize}

\sssubsection{参数方程表示的曲线}

对参数方程
$ \left\{\begin{matrix}
    x = x(t) \\ y = y(t)
\end{matrix}\right. $
表示的曲线,其切线斜率为

$$
    \dfrac{\mathrm{d}y}{\mathrm{d}x}\Big|_{x=x_0} = \dfrac{y'(t)}{x'(t)}\Big|_{t = t_0}
$$ 

注意 $ (x,y) = (x_0,y_0) $ 时 $ t $ 的取值。

\sssubsection{极坐标表示的曲线}

对极坐标 
$ \rho = \rho(\theta) $ 
表示的曲线,可以将其表示为
$ \left\{\begin{matrix}
    x = r(\theta)\cos\theta \\ y = r(\theta)\sin\theta
\end{matrix}\right. $

此时其就转化为参数方程。

\section{导数应用求渐近线}

\sssubsection{水平渐近线}

若 $ {\displaystyle\lim_{x\rightarrow +\infty}}f(x)=b $
或 $ {\displaystyle\lim_{x\rightarrow -\infty}}f(x) = b, $ 
则称 $ y=b $ 是 $ f(x) $ 的一条水平渐近线。

\sssubsection{垂直渐近线}

若 $ {\displaystyle\lim_{x\rightarrow x_0^+}}f(x)=\infty $ 
或 $ {\displaystyle\lim_{x\rightarrow x_0^-}}f(x)=\infty, $ 则称 $ x=x_0 $ 是 $ f(x) $ 的一条垂直渐近线,也叫铅直渐近线。

垂直渐近线只需要讨论分母为零的点或者函数无定义的端点。

\sssubsection{斜渐近线} 

若 $ {\displaystyle\lim_{x\rightarrow +\infty}}f(x)-(kx+b)=0 $ $ (x\rightarrow-\infty), $ 则称
$ y=kx+b $ 为斜渐近线。

具体而言,若\begin{enumerate}
    \item $ {\displaystyle\lim_{x\rightarrow +\infty}}\frac{f(x)}{x}=k; $ 
    \item $ {\displaystyle\lim_{x\rightarrow +\infty}}f(x)-kx = b, $ 
\end{enumerate}
则有斜渐近线 $ y=kx+b $ 。$ x\rightarrow-\infty $ 时同理。

注意,
\begin{itemize}
    \item 即使 $ {\displaystyle\lim_{x\rightarrow \infty}} \frac{y}{x} $ 存在,斜渐近线也不一定存在;
    \item 一侧不会同时存在水平渐近线和斜渐近线。
\end{itemize}

\sssubsection{求斜渐近线的简单方法}

对 $ y = f(x), $ 若能凑形式使得其形如 $ y = ax + g(x) $ 使得 $ {\displaystyle\lim_{x\rightarrow \infty}}g(x) = b, $ 
则 $ y = ax + b $ 即为对应方向的斜渐近线。

\section{导数应用求曲率}

曲率为$$
   k=\frac{|f^\pprime(x)|}{(1+(f'(x))^2)^{\frac{3}{2}}}
$$ 

曲率半径$ \rho $ 为 $ k $ 的倒数。

若已知曲率圆,则其切点处与原方程同函数值,同导数值。

\section{导数应用求极值与最值}

\sssubsection{极值第一充分条件}

若 $ f(x) $ 在 $ x = x_0 $ 连续,$ f'(x) $ 在 $ x = x_0 $ 的左右去心邻域内异号,则 $ f(x_0) $ 为极值点。

\sssubsection{极值第二充分条件}

若 $ f(x) $ 在 $ x = x_0 $ 处有 $ f'(x) = 0, $ 则若
$f^\pprime(x)> 0, $ 则 $ f(x_0) $ 为极小值,$f^\pprime(x)< 0, $ 则 $ f(x_0) $ 为极大值。

\sssubsection{极值第三充分条件}

若对 $ f(x) $ 和任意偶数 $ n $ 有 $ \forall i < n, f^{(i)}(x) = 0,f^{(n)} \neq 0, $ 
则若 $f^{(n)}(x)> 0, $ 则 $ f(x_0) $ 为极小值,若 $f^{(n)}(x)< 0, $ 则 $ f(x_0) $ 为极大值。

\sssubsection{费马引理}

可导函数的每一个可导的极值点都是驻点。

\section{导数应用求凹凸性与拐点}

注意,拐点确实是点。

\sssubsection{拐点第一充分条件}

若 $ f(x) $ 在 $ x = x_0 $ 连续,$ f^\pprime(x) $ 在 $ x = x_0 $ 的左右去心邻域内异号,则 $ f(x_0) $ 为拐点。

\sssubsection{拐点第二充分条件}

若 $ f(x) $ 在 $ x = x_0 $ 处有 $ f^\pprime(x) = 0, $ 则 $ (x_0,f(x_0)) $ 为拐点。

\sssubsection{拐点第三充分条件}

若对 $ f(x) $ 和任意奇数 $ n $ 有 $ \forall i < n, f^{(i)}(x) = 0,f^{(n)} \neq 0, $ 
则其为拐点。

\section{导数应用证明不等式}

其主要分为三种方法。

\begin{itemize}
    \item 单调性
    \item 凹凸性
    
    设 $ f(x) $ 可导,则其为凹函数等价于下面任一情况。
    \begin{itemize}[topsep = 0pt]
        \item $ f'(x) $ 单调递增;
        \item 曲线在其切线上方,即
        
        $ f(x) > f(x_0) + f'(x_0)(x-x_0),x \neq x_0 $ 
        \item 曲线在其割线下方,即
        
        $ f(x) < f(x_0) - \dfrac{f(b)-f(a)}{b-a}(x-a), x\in (a,b) $ 
    \end{itemize}
\end{itemize}

\section{导数应用求方程的根}

应用导数求方程的根时,以单调性结合零点定理。

\section{微分中值定理证明}

\sssubsection{含有一个点 $ \xi $ 的等式}

\begin{itemize}
    \item 若待证式中不含导数,则适用零点定理;
    \item 若待证式中含有导数,则应用零点定理。
    
    构造函数时,可以\begin{itemize}
        \item 观察待证式,如 $ f'(x_0)+g'(x_0)f(x_0) = 0 \Rightarrow \left[e^{g(x_0)}f(x_0)\right]' = 0. $ 
        \item 强行构造原函数,即
        \begin{itemize}
            \item 将待证式中的$ \xi $ 改为 $ x; $ 
            \item 积分以去导数符号并令 $ C = 0; $ 
            \item 移项至待证式左边并构造辅助函数。
        \end{itemize}
    \end{itemize}
\end{itemize}

\sssubsection{含有 $ \eta,\xi $ 两个点的等式}

\begin{itemize}
    \item 题设 $ \xi\neq\eta $ 时,分区间 $ (a,c),(c,b) $ 并用两次拉格朗日;
    
    对于 $ c, $ 需要先在题干结论中引入 $ c $ 并将其反解。
    \item 未明示 $ \xi\neq\eta $ 时,对待证式,若其两个变量能分离至两边,则
    将其分离至两边之后,使用拉格朗日或柯西将两边联系至同一个值,以证明其相等。
\end{itemize}

\sssubsection{含有高阶导数 $ (n \geq 2) $ 的等式或不等式}

当 $ n \geq 2 $ 时就可以考虑使用泰勒,若 $ n > 2, $ 必定使用泰勒。

泰勒展开时,$ x_0 $ 可以取中点和端点,但更常用的还是\textbf{极值与最值}等有性质的点。

