\chapter{函数~~极限~~连续}

\section{函数的性态}

\subsection{有界性的判定}

\begin{itemize}
    \item 若 $ {\displaystyle\lim_{x\rightarrow x_0}}f(x)=A, $ 则存在 $ \delta>0, $ 当
    $ 0<|x-x_0|<\delta $ 时,$ f(x) $ 有界;
    \item  若 $ f(x) $ 在 $ [a,b] $ 连续,则其在 $ [a,b] $ 有界;
    \item[\important]  若 $ f(x) $ 在 $ (a,b) $ 连续,且 $ {\displaystyle\lim_{x\rightarrow a^+}}f(x),
    {\displaystyle\lim_{x\rightarrow b^-}}f(x) $ 均存在,则其在 $ (a,b) $ 有界;
    \item $ f'(x) $ 在\underline{有限区间}有界 $ \Rightarrow f(x) $ 在该区间有界。
\end{itemize}

\subsection{导函数、原函数的奇偶性与周期性}

\sssubsection{导函数的奇偶性与周期性}

\begin{itemize}
    \item 可导奇函数的导函数为偶函数;
    \item 可导偶函数的导函数为奇函数;
    \item 可导周期函数的导函数为周期函数;
\end{itemize}

\sssubsection{原函数的奇偶性与周期性}

\begin{itemize}
    \item 连续奇函数的原函数均为偶函数;
    \item 连续偶函数的原函数仅有一个为奇函数,即 $ C = 0 $ 时;
    \item 周期函数的原函数为周期函数 $ \Rightarrow \dis \int_0^T f(t)\mathrm{d}t = 0. $ 
\end{itemize}

\section{极限的概念}

讨论数列最值,将其拆分为前 $ N $ 个与后无穷个,前者求最值,后者利用极限定义
可知其接近极限值。

讨论同时包含 $ \sin(x_n),\cos(x_n) $ 的抽象数列时,可以考虑
令 $ x_n = \begin{cases}
    \pi/2,& 2i + 1\\ -\pi/2,& 2i
\end{cases}, $ 利用 $ \sin,\cos $ 奇偶性的不同。

\section{重点 - 函数极限的计算}

\subsection{$ 0/0 $ 形}

\sssubsection{洛必达法则}

若 $ f(x),g(x) $ 
\begin{itemize}[topsep = 0pt]
    \item $ \lim f(x)=\lim g(x)=0/\infty; $
    
    可以推广为 $ \dfrac{\blacksquare}{\infty}; $
    \item $ f(x),g(x) $ 在 $ x_0 $ 某去心邻域内可导,且 $ g'(x)\neq 0 $ ;
    
    此处注意, $ \begin{cases}
        n\textrm{阶可导}&\Rightarrow \textrm{洛}n-1\textrm{次}+\textrm{导数定义}\\
        n\textrm{阶连续导数}&\Rightarrow \textrm{洛}n\textrm{次}
    \end{cases} $ 
    \item $\dis \frac{\lim f'(x)}{\lim g'(x)}=A(\textrm{或}\infty), $ 
\end{itemize}
则 $\dis \frac{\lim f(x)}{\lim g(x)}=A(\textrm{或}\infty). $ 

~

\sssubsection{等价代换}

当 $ x\rightarrow0 $ 时,有

\begin{itemize}
    \item $ \sin x \sim \tan x \sim \arcsin x \sim \arctan x \sim e^x - 1 \sim \ln(1+x) \sim x; $ 
    \item $ e^x - 1 -x \sim x - \ln(1+x) \sim 1 - \cos x \sim \dfrac{x^2}{2}; $ 
    \item $ (1+x)^{\alpha}-1\sim \alpha x;$
    \item $ x - \sin x \sim \arcsin x - x \sim \dfrac{x^3}{6};$ 
    \item $ \tan x - x \sim x - \arctan x \sim \dfrac{x^3}{3};$ 
    \item $ \tan x - \sin x \sim \arcsin x - \arctan x \sim \dfrac{x^3}{2}; $ 
\end{itemize}

对于以上等价无穷小,有

\begin{enumerate}
    \item 可变量代换,如 $ \sin \square \sim \square,\ \tan \square \sim \square,\cdots $ 
    \item $ x\rightarrow0 $ 时,
    $\dis a^x-1=e^{x\ln a} -1\sim x\ln a,\ \log_a(1+x)=\frac{\ln(x+1)}{\ln a}\sim \frac{x}{\ln a};$
    \item 若 $ x\rightarrow a $ ,可以令 $ t = x - a \rightarrow 0. $ 
    \item 不能在复合函数的自变量处做等价代换,如 $ x\rightarrow 0\nRightarrow f(x)\sim f(\sin x). $ 
\end{enumerate}

\sssubsection{泰勒公式}

\begin{itemize}
    \item $\dis e^x = \sum_{i=0}^n \dfrac{x^n}{n!}+ o(x^n)$;
    \item $\dis \cos x = 1 - \frac{x^2}{2} + \dfrac{x^4}{24} +\dots + \frac{(-1)^{n}x^{2n}}{(2n)!} + o(x^{2n})$ ;
    \item $\dis \sin x = x - \frac{x^3}{6} + \dots + \frac{(-1)^{n}x^{2n+1}}{(2n+1)!} + o(x^{2n+1}) $ ;
    \item $\dis \arcsin x = x + \frac{x^3}{6} +o(x^3)$ ;
    \item $\dis \tan x = x + \frac{x^3}{3} + o(x^3)$ ;
    \item $\dis \arctan x = x - \frac{x^3}{3} + o(x^3)$ ;
    \item $\dis \ln (1+x) = x - \frac{x^2}{2} + \frac{x^3}{3} + \dots + \frac{(-1)^{n-1}x^n}{n} + o(x^n) $;
    \item $\dis \ln(1-x) = -(x+\frac{x^2}{2} + \frac{x^3}{3}) + o(x^3)$;
    \item $\dis (1+x)^\alpha = 1+\sum_{k=1}^n C_\alpha^kx^k + o(x^n) $ ,
    其中 $\dis C_\alpha^k=\frac{\prod_{i = 0}^{k-1}(\alpha - i)}{k!} $ 

    如,$ \dis \sqrt{1+x} = 1 + \dfrac{1}{2}x - \dfrac{1}{8}x^2 + o(x^2); $ 
    \item $\dis \frac{1}{1-x} = \sum_{i=0}^n x^i + o(x^n) $ ;
    \item $\dis \frac{1}{1+x} = \sum_{i=0}^n (-1)^i x^i + o(x^n) $;
\end{itemize}

泰勒公式求极限时,\begin{itemize}
    \item 分子阶数不小于分母阶数;
    \item 加减不抵消,“齐头并进”;
    \item 可推广为 $ \square\rightarrow 0. $
\end{itemize}

\subsection{$ \infty/\infty $ 形}

主要方法有\begin{itemize}
    \item 洛必达;
    \item 抓大头,即每个因式保留高阶无穷大;
    
    $\dis x\rightarrow 0 \Rightarrow \ln^\alpha(x)\ll x^\beta \ll a^x \ll x^x, $ 
    其中 $ \alpha,\beta > 0, a > 1. $ 
\end{itemize}

\subsection{$ \infty - \infty $ 形}

主要方法有\begin{itemize}
    \item 通分(有分式时);
    \item 有理化(有根号时);
    \item 倒代换,即令 $ t = \dfrac{1}{x}. $ 
\end{itemize}

\subsection{$ 0^0 $ 与 $ \infty^0 $ 形}

若 $ {\displaystyle\lim_{x\rightarrow x_0}}u(x) = 0(\infty),{\displaystyle\lim_{x\rightarrow x_0}}v(x) = 0, $ 则
$ {\displaystyle\lim_{x\rightarrow x_0}}u(x)^{v(x)} = \exp\left({\displaystyle\lim_{x\rightarrow x_0}}v(x)\ln u(x)\right). $ 

\subsection{$ 1^\infty $ 形}

\begin{itemize}
    \item 若 $ {\displaystyle\lim_{x\rightarrow x_0}}u(x) = 0,{\displaystyle\lim_{x\rightarrow x_0}}v(x) = \infty, $ 则
    $ {\displaystyle\lim_{x\rightarrow x_0}}[1+u(x)]^{v(x)} = 
    \exp\left({\displaystyle\lim_{x\rightarrow x_0}}v(x)u(x)\right). $ 
    \item 若 $ {\displaystyle\lim_{x\rightarrow x_0}}u(x) = 1,{\displaystyle\lim_{x\rightarrow x_0}}v(x) = \infty, $ 则
    $ {\displaystyle\lim_{x\rightarrow x_0}}u(x)^{v(x)} = 
    \exp\left({\displaystyle\lim_{x\rightarrow x_0}}v(x)[u(x)-1]\right). $ 
\end{itemize}

事实上,有
\begin{equation*}
    \begin{aligned}
        {\displaystyle\lim_{x\rightarrow 0}}
        \left(\dfrac{\sum_{i=0}^n a_i^x}{n}\right)^\frac{1}{x}
        = \sqrt{\prod a_i}
    \end{aligned}
\end{equation*}

\section{已知极限反求参数}

若 $ {\displaystyle\lim_{x\rightarrow x_0}}\dfrac{f(x)}{g(x)} $ 存在且
$ g{\displaystyle\lim_{x\rightarrow x_0}}g(x) = 0, $ 则 $ {\displaystyle\lim_{x\rightarrow x_0}}f(x) = 0. $ 

若 $ {\displaystyle\lim_{x\rightarrow x_0}}\dfrac{f(x)}{g(x)} = A \Attention{\neq 0} $ 且
$ g{\displaystyle\lim_{x\rightarrow x_0}}f(x) = 0, $ 则 $ {\displaystyle\lim_{x\rightarrow x_0}}g(x) = 0. $ 

\sssubsection{例}

$ {\displaystyle\lim_{x\rightarrow 0}}
\dis \int_b^x \dfrac{\ln(1+t^3)}{t}\mathrm{d}t = 0 \Leftrightarrow b = 0. $

\begin{itemize}
    \item[\textbf{证明}]
    $ b = 0 $ 时原式显然成立。

    $ \dfrac{\ln(1+t^3)}{t} > 0 (t\neq 0)\Rightarrow b \neq 0 $ 时原式不成立。
    
    因此,$ b $ 只能为零。
\end{itemize}

\section{无穷小阶的比较}

\sssubsection{例}

设函数 $ f(x) $ 在 $ x = 0 $ 的某邻域内具有二阶连续导数,且 $ f(0)\neq 0,f'(0)\neq 0,f\pprime(0)
\neq 0, $ 则存在一组唯一的 $ \lambda_i,i=1,2,3 $ 使得 $ h\rightarrow 0 $ 时,有
$ \sum \lambda_if(ih) - f(0) $ 是 $ h^2 $ 的高阶无穷小。

\begin{itemize}
    \item[\textbf{一般法}]
    $ \sum \lambda_if(ih) - f(0) $ 是 $ h^2 $ 的高阶无穷小 $ \Rightarrow \sum \lambda_if(ih) - f(0) = 0; $ 

    对上式两边求导,有 $ \sum \lambda_i if'(ih) = 0; $ 

    对上式两边求导,有 $ \sum \lambda_i^2 if^\pprime(ih) = 0; $ 

    因此,有 $ \begin{pmatrix}
        1&1&1\\ 1&2&3\\ 1&4&9
    \end{pmatrix}\begin{pmatrix}
        \lambda_1 \\
        \lambda_2 \\
        \lambda_3 \\
    \end{pmatrix} = 
    \begin{pmatrix}
        1 \\
        2 \\
        3 \\
    \end{pmatrix}, $ 由于系数矩阵满秩,其有唯一解,因而得证。
    \item[\textbf{泰勒法}]
    将 $ f(h),f(2h),f(3h) $ 展开至二阶,代入
    $ {\displaystyle\lim_{h\rightarrow 0}}\dfrac{\sum \lambda_i f(ih) - f(0)}{h^2}, $ 
    然后和前述做法一致。
\end{itemize}

\section{重点 - 数列极限的计算}

\subsection{夹逼定理}

左边缩,右边放,两边极限相等。

放缩时,有不等式

\begin{itemize}
    \item $ 0<x<\pi/2, $ 则 $ \sin x<x<\tan x;
    \sin x < x < \pi/2 \sin x; 2/\pi x < \sin x < x; $ 
    
    利用 $ f(x) = \dfrac{\sin x}{x} $ 的性质证明。
    \item $ x > 0, x > \sin x; x < 0, x < \sin x; $ 
    \item $ e^x > 1 + x, x\neq 0; $ 
    \item $ \dfrac{x}{1+x}< \ln (x+1)< x, x > -1, x \neq 0. $ 
\end{itemize}

\subsection{单调有界定理}

对数列 $ x_{n+1} = f(x_n) $ 求极限,方法如下。
\begin{itemize}
    \item 适当放缩以证明有界性;
    \item 做差、做商或求导证明单调性;
    \item 若其单调,由单调有界知$ \lim x_n $ 存在;
    \item 令 $ \lim x_n = a, $ 对原式两端取极限,有 $ a = f(a), $ 因此可以解得 $ a; $ 
    \item 若其不单调,则设 $ \lim x_n = a, $ 再利用夹逼定理证明前者确实成立。
\end{itemize}

\subsection{定积分}

\begin{equation*}
    \begin{aligned}
        \int_a^b f(x)\mathrm{d}x&={\displaystyle\lim_{d\rightarrow 0}}\sum_{i=i}^nf(\xi_i)\dfrac{b-a}{n}
    \end{aligned}
\end{equation*}

其中,$ \xi_i\in\left[a+\dfrac{i-1}{n}(b-a),a+\dfrac{i}{n}(b-a)\right]. $ 

\section{间断点的判定}

设 $ x=a $ 为 $ f(x) $ 的一间断点,
\begin{enumerate}
    \item 若 $ {\displaystyle\lim_{x\rightarrow a^+}}f(x) $ 与 $ {\displaystyle\lim_{x\rightarrow a^-}}f(x) $ 均存在,
    则称 $ x=a $ 为 $ f(x) $ 的一个第一类间断点,其还能\textbf{且必须要}分为\begin{itemize}
        \item 可去间断点 - 
        $ {\displaystyle\lim_{x\rightarrow a^+}}f(x)={\displaystyle\lim_{x\rightarrow a^-}}f(x); $
        \item 跳跃间断点 - 
        $ {\displaystyle\lim_{x\rightarrow a^+}}f(x)\neq{\displaystyle\lim_{x\rightarrow a^-}}f(x); $
    \end{itemize}
    \item 若 $ {\displaystyle\lim_{x\rightarrow a^+}}f(x) $ 与 $ {\displaystyle\lim_{x\rightarrow a^-}}f(x) $ 
    中有至少一个不存在,则称其为第二类间断点。第二类间断点不用强制细分。

    第二类间断点可以分为
    \begin{itemize}
        \item 无穷间断点 - 左右极限至少有一个为无穷;
        \item 震荡间断点 - 左右极限至少有一个不存在,但不是无穷;
    \end{itemize}
\end{enumerate}

可能存在间断点的地方:
\begin{itemize}
    \item 初等函数的无定义点;
    \item 分段函数的分段点。
\end{itemize}

