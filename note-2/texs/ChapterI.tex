\chapter{函数~~极限~~连续}

\section{函数的性态}

\subsection{有界性的判定}

\begin{itemize}
    \item 若 $ {\displaystyle\lim_{x\rightarrow x_0}}f(x)=A, $ 则存在 $ \delta>0, $ 当
    $ 0<|x-x_0|<\delta $ 时,$ f(x) $ 有界;
    \item  若 $ f(x) $ 在 $ [a,b] $ 连续,则其在 $ [a,b] $ 有界;
    \item[\important]  若 $ f(x) $ 在 $ (a,b) $ 连续,且 $ {\displaystyle\lim_{x\rightarrow a^+}}f(x),
    {\displaystyle\lim_{x\rightarrow b^-}}f(x) $ 均存在,则其在 $ (a,b) $ 有界;
    \item $ f'(x) $ 在\underline{有限区间}有界 $ \Rightarrow f(x) $ 在该区间有界。
\end{itemize}

\subsection{导函数、原函数的奇偶性与周期性}

\sssubsection{导函数的奇偶性与周期性}

\begin{itemize}
    \item 可导奇函数的导函数为偶函数;
    \item 可导偶函数的导函数为奇函数;
    \item 可导周期函数的导函数为周期函数;
\end{itemize}

\sssubsection{原函数的奇偶性与周期性}

\begin{itemize}
    \item 连续奇函数的原函数均为偶函数;
    \item 连续偶函数的原函数仅有一个为奇函数,即 $ C = 0 $ 时;
    \item 周期函数的原函数为周期函数 $ \Rightarrow \dis \int_0^T f(t)\mathrm{d}t = 0. $ 
\end{itemize}

\section{极限的概念}

讨论数列最值,将其拆分为前 $ N $ 个与后无穷个,前者求最值,后者利用极限定义
可知其接近极限值。

讨论同时包含 $ \sin(x_n),\cos(x_n) $ 的抽象数列时,可以考虑
令 $ x_n = \begin{cases}
    \pi/2,& 2i + 1\\ -\pi/2,& 2i
\end{cases}, $ 利用 $ \sin,\cos $ 奇偶性的不同。

\section{重点 - 函数极限的计算}

\subsection{$ 0/0 $ 形}

\sssubsection{洛必达法则}

若 $ f(x),g(x) $ 
\begin{itemize}[topsep = 0pt]
    \item $ \lim f(x)=\lim g(x)=0/\infty; $
    
    可以推广为 $ \dfrac{\blacksquare}{\infty}; $
    \item $ f(x),g(x) $ 在 $ x_0 $ 某去心邻域内可导,且 $ g'(x)\neq 0 $ ;
    
    此处注意, $ \begin{cases}
        n\textrm{阶可导}&\Rightarrow \textrm{导}n-1\textrm{次}+\textrm{导数定义}\\
        n\textrm{阶连续导数}&\Rightarrow \textrm{导}n\textrm{次}
    \end{cases} $ 
    \item $\dis \frac{\lim f'(x)}{\lim g'(x)}=A(\textrm{或}\infty), $ 
\end{itemize}
则 $\dis \frac{\lim f(x)}{\lim g(x)}=A(\textrm{或}\infty). $ 

~

\sssubsection{等价代换}

当 $ x\rightarrow0 $ 时,有

\begin{itemize}
    \item $ \sin x \sim \tan x \sim \arcsin x \sim \arctan x \sim e^x - 1 \sim \ln(1+x) \sim x; $ 
    \item $ e^x - 1 -x \sim x - \ln(1+x) \sim 1 - \cos x \sim \dfrac{x^2}{2}; $ 
    \item $ (1+x)^{\alpha}-1\sim \alpha x;$
    \item $ x - \sin x \sim \arcsin x - x \sim \dfrac{x^3}{6};$ 
    \item $ \tan x - x \sim x - \arctan x \sim \dfrac{x^3}{3};$ 
    \item $ \tan x - \sin x \sim \arcsin x - \arctan x \sim \dfrac{x^3}{2}; $ 
\end{itemize}

对于以上等价无穷小,有

\begin{enumerate}
    \item 可变量代换,如 $ \sin \square \sim \square,\ \tan \square \sim \square,\cdots $ 
    \item $ x\rightarrow0 $ 时,
    $\dis a^x-1=e^{x\ln a} -1\sim x\ln a,\ \log_a(1+x)=\frac{\ln(x+1)}{\ln a}\sim \frac{x}{\ln a};$
    \item 若 $ x\rightarrow a $ ,可以令 $ t = x - a \rightarrow 0. $ 
\end{enumerate}

\sssubsection{泰勒公式}

\begin{itemize}
    \item $\dis e^x = \sum_{i=0}^n \dfrac{x^n}{n!}+ o(x^n)$;
    \item $\dis \cos x = 1 - \frac{x^2}{2} + \dfrac{x^4}{24} +\dots + \frac{(-1)^{n}x^{2n}}{(2n)!} + o(x^{2n})$ ;
    \item $\dis \sin x = x - \frac{x^3}{6} + \dots + \frac{(-1)^{n}x^{2n+1}}{(2n+1)!} + o(x^{2n+1}) $ ;
    \item $\dis \arcsin x = x + \frac{x^3}{6} +o(x^3)$ ;
    \item $\dis \tan x = x + \frac{x^3}{3} + o(x^3)$ ;
    \item $\dis \arctan x = x - \frac{x^3}{3} + o(x^3)$ ;
    \item $\dis \ln (1+x) = x - \frac{x^2}{2} + \frac{x^3}{3} + \dots + \frac{(-1)^{n-1}x^n}{n} + o(x^n) $;
    \item $\dis \ln(1-x) = -(x+\frac{x^2}{2} + \frac{x^3}{3}) + o(x^3)$;
    \item $\dis (1+x)^\alpha = 1+\sum_{k=1}^n C_\alpha^kx^k + o(x^n) $ ,
    其中 $\dis C_\alpha^k=\frac{\prod_{i = 0}^{k-1}(\alpha - i)}{k!} $ 

    如,$ \dis \sqrt{1+x} = 1 + \dfrac{1}{2}x - \dfrac{1}{8}x^2 + o(x^2); $ 
    \item $\dis \frac{1}{1-x} = \sum_{i=0}^n x^i + o(x^n) $ ;
    \item $\dis \frac{1}{1+x} = \sum_{i=0}^n (-1)^i x^i + o(x^n) $;
\end{itemize}

泰勒公式求极限时,\begin{itemize}
    \item 分子阶数不小于分母阶数;
    \item 加减不抵消,“齐头并进”;
    \item 可推广为 $ \square\rightarrow 0. $
\end{itemize}

\subsection{$ \infty/\infty $ 形}

主要方法有\begin{itemize}
    \item 洛必达;
    \item 抓大头,即每个因式保留高阶无穷大;
    
    $\dis x\rightarrow 0 \Rightarrow \ln^\alpha(x)\ll x^\beta \ll a^x \ll x^x, $ 
    其中 $ \alpha,\beta > 0, a > 1. $ 
\end{itemize}

\subsection{$ \infty - \infty $ 形}

主要方法有\begin{itemize}
    \item 通分(有分式时);
    \item 有理化(有根号时);
    \item 倒代换,即令 $ t = \dfrac{1}{x}. $ 
\end{itemize}

