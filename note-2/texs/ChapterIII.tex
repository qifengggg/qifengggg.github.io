\chapter{一元函数积分学}

\section{定积分的概念}

判定含有变限积分函数的不等式时,常可以将非变限积分函数放入积分号;
此时,可以比较被积函数。

可以使用凹凸性判定含有变限积分函数的不等式。利用曲线割线与凹凸性的关系,可以
很容易地比较一函数 $ f(x) $ 与 $ x $ 之间的几何上的关系,或者说 $ \dfrac{f(x)}{x} $ 与 $ 1 $ 的关系。

\section{不定积分的计算}

不定积分常见的计算方式如下。

\begin{itemize}
    \item 不定积分凑微分
    
    对 $ f(u) $ 及其原函数 $ F(u), $ 若 $ u = u(x) $ 可导,则有
    
    $ \dis \int f(u(x))u'(x)\mathrm{d}x = \int f(u(x))\mathrm{d}u(x) = F(u(x)) + C $ 
    \item 分布积分法
    
    设 $ u = u(x), v = v(x) $ 可导,则 $ \dis \int u\mathrm{d}v = uv - \int v\mathrm{d}u $ 
    \item 换元法
    
    设 $ x = \varphi(t) $ 可导,且 $ \phi'(t) \neq 0, $ 
    若 $ \dis \int f(\varphi(t))\mathrm{d}t = F(t) + C, $

    则 $ \dis \int f(x)\mathrm{d}x = \int f(\varphi(t))\mathrm{d}t = F(t) + C = F(\varphi^{-1}(x)) + C $ 
    \item 三角代换
    \begin{itemize}
        \item 对 $ \dsqrt{a^2 - x^2}, $ 令 $ x = a\sin t; $ 
        \item 对 $ \dsqrt{x^2 - a^2}, $ 令 $ x = a\sec t; $ 
        \item 对 $ \dsqrt{a^2 + x^2}, $ 令 $ x = a\tan t; $ 
    \end{itemize}
    \item 根式代换
    \begin{itemize}
        \item 令 $ \dsqrt[n]{ax+b} = t; $ 
        \item 令 $ \dsqrt[n]{\dfrac{ax+b}{cx+d}} = t; $ 
        \item 对同时有 $ \dsqrt[n]{ax+b} $ 和 $ \dsqrt[m]{ax+b} $ 的,令
        $ \dsqrt[l]{ax+b} = t, $ 其中 $ l $ 为 $ m,n $ 的最小公倍数;
    \end{itemize}
    \item 倒代换
    
    令 $ x = \dfrac{1}{t}, $ 仅在系数 $ \geq 2 $ 时予以考虑。
    \item 万能代换 - 三角有理式
    
    令 $ t = \tan \dfrac{x}{2}, $ 则 $ x = 2\arctan t, \sin x = \dfrac{2t}{1+t^2}, 
    \cos x = \dfrac{1-t^2}{1+t^2}; $
    \item 整体代换
    
    令复杂函数整体 $  = t. $ 
\end{itemize}

\section{定积分的计算}

\begin{itemize}
    \item 定积分凑微分
    
    对 $ f(u) $ 及其原函数 $ F(u), $ 若 $ u = u(x) $ 在 $ [a,b] $ 可导,则有
    
    $ \dis \int_a^b f(u(x))u'(x)\mathrm{d}x = \int_a^b f(u(x))\mathrm{d}u(x) = F(u(x))\big|_a^b + C $ 
    \item 分布积分法
    
    设 $ u = u(x), v = v(x) $ 在 $ [a,b] $ 可导,
    则 $ \dis \int_a^b u\mathrm{d}v = uv - \int_a^b v\mathrm{d}u $ 
    \item 换元法
    
    设 $ x = \varphi(t) $ 在 $ [a,b] $ 连续,$ x = \varphi(t) $ 在 $ [\alpha,\beta] $ 上有一阶连续导数,
    且 $ \varphi(\alpha) = a,\varphi(\beta) = b,\varphi $ 的值域为 $ [a,b], $ 则

    $ \dis \int_a^b f(x)\mathrm{d}x = \int_\alpha^\beta f(\varphi(t))\varphi'(t)\mathrm{d}t. $ 
    \item 奇偶性
    
    若  $ f(x) $ 在 $ [-a,a] $ 连续,则 $ \dis \int_{-a}^af(x)\mathrm{d}x = 
    \begin{cases}
        2\int_0^a f(x)\mathrm{d}x, & f(x)\textrm{是偶函数}\\ 
        0, & f(x)\textrm{是奇函数}
    \end{cases} $ 
    \item 周期性
    
    对周期为 $ T $ 的连续函数 $ f(x), $ 对任意常数 $ a, $ 有 $ \dis \int_a^{a+T}f(x)\mathrm{d}x = 
    \int_0^{T}f(x)\mathrm{d}x $ 
    \item Wallis公式
    
    \begin{equation*}
        \begin{aligned}
            \int_0^{\dfrac{\pi}{2}} \sin^n x\mathrm{d}x = 
            \int_0^{\dfrac{\pi}{2}} \cos^n x\mathrm{d}x = 
            \dfrac{(n-1)!!}{n!!}\left(\dfrac{\pi}{2}\right)^{\texttt{(int)!(n\%2)}}
        \end{aligned}
    \end{equation*}

    设 $ f(x) $ 在 $ [0,1] $ 连续,则 $ \dis \int_0^\pi xf(\sin x)\mathrm{d}x = \dfrac{\pi}{2}
    \int_0^\pi f(\sin x)\mathrm{d}x = \pi\int_0^{\frac{\pi}{2}} f(\sin x)\mathrm{d}x.$
    \item 区间再现公式
    
    在分子为分母的其中一项时,常用区间再现公式。进退维谷时,也可以考虑应用区间再现公式。
    
    $ \dis \int_a^b f(x)\mathrm{d}x \xlongequal{t = a + b - x} \int_a^b f(a + b - t)\mathrm{d}t
    = \dfrac{1}{2} \int_a^b \left(f(x)+f(a + b - t)\right)\mathrm{d}x $ 
    \item 平移变换
    
    令 $ x = t + b. $ 题目给出(类)周期函数条件,待求式积分上下限是周期整数倍但不重合,
    可以考虑平移变换。
\end{itemize}

\section{反常积分的计算}

反常积分实质上是定积分的极限,其可能不存在。

对于瑕点在上下限区间内的,利用区间可加性拆开。如此做时,必须满足极限的
四则运算法则,否则需要考虑规避未定式的化简方法。

\section{反常积分敛散性的判断}

\begin{enumerate}
    \item 反常积分定义
    \item 比较判别法
    \begin{itemize}
        \item 无穷积分,如 $ \dis \int_1^\infty f(x)\mathrm{d}x $ 
        
        此时比较 $ p $ 积分 $ \dis \int_1^\infty \dfrac{1}{x^p}\mathrm{d}x
        \begin{cases}
            p > 1, \textrm{收敛}\\p \leq 1, \textrm{发散}
        \end{cases} $ 
        对$ \dis {\displaystyle\lim_{x\rightarrow +\infty}} \dfrac{f(x)}{1/x^p} = 
        {\displaystyle\lim_{x\rightarrow +\infty}}x^pf(x) = l, $
        \begin{itemize}
            \item $ 0 < l < +\infty $ 时二者同阶同收敛;
            \item $ l = 0 $ 时 $ p $ 积分收敛则原积分收敛(大收则小收)
            \item $ l = +\infty $ 时 $ p $ 积分发散则原积分发散(小发则大发)
        \end{itemize} 
        \item 瑕积分,如 $ \dis \int_0^1 f(x)\mathrm{d}x, x = 0 $ 为瑕点
        
        此时比较 $ p $ 积分 $ \dis \int_0^1 \dfrac{1}{x^p}\mathrm{d}x
        \begin{cases}
            (0 < ) p < 1, \textrm{收敛}\\p \geq 1, \textrm{发散}
        \end{cases} $
        
        注意,若题目中明示原积分为瑕积分,则收敛的范围为 $ 0 < p < 1, $ 因为
        $ p \leq 0 $ 时原积分不是瑕积分;

        若题目中未明示原积分为瑕积分,则其范围为 $ p < 1. $ 
        \item 使用比较判别法时,找比较对象的方法是,被积因式无穷小则等价,无穷大则放缩。
    \end{itemize}
\end{enumerate}

\section{变限积分函数}

\sssubsection{变限积分函数的性质}

\begin{itemize}
    \item 若 $ f(x) $ 在 $ [a,b] $ 可积,则 $ F(x) $ 在 $ [a,b] $ 连续。注意,
    变限积分函数不一定可导,如 $ f(x) $ 有跳跃间断点时。
    \item 变限积分函数求导
    
    设 $$
        F(x)=\int_{\varphi_1(x)}^{\varphi_2(x)} f(t)\mathrm{d}t
    $$ 且 $ \varphi_1(x),\varphi_2(x) $ 可导,$ f(x) $ 连续,则有$$
        F'(x) = f(\varphi_2(x))\varphi_2'(x)-f(\varphi_1(x))\varphi_1'(x)
    $$ 

    变限积分求导,被积函数不含 $ x. $ 
    \item 若$ x_0 $ 为 $ f(x) $在$ [a,b] $ 上的一个跳跃间断点,则$ F(x) $在$ x_0 $ 连续,
    但不可导;
    
    若$ x_0 $ 为 $ f(x) $在$ [a,b] $ 上的一个可去间断点,则$ F(x) $在$ x_0 $ 处可导,
    但 $ F(x) $ 不是$ f(x) $ 的原函数。
\end{itemize}

\section{定积分应用求面积}

注意,面积一定是正数,至少不是负数。

\begin{itemize}
    \item 直角坐标
    
    $ \dis S = \int_a^b |y| \mathrm{d}x $
    \item 极坐标 $ \rho = \rho(\theta) $
    
    $ \dis S = \int_{\theta_1}^{\theta_2} \dfrac{\rho^2}{2}\mathrm{d}\theta $ 
    \item 参数方程 $ \dis \begin{cases}
        x = x(t) \\ y = y(t)
    \end{cases} $ 

    $ \dis S = \int_\alpha^\beta |y(t)x'(t)|\mathrm{d}t $ 
\end{itemize}

\section{定积分应用求体积}

\begin{itemize}
    \item 绕 $ x $ 轴旋转
    
    $ \dis V = \pi\int_0^{2\pi}y^2\mathrm{d}x $ 
    \item 绕 $ y $ 轴旋转
    
    $ \dis V = 2\pi\int_0^{2\pi}|xy|\mathrm{d}x $ 
    \item 平移 不绕坐标轴旋转
    
    此时可以\begin{enumerate}
        \item 微元法
        \item 二重积分(冲刺)
    \end{enumerate}
\end{itemize}

求面积、体积时,对单调函数,可以转而利用其反函数。

\section{定积分应用求弧长}

\sssubsection{直角坐标}

对曲线段 $ y = f(x),x\in[a,b] $ ,设 $ f(x) $ 有连续导数,则给定平面曲线段弧长元素
和弧长分别为
$$ \dis \mathrm{d}s = \dsqrt{1+f'^2(x)}\mathrm{d}x;
s = \int_a^b \dsqrt{1+f'^2(x)}\mathrm{d}x. $$

\sssubsection{参数方程}

若曲线能表示为 $ x = x(t),y=y(t),t\in[\alpha,\beta] $,且其在$ (\alpha,\beta) $ 内
有连续导数,则给定平面曲线段弧长元素和弧长分别为

$$ \dis \mathrm{d}s = \dsqrt[]{y'^2(t)+x'^2(t)}\mathrm{d}t;
s = \int_a^b \dsqrt[]{[y^2(t)]^2+[x'(t)]^2}\mathrm{d}t. $$

\sssubsection{极坐标}


若曲线能表示为 $ \rho = \rho(\theta),\theta\in[\theta_1,\theta_2], $
则给定平面曲线段弧长元素和弧长分别为

$$ \dis \mathrm{d}s = \dsqrt[]{\rho^2(\theta)+\rho'^2(\theta)}\mathrm{d}\theta ;
s = \int_a^b \dsqrt[]{\rho^2(\theta)+[\rho'(\theta)]^2}\mathrm{d}\theta. $$


\section{定积分应用求侧面积}

求侧面积的公式为
\begin{equation*}
    \begin{aligned}
        S = 2\pi \int_a^b \overbrace{|y(x)|}^\textrm{离轴距离} 
        \overbrace{\dsqrt{1+(y'(x))^2}\mathrm{d}x}^{\mathrm{d}s}&\textrm{绕}x\textrm{轴}\\ 
        S = 2\pi \int_a^b \overbrace{|x|}^\textrm{离轴距离} 
        \overbrace{\dsqrt{1+(y'(x))^2}\mathrm{d}x}^{\mathrm{d}s}&\textrm{绕}y\textrm{轴}\\ 
    \end{aligned}
\end{equation*}

在不同的坐标系下,只需要将离轴距离和 $ \mathrm{d}s $ 替换为对应的即可。

\section{定积分的物理应用}

\begin{enumerate}
    \item 做功
    
    变力沿直线,即 $ W = FS= \int_a^b f(s)\mathrm{d}s. $ 
    \item 受力
    \begin{itemize}
        \item 液体压力 $ F = \rho ghS $ 
        \item 万有引力 $ F = \dfrac{GMm}{r^2}, G $ 引力常数,$ M,m $ 质量,$ r $ 距离。
    \end{itemize}
\end{enumerate}

\section{证明含有积分的等式或不等式}

\begin{itemize}
    \item 单调性;
    \item 凹凸性;
    \item 微分中值定理;
    \item 定积分概念和六大积分法;
    \item 二重积分(冲刺)。
\end{itemize}