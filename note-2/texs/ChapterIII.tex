\chapter{一元函数积分学}

\section{定积分的概念}

判定含有变限积分函数的不等式时,常可以将非变限积分函数放入积分号;
此时,可以比较被积函数。

可以使用凹凸性判定含有变限积分函数的不等式。利用曲线割线与凹凸性的关系,可以
很容易地比较一函数 $ f(x) $ 与 $ x $ 之间的几何上的关系,或者说 $ \dfrac{f(x)}{x} $ 与 $ 1 $ 的关系。

\section{不定积分的计算}

不定积分常见的计算方式如下。

\begin{itemize}
    \item 不定积分凑微分
    
    对 $ f(u) $ 及其原函数 $ F(u), $ 若 $ u = u(x) $ 可导,则有
    
    $ \dis \int f(u(x))u'(x)\mathrm{d}x = \int f(u(x))\mathrm{d}u(x) = F(u(x)) + C $ 
    \item 分布积分法
    
    设 $ u = u(x), v = v(x) $ 可导,则 $ \dis \int u\mathrm{d}v = uv - \int v\mathrm{d}u $ 
    \item 换元法
    
    设 $ x = \varphi(t) $ 可导,且 $ \phi'(t) \neq 0, $ 
    若 $ \dis \int f(\varphi(t))\mathrm{d}t = F(t) + C, $

    则 $ \dis \int f(x)\mathrm{d}x = \int f(\varphi(t))\mathrm{d}t = F(t) + C = F(\varphi^{-1}(x)) + C $ 
    \item 三角代换
    \begin{itemize}
        \item 对 $ \dsqrt{a^2 - x^2}, $ 令 $ x = a\sin t; $ 
        \item 对 $ \dsqrt{x^2 - a^2}, $ 令 $ x = a\sec t; $ 
        \item 对 $ \dsqrt{a^2 + x^2}, $ 令 $ x = a\tan t; $ 
    \end{itemize}
    \item 根式代换
    \begin{itemize}
        \item 令 $ \dsqrt[n]{ax+b} = t; $ 
        \item 令 $ \dsqrt[n]{\dfrac{ax+b}{cx+d}} = t; $ 
        \item 对同时有 $ \dsqrt[n]{ax+b} $ 和 $ \dsqrt[m]{ax+b} $ 的,令
        $ \dsqrt[l]{ax+b} = t, $ 其中 $ l $ 为 $ m,n $ 的最小公倍数;
    \end{itemize}
    \item 倒代换
    
    令 $ x = \dfrac{1}{t}, $ 仅在系数 $ \geq 2 $ 时予以考虑。
    \item 万能代换 - 三角有理式
    
    令 $ t = \tan \dfrac{x}{2}, $ 则 $ x = 2\arctan t, \sin x = \dfrac{2t}{1+t^2}, 
    \cos x = \dfrac{1-t^2}{1+t^2}; $
    \item 整体代换
    
    令复杂函数整体 $  = t. $ 
\end{itemize}



