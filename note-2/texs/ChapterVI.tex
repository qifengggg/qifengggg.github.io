\chapter{二重积分}

\section{交换积分次序}

对直角坐标系 $ x,y $ 的交换,利用穿针法;

对极坐标 $ \theta,\rho $ 的交换,
\begin{itemize}
    \item 先 $ \theta $ 后 $ \rho, $ 则引一条射线;
    \item 先 $ \rho $ 后 $ \theta, $ 则可以做不同半径的圆弧,并考虑每个圆弧所
    照射到的角度;

    也可以将 $ \rho,\theta $ 理解(并暂时地符号替换)为直角坐标的 $ x,y, $ 
    此时按照直角坐标系交换积分次序的方式交换其次序。
\end{itemize}

\section{二重积分的计算}

\sssubsection{直角坐标化累次积分}

可以将二重积分化为直角坐标下的二次积分计算。

设有界闭区域 $ D = \{(x,y)|a\leq x\leq b,\varphi_1(x)\leq y\leq \varphi_2(x)\} $ ,其中
$ \varphi_1(x),\varphi_2(x) $ 在 $ [a,b] $ 上连续,$ f(x,y) $ 在 $ D $ 上连续,则有
$ \dis \iint\limits_{D}f(x,y)\mathrm{d}\sigma = \iint\limits_{D}f(x,y)\mathrm{d}x\mathrm{d}y
 = \int_a^b\mathrm{d}x\int_{\varphi_1(x)}^{\varphi_2(x)}f(x,y)\mathrm{d}y $ .

确定上下限时,先确定 $ x $ 的范围,然后沿 $ y $ 正方向穿针确定 $ y $ 的范围,穿入为下限,穿出为上限。

在 $ D = \{(x,y)|c\leq y\leq d,\varPsi_1(y)\leq x\leq \varPsi_2(y)\} $ 上同理。

\sssubsection{极坐标变换}

当被积函数为 $ f(x^2+y^2) $ 或者 $ D $ 为圆域,考虑极坐标变换 $ \left\{\begin{matrix}
    x = \rho\cos\theta\\ y = \rho\sin\theta
\end{matrix}\right. $ 使得 $ \mathrm{d}x\mathrm{d}y = \rho\mathrm{d}\rho\mathrm{d}\theta $ .

先确定 $ \theta $ 的范围,从原点引射线确定 $ \rho $ 范围,穿入为下限,穿出为上限,此时有$$
    \iint\limits_{D}f(x,y)\mathrm{d}x\mathrm{d}y = \int_\alpha^\beta \mathrm{d}\theta
    \int_{\rho_1(\theta)}^{r_2(\theta)}f(\rho\cos\theta,\rho\sin\theta)\rho\mathrm{d}\rho
$$ 

\sssubsection{奇偶性}

若 $ f(x,y) $ 在有界闭区域 $ D $ 上连续,若$ D $ 关于 $ x $ 轴对称,则$$
    \iint\limits_{D}f(x,y)\mathrm{d}\sigma = \begin{cases}
        0,& f(x,y)\textrm{对}y\textrm{为奇函数}\\
        \dis 2\iint\limits_{D_1}f(x,y)\mathrm{d}\sigma,& f(x,y)\textrm{对}y\textrm{为偶函数}\\
    \end{cases}
$$ 
$ D $ 关于 $ y $ 轴对称,函数关于 $ x $ 轴对称时类似。

区域对于 $ x $ 轴对称,函数看 $ y $ 轴;
区域对于 $ y $ 轴对称,函数看 $ x $ 轴;
 
\sssubsection{轮换对称性}

若积分域 $ D $ 关于 $ y = x $ 对称,或相对于积分域两坐标轴的相对位置相同,又或者将 $ x,y $ 对调后,
积分域的边界方程不变,则二重积分具有轮换对称性,即$$
    \iint\limits_{D}f(x,y)\mathrm{d}x\mathrm{d}y = \iint\limits_{D}f(y,x)\mathrm{d}x\mathrm{d}y
    =\red{\dfrac{1}{2}\iint\limits_{D}[f(x,y)+f(y,x)]\mathrm{d}x\mathrm{d}y}
$$ 
 
\sssubsection{形心公式}

若区域 $ D $ 的形心或几何中心坐标为 $ (\bar x,\bar y), $ 则
\begin{equation*}
    \begin{aligned}
        \iint\limits_{D}x\mathrm{d}x\mathrm{d}y = \bar xS_D\\ 
        \iint\limits_{D}y\mathrm{d}x\mathrm{d}y = \bar yS_D
    \end{aligned}
\end{equation*}

若 $ f(x,y) = ax + by + c $ 时,考虑使用形心公式。

\begin{itemize}
    \item[\textbf{例题}] 已知函数 $ f(x,y) $ 具有二阶连续偏导数,且 $ f(1,y) = f (x,1) = 0,$
    $ \dis \iint\limits_{D}f(x,y)\mathrm{d}x\mathrm{d}y = a, $ 

    其中 $ D = \{(x,y)|0\leq x\leq 1, 0\leq y \leq 1\}, $ 
    计算二重积分
    $ \dis \iint\limits_{D}xyf^\pprime_{xy}(x,y)\mathrm{d}x\mathrm{d}y. $ 
    \item[\textbf{方法}] 
    由题,由于 $ f(1,y) = f (x,1) = 0 $ 恒定,得其导数 $ f'(1,y) = f'(x,1) = 0.$

    由于原式中二阶偏导数 $ x $ 在前,
    \begin{equation*}
        \begin{aligned}
            \textrm{原式}&= \int_0^1 x\mathrm{d}x\int_0^y y\mathrm{d}f'_x(x,y) \\ 
            &= \int_0^1 \left[yf_x'(x,y)\Big|_0^1 - \int_0^1f'_x(x,y)\mathrm{d}y\right] 
            \quad{}(\textrm{交换积分次序处理}f_x') \\ 
            &= -\int_0^1\mathrm{d}y\int_0^1 x\mathrm{d}f(x,y)\\ &= 
            -\int_0^1\left[xf(x,y)\Big|_0^1 - \int_0^1f(x,y)\mathrm{d}x\right]\mathrm{d}y \\&= 
            \int_0^1\int_0^1f(x,y)\mathrm{d}x\mathrm{d}y \\&= 
            \iint\limits_{D}f(x,y)\mathrm{d}x\mathrm{d}y = a
        \end{aligned}
    \end{equation*}
\end{itemize}

\begin{itemize}
    \item[\textbf{例题}] 包含参数方程的二重积分求解
    
    设平面区域 $ D $ 由 $ \left\{\begin{matrix}
        x = t - \sin t \\ y = 1 - \cos t
    \end{matrix}\right. (0 \leq t \leq 2\pi)  $ 与 $ x $ 轴围成,

    计算二重积分 $ \iint\limits_{D}(x+2y)\mathrm{d}x\mathrm{d}y. $ 
    \item[\textbf{方法}] 
    \begin{equation*}
        \begin{aligned}
            \textrm{原式}&= \iint\limits_{D}x\mathrm{d}x\mathrm{d}y + \iint\limits_{D}2y\mathrm{d}x\mathrm{d}y
            \\&= \iint\limits_{D}(\pi+2y)\mathrm{d}x\mathrm{d}y\quad{}\textrm{(利用了形心公式)}
            \\ &= \int_0^{2\pi}\mathrm{d}x\int_{0}^{y(x)}(\pi+2y)\mathrm{d}y \quad
            \textrm{(此处}y\textrm{是哑变量,}y(x)\textrm{是}x\textrm{的函数)}
            \\ &= \int_0^{2\pi} \left(\pi y+2y^2\right) \mathrm{d}x\quad{} \textrm{令} x = \sin t 
            \\ &= \int_0^{2\pi}\left[\pi(1-\cos t) + (1-\cos t)^2\right](1-\cos t)\mathrm{d}t 
            \\ &= 3\pi^2 + 5\pi 
        \end{aligned}
    \end{equation*}
\end{itemize}

\begin{itemize}
    \item[\textbf{例题}] 设平面区域 $ D = \left\{ (x,y)||x|<y, (x^2+y^2)^3\leq y^4 \right\}, $ 
    
    计算二重积分 $ \dis \iint\limits_{D}\dfrac{x+y}{\dsqrt{x^2+y^2}}\mathrm{d}x\mathrm{d}y $ 
    \item[\textbf{方法}] 注意到 $ D $ 关于 $ y $ 轴对称,故有
    \begin{equation*}
        \begin{aligned}
            \textrm{原式} &= \iint\limits_{D} \dfrac{y}{\dsqrt{x^2+y^2}}\mathrm{d}y\mathrm{d}x
            \\ &= \int_\frac{\pi}{4}^\frac{\pi}{2}\mathrm{d}\theta \int_0^{\sin^2 \theta} \sin \theta 
            \rho\mathrm{d}\rho 
            \\ &= \int_\frac{\pi}{4}^\frac{\pi}{2} \sin^5 \theta\mathrm{d}\theta
        \end{aligned}
    \end{equation*}
\end{itemize}

\begin{itemize}
    \item[例题] 计算 $ \dis \iint\limits_{D}\dsqrt{|y-x^2|} \mathrm{d}x\mathrm{d}y, $ 其中
    $ D = \{(x,y)| -1\leq x\leq 1, 0\leq y\leq 2\}. $ 
    \item[\textbf{方法}] 注意到 $ D $ 关于 $ y $ 轴对称,且 $ y - x^2 $ 在
    曲线 $ y = x^2 $ 上方大于零,下方小于零。设在第一象限的 $ D $ 中,在曲线下方的部分为 $ D_1, $ 其他的为 $ D_2, $ 
    则
    \begin{equation*}
        \begin{aligned}
            \textrm{原式} = 2(\iint\limits_{D_1}\dsqrt{x^2-y}\mathrm{d}x\mathrm{d}y
            +\iint\limits_{D_2}\dsqrt{y-x^2}\mathrm{d}x\mathrm{d}y)
        \end{aligned}
    \end{equation*}
\end{itemize}

\begin{itemize}
    \item[\textbf{例题}] 设 $ J_i = \iint_{D_i}\dsqrt[3]{x-y}\mathrm{d}x\mathrm{d}y, $ 
    其中 $ D_1 = \{(x,y)|0\leq x\leq 1,0\leq y\leq 1\},$

    $D_2 = \{(x,y)|0\leq x\leq 1,0\leq y\leq \dsqrt{x}\},
    D_3 = \{(x,y)|0\leq x\leq 1,x^2\leq y\leq 1\}, $ 
    比较 $ J_i $ 的大小。
    \item[\textbf{方法}] 由轮换对称性知 $ J_1 = 0. $ 根据 $ J_i $ 被积函数,$ D_1 - D_2 $ 区域内部分
    的积分 $ <0, $ 因此 $ J_2  > 0, $ 同理 $ J_3 < 0. $ 
\end{itemize}

\begin{itemize}
    \item[\textbf{例题}] 设平面区域 $ D = \{(x,y)|1\leq x^2+y^2\leq 4, x\geq 0,y\geq 0\}, $ 
    求 $\dis \iint\limits_{D}\dfrac{x\sin (\pi \dsqrt{x^2+y^2})}{x+y}\mathrm{d}x\mathrm{d}y. $ 
    \item[\textbf{方法}] 由于轮换对称性,利用极坐标,有
    \begin{equation*}
        \begin{aligned}
            \textrm{原式} &= \dfrac{1}{2}\iint\limits_{D}\rho\sin(\pi\rho)\rho \mathrm{d}\rho\mathrm{d}\theta \\
            &= \int_0^\frac{\pi}{2} \mathrm{d}\theta \int_1^2 \rho\sin(\pi\rho)\mathrm{d}\rho = -\dfrac{3}{4}
        \end{aligned}
    \end{equation*}
\end{itemize}

\begin{itemize}
    \item[\textbf{例题}] \textbf{极坐标下的二重积分转化回直角坐标}
    
    计算二重积分 $ \dis \iint\limits_{D}\rho^2\sin \theta \dsqrt{1-\rho^2\cos 2\theta}\mathrm{d}r\mathrm{d}\theta, $ 
    其中 $ D = \left\{(\rho,\theta)|0\leq \rho \leq \sec\theta, 0\leq \theta\leq \dfrac{\pi}{4}\right\}. $ 
    \item[\textbf{方法}] 当题给极坐标式不好化简时,考虑将其化回直角坐标。
    \begin{equation*}
        \begin{aligned}
            \textrm{原式} &= \iint\limits_{D}y\dsqrt{1+y^2-x^2}\mathrm{d}x\mathrm{d}y \\ 
            &=\dfrac{1}{3}-\dfrac{\pi}{16}
        \end{aligned}
    \end{equation*}
\end{itemize}

\begin{itemize}
    \item[\textbf{例题}] 计算二重积分 
    $\dis \iint\limits_{D}(x-y)\mathrm{d}x\mathrm{d}y, $ 

    其中 $ D = \left\{(x,y)|(x-1)^2+(y-1)^2 \leq 2, y\geq x\right\}.$ 
    \item[\textbf{法一}] 正常求解(利用极坐标)
    \begin{equation*}
        \begin{aligned}
            \textrm{原式} &= \int_\frac{\pi}{4}^\frac{3\pi}{4} (\cos \theta -\sin \theta) \mathrm{d}\theta 
            \int_0^{2(\sin \theta + \cos \theta)} \rho^2\mathrm{d}\rho \\ &= 
            \dfrac{8}{3}\int_\frac{\pi}{4}^\frac{3\pi}{4} 
            (\cos \theta -\sin \theta)(\cos \theta + \sin \theta)^3 \mathrm{d}\theta 
            \\&= \dfrac{8}{3}\int_\frac{\pi}{4}^\frac{3\pi}{4} 
            (\cos \theta + \sin \theta)^3 \red{\mathrm{d} (\cos \theta + \sin \theta)}
        \end{aligned}
    \end{equation*}
    \item[\textbf{法二}] \textbf{平移变换}
    
    圆心不在原点时,考虑做平移变换将其移动至原点。

    令 $ u = x - 1, v = y - 1, $ 则题设变为

    $\dis \iint\limits_{D}(u-v)\mathrm{d}u\mathrm{d}v, $ 
    其中 $ D = \left\{(u,v)|u^2+v^2 \leq 2, v\geq u\right\}, $ 易得原式 $= -\dfrac{8}{3}. $ 
\end{itemize}

