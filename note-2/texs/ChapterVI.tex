\chapter{二重积分}

\section{交换积分次序}

对直角坐标系 $ x,y $ 的交换,利用穿针法;

对极坐标 $ \theta,\rho $ 的交换,
\begin{itemize}
    \item 先 $ \theta $ 后 $ \rho, $ 则引一条射线;
    \item 先 $ \rho $ 后 $ \theta, $ 则可以做不同半径的圆弧,并考虑每个圆弧所
    照射到的角度;

    也可以将 $ \rho,\theta $ 理解(并暂时地符号替换)为直角坐标的 $ x,y, $ 
    此时按照直角坐标系交换积分次序的方式交换其次序。
\end{itemize}

\section{二重积分的计算}

\sssubsection{直角坐标化累次积分}

可以将二重积分化为直角坐标下的二次积分计算。

设有界闭区域 $ D = \{(x,y)|a\leq x\leq b,\varphi_1(x)\leq y\leq \varphi_2(x)\} $ ,其中
$ \varphi_1(x),\varphi_2(x) $ 在 $ [a,b] $ 上连续,$ f(x,y) $ 在 $ D $ 上连续,则有
$ \dis \iint\limits_{D}f(x,y)\mathrm{d}\sigma = \iint\limits_{D}f(x,y)\mathrm{d}x\mathrm{d}y
 = \int_a^b\mathrm{d}x\int_{\varphi_1(x)}^{\varphi_2(x)}f(x,y)\mathrm{d}y $ .

确定上下限时,先确定 $ x $ 的范围,然后沿 $ y $ 正方向穿针确定 $ y $ 的范围,穿入为下限,穿出为上限。

在 $ D = \{(x,y)|c\leq y\leq d,\varPsi_1(y)\leq x\leq \varPsi_2(y)\} $ 上同理。

\sssubsection{极坐标变换}

当被积函数为 $ f(x^2+y^2) $ 或者 $ D $ 为圆域,考虑极坐标变换 $ \left\{\begin{matrix}
    x = \rho\cos\theta\\ y = \rho\sin\theta
\end{matrix}\right. $ 使得 $ \mathrm{d}x\mathrm{d}y = \rho\mathrm{d}\rho\mathrm{d}\theta $ .

先确定 $ \theta $ 的范围,从原点引射线确定 $ \rho $ 范围,穿入为下限,穿出为上限,此时有$$
    \iint\limits_{D}f(x,y)\mathrm{d}x\mathrm{d}y = \int_\alpha^\beta \mathrm{d}\theta
    \int_{\rho_1(\theta)}^{r_2(\theta)}f(\rho\cos\theta,\rho\sin\theta)\rho\mathrm{d}\rho
$$ 

\sssubsection{奇偶性}

若 $ f(x,y) $ 在有界闭区域 $ D $ 上连续,若$ D $ 关于 $ x $ 轴对称,则$$
    \iint\limits_{D}f(x,y)\mathrm{d}\sigma = \begin{cases}
        0,& f(x,y)\textrm{对}y\textrm{为奇函数}\\
        \dis 2\iint\limits_{D_1}f(x,y)\mathrm{d}\sigma,& f(x,y)\textrm{对}y\textrm{为偶函数}\\
    \end{cases}
$$ 
$ D $ 关于 $ y $ 轴对称,函数关于 $ x $ 轴对称时类似。

区域对于 $ x $ 轴对称,函数看 $ y $ 轴;
区域对于 $ y $ 轴对称,函数看 $ x $ 轴;
 
\sssubsection{轮换对称性}

若积分域 $ D $ 关于 $ y = x $ 对称,或相对于积分域两坐标轴的相对位置相同,又或者将 $ x,y $ 对调后,
积分域的边界方程不变,则二重积分具有轮换对称性,即$$
    \iint\limits_{D}f(x,y)\mathrm{d}x\mathrm{d}y = \iint\limits_{D}f(y,x)\mathrm{d}x\mathrm{d}y
    =\red{\dfrac{1}{2}\iint\limits_{D}[f(x,y)+f(y,x)]\mathrm{d}x\mathrm{d}y}
$$ 
 
\sssubsection{形心公式}

若区域 $ D $ 的形心或几何中心坐标为 $ (\bar x,\bar y), $ 则
\begin{equation*}
    \begin{aligned}
        \iint\limits_{D}x\mathrm{d}x\mathrm{d}y = \bar xS_D\\ 
        \iint\limits_{D}y\mathrm{d}x\mathrm{d}y = \bar yS_D
    \end{aligned}
\end{equation*}

若 $ f(x,y) = ax + by + c $ 时,考虑使用形心公式。

\begin{itemize}
    \item[\textbf{例题}] 已知函数 $ f(x,y) $ 具有二阶连续偏导数,且 $ f(1,y) = f (x,1) = 0,$
    
    $ \dis \iint\limits_{D}f(x,y)\mathrm{d}x\mathrm{d}y = a, $ 
    其中 $ D = \{(x,y)|0\leq x\leq 1, 0\leq y \leq 1\}, $ 
    计算二重积分

    $ \dis \iint\limits_{D}xyf^\pprime_{xy}(x,y)\mathrm{d}x\mathrm{d}y. $ 
    \item[\textbf{解}] 
    由题,由于 $ f(1,y) = f (x,1) = 0 $ 恒定,得其导数 $ f'(1,y) = f'(x,1) = 0.$

    由于原式中二阶偏导数 $ x $ 在前,
    \begin{equation*}
        \begin{aligned}
            \textrm{原式}&= \int_0^1 x\mathrm{d}x\int_0^y y\mathrm{d}f'_x(x,y) \\ 
            &= \int_0^1 \left[yf_x'(x,y)\big|_0^1 - \int_0^1f'_x(x,y)\mathrm{d}y\right] 
            \quad{}\textrm{交换积分次序处理}f_x' \\ 
            &= -\int_0^1\mathrm{d}y\int_0^1 x\mathrm{d}f(x,y)\\ &= 
            -\int_0^1\left[xf(x,y)\big|_0^1 - \int_0^1f(x,y)\mathrm{d}x\right]\mathrm{d}y \\&= 
            \int_0^1\int_0^1f(x,y)\mathrm{d}x\mathrm{d}y \\&= 
            \iint\limits_{D}f(x,y)\mathrm{d}x\mathrm{d}y = a
        \end{aligned}
    \end{equation*}
\end{itemize}