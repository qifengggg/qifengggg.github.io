\chapter{常微分方程}

\section{一阶微分方程}

\sssubsection{可分离变量}

对形如 $ g(y)\mathrm{d}y = f(x)\mathrm{d}x $ 的,
对两边求积分,得到 $ \dis \int g(y) \mathrm{d}y = \int f(x)\mathrm{d}x. $ 

\sssubsection{一阶齐次方程}

对形如 $ \dis \dfrac{\mathrm{d}y}{\mathrm{d}x} = f(\dfrac{y}{x}) $ 的方程,取
$ \dis u = \dfrac{y}{x}, $ 则有 $ y = xu \Rightarrow \dfrac{\mathrm{d}y}{\mathrm{d}x}
= ux + \dfrac{\mathrm{d}u}{\mathrm{d}x}, $ 代回原式,得
$ \dis \int \dfrac{\mathrm{d}u}{f(u) - u} = \int \dfrac{\mathrm{d}x}{x}, $ 
积分求得 $ u = \dfrac{y}{x} $ 后化简得到原方程的通解。

\subsection{一阶线性微分方程}

其形如 $ \dis \dfrac{\mathrm{d}y}{\mathrm{d}x} + P(x)y = Q(x), $ 其中未知数及其导数次数均为1.

其有公式$$
    y = e^{-\int P(x)\mathrm{d}x}
    \left(\int Q(x)e^{\int P(x)\mathrm{d}x}\mathrm{d}x+C\right)
$$ 

其中包含三个不定积分,其常数 $ C $ 已经事先提出。

括号中的不定积分事实上表示其被积函数的一个原函数。事实上,由于后面有任意常数 $ C, $ 
可以将括号中内容改写为变限积分函数 $ \dis \int_b^x f(x)\mathrm{d}x + C, $ 
此时可以进行需要代入具体值的操作;其中 $ b $ 可以为任意常数。

对于非常见类型的一阶微分方程,或者$ x $ 少 $ y $ 很多的方程,考虑转化为反函数求解。

\begin{itemize}
    \item[\textbf{例题} ] 对微分方程 $ y' + y = f(x), $ 其中 $ f(x) $ 是 $ R $ 上的连续周期函数,其周期为 $ T, $ 
    
    则方程存在唯一的以 $ T $ 为周期的解。
    \item[\textbf{证明}] 
    方程的通解是 $ \dis y = e^{-x}\left(\int e^xf(x)\mathrm{d}x + C\right), $ 注意到其
    可以写为 $ \dis e^{-x}\left(\int_0^x e^tf(t)\mathrm{d}t + C\right). $ 
    
    那么,当且仅当 $ y(T-x) - y(x) = 0 $ 
    时,$ y $ 以 $ T $ 为周期。
    
    \begin{equation*}
        \begin{aligned}
            y(T+x) - y(x) &= e^{-x-T}\left(\int_0^{x+T} e^tf(t)\mathrm{d}t + C\right)
            -e^{-x}\left(\int_0^{x} e^tf(t)\mathrm{d}t + C\right)
            \\&= e^{-x}\left[e^{-T}\left(\int_0^{T} + \int_{T}^{x+T}\right) e^tf(t)\mathrm{d}t + 
            \dfrac{C}{e^T} - C - \int_0^x e^tf(t)\mathrm{d}t \right]
            \\&= e^{-x}\left[e^{-T}\left(\int_0^{T} + e^T\int_{0}^{x}\right) e^tf(t)\mathrm{d}t + 
            \dfrac{C}{e^T} - C - \int_0^x e^tf(t)\mathrm{d}t \right]
            \\&= e^{-x}\left[e^{-T}\int_0^{T} e^tf(t)\mathrm{d}t + 
            (\dfrac{1}{e^T}-1)C\right] = 0
            \\&\Rightarrow C = \dfrac{\dis \int_0^T e^tf(t)\mathrm{d}t}{1-e^T}
        \end{aligned}
    \end{equation*}
    
    因此存在且仅存在一个 $ C $ 使得方程以 $ T $ 为周期。
\end{itemize}




\section{二阶线性常微分方程}

\sssubsection{齐次}

对形如 $ y^\pprime + py' + qy = 0 $ 的方程,解特征方程
$ r^2 + pr + q = 0, $ 得到特征根 $ r_1,r_2. $ 
根据特征根性质得到通解。具体而言,若 $ r_1,r_2 $ 为
\begin{itemize}
    \item 两互异实根 - 通解 $ C_1e^{r_1x} + C_2e^{r_2x}; $ 
    \item 两相等实根 - 通解 $ (C_1 + C_2x)e^{r_x}; $ 
    \item 一组共轭复根 $ \alpha\pm\beta i $ - 通解 $ e^{\alpha x}\left(C_1\cos \beta x + C_2\sin \beta x\right). $ 
\end{itemize}

\sssubsection{非齐次}

对形如 $ y^\pprime + py' + qy = f(x) $ 的方程,解特征方程得到齐次方程的通解;
根据 $ f(x) $ 性质确定一特解形状,“指数照抄,次数最高”。

对方程形如$
    y^\pprime + py'+qy = P_m(x)e^{ax},
$ 
则有一特解形如$$
    y^* = x^k Q_m(x) e^{ax}.
$$ 

对方程形如$
    y^\pprime + py'+qy = e^{\alpha x}\left[P_l(x)\cos \beta x+P_n(x)\sin \beta x\right],
$ 
则有一特解形如$$
    y^* = x^ke^{ax}\left[Q_m(x)\cos\beta x+R_m(x)\sin \beta x\right].
$$ 

代入原式求待定系数。

\sssubsection{高阶常系数线性齐次微分方程}

对方程形如 $ y^{(n)} +p_1y^{(n-1)} + \cdots + p_n y = 0, $ 
解对应特征方程 $ r^n + p_1r^{n-1}+ \cdots + p_n = 0, $ 一般 $ n \leq 4. $ 

对特征方程的根,若其
\begin{itemize}
    \item 有 $ S $ 个互异实根,则通解包含
    
    $ \dis \sum_{i = 1}^S C_ie^{r_ix} $ 
    \item 有 $ S $ 重实根,则通解包含
    
    $ \dis \sum_{i=1}^S (C_i x^{i-1})e^{rx} $ 
    \item 有共轭复根 $ \alpha \pm \beta i, $ 则通解包含
    
    $ \dis e^{\alpha x}\left(A\cos \beta x + B\sin \beta x\right) $ 
\end{itemize}

变限积分函数结合微分方程时,
\begin{itemize}
    \item 其被积函数 $ f(x) $ 无穷阶可导;
    \item 其有初始条件 $ \dis \int_a^a f(x)\mathrm{d}x = 0. $ 
\end{itemize}