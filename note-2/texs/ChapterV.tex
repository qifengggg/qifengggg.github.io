\chapter{多元函数微分学}

\section{多元函数概念}

\sssubsection{全微分定义}

设 $ z = f(x,y) $ 在 $ (x_0,y_0) $ 处有全增量 
$ \Delta z = f(x_0+\Delta x,y_0 + \Delta y) - f(x_0,y_0), $ 
若 $ \Delta z = A\Delta x + B\Delta y + o(\rho),\rho = \dsqrt{(\Delta x)^2+(\Delta y)^2}\rightarrow0 $ 
其中 $ A,B $ 不依赖于 $ \Delta x,\Delta y $ ,仅与 $ x_0,y_0 $ 有关,则称 $ z = f(x,y) $ 
在 $ (x_0,y_0) $ 处可微,而 $ A\Delta x+B\Delta y $ 称为 $ z = f(x,y) $ 在 $ (x_0,y_0) $ 处的
全微分,记为 $ \mathrm{d}z = A\Delta x + B\Delta y = A\mathrm{d}x + B\mathrm{d}y. $ 

其中 $ A = f'_x(x_0,y_0), B = f'_y(x_0,y_0), A\Delta x + B\Delta y $ 是全增量的线性主部。

求重极限时,考虑使用极坐标。

\sssubsection{去极限号}

对重极限,可以去极限号构造微分定义形式,如

$ \dis {\displaystyle\lim_{\begin{subarray}{c}x\rightarrow x_0\\y\rightarrow y_0\end{subarray}}} 
\dfrac{f(x,y)}{\rho^2} = l < +\infty \Rightarrow f(x,y) = o(\rho) = 0\cdot x + 0\cdot y + o(\rho), $  

从而推出函数在该点可微。

\section{多元复合函数求偏导数与全微分}

\begin{itemize}
    \item 一阶偏导数复合结构不变;
    \item 二阶混合偏导数连续必相等;
\end{itemize}

\section{多元隐函数求偏导数与全微分}

计算偏导数有三种方法,
\begin{itemize}
    \item 代入;
    \item 求导公式;
    \item 全微分。
\end{itemize}

\sssubsection{多元隐函数求导公式}

设 $ F(x,y,z) $ 在点 $ (x_0,y_0,z_0) $ 一邻域内有一阶连续偏导数,且 $ F(x_0,y_0,z_0) = 0, F_z'(x_0,y_0,z_0) \neq 0, $ 
则方程 $ F(x,y,z) = 0 $ 的某邻域内确定唯一的有一阶连续偏导数的函数 

$ z = f(x,y), z_0 = f(x_0,y_0),
\dfrac{\partial z}{\partial x} = - \dfrac{F'x}{F'z},\dfrac{\partial z}{\partial y} = - \dfrac{F'y}{F'z}.  $ 

在考虑一个方程确定的隐函数时,可以利用方程组思想。如,
\begin{itemize}
    \item 一个方程,三个变量,确定一二元函数;考虑系数矩阵 $ \begin{pmatrix}
        1&1&1
    \end{pmatrix} $ 
    \item 两个方程,三个变量,可确定两个一元函数;考虑系数矩阵 $ \begin{pmatrix}
        1&1&1\\0&1&1
    \end{pmatrix}. $ 
\end{itemize}

\begin{itemize}
    \item[\textbf{例题}] 设 $ y = y(x),z = z(x) $ 是由方程 $ z = xf(x + y) $ 和 
    $ F(x,y,z) = 0 $ 确定的函数,
    
    其中 $ f,F $ 分别具有一阶连续导数和
    一阶连续偏导数,求 $ \dfrac{\mathrm{d}z}{\mathrm{d}x}. $ 
    \item[\textbf{证明}]
    题目给出两题设,关于 $ x, $ 对前者求导数,对后者求偏导数。此时,
    \begin{equation*}
        \begin{aligned}
            &\dfrac{\mathrm{d}z}{\mathrm{d}x} = f(x+y)+xf'(x+y)(1+\dfrac{\mathrm{d}y}{\mathrm{d}x}) \\ 
            &\dfrac{\partial F}{\partial x}+\dfrac{\partial F}{\partial y}\dfrac{\mathrm{d}y}{\mathrm{d}x}
            + \dfrac{\partial F}{\partial z}\dfrac{\mathrm{d}z}{\mathrm{d}x} = 0
        \end{aligned}
    \end{equation*}
    联立以上两式,得到$ \dis \dfrac{\mathrm{d}z}{\mathrm{d}x} = \dfrac{(f+xf')F_2 - xf'F_1'}{F_2'+xf'F_3'}. $ 

    同理也可得到 $ \dfrac{\mathrm{d}y}{\mathrm{d}x}. $ 
\end{itemize}

\section{变量代换化简偏微分方程}

在变量代换时,换元变量视为中间变量。

\begin{itemize}
    \item[\textbf{例题}] 设函数 $ u = f(x,y) $ 具有二阶偏导数,且其满足等式
    
    $ \dis 4\dfrac{\partial^2 u}{\partial x^2}+12\dfrac{\partial^2 u}{\partial x\partial y}+
    5\dfrac{\partial^2 u}{\partial y^2} = 0. $
    
    确定 $ a,b $ 的值,使得等式在变换 $ \xi = x + ay, \eta = x + by $ 下简化为
    $ \dfrac{\partial u^2}{\partial \xi \partial \eta} = 0. $ 
    \item[\textbf{证明}]
    
    \begin{equation*}
        \begin{aligned}
            &\dfrac{\partial u}{\partial x} = 
            \dfrac{\partial u}{\partial \xi}\dfrac{\partial \xi}{\partial x} + 
            \dfrac{\partial u}{\partial \eta}\dfrac{\partial \eta}{\partial x} = 
            \dfrac{\partial u}{\partial \xi} + \dfrac{\partial u}{\partial \eta} \\
            &\dfrac{\partial u}{\partial y} = 
            a\dfrac{\partial u}{\partial \xi} + b\dfrac{\partial u}{\partial \eta} \\
            &\dfrac{\partial^2 u}{\partial x^2} = 
            \dfrac{\partial^2 u}{\partial \xi^2} + 
            \dfrac{\partial^2 u}{\partial \eta^2} +
            2\dfrac{\partial^2 u}{\partial \xi\eta} \\
            &\dfrac{\partial^2 u}{\partial y^2} = 
            a^2\dfrac{\partial^2 u}{\partial \xi^2} + 
            b^2\dfrac{\partial^2 u}{\partial \eta^2} +
            2ab\dfrac{\partial^2 u}{\partial \xi\eta} \\
            &\dfrac{\partial^2 u}{\partial x\partial y} = 
            a\dfrac{\partial^2 u}{\partial \xi^2} + 
            b\dfrac{\partial^2 u}{\partial \eta^2} +
            (a+b)\dfrac{\partial^2 u}{\partial \xi\eta} \\
        \end{aligned}
    \end{equation*}
    此时将其代入题设式,则满足
    \begin{equation*}
        \begin{aligned}
            &5a^2 + 12a + 4 = 5b^2 + 12b + 4 = 0\\ 
            &8 + 10ab + 12(a+b) = 0
        \end{aligned}
    \end{equation*}
    的 $ a,b $ 即为所求。
\end{itemize}

\section{求无条件极值}

求无条件极值有以下几种方法。

\begin{itemize}
    \item $ AC - B^2 $ 判别法
    
    函数 $ z = f(x,y) $ 在 $ P_0 $ 某邻域内有二阶连续偏导数,且 $ P_0 $ 为驻点,

    设 $ A = f^\pprime_{xx}(x_0,y_0),B = f^\pprime_{xy}(x_0,y_0),C = f^\pprime_{yy}(x_0,y_0), $ 则
    \begin{itemize}
        \item $ AC - B^2 > 0 $ 时, $ P_0 $ 是极值点,且若 $ A>0 $ ,其为极小值点,
        $ A<0 $ 时为极大值点;
        \item $ AC-B^2<0 $ 时其不是极值点;
        \item $ AC-B^2 = 0 $ 时该判别法失效。
    \end{itemize}
    \item 多元函数极值定义
    
    当前者失效时,利用定义。
    \begin{itemize}
        \item 证明是极值 - 利用多元函数保号性,有保号性则根据定义直接证明完毕;
        \item 证明不是极值 - 取不同路径证明其没有保号性。
    \end{itemize}
\end{itemize}

\section{求条件极值(边界条件)}

求条件极值,有以下几种方法。

\begin{itemize}
    \item 拉格朗日乘数法;
    \item 解 $ \varphi(x,y) = 0 $ 得到 $ y = y(x), $ 代入 $ f(x,y) $ 将其
    转化为一元函数极值;
    \item 极坐标 - 当边曲线为圆或者椭圆时考虑引入极坐标;
    \item 均值不等式;
    \item 柯西不等式(150);
\end{itemize}

\sssubsection{拉格朗日乘数法}

令 $ L = f(x,y) + \lambda\varphi(x,y), $ 前者称为拉格朗日函数,$ \lambda $ 是拉格朗日乘数。
对 $ L $ 关于 $ x,y,\lambda $ 都求偏导,得到
$$
    \begin{cases}
        f'_x(x,y) + \lambda\varphi'_x(x,y) = 0\\ 
        f'_y(x,y) + \lambda\varphi'_y(x,y) = 0\\ 
        \varphi(x,y) = 0
    \end{cases}
$$ 

此时消去 $ \lambda, $ 得到驻点。比较各处驻点处的函数值,得到最值。注意,

\begin{itemize}
    \item 目标函数的等价转化 - 对复合的目标函数,若外函数单调,则
    只需研究内函数;
    \item 消 $ \lambda $ 时,只能乘除非零因子,对可能为零的因子,
    需要讨论其取值;
    \item 边界曲线不封闭的,需要讨论其端点。
    \item 拉格朗日函数有多个的,考虑通过化简减少拉格朗日函数的数量。
\end{itemize}

\section{求闭区域最值}

求闭区域 $ D $ 上的连续函数 $ f(x,y) $ 的最大值与最小值时,
\begin{itemize}
    \item 内部驻点 - 求 $ f(x,y) $ 在 $ D $ 内的驻点即可能的极值点;
    \item 边界最值(条件极值)- 求 $ f(x,y) $ 在 $ D $ 边界上的条件极值;
    \item 比较以上各值,最大的为最大值,最小的为最小值。
\end{itemize}

