\chapter{真题}

\begin{quest}[2009T5]
    若 $ f^\pprime $ 不变号,且曲线 $ y = f(x) $ 在点 $ (1,1) $ 处的曲率圆为 $ x^2+y^2 = 2, $ 则函数
    在区间 $ (1,2) $ 内 $ \qline. $ 
    \begin{enumerate}
        \item 有极值点,无零点.
        \item 无极值点,有零点.
        \item 有极值点,有零点.
        \item 无极值点,无零点.
    \end{enumerate}
\end{quest}

\begin{quest}[2009T7]
    设 $ A,B $ 均为二阶矩阵,$ A^*,B^* $ 是其伴随矩阵,若有 $ |A| = 2,|B| = 3, $ 则
    $ \begin{bmatrix}
        O&A\\B&O
    \end{bmatrix} $ 的伴随矩阵是 $ \qline. $ 

    \begin{enumerate}
        \item     
        $ \begin{bmatrix}
            O&3B^*\\2A^*&O
        \end{bmatrix}. $ 
        \item     
        $ \begin{bmatrix}
            O&2B^*\\3A^*&O
        \end{bmatrix}. $
        \item     
        $ \begin{bmatrix}
            O&3A^*\\2B^*&O
        \end{bmatrix}. $
        \item     
        $ \begin{bmatrix}
            O&2A^*\\3B^*&O
        \end{bmatrix}. $
    \end{enumerate}
\end{quest}

\begin{quest}[2009T9]
    曲线 $ \begin{cases}
        x = \dis\int_0^{1-t}e^{u^2}\mathrm{d}u \\ y = t^2\ln(2-t^2)
    \end{cases} $ 在点 $ (0,0) $ 处的切线方程为 $ \qline. $ 
\end{quest}

\begin{quest}[2009T16]
    计算不定积分 $ \dis \int \ln(1+\dsqrt{\dfrac{1+x}{x}})\mathrm{d}x\,(x > 0). $ 
\end{quest}

\begin{quest}[2009T7]
    设 $ z = f(x+y,x-y,xy), $ 其中 $ f $ 具有二阶连续偏导数,
    求 $ \mathrm{d}z $ 与 $ \dfrac{\partial^2 z}{\partial x\partial y}. $ 
\end{quest}

\begin{quest}[2009T19]
    计算二重积分 
    $\dis \iint\limits_{D}(x-y)\mathrm{d}x\mathrm{d}y, $ 
    其中 $ D = \left\{(x,y)|(x-1)^2+(y-1)^2 \leq 2, y\geq x\right\}.$ 
\end{quest}

\begin{quest}[2009T20]
    设 $ y = y(x) $ 是区间 $ (-\pi,\pi) $ 过点 $ (-\frac{\pi}{\sqrt2},\frac{\pi}{\sqrt2}) $ 的
    光滑曲线;当 $ -\pi<x<0 $ 时,曲线上任一点法线过原点;当 $ 0 \leq x < \pi $ 时,函数 $ y(x) $ 
    满足 $ y^\pprime + y + x = 0, $ 求函数 $ y(x) $ 表达式。
\end{quest}

\begin{quest}[2009T22]
    设 $ A = \begin{bmatrix}
        1&-1&-1\\-1&1&1\\0&-4&-2\\
    \end{bmatrix},\xi_1 =\begin{bmatrix}
        -1\\ 1\\-2 
    \end{bmatrix}, $ 
    \begin{itemize}
        \item 求满足 $ A\xi_2 = \xi_1, A^2\xi_3 = \xi_1 $ 的所有向量;
        \item 对任意上述的 $ \xi_2,\xi_3, $ 证明 $ \xi_i,i = 1,2,3 $ 线性无关。
    \end{itemize}
\end{quest}

\begin{quest}[2009T23]
    设有二次型
    $$
        f(x_1,x_2,x_3) = ax_1^2 + ax_2^2 + (a-1)x_3^2
        +2x_1x_3 - 2x_2x_3
    $$
    \begin{itemize}
        \item 求二次型 $ f $ 的矩阵的所有特征值;
        \item 若二次型 $ f $ 的规范形为 $ y^2_1 + y^2_2, $ 求 $ a $ 的值。
    \end{itemize}
\end{quest}

