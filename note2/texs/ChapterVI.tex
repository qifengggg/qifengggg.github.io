\chapter{二次型}

\section{定义}

\sssubsection{二次型}

形如 
\begin{equation*}
    \begin{aligned}
        f(x_1,\cdots,x_n) &= 
        \sum_{i = 1}^n a_{ii}x_i^2 + 
        \sum_{1\leq i < j \leq n} 2 a_{ij}x_ix_j \\ 
        &\whichis X^\top AX,\quad X = (x_1,\cdots, x_n)^\top,\\ 
        & A = A^\top = \begin{pmatrix}
            a_{11}&a_{12}&\cdots&a_{1n}\\ 
            a_{12}&a_{22}&\cdots&a_{2n}\\ 
            \vdots&\vdots&\ddots&\vdots\\ 
            a_{1n}&a_{2n}&\cdots&a_{nn}\\ 
        \end{pmatrix}
    \end{aligned}
\end{equation*}

的二次齐次函数是二次型。

对称矩阵 $ A $ 称为二次型的矩阵。

二次型 $ f $ 的秩 $ r(f) = r(A) $ .

\sssubsection{标准型}

只含有平方项的二次型$ X^\top AX = \sum d_{i}x_i^2 $ 是标准型。二次型总能化为标准型。

标准型的矩阵是对角矩阵 $ \Lambda $ .

\sssubsection{规范型}

系数为 $ 0,\pm 1 $ 的标准型是规范型。

\sssubsection{惯性指数}

(二次型对应的)标准型中正(负)系数的个数为其正(负)惯性指数。

\section{矩阵合同}

\subsection{定义}

对 $ n $ 阶方阵 $ A,B, $
若存在可逆矩阵 $ P $ 使得 $ P^\top AP = B, $ 
则称 $ A,B $ 合同。

\subsection{可逆变换}

设 $ f = X^\top AX, $ 
令可逆矩阵 $ C: X = CY, $ 
代回,有 $ f = Y^\top C^\top ACY. $ 

因此,二次型的可逆变换即为合同变换。

\subsection{判定}

矩阵合同的判定方式如下。
\begin{equation*}
    \begin{aligned}
        \lambda_A = \lambda_B\Leftrightarrow A\sim B &\Rightarrow A,B \textrm{合同} \\
        & \Leftrightarrow X^\top AX, X^\top BX \textrm{的惯性指数相同}\\ 
        & \Leftrightarrow \lambda_A,\lambda_B  \textrm{正负数的个数相同} \\ 
    \end{aligned}
\end{equation*}

\subsection{二次型化为标准型}

\sssubsection{正交变换}

由正交变换将二次型化为标准型,即寻找正交矩阵 $ Q:X = QY $ 以将对称矩阵转化为对角矩阵。

事实上,第五章对称矩阵相似对角化和第六章正交变换很相似,但有些许不同。

\begin{itemize}
    \item 第五章:已知对称矩阵;
          
          第六章:已知二次型 $ \Rightarrow $ 对应的对称矩阵;
    \item 第五章:\begin{itemize}
            \item 求解特征值、特征向量;
            \item 施密特正交单位化;
        \end{itemize}
        
          第六章:同上
    \item 第五章:令 $ Q = (\gamma_1,\cdots,\gamma_n) $ 时
        有 $ Q^{-1}AQ = Q^\top AQ = \Lambda; $
          
          第六章:令 $ Q = (\gamma_1,\cdots,\gamma_n) $ \Attention{有 $ X = QY $ 时},
        二次型 $ f = \sum \lambda_iy_i^2, $
        此时特征值的正负个数为正负惯性指数。
\end{itemize}

\sssubsection{配方法}

由配方法将二次型化为标准型,即通过配平方将多项式变形至只有平方项。

具体而言,常用的公式有
$$
    \begin{cases}
        (a+b)^2 = a^2 + b^2 + 2ab \\
        (a+b+c)^2 = a^2 + b^2 + c^2 + 2ab + 2bc + 2ac \\
        (a+b)(a-b) = a^2 - b^2
    \end{cases}
$$ 

