\chapter{特征值与特征向量}

\section{定义与性质}

\begin{Def}[特征值与特征向量]

    对 $ n $ 阶方阵 $ A $ ,若其满足 $ A\vec\alpha = \lambda\vec\alpha $ ,
则称 $ \lambda $ 为 $ A $ 的特征值,$ \vec\alpha $ 为对应 $ \lambda $ 的
特征向量。
\end{Def}

\sssubsection{性质}

注意,对$ n $ 阶方阵 $ A $ ,有如下性质。
\begin{itemize}
    \item 其有 $ n $ 个特征值(包括重数);
    \item 特征向量 $ \alpha \neq \vec 0 $ ;
    \item 任意特征值 $ \lambda $ 对应无数个特征向量;
    
    若 $ \lambda $ 为单根,其对应的特征向量都线性相关;

    若其为 $ k $ 重根,其最多有 $ k $ 个相互无关的特征向量;
    \item 不同特征值对应的特征向量必定线性无关;
\end{itemize}

\section{特征值与特征向量的求解}

\sssubsection{未知矩阵}

当 $ A $ 未知的时候,可以凑定义,即强行构造 $ A\alpha = \lambda \alpha $ .

\sssubsection{已知矩阵}

当 $ A $ 元素已知时,可以使用公式法。

具体而言,
\begin{itemize}
    \item 通过特征多项式 $ |\lambda E-A| = 0 $ 解出所有$ n $ 个 $ \lambda $ ;
    \begin{itemize}
        \item 加减·消零·得公因式
    \end{itemize}
    \item 对解得的每一个 $ \lambda_0 $ ,求解 $ (\lambda_0 E-A)X = 0 $ 
    以得到\begin{itemize}
        \item 基解 - 其对应的$ k $ 个相互无关的特征向量,此处尽量将
        特征向量整数化;
        \item 通解 - 其对应的全部特征向量,此处 $ k_i \neq 0 $ ;
    \end{itemize}
\end{itemize}

求特征向量时,若未明示求无关的特征向量,则需要求出特征值对应的全部特征向量。

\sssubsection{秩为1的矩阵}

特别地,对 $ A_n:r(A) = 1 $ ,有 $ A = \alpha\beta^\top = \beta\alpha^\top $ ,
此时其特征值必定为 $ \lambda_1 = tr(A),\lambda_i = 0(i\neq 1) $ .

事实上,假设 $ n = 3 $ ,则有
\begin{itemize}
    \item 多项式 - $ |\lambda E - A | = \sum (\lambda - \lambda_i) $ ;
    \item 行列式拆分 - 
    
    $ |\lambda E - A| = \lambda^3 - (a_{11}+a_{22}+a_{33})\lambda^2 + S_2\lambda - |A| = 0 $,
    $ S_2 $ 是某些二阶子式,
    故有
    \begin{itemize}
        \item $ \sum \lambda = tr(A) $ ;
        \item $ \prod \lambda = |A| $ ;
    \end{itemize}
    又因 $ r(A) = 1 \Rightarrow S_2 = 0, |A| = 0 $ ,故 $ \lambda_1 = tr(A), \lambda_2 = \lambda_3 = 0$ .
\end{itemize} 

\sssubsection{表格法}

给定矩阵 $ A $ 及其特征值、特征向量 $ A\alpha = \lambda\alpha $ ,推断其他矩阵的特征值和特征向量时,适用表格法。此时注意,
$ \lambda_A $ 与 $ A $ 具有相同形式。

如,若求 $ A^2 $ 的特征值,有
$$
    A^2\alpha = AA\alpha = \lambda A\alpha = \lambda^2\alpha
$$ 

因此,有表格

\begin{table}[!htbp]\centering
    \begin{tabular}{ccc}
    \toprule
    $A        $&$ \lambda      $& $\alpha                  $ \\ \midrule
    $A^k      $&$ \lambda^k    $& $\alpha                  $ \\
    $A^m+kE   $&$ \lambda^m+kE $& $\alpha                  $ \\
    $A^{-1}   $&$ 1 /\lambda   $& $\alpha                  $ \\
    $A^*      $&$ |A|/\lambda  $& $\alpha                  $ \\
    $A^\top   $&$ \lambda      $&  无法断定                   \\
    $P^{-1}AP $&$ \lambda      $& $\Attention{P^{-1}\alpha}$ \\ \bottomrule
    \end{tabular}
\end{table}

\section{矩阵相似}

\begin{Def}[矩阵相似]

    对方阵 $ A_n,B_n $ ,若有可逆矩阵 $ P $ 使得 $ P^{-1}AP = B $ ,称 $ A,B $ 相似。    
\end{Def}

若矩阵$ A,B $ 相似,以下几个性质成立。

\begin{itemize}
    \item $ |A| = |B| $ ;
    \item $ r(A) = r(B) $ ;
    \item $ \lambda_A = \lambda_B $ ;

    事实上,$ |\lambda E - B| = |\lambda P^{-1}P - P^{-1}AP| = |P^{-1}(\lambda E - A)P| $. 
    \item $ tr(A) = tr(B) $ ;
    \item $ A + kE \sim B + kE $ ;
    \item $ P^{-1}A^nP = B^n, $ 注意此时 $ P $ 没有变化;
    \item $ A\sim B, B\sim C \Rightarrow A\sim C $ .
\end{itemize}

\section{相似对角化}

\begin{Def}[对角化]

    若存在可逆矩阵 $ P $ 使得 $ P^{-1}AP = \Lambda $ ,
    则称 $ A $ 可相似对角化。
\end{Def}

考虑三阶方阵 $ A_{3\times 3} $ ,由 $ P^{-1}AP = \Lambda \Rightarrow AP = P\Lambda $ ,
若设 
$$ 
P = (\alpha_1,\alpha_2,\alpha_3), \Lambda = \begin{pmatrix}
    k_1&&\\&k_2&\\&&k_3
\end{pmatrix} 
$$ 
则有
$$
    (A\alpha_1,A\alpha_2,A\alpha_3) = (k\alpha_1,k\alpha_2,k\alpha_3)
$$ 

由此,$ k_i $ 为 $ A $ 的特征值,$ \alpha_i $ 为其对应的特征向量。

推广至 $ n $ 阶,
\begin{itemize}
    \item $ \Lambda \xlongequal{\textrm{必定存在}} \begin{pmatrix}
        \lambda_1 &&& \\ &\lambda_2 &&\\ &&\ddots&\\ &&&\lambda_n
    \end{pmatrix}$ ;
    \item $ P = (\alpha_1,\cdots,\alpha_n) $ , $ \alpha_i $ 有$ n $ 个,是线性无关的特征向量;

    没有 $ n $ 个线性无关的特征向量时不能相似对角化。
\end{itemize}


\sssubsection{相似对角化的判定}

\begin{equation*}
    \begin{array}{rl}
        \left.\begin{matrix}
            \textrm{$ A $ 是对称矩阵}  \\
            \textrm{$ A $ 有 $ n $ 个互异特征值}
        \end{matrix}\right\}&\Rightarrow A\sim \Lambda \\
        &\Leftrightarrow
        \textrm{可逆矩阵 $ P $ 存在} \\ 
        &\Leftrightarrow
        \textrm{$ A $ 有 $ n $ 个无关特征向量}  \\
        &\Leftrightarrow
        \textrm{对 $ k $ 重根 $ \lambda_0, $ } 
        \textrm{$ \Attention{r(\lambda_0 E - A) = n - k} $ }
    \end{array}
\end{equation*}

\section{相似对角化的求解}

\subsection{普通方阵}

\begin{enumerate}
    \item 化简矩阵 $ A $ (抽象,含参)的具体元素,注意此时不适用初等变换;
    \item 求解特征值、特征向量;
    \item 令 $ P = (\alpha_1,\cdots,\alpha_n) $ ,此时有
    $$
        P^{-1}AP = \Lambda = \begin{pmatrix}
            \lambda_1&&&\\&\lambda_2&&\\&&\ddots&\\&&&\lambda_n
        \end{pmatrix}
    $$ 
    此处可以通过 $ \Lambda $ 反推 $ A $ 进行验证。
\end{enumerate}

\subsection{对称矩阵}

\sssubsection{对称矩阵的性质}

对称矩阵 $ A^\top = A $ 有如下的性质。

\begin{itemize}
    \item 不同特征值对应的特征向量必定正交;
        \begin{itemize}
            \item 事实上,若设 $ \lambda_1\neq\lambda_2, A\alpha_i = \lambda_i\alpha_i, i = 1,2, $
            则有 
            \begin{equation*}
                \begin{aligned}
                    A\alpha_1 = \lambda_1\alpha_1  &\Rightarrow
                    \alpha_1^\top A^\top = \alpha_1^\top A = \lambda_1 \alpha_1^\top \\ 
                    &\Rightarrow \alpha_1^\top A\alpha_2 = \lambda_2\alpha_1^\top\alpha_2 
                    = \lambda_1\alpha_1^\top\alpha_2 \\ 
                    &\Rightarrow (\lambda_2 - \lambda_1) \alpha_1^\top\alpha_2 = 0\\ 
                    &\because \lambda_1 \neq \lambda_2, \quad
                    \therefore \alpha_1^\top \alpha_2 = 0.
                \end{aligned}
            \end{equation*}
        \end{itemize}
    \item 必定存在可逆矩阵 $ P $ 使得 $ P^{-1}AP = \Lambda $ ;
    \item $ k $ 重特征值 $ \lambda_0 $ 必定对应 $ k $ 个线性无关的特征向量;
\end{itemize}

\sssubsection{施密特正交法}

对一组线性无关向量 $ \alpha_i,i = 1,2,3, $ 通过施密特正交法将其化为一组正交向量 $ \beta_i $ 的方法如下。

\begin{itemize}
    \item 令 $ \beta_1 = \alpha_1 $ ;
    \item 令 $ \beta_2 = \alpha_2 - \dfrac{(\alpha_2,\beta_1)}{(\beta_1,\beta_1)}\beta_1 $ ;
    \item 令 $ \beta_3 = \alpha_3 - \dfrac{(\alpha_3,\beta_1)}{(\beta_1,\beta_1)}\beta_1
    - \dfrac{(\alpha_3,\beta_2)}{(\beta_2,\beta_2)}\beta_2 $ ;
\end{itemize}

$ \beta_i $ 应当尽量整数化。此时得到的 $ \beta_i $ 两两正交。

当知道一组正交的 $ \alpha_1,\alpha_2 $ 时,则
\begin{equation*}
    \begin{aligned}
        \alpha_3 &\whichis \begin{pmatrix}
            \vec i&\vec j&\vec k\\ 
            a_1&a_2&a_3\\ 
            b_1&b_2&b_3
        \end{pmatrix}
        \begin{matrix}
            \\ \leftarrow \alpha_1 \\ \leftarrow\alpha_2
        \end{matrix}
        \\ &= c_i\vec i + c_j\vec j + c_k\vec k
    \end{aligned}
\end{equation*}

此时有 $ \alpha_3 = (c_1,c_2,c_3) $ 与 $ \alpha_1,\alpha_2 $
两两正交。

\sssubsection{对称矩阵的相似对角化}

对对称矩阵 $ A $ ,
\begin{itemize}
    \item 必定存在可逆 $ P $ 使得 $ P^{-1}AP = \Lambda $ ,其求解方法同普通方阵;
    \item 必定存在正交矩阵 $ Q $ 使得 $ Q^{-1}AQ = Q^\top AQ = \Lambda $ ;
    \begin{itemize}
        \item 化简矩阵 $ A $ ;
        \item 通过 $ |\lambda E - A| = 0, (\lambda_0E-A)X = 0 $ 得到特征值与特征向量;
        \item 处理得到的特征值和特征向量;
        \begin{itemize}
            \item 若特征值都互异,将特征向量单位化;
            
            $\dis \forall \alpha_i,\gamma_i = \dfrac{\alpha_i}{|\alpha_i|}$ 
            \item 对 $ k $ 重根 $ \lambda_0 $ 及其对应的 $ \alpha_1,\cdots,\alpha_k $ ,
            \begin{itemize}
                \item 若其全部正交,将全部单位化;
                \item 若其不正交,先对 $ \alpha_1,\cdots,\alpha_k $ 做施密特正交化,
                再将\textbf{全部}向量单位化为 $ \gamma_i $ .
            \end{itemize}
        \end{itemize}
        \item 令 $ Q = (\gamma_1,\cdots,\gamma_n) $ ,其中 $ \gamma_i $ 必定
        单位且正交,
        
        则必有 $ Q^{-1}AQ = Q^\top AQ = \Lambda = \begin{pmatrix}
            \lambda_1&&&\\&\lambda_2&&\\&&\ddots&\\&&&\lambda_n
        \end{pmatrix} $ 
    \end{itemize}
\end{itemize}

