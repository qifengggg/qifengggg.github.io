\chapter{向量}

\section{定义与性质}

\subsection{定义}

形如 $ \alpha = (a_1,a_2,\cdots,a_n)^\top $ 的称为$ n $ 维列向量,
形如 $ \alpha = (a_1,a_2,\cdots,a_n) $ 的称为$ n $ 维行向量。

\subsection{模}

$ |\alpha|\whichis \sqrt{a_1^2+\cdots a_n^2} $ 是向量 $ \alpha $ 的模。

注意,
\begin{itemize}
    \item 对 $ n $ 维向量 $ \alpha $ ,有
    $ \forall a_1 \equiv 0 \Leftrightarrow \alpha = \vec 0 \Leftrightarrow |\alpha| = 0 $ ;
    \item 若一向量 $ \alpha $ 的模为1,则称其为单位向量。非单位向量可以通过单位化变为单位向量,具体而言,
    只需将原向量的每一个分量除以其模。
\end{itemize}

\subsection{内积}

对$ n $ 维向量 $ \alpha,\beta $ ,称 $ (\alpha,\beta) \whichis \dis\sum_{i=1}^n a_ib_i $ 为
$ \alpha,\beta $ 的内积。

注意,若 $ A = \alpha\beta^\top \Leftrightarrow r(A) = 1 $ ,必有
$ (\alpha,\beta) = \alpha^\top\beta = \beta^\top\alpha = tr(A) $ .

\subsection{正交性}

若有 $ (\alpha,\beta) = 0 $ ,则称 $ \alpha $ 和 $ \beta $ 两向量正交。

\sssubsection{正交矩阵}

对矩阵 $ A $ ,若 $ AA^\top = A^\top A = E $ 则其为正交矩阵。

对正交矩阵,\begin{itemize}
    \item 每一内部向量都为单位向量;
    \item 任意两内部向量正交。
\end{itemize}

\subsection{性质}

对向量 $ \alpha = (a_1,\cdots,a_n)^\top,\beta = (b_1,\cdots,b_n)^\top $ ,
\begin{itemize}
    \item $ \alpha\pm\beta = (a_1\pm b_1,\cdots,a_n\pm b_n)^\top $ ;
    \item $ k\alpha = (ka_1,\cdots,ka_n)^\top $ ;
    \item 向量乘法没有交换律,没有消去律;
    \item $\forall \alpha, (\vec 0,\alpha) = 0, $ ;
    \item $ (\alpha,\beta) = (\beta,\alpha) = \alpha^\top\beta = \beta^\top\alpha = tr(A) $ ;
    \item $ (\sum k_i\alpha_i,\beta) = \sum k_i(\alpha_i,\beta) $ .
\end{itemize}

\section{向量组的线性相关与线性无关}

\subsection{定义}

对一组列向量 $ \alpha_1,\alpha_2,\cdots,\alpha_s $ ,对于 $ \sum k_i\alpha_i = 0 $ ,若存在
不全为零的一组 $ k_i $ 使其成立,则称 $ \alpha_i $ 线性相关,否则称其线性无关。

\subsection{性质}

\begin{itemize}
    \item 以下五点等价;
    \begin{itemize}
        \item 一组列向量 $ \alpha_1,\alpha_2,\cdots,\alpha_s $ 线性无关;
        \item $ \sum k_i\alpha_i = \vec 0,$当且仅当$ \forall k_i \equiv 0 $ ;
        \item $ (\alpha_1,\cdots,\alpha_s)(k_1,\cdots,k_s)^\top = 0,$当且仅当$ \forall k_i \equiv 0 $ ;
        \item 方程组 $ (\alpha_1,\cdots,\alpha_s)(x_1,\cdots,x_s)^\top = 0 $ 只有零解;
        \item $ r(\alpha_1,\cdots,\alpha_s) = s $ (\textrm{向量个数}) ;
    \end{itemize}
    \item 以下三种向量组组线性相关;
    \begin{itemize}
        \item 向量组中含有零向量;
        \item 向量组中含有成比例的向量;
        \item 向量组中含有向量能被同组向量线性表出;
    \end{itemize}
    \item 向量组中向量个数大于维数时,向量组也可线性表出。
    
    事实上,若设 $ A_{m\times n} = (\alpha_1,\cdots,\alpha_n) $ ,
    若 $ n > m $ 则有 $ r(A) \leq min(m,n) = m < n \Rightarrow r(\alpha_1,\cdots,\alpha_n) < n$,
    故其线性相关;
    \item 整体无关 $ \Rightarrow $ 部分无关;
    
    部分相关 $ \Rightarrow $ 整体相关;
    
    整体与部分是指个数。
    \item 原本无关 $ \Rightarrow $ 延长必无关;

    原本相关 $ \Rightarrow $ 缩短必相关;

    延长与缩短的是维数。
\end{itemize}

\subsection{向量组线性无关性的证明}

\sssubsection{向量组已知}

\begin{itemize}
    \item 若向量组可构成方阵,只需证 $ |\alpha_i| \neq 0 $ ;
    \item 若不能构成方阵,则需证明 $ r(\alpha_i) = s $ .
\end{itemize}

\sssubsection{向量组抽象未知}

考虑使用定义,即令$$
    \sum_{i=1}^s k_i\alpha_i = 0
$$ 
并证明 $ \forall k_i, k_i = 0 $ .

具体而言,可以
\begin{itemize}
    \item 重组法,即套定义,代入已知的无关向量组,并重组系数证明等式成立时系数必全为零;
    \item 等式乘,即将向量组组成的矩阵转化为多个矩阵的积,
    并证明其满秩。
\end{itemize}

\section{线性表示}

\subsection{定义}

设有向量 $ \alpha_1,\cdots,\alpha_s,\beta $ ,若存在一组 $ k_i $ 使得
$$
    \beta = \sum_{i = 1}^s k_i\alpha_i
$$ 
则称 $ \beta $ 可由 $ \alpha_i $ 线性表出。

\subsection{性质}

\begin{itemize}
    \item $ \alpha_1,\cdots,\alpha_m $ 线性无关
        $ \Leftrightarrow \forall \alpha_i $ 不可由剩余向量线性表出;
    \item 以下五点等价;
    \begin{itemize}
        \item $ \beta $ 可由 $ \alpha_1,\cdots,\alpha_s $ 线性表出;
        \item $ \beta = \sum_{i = 1}^s k_i\alpha_i $ 中任意 $ k_i $ 均存在;
        \item $ (\alpha_1,\cdots,\alpha_s)(k_1,\cdots,k_s)^\top = \beta $ 中任意 $ k_i $ 均存在;
        \item $ (\alpha_1,\cdots,\alpha_s)(x_1,\cdots,x_s)^\top = \beta $ 必有解;
        \item $ r(\alpha_1,\cdots,\alpha_s) = r(\alpha_1,\cdots,\alpha_s,\beta) $ .
    \end{itemize}
    \item 以少表多,多必相关。
    
    具体而言,若 $ \beta_1,\cdots,\beta_t $ 可由 $ \alpha_1,\cdots,\alpha_s $ 
    线性表出,且 $ s<t $ ,则 $ \beta_i $ 必线性相关。
\end{itemize}

\section{向量组等价}

\sssubsection{矩阵等价}

设有矩阵 $ A_{m\times n},B_{m\times n} $ ,若 $ A $ 可经有限次变换得到 $ B $ ,则称
矩阵 $ A,B $ 等价。

注意,
\begin{itemize}
    \item 存在可逆 $ P,Q $ 使得 $ PAQ = B \Leftrightarrow A,B $ 等价;
    \item $ A,B $ 等价 $ \Leftrightarrow r(A) = r(B) $ ;
\end{itemize}

\sssubsection{向量组等价}

设有向量组 $ {\rm (I)}: \alpha_1,\cdots,\alpha_s $ 与 $ {\rm (II)}: \beta_1,\cdots,\beta_t $ .
若
\begin{itemize}
    \item $ \rm (I) $ 可由 $ \rm (II) $ 表示,
    
    即对任意 $ \alpha_i $ ,其可由 $ \beta_1,\cdots,\beta_t $ 线性表出,

    即对矩阵 $ (\beta_1,\cdots,\beta_t|\alpha_1,\cdots,\alpha_s) $ ,每一列均有解;
    \item $ \rm (II) $ 可由 $ \rm (I) $ 表示,
    
    即对任意 $ \beta_j $ ,其可由 $ \alpha_1,\cdots,\alpha_s $ 线性表出,

    即对矩阵 $ (\alpha_1,\cdots,\alpha_s|\beta_1,\cdots,\beta_t) $ ,每一列均有解;
\end{itemize}

则称向量组 $ \rm (I),\rm (II) $ 等价,
此时,有充要条件$$
    r({\rm I}) = r({\rm II}) = r({\rm I,II})
$$ 

此时还有向量组 $ \rm (I),\rm (II) $ 等价 $ \Rightarrow $ 
矩阵 $ \rm (I),\rm (II) $ 等价。

\section{极大无关组}

\sssubsection{定义}

对$ {\rm (I)}: \alpha_1,\cdots,\alpha_s $中的一组向量 $ \alpha_1,\cdots,\alpha_t $ ,
若其满足
\begin{itemize}
    \item $ \alpha_1,\cdots,\alpha_t $ 线性无关;
    \item $ \rm (I) $ 中任意剩余向量与 $ \alpha_1,\cdots,\alpha_t $ 
    组成的向量组线性相关;
\end{itemize}
则称$ \alpha_1,\cdots,\alpha_t $ 为向量组 $ \rm (I) $ 的极大无关组。

\sssubsection{性质}

若 $ \alpha_1,\cdots,\alpha_t $ 为 $ \rm (I) $ 的极大无关组,则
\begin{itemize}
    \item $ \rm (I) $ 中剩余向量可由该组向量线性表出;
    \item $ r({\rm I}) = t $ .
\end{itemize}

\sssubsection{求极大无关组}

已知向量组 $ {\rm (I)}: \alpha_1,\cdots,\alpha_s $ 时求其极大无关组的方法如下。

\begin{enumerate}
    \item 将 $ \rm (I) $ 以列向量的形式构成一矩阵 $ A $ ;
    \item 通过初等行变换将 $ A $ 转化为阶梯矩阵 $ B $ ;
    \item 每一阶取一列,构成极大无关组。
\end{enumerate}

