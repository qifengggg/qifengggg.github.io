\chapter{矩阵}

\section{矩阵定义及特殊矩阵}

\subsection{矩阵}

由 $ m\times n $ 个元素组成的$ m $ 行 $ n $ 列数表 $ (a_{ij})_{m\times n}$ 称为一个 $ m\times n $ 阶矩阵。

若矩阵 $ A,B $ 的行列数相等,则称其为同型矩阵。若对任意 $ i=1,2,\dots,m, j = 1,2,\dots,n $ 都有
$ a_{ij} = b_{ij} $ ,则称矩阵 $ A,B $ 相等,记为 $ A = B $ .

\subsection{特殊矩阵}

设矩阵 $ A = (a_{ij})_{m\times n} $,此时一部分特殊矩阵如下。

\begin{itemize}
    \item \textbf{零阵$ O $ }\quad 对任意 $ i,j $ 有 $ a_{ij} = 0 $ ;
    \item \textbf{单位矩阵$ E_n $ }\quad 主对角元素 $ a_{ii} = 1 $ ,其余元素等于零;
    \item \textbf{对称矩阵}\quad $\forall i,j $ 有 $ a_{ij} = a_{ji} $ ;
    \item \textbf{反对称矩阵}\quad $\forall i,j $ 有 $ a_{ij} = -a_{ji} $ ;
    \item \textbf{转置矩阵 $ A^\top $ }\quad $ A^\top = (a_{ji})_{n\times m} $ ;
    \item \textbf{正交矩阵}\quad 对方阵 $ A , AA^\top = A^\top A = E $ ;
\end{itemize}

其中,转置矩阵有如下性质。

\begin{itemize}
    \item $ (A^\top)^\top = A $ ;
    \item $ |A^\top|= |A| $ ;
    \item $ (kA)^\top = kA^\top$ ;
    \item $ (AB)^\top = B^\top A^\top $ ;
    \item $ \begin{pmatrix}
        A&0\\0&B
    \end{pmatrix}^\top = \begin{pmatrix}
        A^\top&0\\0&B^\top
    \end{pmatrix} $ ;
\end{itemize}

\subsection{伴随矩阵}

对方阵 $ A $ ,有
$$
    A^* \whichis \begin{pmatrix}
        A_{11} & A_{21} & \cdots & A_{n1}\\
        A_{12} & A_{22} & \cdots & A_{n2}\\
        \vdots & \vdots & \ddots & \vdots\\
        A_{1n} & A_{2n} & \cdots & A_{nn}\\
    \end{pmatrix} \whichis (A_{ij})^\top
$$ 

其中 $ A_{ij} $ 是 $ a_{ij} $ 的代数余子式。

对于二阶矩阵,有$$
    A = \begin{pmatrix}
        a&b\\c&d
    \end{pmatrix}
    \Rightarrow
    A^* = \begin{pmatrix}
        d & -b \\ -c & a
    \end{pmatrix}
$$ 

伴随矩阵有一重要性质,即
$$
    AA^* = A^*A = \det(A)E 
$$ 

其还有推论$$
    \Delta\Delta^* = \Delta^*\Delta = \det(\Delta)E 
$$ 
其中$ \Delta $ 是含有矩阵的运算式整体。

由上述结论,伴随矩阵还有如下的运算结论。
\begin{itemize}
    \item $ |A^*| = |A|^{n-1} $ ;
    \item $ (kA)^* = k^{n-1}A^*$ ;
    \item $ (A^*)^* = \Attention{\det(A)^{n-2}A} $ ;
    \item $ \begin{pmatrix}
        A&O\\O&B
    \end{pmatrix}^* = \begin{pmatrix}
        |B|A^*&\\&|A|B^*
    \end{pmatrix} $ 
\end{itemize}

\section{矩阵运算}

对矩阵 $ A = (a_{ij})_{m\times n},B = (b_{ij})_{m\times n},D = (d_{ij})_{n\times s} $ ,有

\begin{itemize}
    \item $ kA = (ka_{ij})_{m\times n} $ ;
    \item $ A\pm B = (a_{ij}\pm b_{ij})_{m\times n} $ ;
    \item $ AD = (c_{ij})_{m\times s} $ ,其中 $ \dis c_{ij} = \sum_{t = 1}^n a_{it}d_{tj} $ ;
\end{itemize}

对于两向量 $ \vec \alpha = (a_1,\dots,a_n)^\top, \vec \beta = (b_1,\dots, b_n)^\top, $ 
有如下表格。

\newpage

\begin{table}[!htbp]\centering
    \begin{tabular}{|cc|c}
    \hline
    \multicolumn{1}{|c|}{$\alpha^\top \beta$} & $\beta^\top \alpha$ & 相等的数   \\ \hline
    \multicolumn{1}{|c|}{$\alpha\beta^\top$}  & $\beta\alpha^\top$  & 互为转置矩阵,秩$  = 1 $  \\ \hline
    \multicolumn{2}{|c|}{上为下的迹}                   &       
    \end{tabular}
\end{table}

其中,矩阵 $ A $ 的迹是 $ \dis tr(A)\whichis \sum_{i=1}^n a_{ii} $ .

注意,对矩阵 $ A,B $,有
\begin{itemize}
    \item $ A\neq B \nRightarrow |A|\neq |B| $ ;
    \item $ A\neq O,B\neq O \nRightarrow AB \neq O $ ;
    \item 一般地,$ AB\neq BA $ ,此时 $ (AB)^k \neq A^kB^k $ ;
    \item $ AB = AC \Rightarrow B=C $ 仅当 $ A^{-1} $ 存在。
\end{itemize}

\section{逆矩阵}

对方阵 $ A,B $ ,若有 $ AB = BA = E $ ,
则$ A,B $ 都是可逆矩阵,且有  $ A^{-1} = B, B^{-1} = A $ .

\begin{Theo}[可逆的充要条件]

    $ A $ 是可逆矩阵 $ \Leftrightarrow |A| \neq 0 $ .
\end{Theo}

$ AB = BA = kE $ 时 $ A,B $ 也有可逆性。因此,
$$
    AA^* = |A|E \Rightarrow 
    A^{-1} = \dfrac{1}{|A|} A^*
$$ 

求解逆矩阵时,
\begin{itemize}
    \item 若 $ A $ 抽象未知,则由定义求解,即凑出 $ AB = kE $,其中
    $ A $ 也可以是一个矩阵的运算式;
    \item 若已知 $ A $ ,则有
    \begin{itemize}
        \item $ A^{-1} = \dfrac{1}{|A|} A^* $ ;
        \item $ (A:E)\xrightarrow{\textrm{行变换}}(E:A^{-1}) $ .
    \end{itemize}
\end{itemize}

可逆矩阵还有以下性质。
\begin{itemize}
    \item $ (kA)^{-1} = \dfrac{1}{k}A^{-1} $ ;
    \item $ (A^{-1})^{-1} = A $ ;
    \item $ (AB)^{-1} = B^{-1}A^{-1} $;
    \item $ \begin{pmatrix}
        A&O\\O&B
    \end{pmatrix}^{-1} = \begin{pmatrix}
        A^{-1}&O\\O&B^{-1}
    \end{pmatrix} $;
    \item $ \begin{pmatrix}
        O&A\\B&O
    \end{pmatrix}^{-1} = \begin{pmatrix}
        O&B^{-1}\\A^{-1}&O
    \end{pmatrix} $.
\end{itemize}

显然,对正交矩阵 $ A $ 有 $ A^{-1} = A^\top $ .

\section{初等变换与初等矩阵}

初等矩阵是将 $ E $ 经一次初等变换所得到的矩阵。

初等变换有三种,即
\begin{itemize}
    \item 倍乘;
    \item 倍加;
    \item 交换,
\end{itemize}

因此初等矩阵也有三种,即
\begin{itemize}
    \item $ E_{ij} $ - 将$ E $的$ i,j $ 两排交换;
    \item $ E_{i}(k) $ - 将 $ E $ 的第 $ i $ 排乘$ k $ ;
    \item $ E_{ij}(k) $ - 将$ E $ 的第 $ i $ 行的$ k $ 倍加到第$ j $ 行,
    同时也是将$ E $ 的第 $ j $ 列的$ k $ 倍加到第$ i $ 列。
\end{itemize}

初等矩阵有如下的性质。
\begin{itemize}
    \item 对于一$ m\times n $ 矩阵 $ A $ ,对 $ A $ 进行一次初等$ \begin{matrix}
        \textrm{行}\\\textrm{列}
    \end{matrix} $ 变换相当于$ \begin{matrix}
        \textrm{左}\\\textrm{右}
    \end{matrix} $ 乘对应的初等矩阵;
    \item 有重要结论
        \begin{table}[!htbp]\centering
            \begin{tabular}{|c|c|c|}
            \hline
                        & 行列式 & 逆                   \\ \hline
            $ E_{ij} $    & $ -1 $   & $ E_{ij}(k) $       \\ \hline
            $ E_{i}(k) $  & $ k  $   & $ E_{i}(1/k) $ \\ \hline
            $ E_{ij}(k) $ & $ 1  $   & $ E_{ij}(-k) $      \\ \hline
            \end{tabular}
        \end{table}
\end{itemize}

\section{矩阵的秩}

\begin{Def}[$ r $ 阶子式]

    对一矩阵 $ A_{m\times n} $ ,从其中任取 $ r $ 行和 $ r $ 列,
    按照原顺序构成的行列式是其 $ r $ 阶子式。    
\end{Def}

\begin{Def}[矩阵的秩]

    对一矩阵 $ A $ ,若其有至少一个 $ r $ 阶子式不为零,而全部 $ r+1 $ 阶子式
    均为零,则称 $ r $ 为矩阵 $ A $ 的秩,记为 $ r(A) = r $.
\end{Def}

注意,\begin{itemize}
    \item $ r(A) = 0 \Leftrightarrow A = O $ ;
    \item $ r(A_n) = n \Leftrightarrow |A|\neq 0 $ ;
    \item $ r(A)<k \Leftrightarrow $ 所有 $ r $ 阶子式全为零. 
\end{itemize}

矩阵的秩有如下性质。
\begin{itemize}
    \item $ r(A_{m\times n}) \leq min\{m,n\} $ ;
    \item $ r(kA) = r(A) $ ;
    \item $ r(A+B)\leq r(A)+r(B) $ ;
    \item $ r(A:B) \geq r(A) $ ;
    \item 若 $ AB = 0 $ , $ r(A) + r(B) \leq n $ ;
    \item 对 $ n $ 阶方阵 $ A $ ,有$$
        r(A^*)  = \begin{cases}
            n,&r(A) = n\\ 
            1,& r(A) = n-1\\ 
            0,&r(A)< n-1
        \end{cases}
    $$ 
\end{itemize}

求解秩时,将矩阵 $ A $ 通过初等行变换变为阶梯矩阵 $ B $ ,其非零行数即为秩,
以矩阵表示方程组,则秩也等于约束变量的方程数。

设 $ P,Q $ 为可逆矩阵,若有 $ PAQ = B $ ,则称 $ A $ 与$ B $ 相似。

有性质 $ r(PA) = r(AQ) = r(PAQ) = r(A) $ . 因此,$ A $ 与 $ B $ 等价
$ \Leftrightarrow r(A) = r(B) $.

\section{求解 $ A^n $ 的三种情况}

其一,当 $ r(A) = 1 $ 时,必定能将 $ A $ 表示为一列向量$ \alpha $ 乘一行向量$ \beta^\top $ ,且
\begin{equation*}
    \begin{aligned}
        A^n &= \alpha\beta^\top\cdot\alpha\beta^\top\cdots\alpha\beta^\top
        \\&= \alpha(\beta^\top\cdot\alpha)(\beta^\top\cdots\alpha)\beta^\top
        \\&= \alpha\cdot tr(A)^{n-1}\cdot \beta^\top 
        \\&= tr(A)^{n-1}A
    \end{aligned}
\end{equation*}

其二,若$$
    A = \begin{pmatrix}
        0&a&b\\ 
        0&0&c\\ 
        0&0&0
    \end{pmatrix}
$$ 
则有$$
    A^2 = \begin{pmatrix}
        0&0&ac\\ 
        0&0&0\\ 
        0&0&0
    \end{pmatrix},
    A^i = O (i > 2)
$$ 

其三,对两相似矩阵 $ A\sim B $ ,有
$ B^n = P^{-1}A^nP $ ,即相似矩阵的 $ n $ 次方仍相似。

对第二种类型,可以引申得到对矩阵

$$
    A=\begin{pmatrix}
        0&1&\cdots&\cdots&1\\ 
        &0&\ddots&\ddots&\vdots\\ 
        &&\ddots&\ddots&\vdots\\ 
        &&&0&1\\ 
        &&&&0\\ 
    \end{pmatrix}
$$ 

每对其乘一个 $ A $ ,其结果中的 $ 1 $ 斜列就向右上平移一列,直到成为零阵。

