\chapter{微分方程}

\section{概念}

微分方程中未知函数的导数的最高阶为该微分方程的阶。

满足微分方程的函数称为微分方程的解。

通解为含有相互独立的方程阶数个任意常数的解。通解称为一般解,其不一定是全部解。

不在通解中的解是奇解。

不含有任意常数或任意常数确定后的解为特解。

\section{一阶微分方程}

\subsection{可分离变量的一阶微分方程}

其形如 $ \dis \dfrac{\mathrm{d} y}{\mathrm{d}x} = f(x)g(y) $ ,其微分形式为
$ M_1(x)N_1(y)\mathrm{d}x+M_2(x)N_2(y)\mathrm{d}y = 0 $.

可以使用分离变量法,即将含有 $ x,y $ 分离到等式两边,然后取积分 
$ \dis \int \dfrac{\mathrm{d}y}{g(y)} = \int f(x)\mathrm{d}x $.

\subsection{一阶齐次微分方程}

其形如 $ \dis \dfrac{\mathrm{d}y}{\mathrm{d}x}=f(\dfrac{y}{x}) $ ,此时可
令 $ \dfrac{y}{x} = u $ ,然后应用分离变量法。

事实上,此时有 $ x\dfrac{\mathrm{d}u}{\mathrm{d}x} + u = f(u) $ .

\subsection{一阶线性微分方程}

其形如 $ \dis \dfrac{\mathrm{d}y}{\mathrm{d}x} + P(x)y = Q(x) $ ,其中未知数及其导数次数为1.

其有公式$$
    y = e^{-\int P(x)\mathrm{d}x}
    \left(\int Q(x)e^{\int P(x)\mathrm{d}x}\mathrm{d}x+C\right)
$$ 

其中包含三个不定积分,其常数 $ C $ 已经事先提出。

\subsection{全微分方程}

一阶微分方程形如 $ \dis P(x,y)\mathrm{d}x + Q(x,y)\mathrm{d}y = 0 $ 满足
$ \dis \dfrac{\partial Q}{\partial x} = \dfrac{\partial P}{\partial y} $ 时,
称为全微分方程。

全微分方程的通解为$$
    \int_{x_0}^xP(x,y_0)\mathrm{d}x + \int_{y_0}^y Q(x,y)\mathrm{d}y = C
$$ 

\subsection{伯努利方程}

形如 $ y' + P(x)y = Q(x)y^n $ 的方程是伯努利方程。
此时令 $ z = y^{1-n} $ 将其变为一阶线性方程。

\section{可降阶的高阶微分方程}

\begin{Field}[形如 $ y^{(n)} = f(x) $ 时 ]

    只需要积分 $ n $ 次即可得到方程通解。
\end{Field}

\begin{Field}[不显含函数$ y $ 的二阶可降阶方程 $ y^\pprime = f(x,y') $ ]

    令 $ y = p'(x) $ ,则有 $ y^\pprime = \dfrac{\mathrm{d} p}{\mathrm{d}x}  =p' $ ,因而
    可以将原方程降为一阶。
\end{Field}

\begin{Field}[不显含自变量 $ x $ 的二阶可降阶方程 $  y^\pprime = f(y,y') $ ]

    令 $ y' = p $ ,则有 $ y^\pprime = \dfrac{\mathrm{d}p}{\mathrm{d}y}\cdot \dfrac{\mathrm{d}y}{\mathrm{d}x}
    =p\dfrac{\mathrm{d}p}{\mathrm{d}y} $ ,因而可将其降为一阶。
\end{Field}

在解可降阶高阶微分方程时,应当边解边确定常数,常数确定得越早越好。

\section{高阶线性微分方程}

\subsection{高阶线性微分方程概念}

$ n $ 阶的线性微分方程形如$$
    y^{(n)}+p_{n-1}y^{(n-1)} + \dots + p_1(x)y'+p_0(x)y = f(x)
$$ 

其中若 $ f(x)\equiv 0 $ ,称方程为齐次的,否则为非齐次的。

\subsection{线性微分方程解的结构}

考虑二阶线性微分方程,其有如下性质。

\begin{itemize}
    \item 对二阶齐次线性微分方程的两特解 $ y_1,y_2 $ ,其任意一组线性组合都是
    同一方程的特解;若两特解线性无关,其线性组合为通解;
    \item 若 $ y_1,y_2 $ 为二阶非齐次线性微分方程的两特解,
    则 $ y_1-y_2 $ 是对应齐次方程的一特解;
    \item 若 $ \hat y,y^* $ 分别为一二阶齐次线性微分方程以及与前者
    对应的非齐次方程的特解,则其和为对应非齐次方程的一特解;
    \item 若 $ \hat y,y^* $ 分别为一二阶齐次线性微分方程的通解以及与前者
    对应的非齐次方程的特解,则其和为对应非齐次方程的通解;
    \item 对方程 $ y^\pprime p_1(x)y'+p_0(x)y = f_1(x),y^\pprime p_1(x)y'+p_0(x)y = f_2(x) $
    ,若其分别有特解 $ y_1,y_2, $ 则 $ y_1+y_2 $ 为 $ y^\pprime p_1(x)y'+p_0(x)y = f_1(x)+f_2(x) $ 的特解。
\end{itemize}

\subsection{常系数齐次线性微分方程的求解}

对二阶常系数齐次线性方程 $ y^\pprime +py'+qy = 0 $ ,其中 $ p,q $ 为常数,
其具有特征方程 $ \lambda^2+p\lambda+q = 0 $,根据特征方程的根,可以分为三种情况。

对 $ \Delta = p^2 - 4p $ ,
\begin{enumerate}
    \item 若 $ \Delta > 0 $ ,特征方程有二实根 $ \lambda_1,\lambda_2 $ ,此时方程的通解为$$
        y = C_1e^{\lambda_1x}+C_2e^{\lambda_2x}
    $$ 其中 $ C_1,C_2 $ 是任意常数;
    \item 若 $ \Delta = 0 $ ,特征方程有一二重实根 $ \lambda $ ,此时方程的通解为$$
        y = (C_1+C_2 x)e^{\lambda x}
    $$ 其中 $ C_1,C_2 $ 是任意常数;
    \item 若 $ \Delta < 0 $ ,特征方程有一对共轭复根 $ \alpha\pm\beta i $ ,此时方程的通解为$$
        y = e^{\alpha x}\left(C_1\cos \beta x + C_2 \sin \beta x\right)
    $$ 其中 $ C_1,C_2 $ 是任意常数。
\end{enumerate}

事实上,以上是在复数域中对特征方程进行求解。

对$ n $ 阶常系数齐次线性微分方程是类似的。对 $$
    y^{(n)} + p_1y^{(n-1)} + \dots + p_{n-1}y' + p_ny = 0
$$ 其中 $ p_i,i=1,2,\dots n  $ 为常数,有特征方程$$
    \lambda^n + p_1\lambda^{n-1} +\dots + p_n = 0
$$ 

特征根与方程通解的关系与二阶情形很类似。

\begin{itemize}
    \item 若 $ \lambda_0 $ 为特征方程的 $ k $ 重实根,则特征方程的通解中含有$$
        e^{\lambda_0 x}\sum_{i=1}^k C_ix^{i-1}
    $$ 
    \item 若 $ \lambda_1 $ 为特征方程的 $ k $ 重共轭复根,则方程通解中含有$$
        e^{\alpha x}\left[
            \cos \beta x \sum_{i=1}^k C_ix^{i-1}+
            \sin \beta x \sum_{i=1}^k D_ix^{i-1}
        \right]
    $$ 
\end{itemize}

\subsection{二阶常系数线性非齐次微分方程的求解}

对形如 $ y^\pprime+py'+qy = f(x) $ 的二阶常系数线性非齐次微分方程,其通解的结构为
$ y = \hat y  + y^* $ ,其中 $ \hat y $ 为对应齐次方程的通解, $ y^* $ 为本方程的一个特解。

对方程形如$
    y^\pprime + py'+qy = P_m(x)e^{ax}
$ ,其中 $ P_m(x) $ 为 $ x $ 的 $ m $ 次多项式,
则原方程有一个特解形如$$
    y^* = x^k Q_m(x)e^{ax}
$$ 其中 $ a $ 为$\overset{\textrm{齐次特征方程}}{ \lambda^2 + p\lambda + q = 0  }$的 $ k $ 重根,
将其代入原方程求 $ Q_m(x) $ .

对方程形如$
    y^\pprime + py'+qy = e^{\alpha x}\left[P_l(x)\cos \beta x+P_n(x)\sin \beta x\right]
$ ,其中 $ P_l(x),P_n(x) $ 为 $ x $ 的 $ l,n $ 次多项式,
则原方程有一个特解形如$$
    y^* = x^ke^{ax}\left[Q_m(x)\cos\beta x+R_m(x)\sin \beta x\right]
$$ 其中 $ m = \max(l,n) $ ,且若 $ \alpha+\beta i $ 是齐次特征方程的一对共轭复根,则 $ k = 1 $ ;否则 $ k = 0 $ .

代入原方程,求 $ Q_m(x),R_m(x) $ 。

事实上,令 $ \beta = 0 $ ,则后者转化为前者。

\subsection{欧拉方程}

形如 $\dis x^ny^{(n)} + \sum_{i=1}^np_ix^{n-i}y^{(n-i)} = f(x) $ 的方程为欧拉方程。
令 $ x = e^t $ ,将其转化为以 $ t $ 为自变量的常系数线性方程。

具体而言,若设 $ x = e^t $ ,则有
\begin{equation*}
    \begin{aligned}
        \dfrac{\mathrm{d}y}{\mathrm{d}x} = \dfrac{1}{x}y'_{t};\quad{}
        \dfrac{\mathrm{d}^2y}{\mathrm{d}x^2} = \dfrac{y^\pprime_t - y'_t}{x^2}
    \end{aligned}
\end{equation*}