\chapter{多元函数积分学}

\section{三重积分}

三重积分 $ \dis \iiint\limits_\Omega f(x,y,z)\mathrm{d}V $ 的体积元素为 $ \mathrm{d}V = \mathrm{d}x\mathrm{d}y\mathrm{d}z. $

三重积分有多种计算方法。

\subsection{三重积分的计算方法}

\subsubsection{对称性}

\begin{Theo}[对称性]

    设$ f(x,y,z) $ 在空间有界闭区域 $ \Omega $ 连续,而 $ \Omega $ 关于平面 $ xoy $ 对称,则有
    $$
        \iiint\limits_\Omega f(x,y,z) \mathrm{d}V = \begin{cases}
            0,&f(x,y,z)\textrm{关于}z\textrm{是奇函数}\\
            \dis 2\iiint\limits_{\Omega_1} f(x,y,z)\mathrm{d}V,& f(x,y,z) \textrm{关于}z\textrm{是偶函数}
        \end{cases}
    $$ 
    其中 $ \Omega_1 $ 为 $ \Omega $ 中 $ z\geq0 $ 的一部分。

    对 $ \Omega $ 关于平面 $ yoz $ 与平面 $ zox $ 对称的同理。
\end{Theo}

\begin{Theo}[轮换对称性]

    若相对积分域 $ \Omega $ 三坐标轴的相对位置相同,或应用映射 $ (x,y,z) \Rightarrow (y,z,x) $ 后,积分域 $ \Omega $ 的边界方程不变,
    则有
    $$
    \iiint\limits_\Omega f(x,y,z)\mathrm{d}x\mathrm{d}y\mathrm{d}z
    =\iiint\limits_\Omega f(y,z,x)\mathrm{d}x\mathrm{d}y\mathrm{d}z
    =\iiint\limits_\Omega f(z,x,y)\mathrm{d}x\mathrm{d}y\mathrm{d}z
    $$ 
\end{Theo}

\subsubsection{先计算一个或两个轴}

若 $ \Omega $ 由 $ z_1,z_2 $ 上下两个面夹成,设 $ \forall (x,y,z) \in \Omega, (x,y)\in D_{xy} $ ,则
$$ \dis \iiint\limits_\Omega f\mathrm{d}V = \iint\limits_{D_{xy}}
\mathrm{d}x\mathrm{d}y\int_{z_1(x,y)}^{z_2(x,y)}f\mathrm{d}z $$ 
对由 $ x_1,x_2 $ 前后或者 $ y_1,y_2 $ 左右夹成的积分域同理。

若积分域垂直于一坐标轴(以 $ z $ 轴为例)的截面简单,被积函数 $ f $ 对另外两轴成的平面的积分运算容易,则
$$ \dis \iiint\limits_\Omega f\mathrm{d}V = \int_{z_1}^{z_2}\mathrm{d}z
\iint\limits_{D_{z}}f\mathrm{d}x\mathrm{d}y  $$

\subsubsection{利用三次积分法}

可以直接化为三次积分,即$$
    \iiint\limits_\Omega f\mathrm{d}V =
    \int \mathrm{d}x \int \mathrm{d}y \int f\mathrm{d}z
$$ 

一般情况下,先算先算比较简单的积分。

\subsubsection{利用球坐标系}

做变换 $ (x,y,z)\Rightarrow (\rho,\theta,\varphi) $ ,此时有
$$
    \begin{cases}
        x = \rho\sin \varphi\cos \theta\\ 
        y = \rho\sin \varphi\sin \theta\\ 
        z = \rho\cos \varphi
    \end{cases}
$$ 

其中 $ \varphi\in[0,pi],\theta\in[0,2\pi],\rho\geq0 $ ,
且有 $ \mathrm{d}V = \rho^2\sin\varphi \mathrm{d}\rho \mathrm{d}\varphi\mathrm{d}\theta $ .

若积分域为球的一部或全部,被积函数有 $ x^2+y^2+z^2 $ 的一部或全部,则适用球坐标系。

\subsubsection{利用柱坐标系}

做变换 $ (x,y,z)\Rightarrow (\rho,\theta,z) $ ,此时有
$$
    \begin{cases}
        x = \rho\cos \theta\\ 
        y = \rho\sin \theta\\ 
        z = z
    \end{cases}
$$ 
其中 $ \rho\geq 0,\theta \in R $ ,且有 $ \mathrm{d}V = \rho \mathrm{d}\rho \mathrm{d}\theta \mathrm{d}z $ .

若积分域为圆柱一部或全部,被积函数中含有 $ x^2+y^2,xy $ 或其一部时,适用柱面坐标系。

柱面坐标系的常用积分次序有 $\dis \int \mathrm{d}\theta\int \mathrm{d}\rho \int \mathrm{d}z$ 以及 $ 
\int \mathrm{d}z \int \mathrm{d}\theta \int \mathrm{d}\rho .$ 

\subsection{三重积分的应用}

\begin{Theo}[质心公式]

    一物体在空间占有空间 $ \Omega $ ,其上一点$ (x,y,z) $ 的体密度为 $ \rho(x,y,z) $ ,则
    该密度在 $ \Omega $ 连续,且其有质心 $ (\bar x,\bar y,\bar z) $ ,则有$$
        \bar x = \dfrac{\dis \iiint\limits_{\Omega}x\rho \mathrm{d}V}{\dis \iiint\limits_{\Omega}\rho \mathrm{d}V}
    $$ 
    对 $ \bar y,\bar z $ 同理。
\end{Theo}

\section{曲线积分}

\subsection{第一类曲线积分}

\subsubsection{物理意义}

一条线密度为 $ f(x,y,z) $ 的线的总质量$$
    m = \int\limits_L f(x,y,z)\mathrm{d}s
$$ 

其中 $ \mathrm{d}s $ 是弧长元素。

\subsubsection{第一类曲线积分的性质}

第一类曲线积分有如下性质。
\begin{itemize}
    \item \textbf{线性性}
    
    $\dis \int\limits_L (f\pm g)\mathrm{d}s = \int\limits_L f\mathrm{d}s \pm \int\limits_Lg\mathrm{d}s $ 

    $ \dis k\int\limits_L f\mathrm{d}s = \int\limits_Lkf\mathrm{d}s $ 
    \item \textbf{积分域的分割}
    
    $\dis \int\limits_Lf\mathrm{d}s = \int\limits_{L_1}f\mathrm{d}s + \int\limits_{L_2}f\mathrm{d}s$
    ,其中 $ L=L_1+L_2 $ 
    \item \textbf{保序性}
    
    $\dis f\leq g\Rightarrow \int\limits_L f\mathrm{d}s \leq \int\limits_L g\mathrm{d}s$ 
    \item \textbf{$ L $ 弧长计算公式}
    
    $\dis S_L = \int\limits_L\mathrm{d}s$ 
\end{itemize}

\subsubsection{第一类曲线积分的计算方法}

\sssubsection{利用特殊方法计算}

第一类曲线积分的对称奇偶性、轮换对称性与二、三重积分类似。

计算第一类曲线积分,可以将曲线方程代入被积函数。

第一类曲线积分具有
\begin{itemize}
    \item 形心公式$$
        \bar x = \dfrac{\dis \int x\mathrm{d}s}{\dis \int \mathrm{d}s}
    $$ $ y,z $ 同理;
    \item 形心公式$$
        \bar x = \dfrac{\dis \int xf\mathrm{d}s}{\dis \int f\mathrm{d}s}
    $$ 其中 $ f $ 是密度;对 $ y,z $ 同理。
\end{itemize}

\sssubsection{利用定积分}

有 $ \dis \mathrm{d}s = \sqrt{(\mathrm{d}x)^2+(\mathrm{d}y)^2+(\mathrm{d}z)^2} $ .

设空间曲线 $ L $ 有参数方程 $ x = x(t),y = y(t),z = z(t), t\in [\alpha,\beta] $ ,则有
$$
    \int\limits_L f(x,y,z)\mathrm{d}s =
    \int_\alpha^\beta f(x(t),y(t),z(t))\sqrt{x'(t)^2+y'(t)^2+z'(t)^2}\mathrm{d}t
$$ 

在化为定积分后,要保证积分下限小,积分上限大。

\subsection{第二类曲线积分}

\subsubsection{物理意义}

考虑一质点 $ P $ 因变力 $ F $ 从 $ A $ 移动到 $ B $. 
有 $ \vec{F} = P\vec i+Q\vec j+R\vec k $ ,其中 $ P,Q,R $ 都是
关于$ x,y,z $ 的标量函数。

此时有弧微分向量 $ \vec{\mathrm{d}s} = \vec I\mathrm{d}s = (\mathrm{d}x,\mathrm{d}y,\mathrm{d}z) $ ,
其中 $ \vec I $ 是单位切向量 $ (\cos \alpha,\cos\beta, \cos \gamma) $ ,那么就有
上述质点在移动过程中做的功 $ \dis W = \int\limits_L \vec F \vec{\mathrm{d}s}
= \int\limits_L P\mathrm{d}x+Q\mathrm{d}y+R\mathrm{d}z $ ,其中称 $ P\mathrm{d}x $ 为对 $ x $ 的积分,$ y,z $ 同理。

\subsubsection{基本性质}

第二类曲线积分有如下性质。
\begin{itemize}
    \item \textbf{线性性}
    
    $\dis \int\limits_L (\vec F_1\pm \vec F_2)\vec{\mathrm{d}s} = \
    \int\limits_L F_1·\vec{\mathrm{d}s} \pm \int\limits_LF_2·\vec{\mathrm{d}s} $ 

    $ \dis k\int\limits_L \vec F\vec{\mathrm{d}s}
    = \int\limits_L kF\vec{\mathrm{d}s} $ 
    \item \textbf{积分域的分割}
    
    $\dis \int\limits_L\vec F\vec{\mathrm{d}s} = 
    \int\limits_{L_1}\vec F\vec{\mathrm{d}s} + \int\limits_{L_2}\vec F\vec{\mathrm{d}s}$
    ,其中 $ L=L_1+L_2 $ 
    \item \textbf{反向曲线值变号}
    
    设 $ L^- $ 是 $ L $ 的反向曲线,则有$$
        \int\limits_{L^-}\vec F\vec{\mathrm{d}s}
        = -\int\limits_{L}\vec F\vec{\mathrm{d}s}
    $$ 
\end{itemize}

注意,不考虑对称奇偶性;轮换对称性仍然存在。

\sssubsection{利用定积分计算第二类曲线积分}

设空间有向曲线 $ L $ 有参数方程 $ x = x(t),y = y(t),z = z(t) $ ,起点 $ A $ 对应参数为 $ \alpha, $ 终点 $ B $ 
对应参数为 $ \beta $, 且 $ \beta > \alpha $ 不一定成立,若 $ P,Q,R $ 均连续,$ x'(t),y'(t),z'(t) $ 
也都连续,则有
$$
    \int\limits_{\wideparen{AB}}P(x,y,z)\mathrm{d}x+
    Q\mathrm{d}y + R\mathrm{d}z = 
    \int_\alpha^\beta \{P[x(t),y(t),z(t)]x'(t)+Qy'(t)+Rz'(t)\}\mathrm{d}t
$$ 

化为定积分后,要保证积分下限对应起点,积分上限对应终点。

\subsection{格林公式及曲线积分与路径无关的条件}

\subsubsection{格林公式}

\begin{Def}[正方向]

    对一有界闭区域 $ D $ ,其边界为 $ L $ ,沿着边界线环形,当行进方向是正方向时,区域在前进方向的左侧,
    此时 $ D $ 有正向边界 $ \partial D $ .
\end{Def}

\begin{Theo}[格林公式]

    设 $ xy $ 平面上一有界单连通闭区域 $ D $ 由一条逐段光滑闭曲线 $ L $ 围成,
    $ L $ 取正向,且 $ P(x,y),Q(x,y) $ 在 $ D $ 上有一阶连续偏导数,
    则有
    $$
        \oint\limits_{\partial D} P\mathrm{d}x+Q\mathrm{d}y = \iint\limits_{D}
        \left(\dfrac{\partial Q}{\partial x}-\dfrac{\partial P}{\partial y}\right)
        \mathrm{d}x\mathrm{d}y
    $$ 
\end{Theo}

格林公式可以推广到 $ (n+1) $ 连通区域上,若 $ n > 1 $ 有穷。

当不满足格林公式的适用条件时,构造辅助线创造符合适用条件的正向边界。
此时,需要给出辅助线的符号、表达式和方向。

\subsubsection{平面上曲线积分与路径无关的条件}

\begin{Def}[与路径无关]

    对任意从 $ A $ 到 $ B $ 的有向曲线 $ l_1,l_2 $ 若都有$$
        \int\limits_{L_1}P\mathrm{d}x+Q\mathrm{d}y = 
        \int\limits_{L_2}P\mathrm{d}x+Q\mathrm{d}y
    $$ 则称 $ \dis \int\limits_{L}P\mathrm{d}x+Q\mathrm{d}y $ 与路径无关,
    只与起点、终点有关。此时,将其记为
    $$
        \int_\alpha^\beta P\mathrm{d}x+Q\mathrm{d}y
    $$ 
\end{Def}

设 $ F\{P,Q\} $ 的分量 $ P(x,y),Q(x,y) $ 在单连通区域 $ D $ 内有一阶连续偏导数,
则以下命题等价。

\begin{itemize}
    \item 对 $ D $ 内任意光滑闭曲线 $ L $ ,有 $ \dis \oint\limits_L P\mathrm{d}x+Q\mathrm{d}y = 0 $ 
    \item 对任意 $ L=\wideparen{AB}\subset D, \int\limits_{\wideparen{AB}} $ 仅与起点 $ A $ 和
    终点 $ B $ 有关,与 $ L $ 的取法无关,也称曲线积分与路径无关;
    \item $ \dis P(x,y)\mathrm{d}x+Q(x,y)\mathrm{d}y = \mathrm{d}u(x,y) $ 成立,此时称
    $ u(x,y) $ 为 $ P\mathrm{d}x+Q\mathrm{d}y $ 的原函数;
    \item 在 $ D $ 内 $ \dfrac{\partial Q}{\partial x} = \dfrac{\partial P}{\partial x} $ 处处成立。
\end{itemize}

由于曲线积分与路径无关,可以尽量选择简单的路径。

\subsubsection{原函数的求法}

\begin{Def}[原函数]

    若存在 $ u(x,y) $ 使得 $ \mathrm{d}u(x,y) = P\mathrm{d}x+Q\mathrm{d}y $ 则称 $ u(x,y) $ 
    似乎 $ P(x,y)\mathrm{d}x+Q\mathrm{d}y $ 的一个原函数。
\end{Def}

显然若 $ u(x,y) $ 是  $ P\mathrm{d}x+Q\mathrm{d}y $ 的一个原函数,则 $ u(x,y) + C $ 
也是其一个原函数。

\begin{Theo}[原函数存在定理]

    设 $ D $ 是单连通区域,$ P,Q $ 都具有连续的偏导数,则$$
        \dfrac{\partial Q}{\partial x} \equiv \dfrac{\partial P}{\partial x}
        \Leftrightarrow
        P(x,y)\mathrm{d}x + Q(x,y)\mathrm{d}y \textrm{在}D\textrm{内存在原函数} u(x,y)
    $$ 
    且 $$
        u(x,y) = \int_{(x,y)}^{(x_0,y_0)}P(x,y)\mathrm{d}x+Q(x,y)\mathrm{d}y + C
    $$ 
\end{Theo}

一般令上述 $ (x_0,y_0) $ 为原点,若在原点没有定义,则选择坐标轴上一点。

\subsection{两类曲线积分的关系}

设 $ L = \wideparen{AB} $ 平面上一个逐段光滑有定向的曲线,向量函数 $ \vec A = P(x,y)\vec i + Q(x,y)\vec j $ 
在 $ L $ 上连续,$ \vec T = (\cos\alpha,\cos\beta) $ 为曲线弧在点 $ (x,y) $ 处沿从 $ A $ 到 $ B $ 方向的单位切向量
,即方向余弦,且 $ \vec{\mathrm{d}s}=\vec T\mathrm{d}s = (\mathrm{d}x,\mathrm{d}y) $ 
为该点的弧微分向量,则
$$
    \int\limits_L P(x,y)\mathrm{d}x + Q(x,y)\mathrm{d}y = 
    \int\limits_L \vec A ·\vec{\mathrm{d}s} = 
    \int\limits_L \vec A ·\vec{T}\mathrm{d}s = 
    \int\limits_L \left[P(x,y)\cos\alpha+Q(x,y)\cos\beta \right]\mathrm{d}s
$$ 

在空间上同理。

\section{曲面积分}

\subsection{第一类曲面积分}

\subsubsection{物理意义}

对一面密度为 $ f(x,y,z) $ 的立体,其质量为$$
    m = \iint\limits_{\Sigma} f(x,y,z)\mathrm{d}S
$$ 

其中 $ \mathrm{d}S $ 是体积元素。

\subsubsection{基本性质}

第一类曲面积分有以下性质。

\begin{itemize}
    \item \textbf{线性性}
    
    $\dis \int\limits_\Sigma (f\pm g)\mathrm{d}S = \int\limits_\Sigma f\mathrm{d}S \pm \int\limits_\Sigma g\mathrm{d}S $ 

    $ \dis k\int\limits_\Sigma f\mathrm{d}S = \int\limits_\Sigma f\mathrm{d}S $ 
    \item \textbf{积分域的分割}
    
    $\dis \int\limits_\Sigma f\mathrm{d}S = \int\limits_{\Sigma_1}f\mathrm{d}S + \int\limits_{\Sigma_2}f\mathrm{d}S$
    ,其中 $ \Sigma=\Sigma_1+\Sigma_2 $ 
    \item \textbf{$ \Sigma $ 面积计算公式}
    
    $\dis S_\Sigma = \int\limits_\Sigma\mathrm{d}S$ 
\end{itemize}

