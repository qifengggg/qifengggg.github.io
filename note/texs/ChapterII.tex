\chapter{一元函数的微分学}

\section{导数}

\subsection{导数的概念}

\begin{Def}[导数]

    设函数 $ y=f(x) $ 在点 $ x_{0} $ 的一邻域内有定义,自变量 $ x $ 在$ x_0 $ 处有增量 $ \Delta x $,因变量 $ y $ 
    相应地有增量 $ \Delta y=f(x_0+\Delta x)-f(x_0) $ ,若 
    $ {\displaystyle\lim_{\Delta x\rightarrow 0}}\frac{\Delta y}{\Delta x}=
    {\displaystyle\lim_{\Delta x\rightarrow 0}}\frac{f(x+\Delta x)-f(x)}{\Delta x} $ 存在,则称此
    极限值为函数$ y=f(x) $ 在 $ x=x_0 $ 点处的导数,也称微商,记作$$
        f'(x)\textrm{或}\left.y'\right|_{x=x_0}\textrm{或}\left.\frac{\mathrm{d}f(x)}{\mathrm{d}x}\right|_{x=x_0}
    $$ 
    称函数 $ y=f(x) $ 在 $ x=x_0 $ 处可导。若上述极限不存在,则称 $ f(x) $ 在 $ x=x_0 $ 处不可导。 
\end{Def}

\begin{Field}[导数定义的等价变形]

    导数定义可以等价变形为
    $$
        f'(x)={\displaystyle\lim_{\Delta x\rightarrow 0}}\frac{f(x+\Delta x)-f(x)}{\Delta x}=
        {\displaystyle\lim_{x\rightarrow x_0}} \frac{f(x+x_0)-f(x)}{x-x_0}
    $$ 
\end{Field}
前者适合用于求导函数。

注意:
\begin{itemize}
    \item 定义式中分子和分母的增量必须相对应,如$$
        {\displaystyle\lim_{\Delta x\rightarrow 0}\frac{f(x+2\Delta x)-f(x)}{\Delta x}}\neq f'(x)
    $$ 
    \item 在定义式中,分子中两项应当“一动一静”,否则无法推出可导性。注意,$$
        f(x)\textrm{在}x_0\textrm{处可导}\Rightarrow 
        {\displaystyle\lim_{\Delta x\rightarrow 0}}\frac{f(x+\Delta x)-f(x-\Delta x)}{\Delta x}
    $$ 
    $ \Leftarrow $ 反例 - $ f(x)=|x|, x_0 = 0; $ 
    \item 若已知 $ f(x) $ 在 一点 $ x_0 $ 可导,则可能可以利用导数定义。
\end{itemize}

\begin{Def}[左、右导数]

    若$$
        {\displaystyle\lim_{x\rightarrow x_0^-}}\frac{f(x)-f(x_0)}{x-x_0}\mathop{=}\limits^\exists f_-'(x_0)
    $$ 
    则称 $ f(x) $ 在  $ x_0 $ 处左可导, $ f_-'(x_0) $ 称为其左导数;
    若$$
    {\displaystyle\lim_{x\rightarrow x_0^+}}\frac{f(x)-f(x_0)}{x-x_0}\mathop{=}\limits^\exists f_+'(x_0)
$$ 
则称 $ f(x) $ 在  $ x_0 $ 处右可导, $ f_+'(x_0) $ 称为其右导数;
\end{Def}

\begin{Field}[函数在区间上的可导性]

    \begin{itemize}
        \item 若对 $ \forall x\in (a,b) $ 有 $ f(x) $ 在 $ x_0 $ 可导,则称 $ f(x) $ 在 $ (a,b) $ 上可导;
        \item 若对 $ \forall x\in (a,b) $ 有 $ f(x) $ 在 $ x_0 $ 可导,
        且其在 $ a $ 点右可导,在 $ b $ 点左可导,则称 $ f(x) $ 在 $ [a,b] $ 上可导。
    \end{itemize}
\end{Field}

\subsection{函数可导的条件}

\begin{Theo}[函数可导的必要条件]

    可导必连续。
\end{Theo}
$ \Leftarrow $ 反例如下。
\begin{itemize}
    \item $ f(x)=|x|,x_0=0 $ ;
    \item $ f(x)=\sqrt[3]x,x_0=0 $ 此时其左右导数均无穷大。
\end{itemize}

\begin{Theo}[函数可导的充要条件]

    $ f(x) $ 在 $ x=x_0 $ 处可导$ \Leftrightarrow f_-'(x) $ 与 $ f_+'(x) $  均存在且相等。
\end{Theo}

其常用于:\begin{enumerate}
    \item 已知可导性求参数;
    \item (后者)分段函数分段点的可导性。
\end{enumerate}

\subsection{导数的几何意义}

若函数 $ y=f(x) $ 在 $ x_0 $ 处导数 $ f'(x_0) $ \textbf{存在},则在几何上, $ f'(x_0) $ 表示曲线
$ y=f(x) $ 在点 $ (x_0,f(x_0)) $ 处的切线斜率,即\begin{itemize}
    \item 切线方程$ y-f(x_0)=f'(x_0)(x-x_0) $ 
    \item 法线方程$ y-f(x_0)=-\frac1{f'(x_0)}(x-x_0) $ 
\end{itemize}

注意:
\begin{itemize}
    \item 求切法线的步骤:\begin{enumerate}
        \item 求切点坐标;
        \item 求斜率;
        \item 写方程。
    \end{enumerate}
    \item 可用于处理\begin{itemize}
        \item 显函数 $ y=f(x) $ ;
        \item 隐函数 $ F(x,y)=0 $ ;
        \item 参数方程、极坐标方程,极坐标时可以代$ r $ 入$ \theta. $
    \end{itemize}
\end{itemize}

\subsection{导数的运算}

以下是一些基本初等函数的导数公式:

\begin{equation*}
    \begin{array}{ll}
        (c)'=0&(x^a)'=ax^{a-1}\\
        (\sin x)'=\cos x&(\cos x)'=-\sin x\\ 
        (\tan x)'=\sec^2 x&(\cot x)'=-\csc^2 x\\ 
        (\sec x)'=\tan x\sec x&(\csc x)'=-\cot x\csc x\\ 
        (\arcsin x)'=\frac{1}{\sqrt{(1-x^2)}}&(\arccos x)'=-\frac{1}{\sqrt{(1-x^2)}}\\
        (\arctan x)'=\frac1{1+x^2} & (\textrm{arccot} x)'=-\frac1{1+x^2}\\
        (a^x)'=a^x\ln a&(\log_a x)'=\frac1{x\ln a}\\
    \end{array}
\end{equation*}

注意到对于三角函数,带有“正”字的正弦、正切、正割、反正弦、反正切的导数都不包含负号,
而带有“余”字的余弦、余切、余割、反余弦、反余切的导数都包含负号。

\begin{Field}[导数的四则运算法则]

    设 $ f(x) $ 与 $ g(x) $ 均在 $ x $ 点可导,则
    \begin{enumerate}
        \item $ \left[f(x)\pm g(x)\right]'=f'(x)\pm g'(x) $ ;
        \item $ \left[f(x)g(x)\right]'=f'(x)g(x)+g'(x)f(x) $ ;
        \item $\dis \left[\frac{f(x)}{g(x)}\right]'=\frac{f'(x)g(x)-g'(x)f(x)}{g^2(x)}$,其中要求 $ g(x)\neq 0 $  ;
    \end{enumerate}
\end{Field}

注意,若 $ f(x) $ 可导但是 $ g(x) $ 不可导于 $ x=x_0, $ 则$ f(x)\pm g(x) $ 也不可导于 $ x=x_0. $ 

可以将和、差、积的公式推广至有限多个。对乘法,有
$$
    \left(\prod_{i=1}^nf_i(x)\right)'=\sum_{i=1}^n\left(f'_i(x)\prod_{\begin{subarray}{align}j=1\\j\neq i\end{subarray}}^n f_j(x)\right)
$$ 

\begin{Field}[行列式函数求导]

    行列式函数求导可以按行或按列求导。
    
    如对二阶行列式函数,有
    \begin{equation*}
        \begin{aligned}
            \left[\begin{matrix}
                a_{11}&a_{12}\\a_{21}&a_{22}
            \end{matrix}\right]'&=(a_{11}a_{22}-a_{12}a_{21})'\\
            &=a_{11}a_{12}'+a_{11}'a_{22}-a_{12}a_{21}'-a_{12}'a_{21}\\ 
            &=\begin{bmatrix}
                a_{11}'&a_{12}'\\a_{21}&a_{22}
            \end{bmatrix}+
            \begin{bmatrix}
                a_{11}&a_{12}\\a_{21}'&a_{22}'
            \end{bmatrix}\\
            &=\begin{bmatrix}
                a_{11}'&a_{12}\\a_{21}'&a_{22}
            \end{bmatrix}+
            \begin{bmatrix}
                a_{11}&a_{12}'\\a_{21}&a_{22}'
            \end{bmatrix}\\
        \end{aligned}
    \end{equation*}
\end{Field}

\begin{Field}[复合函数求导]

    设 $ y=f(u) $ ,$ u=\varphi(x) $ ,若 $ \varphi(x) $ 在 $ x $ 处可导,且 $ f(u) $ 在对应的 $ u $ 处可导,
    则复合函数 $ y=f(\varphi(x)) $ 在 $ x $ 处可导,且有$$
        \frac{\mathrm{d}y}{\mathrm{d}x}=\frac{\mathrm{d}y}{\mathrm{d}u}\cdot
        \frac{\mathrm{d}u}{\mathrm{d}x}=f'(\varphi(x))\varphi'(x)
    $$  
    对应地,有$$
        \mathrm{d}y=f'(u)\mathrm{d}u=f'(\varphi(x))\varphi'(x)\mathrm{d}x
    $$ 
\end{Field}

\begin{Field}[隐函数求导数]

    设 $ y=y(x) $ 由方程 $ F(x,y)=0 $ 确定,则$$
        \frac{\mathrm{d}y}{\mathrm{d}x}={\dfrac{\mathrm{d}F}{\mathrm{d}x}}\Big\slash{\dfrac{\mathrm{d}F}{\mathrm{d}y}}
        ,\quad{}\frac{\mathrm{d}^2F}{\mathrm{d}x^2}=\frac{\mathrm{d}}{\mathrm{d}x}(\frac{\mathrm{d}F}{\mathrm{d}x})
    $$ 
\end{Field}

\sssubsection{反函数求导数}

设 $ x = f^{-1}(y) $ 由 $ y = f(x) $ 确定,则
\begin{itemize}
    \item 若 $ f(x) $ 可导且 $ f'(x)\neq 0, $ 则$ \dis \dfrac{\mathrm{d}x}{\mathrm{d}y} = 
    \dfrac{1}{\mathrm{d}y/ \mathrm{d}x} = \dfrac{1}{f'(x)}. $ 
    \item 若 $ f(x) $ 二阶可导且 $ f'(x)\neq 0 $ 则 $ \dfrac{\mathrm{d}^2y}{\mathrm{d}x^2} = 
    -\dfrac{f^\pprime(x)}{[f'(x)]^3}. $ 
\end{itemize}

$ f(x) $ 可导且 $ f'(x)\neq 0 $ 说明 $ f(x) $ 单调且存在反函数。

\sssubsection{参数方程求导数}

设 $ y = f(x) $ 由参数方程 $ \begin{cases}
    x = x(t) \\ y = y(t)
\end{cases} $ 确定。此时,$ t = t(x),y = y(t(x)). $ 

对于其导数,有
\begin{itemize}
    \item 若 $ x(t),y(t) $ 均可导,且$ x'(t)\neq 0, $ 则 $ \dfrac{\mathrm{d}y}{\mathrm{d}x} = 
    \dfrac{\mathrm{d}y/\mathrm{d}t}{\mathrm{d}x/\mathrm{d}t} = \dfrac{y'(t)}{x'(t)}; $ 
    \item 若 $ x(t),y(t) $ 均二阶可导,且$ x'(t)\neq 0, $ 则 $ \dfrac{\mathrm{d}^2y}{\mathrm{d}x^2} = 
    \dfrac{y^\pprime(t)x'(t) - y'(t)x^\pprime(t)}{[x'(t)]^3}. $ 
\end{itemize}


\begin{Field}[常见初等函数的$ n $ 阶导数公式]

    \begin{enumerate}
        \item $ y=e^x \Rightarrow y^{(n)}=e^x; $ 
        \item $ y=a^x(a>0,a\neq1)\Rightarrow y^{(n)}=a^x(\ln a)^n; $ 
        \item $ y=\sin x \Rightarrow y^{(n)}=\sin(x+\frac{n\pi}{2});$
        \item $ y=\cos x \Rightarrow y^{(n)}=\cos(x+\frac{n\pi}{2});$  
        \item $ (\frac{1}{x+a})^{(n)} = \frac{(-1)^{n}n!}{(x+a)^{n+1}}$ 
        \item $ y=\ln(x+a)\Rightarrow y^{(n)}=(\frac{1}{x+a})^{(n-1)} = \frac{(-1)^{n-1}(n-1)!}{(x+a)^{n}} $ 
    \end{enumerate}
\end{Field}


\section{微分}

\subsection{微分的概念}

\begin{Def}[微分]

    设函数 $ y=f(x) $ 在 $ x=x_0 $ 处有增量 $ \Delta x $ 时,若函数增量 $ \Delta y=f(x_0+\Delta x)=f(x_0) $ 
    可以表为 $ \Delta y=A(x_0)\Delta x+o(\Delta x)(\Delta x\rightarrow 0) $ ,其中 $ A(x_0) $ 与 $ \Delta x $ 
    无关,则称 $ f(x) $ 在 $ x=x_0 $ 处可微,$ \Delta y $ 中的主要线性部分(也称\textbf{线性主部})
    $ A(x_0)\Delta x $ 称为 $ f(x) $ 在 $ x=x_0 $ 处的微分,记为
    $$
        \mathrm{d}y\arrowvert_{x=x_0}\ \textrm{或}\ \mathrm{d}f(x)\arrowvert_{x+0}
    $$ 
\end{Def}

\subsection{可微与导数的的关系}

\begin{Theo}[可微与导数的关系]

    $ f(x) $ 在 $ x=x_0 $ 处可微 $ \Leftrightarrow f(x) $ 在 $ x=x_0 $ 处可导,且
    $ \mathrm{d}x|_{x=x_0} =A(x_0)\Delta x=f'(x_0)\mathrm{d}x $ 
\end{Theo}

一般地,$ y=f(x) $ 则有 $ \mathrm{d}y=f'(x)\mathrm{d}x, $ 故导数 $ f'(x)=\frac{\mathrm{d}y}{\mathrm{d}x} $ 也称微商,即微分之商。

\subsection{微分公式与法则}

\begin{itemize}
    \item $ \mathrm{d}[f(x)\pm g(x)]=\mathrm{d}f(x)\pm\mathrm{d}g(x) $ ;
    \item $ \mathrm{d}[f(x)\cdot g(x)]=g(x)\mathrm{d}f(x)+f(x)\mathrm{d}g(x) $ ;
    \item $ g(x)\neq0 $ 时,$ \mathrm{d}\frac{f(x)}{g(x)}=\frac{g(x)\mathrm{d}f(x)-f(x)\mathrm{d}g(x)}{g^2(x)} $ .
\end{itemize}

\section{微分中值定理}

\subsection{罗尔定理}

\begin{Theo}[罗尔定理]

    若函数满足
    \begin{itemize}
        \item 在闭区间 $ [a,b] $ 连续;
        \item 在开区间 $ (a,b) $ 可导;
        \item $ f(a)=f(b), $ 
    \end{itemize}
    则至少存在一点 $ \xi\in(a,b) $ 使得 $ f'(\xi)=0. $ 
\end{Theo}
注意:
\begin{itemize}
    \item 常用于证明\textbf{形如}$ f(x)=g(x) $ 的方程在一开区间上有根;
    \item 难点:\begin{itemize}
        \item 构造区间;
        \item \textbf{构造辅助函数},其中 $ F(x) $ 一般为 $ f(x)-g(x) $ 的原函数,或 $ f(x)-g(x) $ 的解。
    \end{itemize}
\end{itemize}

\begin{Field}[构造辅助函数的方法]

    \begin{enumerate}
        \item 将等式化为方程;
        \item 变形,直到能够积分为止;具体而言,可以\begin{itemize}
            \item 同乘同除一因式;
            \item 添加能正负相消的两项。
        \end{itemize}
        \item 取不定积分;
        \item 整理,直到满足其在 $ [a,b] $ 上连续,$ (a,b) $ 上可导为止。
    \end{enumerate}
    注意,以上方法仅用于发掘可能的辅助函数,不注重严谨性。
\end{Field}

不难发现,\begin{itemize}
    \item $ xf'(x) + nf(x) = 0 \Rightarrow F(x) = x^n f(x); $ 
    \item $ f(x)g'(x)-f'(x)=0\Rightarrow F(x)=f(x)e^{-g(x)} $ ;
    \begin{itemize}
        \item $ f(x)+f'(x)=0\Rightarrow F(x)=f(x)e^x $ ;
        \item $ f(x)-f'(x)=0\Rightarrow F(x)=f(x)e^{-x} $ ;
        \item $ f(x)+\lambda f'(x)=0\Rightarrow F(x)=f(x)e^{\lambda x} $ ;
    \end{itemize}

\end{itemize}

\subsection{拉格朗日中值定理}

\begin{Theo}[拉格朗日中值定理]

    若函数 $ f(x) $ 满足
    \begin{enumerate}
        \item 在 $ [a,b] $ 上连续;
        \item 在 $ (a,b) $ 上可导, 
    \end{enumerate}
    则至少存在一点 $ \xi \in (a,b) $ 使得 $ f'(\xi)=\dfrac{f(b)-f(a)}{b-a} 
    \Rightarrow f'(\xi)(b-a)=f(b)-f(a).$
\end{Theo}

注意:\begin{enumerate}
    \item 证明:令 $\dis F(x)=f(x)-\frac{f(b)-f(a)}{b-a}x $ 并且利用罗尔定理;
    \item 可证方程在开区间 $ (a,b) $ 上有根;
    \item 可证不等式;
    \item 出现一函数增量 $ \Rightarrow $ 利用拉格朗日中值定理;
    \item 联系 $ f(x) $ 与 $ f'(x) $ 的桥梁;
    \item 求极限时,若出现 $ f(b) - f(a), $ 可能应用拉格朗日得
    $ f'(\xi)(b-a), $ 再利用夹逼定理处理 $ \xi; $ 
    \item 与割线有关。
\end{enumerate}

\begin{Infer}[拉格朗日中值定理推论]

    \begin{itemize}
        \item 若 $ f(x) $ 在 $ (a,b) $ 内可导,且 $ f'(x)\equiv0, $ 则 $ f(x) $ 在 $ (a,b) $ 内为常数;
        \item 若 $ f(x) $ 在 $ (a,b) $ 内可导,且 $ f'(x)\equiv k, $ 则在 $ (a,b) $ 上 $ f(x)=kx+c, $ 
        其中 $ c $ 为常数。
    \end{itemize}
\end{Infer}

\subsection{柯西中值定理}

\begin{Theo}[柯西中值定理]

    设 $ f(x) $ 和 $ g(x) $ 满足
    \begin{enumerate}
        \item 在闭区间 $ [a,b] $ 上皆连续;
        \item 在开区间 $ (a,b) $ 内皆可导;
        \item $ g'(x)\neq 0, $ 
    \end{enumerate}
    则存在 $ \xi\in (a,b) $ 使得 $ \frac{f(b)-f(a)}{g(b)-g(a)}=\frac{f'(\xi)}{g'(\xi)}(a<\xi<b). $ 
\end{Theo}

注意:\begin{enumerate}
    \item 证明:令 $ F(x)=\frac{f(b)-f(a)}{g(b)-g(a)}g(x)-f(x), $ 利用罗尔定理可证;
    \item 可证方程有根;
    \item 出现两个函数增量 $ \Rightarrow $ 使用柯西中值定理证明;
    \item 若要求两个不同的中值,则寻求一特殊点 $ c $ 将 $ a,b $ 分为两子区间,并分别运用中值定理,如罗尔定理。
\end{enumerate}

\subsection{泰勒中值定理}

\begin{Theo}[带拉格朗日余项的泰勒中值定理]

    设 $ f(x) $ 在包含$ x_0 $ 的一邻域$ U_\delta(x_0) $ 内有 $ n+1 $ 阶导数,则对 $ x\in U_\delta, $ 
    有$$
        f(x) = \sum_{i=0}^n \frac{f^{(i)}(x_0)}{n!}(x-x_0)^n + R_n(x)
    $$ 
    其中,$\dis R_n(x)=\frac{f^{(n+1)}(\xi)}{(n+1)!}(x-x_0)^{n+1}, \xi $ 介于 $ x $ 与 $ x_0, $ 
    $ R_n(x) $ 称为拉格朗日余项。
\end{Theo}

\begin{Theo}[带皮亚诺余项的泰勒定理]

    设 $ f(x) $ 在 $ x=x_0 $ 处有 $ n $ 阶导数,则$$
        f(x) = \sum_{i=0}^n \frac{f^{(i)}(x_0)}{n!}(x-x_0)^n + R_n(x)
    $$ 其中 $ R_n(x)=o[(x-x_0)^n],\ x\rightarrow x_0 $ 称为皮亚诺余项。
\end{Theo}

以下是一些常用函数的带皮亚诺余项的马克劳林展开式。

\begin{itemize}
    \item $\dis e^x = \sum_{i=0}^n \dfrac{x^n}{n!}+ o(x^n)$;
    \item $\dis \cos x = 1 - \frac{x^2}{2} + \dfrac{x^4}{24} +\dots + \frac{(-1)^{n}x^{2n}}{(2n)!} + o(x^{2n})$ ;
    \item $\dis \sin x = x - \frac{x^3}{6} + \dots + \frac{(-1)^{n}x^{2n+1}}{(2n+1)!} + o(x^{2n+1}) $ ;
    \item $\dis \arcsin x = x + \frac{x^3}{6} +o(x^3)$ ;
    \item $\dis \tan x = x + \frac{x^3}{3} + o(x^3)$ ;
    \item $\dis \arctan x = x - \frac{x^3}{3} + o(x^3)$ ;
    \item $\dis \ln (1+x) = x - \frac{x^2}{2} + \frac{x^3}{3} + \dots + \frac{(-1)^{n-1}x^n}{n} + o(x^n) $;
    \item $\dis \ln(1-x) = -(x+\frac{x^2}{2} + \frac{x^3}{3}) + o(x^3)$;
    \item $\dis (1+x)^\alpha = 1+\sum_{k=1}^n C_\alpha^kx^k + o(x^n) $ ,
    其中 $\dis C_\alpha^k=\frac{\prod_{i = 0}^{k-1}(\alpha - i)}{k!} $ 

    如,$ \dis \sqrt{1+x} = 1 + \dfrac{1}{2}x - \dfrac{1}{8}x^2 + o(x^2); $ 
    \item $\dis \frac{1}{1-x} = \sum_{i=0}^n x^i + o(x^n) $ ;
    \item $\dis \frac{1}{1+x} = \sum_{i=0}^n (-1)^i x^i + o(x^n) $;
\end{itemize}

注意到泰勒展开时,对展开式的第二项,有
\begin{equation*}
    \begin{aligned}
        \sin x \xLongrightarrow{\textrm{正负变换}}
        \arcsin x \xLongrightarrow{\textrm{去掉阶乘}}
        \tan x \xLongrightarrow{\textrm{正负变换}}
        \arctan x \xLongrightarrow{\textrm{增加阶乘}}
        \sin x .
    \end{aligned}
\end{equation*}

注意,对于泰勒公式,
\begin{enumerate}
    \item \begin{enumerate}
        \item 可证不等式;
        \item 可证方程有根;
        \item 可求等价函数;
        \item 可求极限;
        \item 可求高阶导数值;
        \item 可近似计算;
    \end{enumerate}
    \item 在不等式证明、方程有根题目中应用拉格朗日余项,否则应用皮亚诺余项;
    \item 题目出现高阶(至少2阶)导数 $ \Rightarrow $ 可能应用泰勒中值定理;
    \item 一般在导数已知的点展开;
    \item 使用泰勒公式时,$ x\rightarrow 0 $ 可以推广为 $ \square \rightarrow 0. $ 
\end{enumerate}

\section{导数应用}

\subsection{函数单调判定定理}

\begin{Theo}[函数单调判定定理]

    设函数 $ f(x) $ 在 $ (a,b) $ 可导,若恒有 $ f'(x)>0(<0) $,则 $ f(x) $ 在 $ (a,b) $ 内单调递增(递减);
    若恒有 $ f'(x)\geq0(\leq 0) $ ,则其在 $ (a,b) $ 内单调不减(不增)。
\end{Theo}

注意,\begin{itemize}
    \item 此为充分不必要条件;
    \item 成立条件:要求一区间,而非一点。反例:$$
        f(x)=\begin{cases}
            \frac{x}{2}+x^2\sin \frac{1}{x},& x\neq 0\\
            0,& x=0
        \end{cases}
    $$ 
    \item 可证不等式。
\end{itemize}

\section{极值}

\subsection{定义}

\begin{Def}[极值]

    设函数 $ f(x) $ 在 $ (a,b) $ 内有定义,$ x_0\in (a,b) $ ,则
    \begin{itemize}
        \item 若 $ x_0 $ 存在一邻域 $ U $ ,使得对 $ U $ 内任一点 $ x \neq x_0 $,
        都有 $ f(x)<f(x_0) $ ,则称 $ f(x_0) $ 为函数 $ f(x) $ 的一个极大值,
        $ x_0 $ 为 $ f(x) $ 一个极大值点;
        \item 若 $ x_0 $ 存在一邻域 $ U $ ,使得对 $ U $ 内任一点 $ x \neq x_0 $,
        都有 $ f(x)>f(x_0) $ ,则称 $ f(x_0) $ 为函数 $ f(x) $ 的一个极小值,
        $ x_0 $ 为 $ f(x) $ 一个极小值点;
    \end{itemize}
    函数的极大值和极小值统称极值。极大值点与极小值点统称极值点。
\end{Def}

\subsection{极值的性质}

\begin{itemize}
    \item 若 $ f'(x_0)=0 $ ,则称 $ x_0 $ 为一个驻点;
    \item 若 $ f(x) $ 在 $ x=x_0 $ 处可导,而$ f(x_0) $ 为极值,则 $ f'(x_0)=0, $
    即可导的极值点导数为零;
    \item 极值点 $ \nLeftrightarrow $ 驻点,如 $ f(x)=|x|, x=0; $ 
    \item 可能的极值点:\begin{itemize}
        \item $ f'(x)=0 $ 处;
        \item $ f'(x) $ 不存在处。
    \end{itemize}
\end{itemize}

\subsection{极值的充分条件}

\begin{Theo}[极值第一充分条件]

    设 $ f(x) $ 在 $ x_0 $ 处连续,在 $ 0<|x-x_0|<\delta $ 内可导,$ f'(x_0) $ 不存在或等于0,则
    \begin{enumerate}
        \item 若在 $ (x_0-\delta,x_0) $ 内任一点$ x $ 处有 $ f'(x)>0 $ ,而在 $ (x_0,x_0+\delta) $ 中
        任一点$ x $ 处有 $ f'(x)<0 $ ,则 $ f(x_0) $ 为极大值,$ x_0 $ 为极大值点;
        \item 若在 $ (x_0-\delta,x_0) $ 内任一点$ x $ 处有 $ f'(x)<0 $ ,而在 $ (x_0,x_0+\delta) $ 中
        任一点$ x $ 处有 $ f'(x)>0 $ ,则 $ f(x_0) $ 为极小值,$ x_0 $ 为极小值点;
        \item 若 $ (x_0-\delta,x_0) $ 内任一点与 $ (x_0,x_0+\delta) $ 中任一点
        处导数同号,则 $ f(x_0) $ 不为极值,$ x_0 $ 不为极值点。
    \end{enumerate}
\end{Theo}

\begin{Theo}[极值第二充分条件]

    设函数 $ f(x) $ 在 $ x_0 $ 处存在二阶导数,且 $ f'(x_0)=0 $ , $ f^\pprime(x_0)\neq 0 $ ,
    则\begin{itemize}
        \item 当 $ f^\pprime(x_0)<0 $ 时,$ f(x_0) $ 为极大值,$ x_0 $ 为极大值点;
        \item 当 $ f^\pprime(x_0)>0 $ 时,$ f(x_0) $ 为极小值,$ x_0 $ 为极小值点。
    \end{itemize}
\end{Theo}

\section{最大值、最小值问题}

极值与最值的区别有\begin{enumerate}
    \item 前者局部,后者整体;
    \item 最值可为区间端点,极值不可以;
    \item 区间内部最值必为极值。
\end{enumerate}

最值可用于证明不等式。

应用问题的最值求法的步骤如下。
\begin{enumerate}
    \item 设出合适的自变量;
    \item 表示出目标函数;
    \item 求目标函数的最值。
\end{enumerate}

\subsection{凹向与拐点}

\begin{Def}[凹凸性]

    设 $ f(x) $ 在区间 $ I $ 连续,若对任意两点 $ x_1 \neq x_2 $ ,都有$$
        f(\frac{x_1+x_2}{2})>(<)\frac{1}{2}[f(x_1)+f(x_2)],
    $$ 则称 $ f(x) $ 在 $ I $ 上是凸(凹)的。
\end{Def}

几何上,若曲线 $ y=f(x) $ 上任意两点割线在曲线下(上),则 $ y=f(x) $ 是凸(凹)的。

若 $ f(x) $ 有切线,而每一点的切线都在曲线之上(下),则称 $ y=f(x) $ 是凸(凹)的。

\begin{Theo}[凹凸判定定理]

    设函数在 $ (a,b) $ 上有二阶导数,\begin{itemize}
        \item 若对任意 $ x\in (a,b) $ 都有 $ f^\pprime(x)>0 $ ,则称 $ f(x) $ 是凹函数;
        \item 若对任意 $ x\in (a,b) $ 都有 $ f^\pprime(x)<0 $ ,则称 $ f(x) $ 是凸函数;
    \end{itemize}
\end{Theo}

\begin{Def}[拐点]

    连续曲线上凹凸的分界点是曲线的拐点。
\end{Def}

\begin{Theo}[二阶可导点为拐点的必要条件]

    若 $ (x_0,f(x_0)) $ 为拐点,且 $ f^\pprime(x_0) $ 存在,则 $ f^\pprime(x_0)=0 $ 。
\end{Theo}

\begin{Theo}[二阶可导点为拐点的充分条件]

    设 $ y=f(x) $ 在 $ x=x_0 $ 连续,在 $ x_0 $ 的一去心邻域内二阶可导,\begin{enumerate}
        \item 若二阶导函数在 $ x=x_0 $ 两侧异号,则 $ (x_0,f(x_0)) $ 为一拐点;若同号,则不是拐点;
        \item 若在 $ x_0 $ 处二阶导数为零,而三阶导数不为零,则其为拐点;若三阶导数也为零,则不是拐点。
    \end{enumerate}
\end{Theo}

注意,$ f'(x) $ 的极值点事实上就是 $ f(x) $ 的拐点横坐标。

求拐点的步骤:\begin{enumerate}
    \item 求 $ x_0 $ 使得 $ f^\pprime(x_0)=0 $ 或不存在;
    \item 判断;
    \item 写拐点。
\end{enumerate}

\subsection{渐近线定义及其求法}

\sssubsection{水平渐近线}

若 $ {\displaystyle\lim_{x\rightarrow +\infty}}f(x)=b $
或 $ {\displaystyle\lim_{x\rightarrow -\infty}}f(x) = b, $ 
则称 $ y=b $ 是 $ f(x) $ 的一条水平渐近线。

\sssubsection{垂直渐近线}

若 $ {\displaystyle\lim_{x\rightarrow x_0^+}}f(x)=\infty $ 
或 $ {\displaystyle\lim_{x\rightarrow x_0^-}}f(x)=\infty, $ 则称 $ x=x_0 $ 是 $ f(x) $ 的一条垂直渐近线,也叫铅直渐近线。

垂直渐近线只需要讨论分母为零的点或者函数无定义的端点。

\sssubsection{斜渐近线} 

若 $ {\displaystyle\lim_{x\rightarrow +\infty}}f(x)-(kx+b)=0 $ $ (x\rightarrow-\infty), $ 则称
$ y=kx+b $ 为斜渐近线。

具体而言,若\begin{enumerate}
    \item $ {\displaystyle\lim_{x\rightarrow +\infty}}\frac{f(x)}{x}=k; $ 
    \item $ {\displaystyle\lim_{x\rightarrow +\infty}}f(x)-kx = b, $ 
\end{enumerate}
则有斜渐近线 $ y=kx+b $ 。$ x\rightarrow-\infty $ 时同理。

注意,
\begin{itemize}
    \item 即使 $ {\displaystyle\lim_{x\rightarrow \infty}} \frac{y}{x} $ 存在,斜渐近线也不一定存在;
    \item 一侧不会同时存在水平渐近线和斜渐近线。
\end{itemize}

\section{曲率及曲率半径}

曲率为$$
   k=\frac{|f^\pprime(x)|}{(1+(f'(x))^2)^{\frac{3}{2}}}
$$ 

曲率半径$ \rho $ 为 $ k $ 的倒数。