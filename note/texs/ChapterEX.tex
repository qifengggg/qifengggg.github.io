\begin{Appendices}

\chapter{补充结论}

\sssubsection{反三角公式}

\begin{itemize}
    \item $ \arcsin x + \arccos x = \dfrac{\pi}{2}; $ 
    \item $ \arctan x + \textrm{arccot} x = \dfrac{\pi}{2}; $ 
    \item $ \arctan x + \arctan \dfrac{1}{x} = \begin{cases}
        \dfrac{\pi}{2},& x>0\\
        -\dfrac{\pi}{2},& x<0\\
    \end{cases} $ 
\end{itemize}

\sssubsection{$ n $ 次根式的极限}

\begin{itemize}
    \item $ {\displaystyle\lim_{n\rightarrow \infty}}\dsqrt[n]a = 1\ (a > 0); $ 
    \item $ {\displaystyle\lim_{n\rightarrow \infty}}\dsqrt[n]n = 1; $ 
    \item $ {\displaystyle\lim_{n\rightarrow \infty}}\dsqrt[n]{a_1^n+\cdots+a_n^n} = a_m\ (a_m = \max\{a_i\}); $ 
\end{itemize}

\sssubsection{递推数列求极限}

对数列 $ x_{n+1} = f(x_n) $ 求极限,方法如下。
\begin{itemize}
    \item 适当放缩以证明有界性;
    \item 做差、做商或求导证明单调性;
    \item 若其单调,由单调有界知$ \lim x_n $ 存在;
    \item 令 $ \lim x_n = a, $ 对原式两端取极限,有 $ a = f(a), $ 因此可以解得 $ a; $ 
    \item 若其不单调,则设 $ \lim x_n = a, $ 再利用夹逼定理证明前者确实成立。
\end{itemize}

\sssubsection{均值不等式}

\begin{equation*}
    \begin{aligned}
        \sqrt[n]{a_1\cdots a_n}\leq \dfrac{a_1+\cdots a_n}{n}\leq \dsqrt{\dfrac{a_1^2+\cdots a_n^2}{n}}
    \end{aligned}
\end{equation*}

\newpage

\sssubsection{函数不等式}

\begin{itemize}
    \item $ \sin x < x <\tan x\ ( x \in (0,\dfrac{\pi}{2})); $ 
    \item $ \sin x < x\ (x > 0),\ \ \sin x > x\ (x < 0); $ 
    \item $ e^x > 1 + x\ (x \neq 0); $ 
    \item $ \dfrac{x}{1+x}<\ln(1+x)<x\ (x \in (-1,0)\cup(0,\infty)); $ 
\end{itemize}

\sssubsection{零点定理应用}

若 $ f(x)\in C[0,1],f(0) = f(1), $ 则对任意 $ n\geq 2,\exists \xi\in [0,1] $ 
使得 $ f(\xi+\dfrac{1}{n}) = f(\xi). $ 
\begin{proof}
    构造 $ F(x) = f(x + \dfrac{1}{n}) - f(x), $ 
    由于 $ F(\dfrac{0}{n}),\cdots,F(\dfrac{n-1}{n}) $ 的平均值为 $ 0, $ 
    说明 $ 0 $ 是 $ F(x) $ 的函数值,因此必定有 $ \xi $ 使得
    $ F(\xi) = f(\xi + \dfrac{1}{n}) - f(\xi) = 0. $ 
\end{proof}

\sssubsection{比值的极限推导数}

设 $ f(x) $ 在 $ x = 0 $ 处连续,且 $ {\displaystyle\lim_{x\rightarrow 0}}\dfrac{f(x)}{x} = A, $ 则
$ f(0) = 0,f'(0) = A. $ 

\sssubsection{一类带绝对值函数的可导性}

设 $ f(x) = (x-x_0)^n|x-x_0|, $ 则 $ f(x) $ 在 $ x = x_0 $ 处 $ n $ 阶可导,但 $ n+1 $ 阶不可导。

\sssubsection{和差化积公式与二倍角公式}

\begin{itemize}
    \item 和差化积公式
    \begin{itemize}
        \item $\dis \sin(\alpha\pm\beta) = \sin\alpha\cos\beta+\cos\alpha\sin\beta $
        \item $\dis \cos(\alpha\pm\beta) = \cos\alpha\cos\beta-\sin\alpha\sin\beta $
        \item $\dis \tan(\alpha\pm\beta) = \dfrac{\tan\alpha+\tan\beta}{1+\tan\alpha\tan\beta}$
    \end{itemize}
    \item 二倍角公式
    \begin{itemize}
        \item $\dis \sin 2\alpha = 2\sin\alpha\cos\alpha $
        \item $\dis \cos 2\alpha = \cos^2\alpha - \sin^2\alpha $
        \item $\dis \tan 2\alpha = \dfrac{2\tan\alpha}{1+\tan^2\alpha}$
    \end{itemize}
    \item 降幂公式
    \begin{itemize}
        \item $\dis \sin^2 \alpha = \dfrac{1-\cos 2\alpha}{2} $ 
        \item $\dis \cos^2 \alpha = \dfrac{1+\cos 2\alpha}{2} $
    \end{itemize}
\end{itemize}

\sssubsection{幂指函数求导公式}

若 $ u = u(x),v = v(x) $ 均可导,且 $ u(x)>0, $ 
则有 $ (u^v)' = (e^{v\ln u})' = u^v(v\ln u)'. $ 

\sssubsection{高阶导数值的求法}

求高阶导数值时, 有如下的求法。
\begin{enumerate}
    \item 奇偶性 - 奇函数求偶阶导或偶函数求奇阶导为奇函数。
    \item 递推公式
    \begin{itemize}[parsep = 6pt]
        \item $ [(ax+b)^\alpha]^{(n)} = \dfrac{\alpha !}{(\alpha - n)!}(ax+b)^{\alpha-n}a^n = 
        \alpha(\alpha - 1)\cdots(a-n+1)(ax+b)^{\alpha-n}a^n, $ 
        
        特别地,$ \left(\dfrac{1}{ax+b}\right)^{(n)} = \dfrac{(-1)^nn!a^n}{(ax+b)^{n+1}}; $ 
        \item $ \dis (e^{ax+b})^{(n)} = a^ne^{ax+b},(a^x)^{(n)} = a^x\ln^n a;$ 
        \item $ [\ln(ax+b)]^{(n)} = a\left(\dfrac{1}{ax+b}\right)^{(n-1)} = \dfrac{(-1)^{n-1}(n-1)!a^n}{(ax+b)^n}; $ 
        \item $ [\sin(ax+b)]^{(n)} = a^n\sin(ax+b+\dfrac{n\pi}{2});
        [\cos(ax+b)]^{(n)} = a^n\cos(ax+b+\dfrac{n\pi}{2}). $ 
    \end{itemize}
    \item 莱布尼茨公式 - 乘积的高阶导数
    
    若 $ u = u(x),v = v(x) $ 均 $ n $ 阶可导,则有$$
        (uv)^{(n)} = \sum_{k=0}^n\mathrm{C}_n^k u^{(k)} v^{(n-k)}.
    $$ 
    \item 泰勒公式 - 一般而言,应用于 $ x = 0 $ 处。
\end{enumerate}

\sssubsection{拉格朗日证明包含两点导数的等式}

对区间 $ [a,c],[c,b] $ 分别应用拉格朗日,其中 $ c $ 根据题干结论确定。

\sssubsection{柯西中值定理证明包含两点导数的等式}

\begin{itemize}
    \item 对 $ f(x) $ 使用拉格朗日;
    \item 对 $ f(x),g(x) $ 使用柯西。
\end{itemize}

\sssubsection{导数与单调性的推断}

\begin{itemize}
    \item 已知一点导数符号 $ \nRightarrow $ 单调区间
    
    $ f(x) = \begin{cases}x+2x^2\sin \dfrac{1}{x},& x\neq 0\\ 0,& x = 0\end{cases} $ 
    \item 若 $ f(x) $ 在 $ x = x_0 $ 有一阶连续函数且 $ f'(x_0) > 0, $ 
    则在 $ x = x_0 $ 某邻域内有 $ f'(x)>0, f(x) $ 单调递增。
\end{itemize}

\sssubsection{一个包含 $ e^x, \sin, \cos $ 的积分的结论}

也即上导下抄。

\begin{equation*}
    \begin{aligned}
        &\int e^{\alpha x} \sin \beta x \mathrm{d}x 
        =& \dfrac{ \dis 
        \left|\begin{matrix}
            (e^{\alpha x})' & (\sin \beta x)' \\ 
            e^{\alpha x} & \sin \beta x 
        \end{matrix}\right|
        }{\alpha^2 + \beta^2} + C \\ 
        &\int e^{\alpha x} \cos \beta x \mathrm{d}x 
        =& \dfrac{ \dis 
        \left|\begin{matrix}
            (e^{\alpha x})' & (\cos \beta x)' \\ 
            e^{\alpha x} & \cos \beta x 
        \end{matrix}\right|
        }{\alpha^2 + \beta^2} + C \\ 
    \end{aligned}
\end{equation*}

\sssubsection{微分方程中“任意常数”的写法}

\begin{itemize}
    \item 等式中无 $ \ln \rightarrow C;  $ 
    \item 等式中有 $ \ln (\square) \rightarrow \ln C; $ 
    \item 等式中有 $ \ln |\square| \rightarrow \ln |C|; $ 
\end{itemize}

\sssubsection{$ \Gamma $ 积分}

$$
    \int_0^{+\infty} x^n e^{-x}\mathrm{d}x = n!
$$ 

\sssubsection{偏导数逆问题}

对偏导数求积分时,常数项为不含该未知量的函数,如
\begin{equation*}
    \begin{aligned}
        f^\pprime_{yy}(x,y) = 2 \Rightarrow
        f'_y(x,y) = \int f^\pprime_{yy}(x,y) \mathrm{d}y
        = 2y + \Attention{c(x)}
    \end{aligned}
\end{equation*}
然后利用其他已知条件求解原函数。

\sssubsection{欧拉积分、泊松积分}

欧拉积分可积,但不可求积,其主要包括
\begin{itemize}
    \item $ \dis e^{\pm x^2},e^{1/x},\dfrac{1}{\ln x}; $ 
    \item $ \sin x^2,\sin \dfrac{1}{x}, \dfrac{\sin x}{x}; $ 
    \item $ \cos x^2,\cos \dfrac{1}{x}, \dfrac{\cos x}{x} $ 等。
\end{itemize}

泊松积分为
$$
    \int_{-\infty}^\infty e^{-x^2}\mathrm{d}x = \sqrt{\pi}    
$$ 

\sssubsection{求极坐标下,曲线在某点的切线}

极坐标下显然有 $ \left\{\begin{matrix}
    x = \rho \cos \theta \\ y = \rho \sin \theta
\end{matrix}\right. $ 成立,此时只需要将曲线方程代入上式,
就转化为参数方程。

\sssubsection{求带指数的复杂方程的导数}

对形如 $ e^{f(x)} = g(x) $ 的复杂方程,可以考虑两端取对数后再计算。

\end{Appendices}