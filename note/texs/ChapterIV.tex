\chapter{多元函数微分学}

\section{多元函数微分法}

\subsection{多元函数}

\begin{Def}[二元函数极限]

    设 $ f(x,y) $ 在点 $ (x_0,y_0) $ 某去心邻域有定义,若 $ \forall \varepsilon > 0, \exists \delta > 0 $ 
    使得只要 $ 0 < \dsqrt{}{(x-x_0)^2+(y-y_0)^2}<\delta $ 就有 $ |f(x,y)-A|<\varepsilon $ ,则称
    当 $ (x,y) $ 趋近于 $ (x_0,y_0) $ 时,$ f(x,y) $ 的极限存在,值为 $ A $ ,记为
    $ {\displaystyle\lim_{\begin{subarray}{c}
        x\rightarrow x_0\\y\rightarrow y_0
    \end{subarray}}}f(x,y) = A $ 或者 $ {\displaystyle\lim_{(x,y)\rightarrow (x_0,y_0)}}f(x,y) = A $ ;
    否则,称为极限不存在。
\end{Def}

注意,\begin{enumerate}
    \item 求二重极限可用的方法有
    \begin{itemize}
        \item 极限的四则运算;
        \item 夹逼定理;
        \item 等价代换;
        \item 重要极限;
        \item 无穷小与有界量积仍为无穷小;
        \item 连续性;
        \item 极坐标变换。
    \end{itemize}
    \item 二重积分存在充要条件$$
        {\displaystyle\lim_{(x,y)\rightarrow (x_0,y_0)}}f(x,y) = A \Leftrightarrow
        \textrm{当} (x,y) \textrm{以任意形式趋近于} (x_0,y_0) \textrm{时都有} f(x,y)\rightarrow A
    $$ 
    其逆否命题常用。
\end{enumerate}

\begin{Def}[二元函数连续]

    若 $ {\displaystyle\lim_{(x,y)\rightarrow (x_0,y_0)}}f(x,y) = f(x_0,y_0) $ 
    或者$ {\displaystyle\lim_{(\Delta x,\Delta y)\rightarrow (0,0)}}
    f(x+\Delta x,y + \Delta y) - f(x,y) = 0 $ ,
    则称 $ f(x,y) $ 在 $ (x_0,y_0) $ 连续。
    若 $ f(x,y) $ 在区域 $ D $ 中每一点均连续,则称 $ f(x,y) $ 在 $ D $ 内连续。
\end{Def}

\subsection{偏导数}

\begin{Def}[偏导数]

    设二元函数 $ z = f(x,y) $ 在 $ (x_0,y_0) $ 某邻域内有定义,令 $ y = y_0 $ ,
    给 $ x $ 以增量 $ \Delta x $ ,\nextline 若 $ {\displaystyle\lim_{\Delta x\rightarrow 0}}
    \dfrac{(x_0+\Delta x,y_0)-f(x_0,y_0)}{\Delta x} $ 存在,则称其为 $ z = f(x,y) $ 
    在 $ (x_0,y_0) $ 处关于 $ x $ 的偏导数,\nextline 记为 $ f_x'(x_0,y_0) $ 
    或者 $ \dfrac{\partial z}{\partial x}\Big|_{(x_0,y_0)} $ 
    或者 $ z'_x\Big|_{(x_0,y_0)} $ 。此外,$$
        f_x'(x_0,y_0) = {\displaystyle\lim_{x\rightarrow x_0}}
        \dfrac{f(x,y_0)-f(x_0,y_0)}{x-x_0}
    $$ 
\end{Def}

$ y $ 的偏导数可以类似地定义。

注意,两偏导数在 $ (x_0,y_0) $ 处存在 $ \not\Leftrightarrow f(x,y) $ 在 $ (x_0,y_0) $ 连续。

\begin{Def}[高阶偏导数]

    设对 $ z = (x,y) $ 有 $ f_x' $ 和 $ f_y' $ ,则有
    \begin{itemize}
        \item $ \dfrac{\partial }{\partial x}\left(\dfrac{\partial z}{\partial x}\right) 
        = \dfrac{\partial^2 z}{\partial x^2} = f^\pprime_{xx} $ ;
        \item $ \dfrac{\partial }{\partial y}\left(\dfrac{\partial z}{\partial y}\right) 
        = \dfrac{\partial^2 z}{\partial y^2} = f^\pprime_{yy} $ ;
        \item $ \dfrac{\partial }{\partial y}\left(\dfrac{\partial z}{\partial x}\right) 
        = \dfrac{\partial^2 z}{\partial x\partial y} = f^\pprime_{xy} $ ;
        \item $ \dfrac{\partial }{\partial x}\left(\dfrac{\partial z}{\partial y}\right) 
        = \dfrac{\partial^2 z}{\partial y\partial x} = f^\pprime_{yx} $ ;
    \end{itemize}
    其中后二者称为混合二阶偏导数。
\end{Def}

\begin{Theo}[二阶混合偏导数相等的充分条件]

    若对 $ z = f(x,y) $ 有 $ f^\pprime_{xy} $ 与 $ f^{\prime\prime}_{yx} $ 于 $ (x_0,y_0) $ 连续,
    则有 $ f^\pprime_{xy}(x_0,y_0) = f^\pprime_{yx}(x_0,y_0) $ 。
\end{Theo}

\subsection{全微分}

\begin{Def}[全微分]

    设 $ z = f(x,y) $ 在 $ (x_0,y_0) $ 处有全增量 
    $ \Delta z = f(x_0+\Delta x,y_0 + \Delta y) - f(x_0,y_0) $ ,
    若 $ \Delta z = A\Delta x + B\Delta y + o(\rho),\rho = \dsqrt{(\Delta x)^2+(\Delta y)^2}\rightarrow0 $ 
    其中 $ A,B $ 不依赖于 $ \Delta x,\Delta y $ ,仅与 $ x_0,y_0 $ 有关,则称 $ z = f(x,y) $ 
    在 $ (x_0,y_0) $ 处可微,而 $ A\Delta x+B\Delta y $ 称为 $ z = f(x,y) $ 在 $ (x_0,y_0) $ 处的
    全微分,记为 $ \mathrm{d}z\big|_{(x_0,y_0)} $ 或 $ \mathrm{d}f\big|_{(x_0,y_0)} $
\end{Def}


注意,
\begin{equation*}
    \begin{array}{c}
        z = f(x,y) \textrm{在} (x_0,y_0) \textrm{可微} \\
        \Updownarrow \\
        {\displaystyle\lim_{(x,y)\rightarrow (x_0,y_0)}} 
        \dfrac{\overbrace{f(x,y)-f(x_0,y_0)}^{\dis \Delta z} - 
        \overbrace{f'_x(x_0,y_0)(x-x_0)+f'_y(x_0,y_0)(y-y_0)}^{\dis \mathrm{d}z}}{\dis \sqrt{(x-x_0)^2+(y-y_0)^2}}
    \end{array}
\end{equation*}

\begin{Theo}[可微的必要条件]

    若 $ f(x,y) $ 在 $ (x_0,y_0) $ 可微,则\begin{itemize}
        \item $ f(x,y) $ 在 $ (x_0,y_0) $ 连续;
        \item $ f'_x(x,y),f'_y(x,y) $ 在 $ (x_0,y_0) $ 处均存在。
    \end{itemize}
\end{Theo}

\begin{Theo}[可微的充分条件]

    若 $ f'_x(x,y) $ 和 $ f'_y(x_0,y_0) $ 在 $ (x_0,y_0) $ 均连续,则
    $ f(x,y) $ 在 $ (x_0,y_0) $ 可微。
\end{Theo}

\sssubsection{多元函数几个概念的联系}

% Please add the following required packages to your document preamble:
% \usepackage{booktabs}
\begin{table}[!htbp]\centering
    \begin{tabular}{@{}lcccc@{}}
        &  &  &  & \begin{tabular}[c]{@{}c@{}}$ f(x,y) $ 在 \\ $ (x_0,y_0) $ 连续\end{tabular} \\
        &  &  & $\nearrow$ &  \\
       \begin{tabular}[c]{@{}l@{}}$ f'_x(x,y) $ 和 $ f'_y(x_0,y_0) $ \\ 在 $ (x_0,y_0) $ 均连续\end{tabular} & $\longrightarrow$ & \begin{tabular}[c]{@{}c@{}}$ f(x,y) $ 在 \\ $ (x_0,y_0) $ 可微\end{tabular} &  &  \\
        & \multicolumn{1}{l}{} & \multicolumn{1}{l}{} & \multicolumn{1}{l}{$\searrow$} & \multicolumn{1}{l}{} \\
        &  &  &  & \begin{tabular}[c]{@{}c@{}}$ f'_x(x,y),f'_y(x,y) $ 在\\  $ (x_0,y_0) $ 处均存在\end{tabular}
    \end{tabular}
\end{table}