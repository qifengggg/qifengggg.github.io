\chapter{多元函数微分学}

\section{多元函数微分法}

\subsection{多元函数}

\begin{Def}[二元函数极限]

    设 $ f(x,y) $ 在点 $ (x_0,y_0) $ 某去心邻域有定义,若 $ \forall \varepsilon > 0, \exists \delta > 0 $ 
    使得只要 $ 0 < \dsqrt{}{(x-x_0)^2+(y-y_0)^2}<\delta $ 就有 $ |f(x,y)-A|<\varepsilon $ ,则称
    当 $ (x,y) $ 趋近于 $ (x_0,y_0) $ 时,$ f(x,y) $ 的极限存在,值为 $ A $ ,记为
    $ {\displaystyle\lim_{\begin{subarray}{c}
        x\rightarrow x_0\\y\rightarrow y_0
    \end{subarray}}}f(x,y) = A $ 或者 $ {\displaystyle\lim_{(x,y)\rightarrow (x_0,y_0)}}f(x,y) = A $ ;
    否则,称为极限不存在。
\end{Def}

注意,\begin{enumerate}
    \item 求二重极限可用的方法有
    \begin{itemize}
        \item 极限的四则运算;
        \item 夹逼定理;
        \item 等价代换;
        \item 重要极限;
        \item 无穷小与有界量积仍为无穷小;
        \item 连续性;
        \item 极坐标变换。
    \end{itemize}
    \item 二重积分存在充要条件$$
        {\displaystyle\lim_{(x,y)\rightarrow (x_0,y_0)}}f(x,y) = A \Leftrightarrow
        \textrm{当} (x,y) \textrm{以任意形式趋近于} (x_0,y_0) \textrm{时都有} f(x,y)\rightarrow A
    $$ 
    其逆否命题常用。
\end{enumerate}

\begin{Def}[二元函数连续]

    若 $ {\displaystyle\lim_{(x,y)\rightarrow (x_0,y_0)}}f(x,y) = f(x_0,y_0) $ 
    或者$ {\displaystyle\lim_{(\Delta x,\Delta y)\rightarrow (0,0)}}
    f(x+\Delta x,y + \Delta y) - f(x,y) = 0 $ ,
    则称 $ f(x,y) $ 在 $ (x_0,y_0) $ 连续。
    若 $ f(x,y) $ 在区域 $ D $ 中每一点均连续,则称 $ f(x,y) $ 在 $ D $ 内连续。
\end{Def}

\subsection{偏导数}

\begin{Def}[偏导数]

    设二元函数 $ z = f(x,y) $ 在 $ (x_0,y_0) $ 某邻域内有定义,令 $ y = y_0 $ ,
    给 $ x $ 以增量 $ \Delta x $ ,\nextline 若 $ {\displaystyle\lim_{\Delta x\rightarrow 0}}
    \dfrac{(x_0+\Delta x,y_0)-f(x_0,y_0)}{\Delta x} $ 存在,则称其为 $ z = f(x,y) $ 
    在 $ (x_0,y_0) $ 处关于 $ x $ 的偏导数,\nextline 记为 $ f_x'(x_0,y_0) $ 
    或者 $ \dfrac{\partial z}{\partial x}\Big|_{(x_0,y_0)} $ 
    或者 $ z'_x\Big|_{(x_0,y_0)} $ 。此外,$$
        f_x'(x_0,y_0) = {\displaystyle\lim_{x\rightarrow x_0}}
        \dfrac{f(x,y_0)-f(x_0,y_0)}{x-x_0}
    $$ 
\end{Def}

$ y $ 的偏导数可以类似地定义。

注意,两偏导数在 $ (x_0,y_0) $ 处存在 $ \not\Leftrightarrow f(x,y) $ 在 $ (x_0,y_0) $ 连续。

\begin{Def}[高阶偏导数]

    设对 $ z = (x,y) $ 有 $ f_x' $ 和 $ f_y' $ ,则有
    \begin{itemize}
        \item $ \dfrac{\partial }{\partial x}\left(\dfrac{\partial z}{\partial x}\right) 
        = \dfrac{\partial^2 z}{\partial x^2} = f^\pprime_{xx} $ ;
        \item $ \dfrac{\partial }{\partial y}\left(\dfrac{\partial z}{\partial y}\right) 
        = \dfrac{\partial^2 z}{\partial y^2} = f^\pprime_{yy} $ ;
        \item $ \dfrac{\partial }{\partial y}\left(\dfrac{\partial z}{\partial x}\right) 
        = \dfrac{\partial^2 z}{\partial x\partial y} = f^\pprime_{xy} $ ;
        \item $ \dfrac{\partial }{\partial x}\left(\dfrac{\partial z}{\partial y}\right) 
        = \dfrac{\partial^2 z}{\partial y\partial x} = f^\pprime_{yx} $ ;
    \end{itemize}
    其中后二者称为混合二阶偏导数。
\end{Def}

\begin{Theo}[二阶混合偏导数相等的充分条件]

    若对 $ z = f(x,y) $ 有 $ f^\pprime_{xy} $ 与 $ f^{\prime\prime}_{yx} $ 于 $ (x_0,y_0) $ 连续,
    则有 $ f^\pprime_{xy}(x_0,y_0) = f^\pprime_{yx}(x_0,y_0) $ 。
\end{Theo}

\subsection{全微分}

\begin{Def}[全微分]

    设 $ z = f(x,y) $ 在 $ (x_0,y_0) $ 处有全增量 
    $ \Delta z = f(x_0+\Delta x,y_0 + \Delta y) - f(x_0,y_0) $ ,
    若 $ \Delta z = A\Delta x + B\Delta y + o(\rho),\rho = \dsqrt{(\Delta x)^2+(\Delta y)^2}\rightarrow0 $ 
    其中 $ A,B $ 不依赖于 $ \Delta x,\Delta y $ ,仅与 $ x_0,y_0 $ 有关,则称 $ z = f(x,y) $ 
    在 $ (x_0,y_0) $ 处可微,而 $ A\Delta x+B\Delta y $ 称为 $ z = f(x,y) $ 在 $ (x_0,y_0) $ 处的
    全微分,记为 $ \mathrm{d}z\big|_{(x_0,y_0)} $ 或 $ \mathrm{d}f\big|_{(x_0,y_0)} $
\end{Def}


注意,
\begin{equation*}
    \begin{array}{c}
        z = f(x,y) \textrm{在} (x_0,y_0) \textrm{可微} \\
        \Updownarrow \\
        {\displaystyle\lim_{(x,y)\rightarrow (x_0,y_0)}} 
        \dfrac{\overbrace{f(x,y)-f(x_0,y_0)}^{\dis \Delta z} - 
        \overbrace{f'_x(x_0,y_0)(x-x_0)+f'_y(x_0,y_0)(y-y_0)}^{\dis \mathrm{d}z}}{\dis \sqrt{(x-x_0)^2+(y-y_0)^2}} = 0
    \end{array}
\end{equation*}

\begin{Theo}[可微的必要条件]

    若 $ f(x,y) $ 在 $ (x_0,y_0) $ 可微,则\begin{itemize}
        \item $ f(x,y) $ 在 $ (x_0,y_0) $ 连续;
        \item $ f'_x(x,y),f'_y(x,y) $ 在 $ (x_0,y_0) $ 处均存在。
    \end{itemize}
\end{Theo}

\begin{Theo}[可微的充分条件]

    若 $ f'_x(x,y) $ 和 $ f'_y(x_0,y_0) $ 在 $ (x_0,y_0) $ 均连续,则
    $ f(x,y) $ 在 $ (x_0,y_0) $ 可微。
\end{Theo}

\sssubsection{多元函数几个概念的联系}

% Please add the following required packages to your document preamble:
% \usepackage{booktabs}
\begin{table}[!htbp]\centering
    \begin{tabular}{@{}lcccc@{}}
        &  &  &  & \begin{tabular}[c]{@{}c@{}}$ f(x,y) $ 在 \\ $ (x_0,y_0) $ 连续\end{tabular} \\
        &  &  & $\nearrow$ &  \\
       \begin{tabular}[c]{@{}l@{}}$ f'_x(x,y) $ 和 $ f'_y(x_0,y_0) $ \\ 在 $ (x_0,y_0) $ 均连续\end{tabular} & $\longrightarrow$ & \begin{tabular}[c]{@{}c@{}}$ f(x,y) $ 在 \\ $ (x_0,y_0) $ 可微\end{tabular} &  &  \\
        & \multicolumn{1}{l}{} & \multicolumn{1}{l}{} & \multicolumn{1}{l}{$\searrow$} & \multicolumn{1}{l}{} \\
        &  &  &  & \begin{tabular}[c]{@{}c@{}}$ f'_x(x,y),f'_y(x,y) $ 在\\  $ (x_0,y_0) $ 处均存在\end{tabular}
    \end{tabular}
\end{table}

\subsection{复合函数微分法}

求二阶偏导数时,二阶偏导和一阶偏导的偏导结构相同。

\subsection{隐函数求导}

\begin{Theo}[隐函数存在定理]

    若有
    \begin{itemize}
        \item 函数 $ F(x,y) $ 在以 $ P_0(x_0,y_0) $ 为内点的某区域 $ D\subset R^2 $ 上连续;
        \item $ F(x_0,y_0) = 0 $ (称为初始条件);
        \item 在 $ D $ 内存在连续的偏导数 $ F'_y(x,y) $ ;
        \item $ F'_y(x_0,y_0)\neq0 $ ,
    \end{itemize}
    则在点 $ P_0 $ 的某邻域 $ \bigcup_{P_0}\subset D $ 内,方程 $ F(x,y)=0 $ 唯一地确定一个定义
    在区间 $ (x_0-\alpha,x_0+\alpha) $ 内的隐函数 $ y = f(x) $ ,使得
    \begin{enumerate}
        \item $ f(x_0) = y_0 $ 且 $ x\in (x_0-\alpha,x_0+\alpha) $ 时,$ F(x_0,f(x_)) \equiv 0 $ ;
        \item 函数 $ f(x) $ 在区间 $ (x_0-\alpha,x_0+\alpha) $ 内连续。
    \end{enumerate}
\end{Theo}

对一方程 $ F(x,y,z) = 0 $ 确定隐函数 $ z = z(x,y) $,求导数时,有三种方法。

\begin{enumerate}
    \item 直接求导 - 对方程两边同时关于自变量求偏导,即\begin{equation*}
        \begin{aligned}
            &F'_x \cdot 1 + F'_y \cdot 0 + F'_z \cdot \dfrac{\partial z}{\partial x} = 0\\
            \Rightarrow & \dfrac{\partial z}{\partial x} = -\dfrac{F'_x}{F'_z}
        \end{aligned}
    \end{equation*}
    对 $ y $ 同理;
    \item 全微分 - 对方程两边同时取全微分,有 $ \mathrm{d}F(x,y,z) = 0 $ ,而\begin{equation*}
        \begin{aligned}
            &\mathrm{d}F(x,y,z) = F'_x \mathrm{d}x + F'_y \mathrm{d}y + F'_z\mathrm{d}z\\ 
            \Rightarrow & \mathrm{d}z = -\dfrac{F_x'}{F_z'}\mathrm{d}x - \dfrac{F_y'}{F_y'}\mathrm{d}y
        \end{aligned}
    \end{equation*}
    又 $ \mathrm{d}z = \dfrac{\partial z}{\partial x}\mathrm{d}x + \dfrac{\partial z}{\partial y}\mathrm{d}y $ ,
    故有 $ \dfrac{\partial z}{\partial x} = -\dfrac{F_x'}{F_z'},
    \dfrac{\partial z}{\partial y} = -\dfrac{F_y'}{F_z'}$;
    \item 公式法 - 利用上述结论,直接求得 $ F_x',F_y',F_z' $ 代入。
\end{enumerate}

其中,直接求导法应用广泛。

对方程组 $ \dis \begin{cases}
    F(x,y,u,v) = 0 \\ G(x,y,u,v) = 0
\end{cases} $ 确定的隐函数 $ u = u(x,y) $ 及 $ v = v(x,y) $ ,求偏导数
时,对每个方程求偏导,然后解关于$ \dfrac{\partial u}{\partial x},\dfrac{\partial v}{\partial x} $ 的、新的方程组。


\section{多元函数极值}

\subsection{无条件极值}

\begin{Def}[无条件极值]
    
    设函数 $ z = f(x,y) $ 在点 $ P_0(x_0,y_0) $ 某邻域内有定义,若对该邻域内除 $ P_0 $ 的任意一点 $ P(x,y) $ 
    都有 $ f(x,y)\leq f(x_0,y_0) $ ,则称点 $ P_0 $ 为函数 $ z = f(x,y) $ 的极大值点;类似地,可以定义极小值点。
    前二者统称为极值点,极大值和极小值统称极值。
\end{Def}

事实上,对 $ (x,y) $ 某去心邻域内的任意 $ (x_0,y_0) $ ,
若总有 $ \Delta f = f(x,y) - f(x_0,y_0) < 0 $ ,则极值为极大值点;若$ \Delta f > 0 $ 时则为极小值点;
若任意去心邻域内上述两条都不成立即不保号,则不是极值点。
    
\begin{Theo}[二元函数极值存在的必要条件]

    设函数 $ z = f(x,y) $ 在 $ P_0 $ 某邻域有定义,且存在一阶偏导数,若 $ P_0 $ 是极值点,则其
    必定为驻点。
\end{Theo}

注意,驻点 $ \nLeftrightarrow $ 极值点,仅当可微时极值点才是驻点。

对二元函数,可能的极值点仅需要注意驻点。

\begin{Theo}[二元函数极值存在的充分条件]

    函数 $ z = f(x,y) $ 在 $ P_0 $ 某邻域内有二阶连续偏导数,且 $ P_0 $ 为驻点,
    设 $ A = f^\pprime_{xx}(x_0,y_0),B = f^\pprime_{xy}(x_0,y_0),C = f^\pprime_{yy}(x_0,y_0) $ ,则
    \begin{itemize}
        \item $ AC - B^2 > 0 $ 时, $ P_0 $ 是极值点,且若 $ A>0 $ ,其为极小值点,
        $ A<0 $ 时为极大值点;
        \item $ AC-B^2>0 $ 时其不是极值点;
        \item $ AC-B^2 = 0 $ 时该判别法失效,此时用定义,即判断 $ \Delta f $ 在
        $ P_0 $ 去心邻域是否保号。
    \end{itemize}
\end{Theo}

事实上,该定义可以理解为$$
    \mathrm{d}f = \begin{pmatrix}
        \mathrm{d}x & \mathrm{d}y
    \end{pmatrix}\begin{pmatrix}
        A&B\\B&C
    \end{pmatrix}\begin{pmatrix}
        \mathrm{d}x\\\mathrm{d}y
    \end{pmatrix}
$$ 

当 $ AC-B^2 $ 判别法失效,判定保号性时,若难以正面判定,可以利用
不同情况下的符号证明保号性不成立。

给定 $ F(x,y,z) $ 求 $ z = z(x,y) $ 极值时,对 $ F $ 关于 $ x,y $ 求偏导,并且
令 $ \dfrac{\partial z}{\partial x},\dfrac{\partial z}{\partial y} $ 
都等于零,将得到的式子反带入 $ f $ 求解得到一组点,利用 $ AC-B^2 $ 判别法确定其
其是否为极值点。

\subsection{条件极值}

给定 $ z = f(x,y) $ ,约束条件 $ \varphi(x,y) = 0 $ ,求极值,
解决方法是将其化为无条件极值,具体而言,可以
\begin{itemize}
    \item 直接代入;
    \item 拉格朗日乘数法。
\end{itemize}

对于后者,令 $ L = f(x,y) + \lambda\varphi(x,y) $ ,前者称为拉格朗日函数,$ \lambda $ 是拉格朗日乘数。
对 $ L $ 关于 $ x,y,\lambda $ 都求偏导,解得驻点,利用实际意义判别驻点是否为极值。

有多个约束条件时,有拉格朗日函数 $ L = f(x,y) + \lambda_1\varphi_1 + \lambda_2\varphi_2 $ ,后面同理。

给定方程,在闭区域内求极值,则在边界与区域内部都求极值,然后比较。

\section{二重积分}

\subsection{性质}

\begin{itemize}
    \item $ \dis \iint\limits_{D}kf(x,y)\mathrm{d}\sigma = k \iint\limits_{D}f(x,y)\mathrm{d}\sigma $ ;
    \item $ \dis \iint\limits_{D}f(x,y)\pm g(x,y)\mathrm{d}\sigma 
     = \iint\limits_{D}f(x,y)\mathrm{d}\sigma + \iint\limits_{D}g(x,y)\mathrm{d}\sigma$ ;
    \item $ \dis \iint\limits_{D}f(x,y)\mathrm{d}\sigma = \iint\limits_{D_1}f(x,y)\mathrm{d}\sigma
    + \iint\limits_{D_2}f(x,y)\mathrm{d}\sigma $ ,其中 $ D = D_1\cup D_2 $ ,且后者
    除公共边界不重合;
    \item 若 $ f(x,y)\leq g(x,y) $ , $ (x,y)\in D $ ,有 $ \dis \iint\limits_{D}f(x,y)\mathrm{d}\sigma
    \leq \iint\limits_{D}g(x,y)\mathrm{d}\sigma $ ;
    \item 对 $ m\leq f(x,y)\leq M,(x,y)\in D $ ,有 $ \dis mS \leq \iint\limits_{D}f(x,y)\mathrm{d}\sigma
    \leq MS $ ,其中 $ S $ 为区域 $ D $ 的面积。
    \item $ \dis \left|\iint\limits_{D}f(x,y)\mathrm{d}\sigma\right| \leq \iint\limits_{D}|f(x,y)|\mathrm{d}\sigma $ ;
    \item 积分中值定理 - 设 $ f(x,y) $ 在有界闭区域 $ D $ 上连续,$ S $ 为 $ D $ 的面积,
    必定存在 $ (\xi,\eta)\in D $ 使得 $ \dis \iint\limits_{D}f(x,y)\mathrm{d}\sigma = f(\xi,\eta)\cdot S $ .
\end{itemize}

注意,$ \dis \iint\limits_{D}f(x,y)\mathrm{d}\sigma $ 与 $ f(x,y) $ 在 $ D $ 上
有限条曲线上的值无关。

\subsection{二重积分的计算}

可以将二重积分化为直角坐标下的二次积分计算。

设有界闭区域 $ D = \{(x,y)|a\leq x\leq b,\varphi_1(x)\leq y\leq \varphi_2(x)\} $ ,其中
$ \varphi_1(x),\varphi_2(x) $ 在 $ [a,b] $ 上连续,$ f(x,y) $ 在 $ D $ 上连续,则有
$ \dis \iint\limits_{D}f(x,y)\mathrm{d}\sigma = \iint\limits_{D}f(x,y)\mathrm{d}x\mathrm{d}y
 = \int_a^b\mathrm{d}x\int_{\varphi_1(x)}^{\varphi_2(x)}f(x,y)\mathrm{d}y $ .

在 $ D = \{(x,y)|c\leq y\leq d,\varPsi_1(y)\leq x\leq \varPsi_2(y)\} $ 上同理。

计算时,为了方便计算,可以考虑交换积分次序。

部分二重积分具有如下性质。

\begin{Theo}[对称性]

    若 $ f(x,y) $ 在有界闭区域 $ D $ 上连续,若$ D $ 关于 $ x $ 轴对称,则$$
        \iint\limits_{D}f(x,y)\mathrm{d}\sigma = \begin{cases}
            0,& f(x,y)\textrm{对}y\textrm{为奇函数}\\
            2\iint\limits_{D_1}f(x,y)\mathrm{d}\sigma,& f(x,y)\textrm{对}y\textrm{为偶函数}\\
        \end{cases}
    $$ 
    $ D $ 关于 $ y $ 轴对称,函数关于 $ x $ 轴对称时类似。
\end{Theo}

\begin{Theo}[轮换对称性]

    若积分域 $ D $ 关于 $ y = x $ 对称,或相对于积分域两坐标轴的相对位置相同,又或者将 $ x,y $ 对调后,
    积分域的边界方程不变,则二重积分具有轮换对称性,即$$
        \iint\limits_{D}f(x,y)\mathrm{d}x\mathrm{d}y = \iint\limits_{D}f(y,x)\mathrm{d}x\mathrm{d}y
    $$ 
\end{Theo}

可以将二重积分化为极坐标下的二次积分计算,此时有 $ \mathrm{d}\sigma = r\mathrm{d}r\mathrm{d}\theta $.

将直角坐标化为极坐标时,有$$
    \iint\limits_{D}f(x,y)\mathrm{d}x\mathrm{d}y = 
    \iint\limits_{D}f(r\cos\theta,r\sin\theta)r\mathrm{d}r\mathrm{d}\theta
$$ 

若积分域为圆或圆的一部分时,或被积函数中含有 $ x^2,y^2,xy,x^2+y^2 $ 等$ x,y $ 的二次函数,
可以考虑应用极坐标。

