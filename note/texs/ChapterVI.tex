\chapter{无穷级数}

\section{数项级数}

\subsection{数项级数概念}

\begin{Def}[数项级数]

    无穷多个数 $ u_1,u_2,\dots,u_n,\dots $ 按照一定的顺序相加得到的表达式
     $ \sum_{n=1}^\infty u_n $ 称为数项级数,简称级数,其中 $ u_n $ 称为通项。
\end{Def}

级数是无数个的,也是有序的。

\begin{Def}[级数收敛]

    称 $ S_n = \sum_{k=1}^n u_k $ 为级数的部分和,数列 $ \{S_n\} $ 即为部分和数列。

    若 $ {\displaystyle\lim_{n\rightarrow \infty}}S_n\xlongequal{\exists}S $ ,
    则称级数 $\dis \sum_{n=1}^\infty u_n $ 收敛,且和为 $ S $ ;反之,若极限不存在,则称其
    为发散的,发散级数没有和的概念。
\end{Def}

以下是几个重要级数及其敛散性。

\begin{itemize}
    \item 几何级数 - $\dis \sum_{n=0}^\infty ar^n $ ,当 $ |r|<1 $ 时,其收敛于 $ \dfrac{a}{1-r} $ ;
    若 $ |r|\geq1 $ ,则其发散。
    \item p级数 - $\dis \sum_{n=1}^\infty \dfrac{1}{n^p} $ ,若 $ p > 1 $ ,其收敛;若 $ p\leq 1 $ ,其发散。
\end{itemize}

事实上,有 $\dis \sum_{n=1}^\infty \dfrac{1}{n^2} = \dfrac{\pi^2}{6}. $ 

\section{级数基本性质及其收敛必要条件}

级数有以下性质。

\begin{itemize}
    \item 若 $ \dis \sum_{n=1}^{\infty}u_n,\sum_{n=1}^{\infty}v_n $ 都收敛,则
    $ \dis \sum_{n = 1}^{\infty}(u_n\pm v_n) $ 也收敛,且有
    $ \dis \sum_{n = 1}^{\infty}(u_n\pm v_n) = \sum_{n=1}^{\infty}u_n\pm\sum_{n=1}^{\infty}v_n $ .
    \item 在级数中增加、减少、改变有限项不会影响级数敛散性。
    \item 收敛级数具有结合律,即可以对级数的项任意地添加括号,新级数仍然收敛于同一个和。
    注意,去括号不成立。
    \item 对 $ k \neq 0 $ 有 $ \dis \sum_{n=1}^{\infty}ku_n $ 与  $ \dis \sum_{n=1}^{\infty}u_n $
    具有相同的敛散性。
    \item 级数  $ \dis \sum_{n=1}^{\infty}u_n $ 收敛的必要条件是 $ {\displaystyle\lim_{n\rightarrow \infty}}u_n = 0 $ .
\end{itemize}

\section{正项级数及其敛散性判别法}

\begin{Def}[正项级数]

    若 $ u_n\geq 0 $ ,则有 $\dis \sum_{n=1}^{\infty}u_n $ 称为正项级数。
\end{Def}

\begin{Theo}[正项级数收敛基本定理]

    正项级数 $\dis \sum_{n=1}^{\infty}u_n $ 收敛 $ \Leftrightarrow  S_n $ 有上界。
\end{Theo}

正项级数有以下几种判别法。

\begin{Theo}[直接比较法]

    设当 $ n\geq N $ 时,有 $ 0\leq u_n\leq v_n $ 总是成立,则\begin{itemize}
        \item 若 $\dis \sum_{n=1}^{\infty}v_n $ 收敛,则 $\dis \sum_{n=1}^{\infty}u_n $ 收敛;
        \item 若 $\dis \sum_{n=1}^{\infty}u_n $ 发散,则 $\dis \sum_{n=1}^{\infty}v_n $ 发散。
    \end{itemize}
\end{Theo}

\begin{Theo}[间接比较法]

    设 $ u_n\geq 0, v_n>0,n = 1,2,3,\dots $,且有 $ {\displaystyle\lim_{n\rightarrow \infty}}
    \dfrac{u_n}{v_n} = A  $ ,则\begin{itemize}
        \item 若 $ 0<A<+\infty $ ,两级数具有相同的敛散性;
        \item 若 $ A = 0 $ ,则若 $\dis \sum_{n=1}^{\infty}v_n $ 收敛,有 $\dis \sum_{n=1}^{\infty}u_n $ 收敛;
        \item 若 $ A = +\infty $ ,若 $\dis \sum_{n=1}^{\infty}u_n $ 收敛,则 $\dis \sum_{n=1}^{\infty}v_n $ 收敛。
    \end{itemize}
\end{Theo}

\begin{Theo}[比值判别法]

    设 $ u_n > 0 $ ,有 $ {\displaystyle\lim_{n\rightarrow \infty}}\dfrac{u_{n+1}}{u_{n}} = \rho $ ,
    则\begin{itemize}
        \item 若 $ \rho > 1 $ ,有 $\dis \sum_{n=1}^{\infty}u_n $ 收敛;
        \item 若 $ \rho < 1 $ ,则其发散;
        \item 若 $ \rho = 1 $ ,该判别法失效。
    \end{itemize}
\end{Theo}

\begin{Theo}[根值判别法]

    设 $ u_n \leq 0 $ ,有 $ {\displaystyle\lim_{n\rightarrow \infty}}\sqrt[n]{u_n} = \rho $ ,
    则\begin{itemize}
        \item 若 $ \rho > 1 $ 甚至为$ +\infty $ 时,有 $\dis \sum_{n=1}^{\infty}u_n $ 收敛;
        \item 若 $ \rho < 1 $ ,则其发散;
        \item 若 $ \rho = 1 $ ,该判别法失效。
    \end{itemize}
\end{Theo}

\begin{Theo}[积分准则]

    若函数在 $ [1,+\infty) $ 上非负、连续、递减,则级数 $\dis \sum_{n=1}^{\infty}f(n) $ 与
    积分 $ \dis \int_1^{+\infty}f(x)\mathrm{d}x $ 具有相同的敛散性。
\end{Theo}

注意,判别正项级数敛散性时,\begin{equation*}
    \begin{matrix}
        \textrm{间接比较法} \\\Downarrow\\ 
        \textrm{比值法}\\\Downarrow\\
         \textrm{根值法}\\\Downarrow\\
          \textrm{积分准则}\\\Downarrow\\
        \textrm{直接比较法}\\\Downarrow\\
         \textrm{定义·部分和数列}\\\Downarrow\\
          \textrm{收敛级数定义}
    \end{matrix}
\end{equation*}

此外,若通项为积分或者是抽象的,从直接比较法开始应用,试图放缩被积函数以去除积分号。

\section{任意项级数}

\subsection{交错级数及其敛散性判别法}

\begin{Def}[交错级数]

    若 $\dis u_n>0,\sum_{n=1}^{\infty}(-1)^{n+1}u_n $ 称为交错级数。
\end{Def}

\begin{Theo}[莱布尼茨判别法]

    交错级数 $\dis \sum_{n=1}^{\infty}(-1)^{n+1}u_n $ 满足
    \begin{itemize}
        \item $ u_{n+1}\leq u_n $ 
        \item $ {\displaystyle\lim_{n\rightarrow \infty}}u_n = 0 $ 
    \end{itemize}
    则该交错级数收敛,且 $\dis 0\leq \sum_{n=1}^{\infty}(-1)^{n+1}u_n\leq u_1 $ .
\end{Theo}

\subsection{条件收敛与绝对收敛}

\begin{Def}[条件收敛与绝对收敛]

    若 $ \dis \sum_{n=1}^{\infty}|u_n| $ 收敛,则称 $ \dis \sum_{n=1}^{\infty}u_n $ 绝对收敛。

    若 $ \dis \sum_{n=1}^{\infty}u_n $ 收敛而 $ \dis \sum_{n=1}^{\infty}|u_n| $ 发散,则
    $ \dis \sum_{n=1}^{\infty}u_n $ 称为条件收敛。
\end{Def}

绝对收敛的级数有交换律,而条件收敛的级数没有。

\begin{Theo}[条件收敛必收敛]

    $ \dis \sum_{n=1}^{\infty}|u_n|$ 收敛 $\dis \Rightarrow \sum_{n=1}^{\infty}u_n $ 收敛。
\end{Theo}

\section{幂级数}

\subsection{函数项级数}

\begin{Def}[函数项级数]

    设 $ u_n(x),n=1,2,\dots $ 定义在 $ I $ 上,则 $ \dis \sum_{n=1}^{\infty}u_n(x) $ 
    称为区间 $ I $ 上的函数项级数。
\end{Def}

\begin{Def}[收敛点与收敛域]

    对 $ x_0\in I $ ,若常数项级数 $\dis \sum_{n=1}^{\infty}u_n(x_0) $ 收敛,则称 $ x_0 $ 为函数项级数
    $ \dis \sum_{n=1}^{\infty}u_n(x) $ 的收敛点;反之,若其发散,则为发散点。

    函数项级数 $ \dis \sum_{n=1}^{\infty}u_n(x) $ 的全部收敛点构成的集合是收敛域,全部发散点构成的集合
    是发散域。
\end{Def}

\begin{Def}[和函数]

    和函数 $\dis S(x) =: \sum_{n=1}^{\infty}u_n(x) $ ,其定义域为收敛域。
\end{Def}

\subsection{幂级数}

\begin{Def}[幂级数]

    形如$ \dis \sum_{n=1}^\infty a_n(x-x_0)^n $ 的级数称为 $ (x-x_0) $ 的幂级数,常数 $ a_n $ 称为
    幂级数的系数。
\end{Def}

显然,当 $ x_0 = 0 $ 时,有 $ \dis \sum_{n=0}^{\infty}a_nx^n $ 称为 $ x $ 的幂级数。

\begin{Theo}[阿贝尔定理]

    若幂级数 $\dis \sum u_n(x) $ 在 $ x = x_0 $ 处收敛,则对任意 $ x $ 满足 $ |x| < |x_0| $ ,
    都有$ \dis \sum u_n(x) $ 绝对收敛;
    若在 $ x = x_0 $ 发散,则对任意 $ x $ 满足 $ |x|>|x_0| $ 都有$\dis \sum u_n(x) $ 发散。
\end{Theo}

\subsection{幂级数收敛半径及其求法}

幂级数的收敛半径 $ R $ 可以由以下公式求出。

\begin{itemize}
    \item $ \dis R = {\displaystyle\lim_{n\rightarrow \infty}}\left|\dfrac{a_n}{a_{n+1}}\right| $ 
    \item $ \dis R = {\displaystyle\lim_{n\rightarrow \infty}}\dfrac{1}{\dsqrt[n]{|a_n|}} $ 
\end{itemize}

注意,公式仅对标准幂级数成立,即要求幂函数的次数逐次增长;

收敛区间的中心为 $ x_0 $. 以 $ x = 0 $ 为例子,若存在收敛半径 $ R\geq 0 $ ,则
$ x\in (-R,R) $ 时级数绝对收敛,$ x\in R\big\backslash(-R,R) $ 时级数发散。
若级数有条件收敛点,则其只能在 $ \pm R $ 处,因而最多有两个。

在求收敛半径后,必须写出收敛区间,然后判断端点处敛散性。

\subsection{幂级数的性质}

\sssubsection{幂级数的加减运算}

设 $ \dis \sum_{n=1}^\infty a_nx^n = f(x),|x|< R_1;\sum_{n=1}^\infty b_nx^n = g(x), |x|< R_2 $,则有
$ \dis \sum_{n=1}^\infty (a_n\pm b_n)x^n = f(x)\pm g(x), |x|<\min(R_1,R_2)  $ .

\sssubsection{幂级数的逐项求导与逐项积分}

设幂级数 $ \dis \sum_{n=0}^\infty a_nx^n $ 的收敛半径 $ R>0 $ 且有和函数 $ \dis S(x) = \sum_{n=1}^\infty a_nx^n $ ,则
成立以下性质。
\begin{itemize}
    \item $ S(x) $ 在 $ \overset{\textrm{收敛区间}}{(-R,R)} $ 内可导,且有逐项求导公式$$
        S'(x) = \left(\sum_{n=0}^\infty a_nx^n\right)' = \sum_{n=0}^\infty (a_nx^n)' = \sum_{n=0}^\infty na_nx^{n-1}
    $$ 
    求导后幂级数收敛半径不变,因而 $ S(x) $ 在 $ (-R,R) $ 内有任意阶导数;
    \item $ S(x) $ 在$ \overset{\textrm{收敛区间}}{(-R,R)} $内有逐项积分公式$$
        \int_0^x S(t)\mathrm{d}t = \sum_{n=0}^\infty\int_0^x a_nt^n\mathrm{d}t = \sum_{n=0}^\infty \dfrac{a_n}{n+1}x^{n+1}
    $$ 
    积分后收敛半径不变;
    \item 幂级数的和函数 $ S(x) $ 在收敛域上连续。
\end{itemize}

\sssubsection{单位幂级数}

$ \dis \sum_{n=0}^\infty x^n $ 称为单位幂级数,显然其收敛半径是1,且收敛区间内,有 $ \dis \sum x^n = \dfrac{1}{1-x} $ .

$ \dis \sum_{n=0}^\infty (-1)^{n+1}x^n $ 收敛半径也是1, 且收敛区间内,有 $ \dis \sum x^n = \dfrac{1}{1+x} $ .


\section{函数展开成幂级数}

\subsection{概念}

设 $ f(x) $ 是给定的函数,若存在幂级数 $ \dis \sum_{n=0}^\infty a_n(x-x_0)^n $ 使得任意 $ x\in I $ 
级数 $ \dis \sum_{n=0}^\infty a_n(x-x_0)^n $ 都收敛到 $ f(x) $ ,即 $\dis \forall x\in I, f(x) = \sum_{n=0}^\infty a_n(x-x_0)^n $ ,
则称 $ f(x) $ 在区间 $ I $ 上可展开为 $ x - x_0 $ 的幂级数,称级数 $ \dis \sum_{n=0}^\infty a_n(x-x_0)^n $ 为$ f(x) $ 的幂级数。

\begin{Def}[幂级数展开唯一性]

    若 $ f(x) $ 在 $ x = x_0 $ 可以展开为幂级数 $ \dis \sum_{n=0}^\infty a_n(x-x_0)^n $ ,则该展开式是唯一的,且为
    泰勒级数,即有 $ \dis f(x) = \sum_{n=0}^\infty a_n(x-x_0)^n = \sum_{n=0}^\infty \dfrac{f^{(n)(x_0)}}{n!}(x-x_0)^n$
\end{Def}

\subsection{函数展开成幂级数的条件及其形式}

\sssubsection{泰勒级数与马克劳林级数的定义}

设函数 $ f(x) $ 在 $ x_0 $ 的某一领域 $ |x-x_0|<\sigma $ 内具有任意阶导数,则级数
$ \dis \sum_{n=0}^\infty \dfrac{f^{(n)(x_0)}}{n!}(x-x_0)^n $ 称为函数 $ f(x) $ 在 $ x = x_0 $ 处的泰勒级数。

特别地,若 $ x_0 = 0 $ ,称级数为$ f(x) $ 的马克劳林级数。

\sssubsection{函数展开成为幂级数的充要条件}

设 $ f(x) $ 在 $ |x-x_0|<R $ 内有任意阶导数,其泰勒公式为$$
    f(x) = \sum_{i=0}^n \dfrac{f^{(i)}(x_0)}{i!}(x-x_0)^i + R_n(x)
$$ 其中$ R_n(x) $为 $ n $ 阶余项,
其拉格朗日型为 $ \dis R_n(x) = \dfrac{f^{(n+1)}[x_0+\theta(x-x_0)]}{(n+1)!}(x-x_0)^{n+1},\theta\in(0,1) $ ,
则 $ \dis f(x) = \sum_{n=0}^\infty \dfrac{f^{(n)}(x_0)}{ni!}(x-x_0)^n,|x-x_0|<R $ 的充要条件为
$ {\displaystyle\lim_{n\rightarrow \infty}}R_n(x) = 0,|x-x_0|<R $ .

特别地,当 $ x_0 = 0 $ 时得到函数展开为马克劳林级数的充要条件。

\subsection{函数展开成为幂级数的方法}

\begin{itemize}
    \item 公式法 $$
     \dis f(x) = \sum_{n=0}^\infty \dfrac{f^{(n)}(x_0)}{ni!}(x-x_0)^n,|x-x_0|<R 
    $$
    \item 变量替换;
    \item 逐项求导;
    \item 逐项积分;
    \item 恒等变形、变量代换等。
\end{itemize}

幂级数有8个常用的展开式。

\begin{itemize}
    \item $ \dis e^x = \sum_{n=0}^{\infty}\dfrac{1}{n!}x^n, |x|<\infty $ 
    \item $ \dis \sin x = \sum_{n=0}^{\infty}(-1)^n\dfrac{x^{2n+1}}{(2n+1)!},|x|<\infty $ 
    \item $ \dis \cos x = \sum_{n=0}^{\infty}(-1)^n\dfrac{x^{2n}}{(2n)!},|x|<\infty $ 
    \item $ \dis (1+x)^\alpha = 1 + \sum_{n=1}^{\infty}\dfrac{\mathrm{C}_\alpha^n}{n!}x^n, |x|<1 $ 
    \item $ \dis \dfrac{1}{1-x} = \sum_{n=0}^{\infty}x^n, |x|<1 $ 
    \item $ \dis \dfrac{1}{1+x} = \sum_{n=0}^{\infty}(-1)^nx^n, |x|<1 $ 
    \item $ \dis \ln(1+x) = \sum_{n=1}^{\infty} \dfrac{(-1)^{n+1}x^n}{n!}, -1<x\leq 1 $ 
    \item $ \dis \ln(1-x) = \sum_{n=1}^{\infty} \dfrac{x^n}{n!}, -1\leq x< 1 $ 
\end{itemize}

以上都展成马克劳林级数。若要展成泰勒级数,经过适当处理后可以利用马克劳林级数的结果。

展开的结果尽量合成一个级数,级数后要注收敛范围。

可以利用幂级数求数项级数,即将数项级数视为幂级数的 $ x $ 取了定值。

\section{傅里叶级数}

\subsection{傅里叶级数及傅里叶系数}

设函数 $ f(x) $ 在区间 $ [-\pi,\pi] $ 上黎曼可积,则称公式
\begin{itemize}
    \item $ \dis a_k = \dfrac{1}{\pi}\int_{-\pi}^\pi f(x)\cos kx\mathrm{d}x $ 
    \item $ \dis b_k = \dfrac{1}{\pi}\int_{-\pi}^\pi f(x)\sin kx\mathrm{d}x $ 
\end{itemize}

为函数 $ f(x) $ 的傅里叶系数。称以 $ a_k,b_k $ 为系数的三角级数$$
    \dfrac{a_0}{2} + \sum_{n=1}^{\infty}\left(a_n\cos nx + b_n\sin nx\right)
$$ 
为 $ f(x) $ 的傅里叶级数,记为
$$
    f(x)\sim \dfrac{a_0}{2} + \sum_{n=1}^{\infty}\left(a_n\cos nx + b_n\sin nx\right)
$$ 

类似地,可以定义周期为 $ 2l $ 的函数的傅里叶级数,即

\begin{itemize}
    \item $ \dis a_k = \dfrac{1}{l}\int_{-l}^l f(x)\cos \dfrac{k\pi x}{l}\mathrm{d}x $ 
    \item $ \dis b_k = \dfrac{1}{l}\int_{-l}^l f(x)\sin \dfrac{k\pi x}{l}\mathrm{d}x $
    \item $ \dis f(x) \sim \dfrac{a_0}{2} + 
    \sum_{n=1}^{\infty}\left(a_n\cos \dfrac{n\pi x}{l} + b_n\sin \dfrac{n\pi x}{l}\right) $ 
\end{itemize}

\subsection{傅里叶级数的收敛定理}

设 $ f(x) $ 是周期为 $ 2\pi $ 的周期函数,并满足迪利克雷条件,即
\begin{itemize}
    \item 在一个周期内连续或有有限个第一类间断点;
    \item 在一个周期内有有限个极值点,即单调区间个数有限,
\end{itemize}
则$ f(x) $ 的傅里叶级数收敛,且有

\begin{equation*}
    \begin{aligned}
        \dfrac{a_0}{2} + \sum_{n=1}^{\infty}\left(a_n\cos nx + b_n\sin nx\right)
        = \begin{cases}
            f(x),&x\textrm{为$ f(x) $ 的连续点}\\\dfrac{f(x-0)+f(x+0)}{2},& x\textrm{为$ f(x) $ 的间断点}
        \end{cases}
    \end{aligned}
\end{equation*}

\subsection{奇偶函数的傅里叶级数}

若 $ f(x) $ 是奇函数,则有 $ \dis a_n = \dfrac{1}{l}\int_{-l}^l f(x)\cos \dfrac{n\pi x}{l}\mathrm{d}x = 0,
f(x) = \sum_{n=1}^{\infty}b_n\sin \dfrac{n\pi x}{l} $ ,是为正弦级数。

若 $ f(x) $ 是偶函数,则有 $ \dis b_n = \dfrac{1}{l}\int_{-l}^l f(x)\sin \dfrac{n\pi x}{l}\mathrm{d}x = 0,
f(x) =\dfrac{a_0}{2}+  \sum_{n=1}^{\infty}a_n\cos \dfrac{n\pi x}{l} $ ,是为余弦级数。

以上二级数,都有 $ x\in\{f\textrm{的连续点}\} $.

\subsection{有限区间上的函数的傅里叶级数}

将有限区间上的函数展为正弦级数时,将$ f(x) $ 延拓为周期为 $ 2\pi $ 的奇函数 $ F(x) $ .
此时 $ a_n = 0,\dis b_n = \int_{-\pi}^{\pi} f(x)\sin nx \mathrm{d}x $ ,用分部积分法求 $ b_n $ .
然后有 $ \dis f(x) \xlongequal{\textrm{f(x)延拓的部分}} F(x) \xlongequal{\textrm{连续点}} \sum_{n = 1}^\infty b_n\sin nx$
并给出 $ x $ 的区间。 

展成余弦级数同理。

